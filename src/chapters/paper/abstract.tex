\begin{abstract}
	Generation Z in Indonesia faces difficulties in manual expense tracking despite having adopted digital payment methods like QRIS. This research develops a mobile-based expense tracking system called TrackMyBills that utilizes the Donut model for data extraction from QRIS, transfer, and receipt payment documents without OCR. The system consists of a Flutter mobile application and DonutAPI backend service using two models: a model fine-tuned on the CORD-v2 dataset for paper receipt documents and a custom model on the QRIS-TF dataset for digital payment documents. Comprehensive evaluation was conducted through functionality testing, model performance assessment, and user experience evaluation. Evaluation results show all functional scenarios successfully executed. The base model achieved 73.43\% accuracy, 90.53\% precision, 79.53\% recall, 84.68\% F1-score, and 18.85\% MCER. The custom model achieved 68.78\% accuracy, 100\% precision, 68.78\% recall, 81.50\% F1-score, and 17.20\% MCER. User experience evaluation using the System Usability Scale (SUS) resulted in an average score of 71.83, exceeding the standard threshold of 68, indicating good user satisfaction and the application being rated as usable and non-confusing by the majority of Gen Z respondents.
\end{abstract}

\begin{IEEEkeywords}
	QRIS, Donut, expense tracking system, Gen Z, computer vision, deep learning, OCR-free
\end{IEEEkeywords}