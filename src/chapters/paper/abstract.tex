\begin{abstract}
	Gen Z in Indonesia faces challenges in manual expense tracking despite widespread adoption of digital payment methods such as QRIS (Quick Response Code Indonesian Standard). This research develops TrackMyBills, a mobile-based expense tracking system that utilizes the Donut model for OCR-free data extraction from QRIS payment receipts, bank transfers, and paper receipts. The system comprises a Flutter mobile application and DonutAPI backend service implementing two specialized models: a base model fine-tuned on the CORD-v2 dataset for paper receipt processing and a custom model trained on the QRIS-TF dataset for digital payment documents. 
	
	The research follows Design Science Research Methodology (DSRM) with comprehensive evaluation encompassing functional testing, model performance assessment, and user experience analysis. Functional testing demonstrated successful execution of all testing scenarios. The base model achieved 73.43\% accuracy, 90.53\% precision, 79.53\% recall, 84.68\% F1-score, and 18.85\% mCER. The custom model attained 68.78\% accuracy, 100\% precision, 68.78\% recall, 81.50\% F1-score, and 17.20\% mCER. User experience evaluation using the System Usability Scale (SUS) yielded an average score of 71.83, exceeding the standard threshold of 68, indicating high user satisfaction and demonstrating that the application is perceived as usable and intuitive by the majority of Gen Z respondents. The results validate the effectiveness of OCR-free document processing approaches for financial document understanding and demonstrate the system's capability to significantly improve expense tracking efficiency for digital-native users.
\end{abstract}

\begin{IEEEkeywords}
	QRIS, Donut, expense tracking system, Gen Z, computer vision, deep learning, OCR-free
\end{IEEEkeywords}