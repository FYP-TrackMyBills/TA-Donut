\begin{abstract}
	Generasi Z di Indonesia menghadapi kesulitan dalam pencatatan pengeluaran manual meskipun telah mengadopsi metode pembayaran digital seperti QRIS. Penelitian ini mengembangkan sistem pencatatan pengeluaran berbasis mobile TrackMyBills yang menggunakan model Donut untuk ekstraksi data dari bukti pembayaran QRIS, transfer, dan struk tanpa OCR. Sistem terdiri dari aplikasi mobile Flutter dan backend service DonutAPI yang menggunakan dua model: model fine-tuned pada dataset CORD-v2 untuk struk pembayaran dan model custom pada dataset QRIS-TF untuk dokumen pembayaran digital. Evaluasi komprehensif dilakukan melalui pengujian fungsionalitas, kinerja model, dan pengalaman pengguna. Hasil evaluasi menunjukkan semua skenario fungsional berhasil dijalankan dengan baik. Base model mencapai accuracy 73,43\%, precision 90,53\%, recall 79,53\%, F1-score 84,68\%, dan MCER 18,85\%. Custom model mencapai accuracy 68,78\%, precision 100\%, recall 68,78\%, F1-score 81,50\%, dan MCER 17,20\%. Evaluasi pengalaman pengguna menggunakan System Usability Scale (SUS) menghasilkan skor rata-rata 71,83, melampaui ambang batas standar 68, menunjukkan tingkat kepuasan pengguna yang baik dan aplikasi dinilai mudah digunakan oleh mayoritas responden Gen Z.
\end{abstract}

\begin{IEEEkeywords}
	QRIS, Donut, sistem pencatatan pengeluaran, Gen Z, computer vision, deep learning, tanpa OCR
\end{IEEEkeywords}