\begin{abstract}
	QRIS has become an increasingly popular payment method in Indonesia, especially among Gen Z. However, many Gen Z still face difficulties in manually recording their expenses. This research aims to develop a mobile-based expense tracking system that can assist users, namely Gen Z, in recording their expenses. This system uses the Donut model to extract important information out of payment and receipt images and save it in a structured format to be displayed to the user. The Donut model is an end-to-end SOTA model that can be used to extract information from documents without requiring OCR. The research implemented the DSRM methodology. Model development was carried out in stages, starting from collecting data, data exploration, modeling, and evaluation of the resulting model. The Donut model is fine-tuned on the CORD-v2 dataset for paper payment receipt documents and the QRIS-TF dataset for QRIS and transfer payment documents. With these two types of models, the DonutAPI backend service was developed with FastAPI as a REST API interface for model inference. The TrackMyBills mobile application was developed with Flutter as the user interface. User experience testing shows that the application has a good user satisfaction level, with a SUS score of 71.83 on average, which is above the SUS score threshold of 68. The base model shows an F1-score of 84.68\%, and an mCER of 18.85\%. The custom model shows an F1-score of 81.50\%, and an mCER of 17.20\%. The evaluation results indicate that the TrackMyBills application has a good level of user satisfaction and is considered usable and not confusing by the majority of respondents. This application can help Gen Z in recording their expenses more easily and efficiently.
\end{abstract}

\begin{IEEEkeywords}
	QRIS, Donut, expense tracking system, Gen Z, computer vision, deep learning, OCR-free
\end{IEEEkeywords}