\section{Introduction}

The digital payment landscape in Indonesia has undergone rapid transformation, with QRIS (Quick Response Code Indonesian Standard) emerging as a dominant payment method since its launch by Bank Indonesia in 2019. This transformation has been particularly pronounced among Gen Z, who have embraced digital payment technologies with remarkable enthusiasm. By April 2024, QRIS had attracted 48.12 million users and 31.61 million merchants, facilitating transactions worth Rp 31.65 trillion, representing a substantial 149.46\% annual increase \cite{CNNqris2024, Tempo2024BIQRIS}. Statistical evidence reveals that 38\% of Gen Z users incorporate QRIS into their daily financial activities, while 57\% of this demographic demonstrates a clear preference for non-cash transactions over traditional payment methods \cite{qris2023goodstats, jawapos2024qris}.

Despite this widespread adoption of digital payment technologies, Gen Z faces significant challenges in personal financial management, particularly in expense tracking and recording. Research conducted by \cite{beck2019managing} demonstrates that while Gen Z possesses limited financial literacy, they recognize the critical importance of effective financial management. This paradox becomes more pronounced when examining actual financial behaviors. A comprehensive study involving 521 Indonesian university students revealed that 84\% of students without formal financial education fail to maintain expense records, and remarkably, even among those who have received financial education, 78\% still do not engage in systematic expense tracking \cite{johan2021effect}. This phenomenon extends beyond Indonesian borders, with similar patterns observed globally. American Gen Z individuals express frustration with digital spending that requires "just a few clicks," often reporting that they fail to recognize how rapidly digital transactions can accumulate \cite{lewis2019follow}.

The underlying causes of these expense tracking difficulties are multifaceted but primarily stem from the complexity of information classification found in payment receipts and transaction records. Research by \cite{kaye2014money} identifies that the challenge of categorizing information from various payment proofs leads to unstructured manual recording processes that are both time-consuming and error-prone. This manual approach to financial tracking creates significant barriers for consistent expense monitoring, despite users' awareness of its importance. The cognitive load required for manual data entry, combined with the variety of payment formats and merchant information structures, contributes to tracking abandonment among users who initially intend to maintain financial records.

The behavioral preferences of Gen Z further compound these challenges while simultaneously pointing toward potential solutions. Research indicates that 75\% of Gen Z individuals utilize mobile devices as their primary computing platform, having never experienced a world without smartphones \cite{Campfire2024GenZ}. This mobile-first orientation suggests that effective financial management solutions for this demographic must be designed with mobile accessibility and user experience as primary considerations \cite{wandhe2024new}. The portability and flexibility that mobile devices offer align naturally with Gen Z's lifestyle patterns and technological expectations.

Given these challenges and user preferences, there exists a clear need for an intelligent mobile-based system that can automatically extract and categorize financial information from various payment documents. Such a system should leverage advanced computer vision and deep learning technologies to eliminate manual data entry requirements while maintaining accuracy and reliability. The ideal solution would seamlessly process diverse document types including QRIS payment confirmations, bank transfer receipts, and paper receipts, transforming them into structured, categorized expense records that enable effortless financial tracking for digital-native users.