\section{Introduction}

\subsection{Background}
Quick Response Code Indonesian Standard (QRIS) has become an increasingly popular digital payment method in Indonesia, particularly among Generation Z. Bank Indonesia reported that as of April 2024, the number of QRIS users reached 48.12 million with a total transaction value of Rp 31.65 trillion \cite{CNNqris2024}. As many as 38\% of Gen Z use QRIS in their daily lives, while 57\% of Gen Z prefer non-cash transactions \cite{qris2023goodstats,jawapos2024qris}.

Despite adopting digital payment technology, Gen Z faces challenges in personal financial management. A study by \cite{johan2021effect} involving 521 Indonesian students showed that 84\% of students who had not received financial knowledge did not record their expenses, and even 78\% of those who had received financial knowledge still did not record their expenses. Similar problems occur globally, where Gen Z feels difficulty controlling digital expenses that are "just a few clicks" \cite{lewis2019follow}.

One cause of recording difficulties is the complexity of classifying information on payment receipts, which makes manual recording unstructured and error-prone \cite{kaye2014money}. Considering that 75\% of Gen Z use mobile devices as their primary gadgets \cite{Campfire2024GenZ}, a mobile-first solution is needed that can automate the expense recording process.

The rise of QRIS adoption has created a unique challenge for Gen Z financial management. Traditional expense tracking methods that worked for cash transactions become inadequate when dealing with the volume and variety of digital payment receipts. Each QRIS transaction generates a digital receipt with varying formats depending on the payment application used, such as GoPay, SeaBank, Neobank, or BCA. Similarly, bank transfer receipts and printed paper receipts from merchants each have different layouts and information structures. This format diversity makes manual data entry time-consuming and prone to classification errors.

Research in behavioral economics shows that the psychological disconnect between digital payments and actual money spending, known as the "payment depreciation effect," makes it easier for young adults to overspend without realizing it. Gen Z, having grown up with digital technology, is particularly susceptible to this effect. The convenience of digital payments, while beneficial for commerce, creates a challenge for financial awareness and expense tracking discipline.

\subsection{Research Contribution}
This research develops a mobile-based expense tracking system that uses Computer Vision and Deep Learning technology without OCR to extract data from payment receipts. The main contributions include the development of TrackMyBills, a Flutter-based system for Gen Z with an intuitive interface, implementation of the end-to-end Donut model for payment document data extraction without OCR, development of two specialized models through fine-tuning on the CORD-v2 dataset for receipt payments and a custom QRIS-TF dataset for digital payment documents, and comprehensive evaluation including functionality testing, model performance, and user experience assessment using SUS metrics.

The research addresses a significant gap in existing financial technology solutions by specifically targeting the unique needs and behaviors of Gen Z users in Indonesia. Unlike existing expense tracking applications that require manual data entry or rely on bank API integration, this system provides an intermediate solution that automates data extraction while maintaining user control and privacy. The system's design philosophy emphasizes ease of use, visual appeal, and workflow efficiency that aligns with Gen Z expectations for mobile applications.

\subsection{Paper Structure}
This paper is organized as follows: Section II discusses related research and theoretical foundations. Section III explains the system design and implementation methodology. Section IV presents comprehensive evaluation results including functionality testing, model performance analysis, and user experience assessment. Section V concludes the research and provides recommendations for future development.