\section{Penelitian Terkait dan Dasar Teori}

\subsection{Penelitian Terkait}
Penelitian sebelumnya dalam bidang ekstraksi dokumen finansial telah mengeksplorasi berbagai pendekatan. \cite{penelitian1} menggunakan kombinasi OCR dan machine learning untuk ekstraksi data dari faktur, namun masih memerlukan preprocessing yang kompleks. \cite{penelitian2} mengembangkan sistem berbasis CNN untuk klasifikasi dokumen keuangan dengan akurasi 87\%, tetapi terbatas pada format dokumen yang homogen.

Penggunaan model transformer untuk document understanding menunjukkan hasil yang menjanjikan. LayoutLM \cite{layoutlm} menggabungkan informasi visual dan tekstual untuk pemahaman dokumen, sementara BERT \cite{bert} memberikan foundation yang kuat untuk natural language processing. Namun, penelitian-penelitian tersebut masih memerlukan tahap OCR yang menambah kompleksitas sistem.

\subsection{Model Donut}
Donut (Document Understanding Transformer) merupakan model state-of-the-art yang dapat melakukan ekstraksi informasi dari dokumen secara end-to-end tanpa memerlukan OCR \cite{donut}. Model ini terdiri dari dua komponen utama:

\begin{enumerate}
    \item \textbf{Encoder Visual}: Menggunakan Swin Transformer untuk mengekstrak fitur visual dari gambar dokumen
    \item \textbf{Decoder Tekstual}: Menggunakan BART untuk menghasilkan output terstruktur dalam format JSON
\end{enumerate}

Keunggulan model Donut dibandingkan pendekatan tradisional terletak pada kemampuannya untuk memahami struktur dokumen secara holistik tanpa tahap OCR yang dapat menimbulkan error propagation.

\subsection{Framework Pengembangan}
Sistem dikembangkan menggunakan Flutter untuk aplikasi mobile yang mendukung cross-platform development dengan performa native. Backend service dibangun menggunakan FastAPI yang menyediakan antarmuka REST API yang efisien dan mendukung automatic documentation. Docker digunakan untuk containerization yang memastikan konsistensi deployment di berbagai environment.

\subsection{Metrik Evaluasi}
Evaluasi kinerja model menggunakan metrik standar information extraction: Accuracy, Precision, Recall, F1-score, dan Match Character Error Rate (MCER). Evaluasi pengalaman pengguna menggunakan System Usability Scale (SUS) yang merupakan standar industri untuk mengukur usability dengan skala 0-100, dimana skor di atas 68 dianggap acceptable.