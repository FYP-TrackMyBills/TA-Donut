\section{Related Work}

\subsection{Financial Document Processing}
Traditional financial document processing systems rely heavily on OCR-based approaches, where text extraction precedes information classification and extraction. Recent advances have introduced hybrid approaches combining visual and textual features. LayoutLM \cite{xu2020layoutlm} pioneered document understanding by integrating visual layout with textual information, though still dependent on OCR preprocessing. While more robust than pure OCR methods, these hybrid systems inherit text recognition limitations when processing low-quality images common in mobile payment scenarios.

\subsection{Transformer Architecture}
Transformer models have revolutionized both natural language processing and computer vision through their attention mechanisms \cite{vaswani2017attention}. The architecture's ability to capture long-range dependencies and enable parallel processing makes it particularly suitable for document understanding tasks. BERT \cite{koroteev2021bert} established transformer effectiveness for language understanding, while Vision Transformers like Swin Transformer \cite{liu2021swin} demonstrated superior computer vision performance through hierarchical window-based attention mechanisms. These developments laid the foundation for unified visual-textual document understanding systems.

\subsection{Donut Model}
Donut (Document Understanding Transformer) represents a paradigm shift by enabling end-to-end document understanding without OCR \cite{kim2021donut}. The architecture combines a Swin Transformer visual encoder for spatial feature extraction with a BART decoder \cite{lewis2019bart} for structured JSON generation. This unified approach eliminates OCR error propagation and enables holistic document understanding. Donut demonstrates state-of-the-art performance across document understanding benchmarks, excelling in complex layouts and diverse document types that challenge traditional OCR-based systems. The model's training involves pre-training on synthetic datasets followed by task-specific fine-tuning, making it adaptable to various document processing applications. \autoref{fig:donut_pipeline} illustrates the Donut pipeline, highlighting its end-to-end processing capabilities while \autoref{fig:non_donut_pipeline} contrasts it with traditional OCR-based approaches.

\begin{figure}[htbp]
    \centerline{\includegraphics[width=0.4\textwidth]{images/donut-pipeline.png}}
    \caption{Donut Pipeline}
    \label{fig:donut_pipeline}
\end{figure}

\begin{figure}[htbp]
    \centerline{\includegraphics[width=0.45\textwidth]{images/non-donut-pipeline.png}}
    \caption{Non Donut Pipeline}
    \label{fig:non_donut_pipeline}
\end{figure}

\subsection{Flutter Framework}
Flutter provides comprehensive cross-platform mobile development using a single codebase \cite{flutter2021}. Its widget-based architecture and Dart programming language enable rapid development while maintaining native performance across Android and iOS platforms. For Gen Z-targeted financial applications, Flutter's reactive programming model and extensive UI component library facilitate intuitive, responsive interfaces. The framework's integration capabilities with REST APIs and native device features like camera access make it particularly suitable for document processing applications requiring real-time image capture and server communication.

\subsection{FastAPI Framework}
FastAPI serves as a modern, high-performance web framework for building REST APIs with Python \cite{ramirez2020fastapi}. Its automatic documentation generation, type validation, and asynchronous request handling capabilities make it ideal for machine learning model deployment. The framework's native support for request/response models and integration with popular ML libraries streamline development of scalable backend services. FastAPI's performance characteristics and built-in security features provide the reliability required for production financial document processing systems while maintaining development simplicity.

\subsection{Evaluation Metrics}
Model performance evaluation employs standard information extraction metrics including Accuracy, Precision, Recall, F1-score, and mean Character Error Rate (mCER) \cite{neudecker2021survey}. These metrics provide comprehensive assessment of extraction quality at both field and character levels. User experience evaluation utilizes the System Usability Scale (SUS) \cite{brooke1996sus}, an industry-standard questionnaire providing scores from 0-100 where values above 68 indicate acceptable usability. The SUS methodology enables quantitative assessment of interface effectiveness and user satisfaction, crucial for validating system design decisions targeting Gen Z users.