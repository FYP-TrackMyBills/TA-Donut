\section{Hasil Evaluasi dan Pembahasan}

\subsection{Evaluasi Fungsionalitas Sistem}
Evaluasi fungsionalitas dilakukan untuk memastikan bahwa seluruh fitur sistem dapat bekerja sesuai dengan spesifikasi yang telah ditetapkan. Sebanyak 19 skenario pengujian telah dirancang untuk menguji berbagai aspek fungsionalitas aplikasi TrackMyBills.

\begin{table}[htbp]
\centering
\caption{Rangkuman Hasil Uji Fungsionalitas}
\label{tab:functional-summary}
\begin{tabular}{|c|c|c|}
\hline
\textbf{Kategori Pengujian} & \textbf{Jumlah Skenario} & \textbf{Status} \\
\hline
Input/Upload Gambar & 8 & Berhasil \\
\hline
Preprocessing Gambar & 2 & Berhasil \\
\hline
Ekstraksi Data & 1 & Berhasil \\
\hline
Validasi dan Editing Data & 4 & Berhasil \\
\hline
Penyimpanan Transaksi & 2 & Berhasil \\
\hline
Navigasi Aplikasi & 2 & Berhasil \\
\hline
\textbf{Total} & \textbf{19} & \textbf{100\% Berhasil} \\
\hline
\end{tabular}
\end{table}

Hasil evaluasi menunjukkan bahwa seluruh skenario pengujian (19 dari 19) berhasil dijalankan sesuai dengan ekspektasi. Aplikasi dapat menangani berbagai jenis input gambar termasuk PNG, JPEG, screenshot, dan foto langsung dari kamera. Sistem juga berhasil menangani kasus edge case seperti gambar dengan resolusi rendah dan pembatalan operasi.

\subsection{Evaluasi Kinerja Model}

\subsubsection{Kinerja Base Model (CORD-v2)}
Base model yang di-fine-tune pada dataset CORD-v2 digunakan untuk memproses struk pembayaran kertas. Evaluasi dilakukan terhadap 98 sampel total transaksi dan 275-368 sampel untuk atribut menu.

\begin{table}[htbp]
\centering
\caption{Hasil Evaluasi Base Model}
\label{tab:base-model-results}
\begin{tabular}{|l|c|c|c|c|c|}
\hline
\textbf{Metrik} & \textbf{Accuracy} & \textbf{Precision} & \textbf{Recall} & \textbf{F1-Score} & \textbf{MCER} \\
\hline
Weighted & 73,43\% & 90,53\% & 79,53\% & 84,68\% & 18,85\% \\
\hline
Unweighted & 72,33\% & 89,85\% & 78,77\% & 83,94\% & 20,40\% \\
\hline
\end{tabular}
\end{table}

\begin{table}[htbp]
\centering
\caption{Analisis Per Atribut Base Model}
\label{tab:base-model-detail}
\begin{tabular}{|l|c|c|c|c|c|}
\hline
\textbf{Atribut} & \textbf{Sampel} & \textbf{TP} & \textbf{FP} & \textbf{FN} & \textbf{MCER} \\
\hline
Total transaksi & 98 & 82 & 3 & 13 & 6,86\% \\
\hline
Nama menu & 275 & 174 & 24 & 77 & 29,41\% \\
\hline
Harga menu & 368 & 274 & 27 & 67 & 20,15\% \\
\hline
Jumlah menu & 275 & 216 & 24 & 35 & 29,27\% \\
\hline
\end{tabular}
\end{table}

Base model menunjukkan performa yang sangat baik untuk ekstraksi total transaksi dengan tingkat akurasi 83,67\% (82/98) dan MCER hanya 6,86\%. Model juga menunjukkan precision yang tinggi (90,53\%) yang mengindikasikan bahwa false positive rate rendah, artinya model jarang memprediksi atribut yang tidak ada.

\subsubsection{Kinerja Custom Model (QRIS-TF)}
Custom model yang dilatih pada dataset QRIS-TF dirancang khusus untuk memproses dokumen pembayaran digital QRIS dan transfer dari berbagai aplikasi.

\begin{table}[htbp]
\centering
\caption{Hasil Evaluasi Custom Model}
\label{tab:custom-model-results}
\begin{tabular}{|l|c|c|c|c|c|}
\hline
\textbf{Metrik} & \textbf{Accuracy} & \textbf{Precision} & \textbf{Recall} & \textbf{F1-Score} & \textbf{MCER} \\
\hline
Weighted & 68,78\% & 100,00\% & 68,78\% & 81,50\% & 17,20\% \\
\hline
Unweighted & 64,79\% & 100,00\% & 64,79\% & 78,63\% & 17,94\% \\
\hline
\end{tabular}
\end{table}

\begin{table}[htbp]
\centering
\caption{Analisis Per Atribut Custom Model}
\label{tab:custom-model-detail}
\begin{tabular}{|l|c|c|c|c|c|}
\hline
\textbf{Atribut} & \textbf{Sampel} & \textbf{TP} & \textbf{FP} & \textbf{FN} & \textbf{MCER} \\
\hline
Total transaksi & 99 & 68 & 0 & 31 & 18,90\% \\
\hline
Waktu transaksi & 79 & 30 & 0 & 49 & 12,79\% \\
\hline
ID transaksi & 93 & 20 & 0 & 73 & 42,23\% \\
\hline
Tipe transaksi & 99 & 91 & 0 & 8 & 9,09\% \\
\hline
Aplikasi & 99 & 91 & 0 & 8 & 28,28\% \\
\hline
Target transaksi & 99 & 68 & 0 & 31 & 15,84\% \\
\hline
\end{tabular}
\end{table}

Custom model menunjukkan precision yang sempurna (100\%) karena tidak ada false positive dalam semua prediksi. Model berperforma sangat baik untuk atribut tipe transaksi dan aplikasi dengan tingkat akurasi masing-masing 91,92\% (91/99). Namun, model masih menghadapi tantangan dalam mengekstrak ID transaksi yang kompleks, dengan tingkat akurasi hanya 21,51\% (20/93).

\subsubsection{Analisis Komparatif Model}
Perbandingan kedua model menunjukkan trade-off yang menarik:
\begin{itemize}
    \item \textbf{Base Model}: F1-score lebih tinggi (84,68\% vs 81,50\%) dengan recall yang lebih baik
    \item \textbf{Custom Model}: Precision sempurna (100\%) dengan MCER yang lebih rendah (17,20\% vs 18,85\%)
    \item \textbf{Konsistensi}: Kedua model menunjukkan performa yang stabil pada atribut kunci seperti total transaksi
\end{itemize}

\subsection{Evaluasi Pengalaman Pengguna}
Evaluasi pengalaman pengguna dilakukan menggunakan System Usability Scale (SUS) terhadap 15 responden yang merupakan target pengguna Gen Z. Setiap responden menggunakan aplikasi TrackMyBills untuk menyelesaikan skenario penggunaan standar kemudian mengisi kuesioner SUS.

\begin{table}[htbp]
\centering
\caption{Distribusi Skor SUS Responden}
\label{tab:sus-distribution}
\begin{tabular}{|c|c|c|}
\hline
\textbf{Rentang Skor} & \textbf{Jumlah Responden} & \textbf{Persentase} \\
\hline
85-100 (Excellent) & 3 & 20\% \\
\hline
70-84 (Good) & 8 & 53,3\% \\
\hline
50-69 (OK) & 4 & 26,7\% \\
\hline
0-49 (Poor) & 0 & 0\% \\
\hline
\end{tabular}
\end{table}

\begin{table}[htbp]
\centering
\caption{Statistik Skor SUS}
\label{tab:sus-statistics}
\begin{tabular}{|l|c|}
\hline
\textbf{Metrik} & \textbf{Nilai} \\
\hline
Rata-rata & 71,83 \\
\hline
Median & 70,00 \\
\hline
Skor Tertinggi & 85,00 \\
\hline
Skor Terendah & 50,00 \\
\hline
Standar Deviasi & 9,78 \\
\hline
\end{tabular}
\end{table}

Hasil evaluasi menunjukkan skor SUS rata-rata 71,83, yang berada di atas ambang batas standar usability (68). Menurut interpretasi standar SUS, aplikasi TrackMyBills termasuk dalam kategori "Good" usability. Sebanyak 73,3\% responden memberikan skor di atas ambang batas, menunjukkan bahwa mayoritas pengguna menilai aplikasi mudah digunakan.

\subsection{Analisis Kualitatif Pengalaman Pengguna}
Selain evaluasi kuantitatif SUS, responden juga memberikan feedback kualitatif:

\begin{itemize}
    \item \textbf{Intensi Penggunaan}: 80\% responden menyatakan akan menggunakan aplikasi untuk mempermudah pencatatan pengeluaran
    \item \textbf{Dampak Behavior}: 60\% responden merasa akan menjadi lebih disiplin dalam mencatat pengeluaran
    \item \textbf{Feedback Positif}: Antarmuka yang intuitif, proses ekstraksi yang cepat, dan kemudahan editing data
    \item \textbf{Area Improvement}: Akurasi ekstraksi untuk beberapa format struk, dan fitur kategorisasi yang lebih detail
\end{itemize}

\subsection{Diskusi dan Limitasi}
Hasil evaluasi menunjukkan bahwa sistem TrackMyBills berhasil mencapai tujuan penelitian dalam membantu Gen Z mencatat pengeluaran dengan lebih mudah. Namun, terdapat beberapa limitasi:

\begin{enumerate}
    \item \textbf{Variasi Format}: Model masih memerlukan improvement untuk menangani variasi format dokumen yang lebih luas
    \item \textbf{Kompleksitas Data}: Atribut kompleks seperti ID transaksi masih sulit diekstrak dengan akurasi tinggi
    \item \textbf{Preprocessing}: Beberapa dokumen masih memerlukan cropping manual untuk hasil optimal
    \item \textbf{Dataset Size}: Dataset custom yang relatif kecil (200 sampel) membatasi generalisasi model
\end{enumerate}

Meskipun demikian, hasil evaluasi secara keseluruhan menunjukkan bahwa pendekatan menggunakan model Donut untuk ekstraksi data dokumen finansial tanpa OCR terbukti efektif dan dapat memberikan pengalaman pengguna yang memuaskan bagi target audiens Gen Z.