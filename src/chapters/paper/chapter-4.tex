\section{Evaluation Results and Analysis}

\subsection{Evaluation Methodology}
The TrackMyBills system evaluation follows a three-dimensional approach encompassing functional validation, model performance assessment, and user experience measurement. This comprehensive methodology ensures the system meets both technical performance standards and real-world usability requirements for Gen Z users, aligning with the Design Science Research Methodology framework.

\subsection{Functional Requirements Evaluation}
Functional testing encompasses 19 comprehensive test scenarios covering the complete user workflow from image input through transaction storage. All scenarios include specific preconditions, execution steps, and success criteria to ensure thorough system coverage.

\begin{table}[htbp]
\centering
\caption{Summary of Functional Testing Results}
\label{tab:functional-summary}
\begin{tabular}{|l|c|c|}
\hline
\textbf{Testing Category} & \textbf{Scenarios} & \textbf{Success Rate} \\
\hline
Image Input and Upload & 8 & 100\% \\
Image Preprocessing & 2 & 100\% \\
Data Extraction & 1 & 100\% \\
Data Validation and Editing & 4 & 100\% \\
Transaction Storage & 2 & 100\% \\
Application Navigation & 2 & 100\% \\
\hline
\textbf{Total} & \textbf{19} & \textbf{100\%} \\
\hline
\end{tabular}
\end{table}

Results demonstrate complete functional compliance across all scenarios. The system successfully handles diverse input sources including camera capture and gallery selection, processes various image formats while rejecting unsupported types, and manages user-directed image cropping and automatic processing workflows. Data validation confirms proper handling of user modifications and error prevention, while storage testing validates both successful operations and appropriate error handling.

\subsection{Model Performance Evaluation}

\subsubsection{Base Model Performance}
The base model, fine-tuned on CORD-v2 dataset, demonstrates strong performance for paper receipt processing. Evaluation on 98 total transaction samples and 275-368 menu item samples reveals consistent accuracy across extraction tasks.

\begin{table}[htbp]
\centering
\caption{Base Model Performance Metrics}
\label{tab:base-model-results}
\begin{tabular}{|l|c|c|c|c|c|}
\hline
\textbf{Metric} & \textbf{Accuracy} & \textbf{Precision} & \textbf{Recall} & \textbf{F1-Score} & \textbf{MCER} \\
\hline
Weighted & 73.43\% & 90.53\% & 79.53\% & 84.68\% & 18.85\% \\
Unweighted & 72.33\% & 89.85\% & 78.77\% & 83.94\% & 20.40\% \\
\hline
\end{tabular}
\end{table>

The base model achieves impressive weighted precision of 90.53\%, indicating rare false positive predictions valuable for expense tracking applications. The F1-score of 84.68\% represents strong precision-recall balance suitable for practical deployment. Total transaction amount extraction performs exceptionally with 83.67\% accuracy and 6.86\% MCER, making it highly reliable for primary expense tracking use cases.

\begin{table}[htbp]
\centering
\caption{Base Model Attribute-Level Analysis}
\label{tab:base-model-detail}
\begin{tabular}{|l|c|c|c|c|c|}
\hline
\textbf{Attribute} & \textbf{Samples} & \textbf{TP} & \textbf{FP} & \textbf{FN} & \textbf{MCER} \\
\hline
Total Transaction & 98 & 82 & 3 & 13 & 6.86\% \\
Menu Name & 275 & 174 & 24 & 77 & 29.41\% \\
Menu Price & 368 & 274 & 27 & 67 & 20.15\% \\
Menu Quantity & 275 & 216 & 24 & 35 & 29.27\% \\
\hline
\end{tabular>
\end{table>

\subsubsection{Custom Model Performance}
The custom model, trained on QRIS-TF dataset, addresses Indonesian digital payment document characteristics with distinct performance patterns featuring perfect precision but conservative recall.

\begin{table}[htbp]
\centering
\caption{Custom Model Performance Metrics}
\label{tab:custom-model-results}
\begin{tabular}{|l|c|c|c|c|c|}
\hline
\textbf{Metric} & \textbf{Accuracy} & \textbf{Precision} & \textbf{Recall} & \textbf{F1-Score} & \textbf{MCER} \\
\hline
Weighted & 68.78\% & 100.00\% & 68.78\% & 81.50\% & 17.20\% \\
Unweighted & 64.79\% & 100.00\% & 64.79\% & 78.63\% & 17.94\% \\
\hline
\end{tabular>
\end{table>

Perfect precision (100\%) indicates extremely conservative extraction behavior, eliminating false positive errors crucial for financial applications. The weighted F1-score of 81.50\% remains strong despite conservative approach. Transaction type and payment application extraction achieve exceptional 91.92\% accuracy, while transaction ID extraction faces challenges at 21.51\% accuracy due to format complexity.

\begin{table}[htbp]
\centering
\caption{Custom Model Attribute-Level Analysis}
\label{tab:custom-model-detail}
\begin{tabular}{|l|c|c|c|c|c|}
\hline
\textbf{Attribute} & \textbf{Samples} & \textbf{TP} & \textbf{FP} & \textbf{FN} & \textbf{MCER} \\
\hline
Total Transaction & 99 & 68 & 0 & 31 & 18.90\% \\
Transaction Time & 79 & 30 & 0 & 49 & 12.79\% \\
Transaction ID & 93 & 20 & 0 & 73 & 42.23\% \\
Transaction Type & 99 & 91 & 0 & 8 & 9.09\% \\
Payment Application & 99 & 91 & 0 & 8 & 28.28\% \\
Transaction Target & 99 & 68 & 0 & 31 & 15.84\% \\
\hline
\end{tabular>
\end{table>

\subsubsection{Comparative Analysis}
The dual-model approach proves justified through complementary strengths. The base model demonstrates superior recall (79.53\% vs 68.78\%) and F1-score (84.68\% vs 81.50\%) for comprehensive paper receipt extraction. The custom model's perfect precision eliminates false information risk crucial for financial applications. Both models achieve acceptable MCER values (18.85\% vs 17.20\%) with consistent performance across weighted and unweighted metrics.

\subsection{User Experience Evaluation}

\subsubsection{System Usability Scale Assessment}
User experience evaluation employs SUS methodology with 15 Gen Z participants completing standardized usage scenarios. The evaluation covers the complete expense tracking workflow from image capture through transaction storage.

\begin{table}[htbp]
\centering
\caption{SUS Score Distribution}
\label{tab:sus-distribution}
\begin{tabular}{|l|c|c|}
\hline
\textbf{Score Range} & \textbf{Participants} & \textbf{Percentage} \\
\hline
85-100 (Excellent) & 3 & 20.0\% \\
70-84 (Good) & 8 & 53.3\% \\
50-69 (Acceptable) & 4 & 26.7\% \\
0-49 (Poor) & 0 & 0.0\% \\
\hline
\end{tabular>
\end{table>

SUS evaluation demonstrates strong user acceptance with average score of 71.83, significantly exceeding the standard threshold of 68. Score distribution shows 73.3\% of participants rated the system above acceptability threshold, with no poor ratings indicating consistent positive user experience.

\begin{table}[htbp]
\centering
\caption{SUS Statistical Analysis}
\label{tab:sus-statistics}
\begin{tabular}{|l|c|}
\hline
\textbf{Measure} & \textbf{Value} \\
\hline
Mean Score & 71.83 \\
Median Score & 70.00 \\
Standard Deviation & 9.78 \\
Score Range & 35.00 \\
\hline
\end{tabular>
\end{table>

\subsubsection{Qualitative Feedback Analysis}
Qualitative feedback reveals strong adoption intentions with 80\% of participants expressing willingness to use TrackMyBills for expense tracking. Behavioral impact assessment indicates 60\% believe the system would improve their expense tracking discipline. Participants appreciated automation compared to manual entry methods and praised intuitive interface design. Areas for improvement include enhanced categorization options and improved accuracy for certain receipt types.

\subsection{Performance and Scalability}
System performance testing shows typical extraction requests complete within 3-5 seconds on standard cloud infrastructure, meeting mobile application expectations. The containerized architecture supports horizontal scaling with individual model instances processing multiple concurrent requests. Error handling mechanisms successfully manage various failure modes while maintaining data integrity and application stability.