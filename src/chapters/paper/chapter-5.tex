\section{Kesimpulan dan Saran}

\subsection{Kesimpulan}
Penelitian ini berhasil mengembangkan sistem pencatatan pengeluaran berbasis mobile TrackMyBills yang menggunakan model Donut untuk ekstraksi data dari bukti pembayaran QRIS, transfer, dan struk tanpa OCR. Berdasarkan evaluasi komprehensif yang dilakukan, dapat disimpulkan sebagai berikut:

\begin{enumerate}
    \item \textbf{Fungsionalitas Sistem}: Seluruh 19 skenario pengujian fungsionalitas berhasil dijalankan dengan baik, menunjukkan bahwa sistem dapat menangani berbagai jenis input dan kasus penggunaan sesuai dengan spesifikasi yang ditetapkan.
    
    \item \textbf{Kinerja Model}: Kedua model yang dikembangkan menunjukkan performa yang memuaskan. Base model (CORD-v2) mencapai F1-score 84,68\% dengan MCER 18,85\%, sementara custom model (QRIS-TF) mencapai F1-score 81,50\% dengan MCER 17,20\% dan precision sempurna 100\%.
    
    \item \textbf{Pengalaman Pengguna}: Evaluasi menggunakan System Usability Scale (SUS) menghasilkan skor rata-rata 71,83, melampaui ambang batas standar usability (68). Sebanyak 73,3\% responden memberikan penilaian positif, dengan 80\% menyatakan akan menggunakan aplikasi untuk mempermudah pencatatan pengeluaran.
    
    \item \textbf{Kontribusi Teknologi}: Pendekatan end-to-end menggunakan model Donut terbukti efektif untuk ekstraksi data dokumen finansial tanpa memerlukan tahap OCR yang dapat menimbulkan error propagation.
    
    \item \textbf{Dampak Behavioral}: 60\% responden merasa akan menjadi lebih disiplin dalam mencatat pengeluaran dengan adanya aplikasi TrackMyBills, menunjukkan potensi dampak positif terhadap kebiasaan pengelolaan keuangan Gen Z.
\end{enumerate}

Hasil penelitian ini membuktikan bahwa teknologi Computer Vision dan Deep Learning dapat diimplementasikan secara efektif untuk mengatasi masalah pencatatan pengeluaran manual yang dihadapi Gen Z dalam era pembayaran digital.

\subsection{Saran Pengembangan Selanjutnya}
Berdasarkan hasil evaluasi dan limitasi yang ditemukan, penelitian selanjutnya dapat difokuskan pada area-area berikut:

\begin{enumerate}
    \item \textbf{Unified Model}: Mengembangkan satu model tunggal yang dapat menangani semua jenis dokumen pembayaran (QRIS, transfer, dan struk) dengan task prompt yang sesuai untuk mengurangi kompleksitas sistem.
    
    \item \textbf{Dataset Enhancement}: Memperbesar dan memperbanyak variasi dataset, terutama untuk dokumen pembayaran digital, agar model dapat menangani format yang lebih beragam dengan akurasi yang lebih tinggi.
    
    \item \textbf{Holistic Document Understanding}: Mengembangkan model yang dapat memahami dokumen secara holistik tanpa memerlukan cropping manual, dengan melatih pada dataset yang mencakup seluruh konteks dokumen.
    
    \item \textbf{Fitur Lanjutan}: Menambahkan fitur integrasi dengan layanan perbankan, penyimpanan cloud, dan analisis pengeluaran berbasis AI untuk memberikan insight yang lebih mendalam kepada pengguna.
    
    \item \textbf{Cross-Platform Support}: Mengembangkan versi iOS dan web application untuk menjangkau audiens yang lebih luas.
    
    \item \textbf{Real-world Deployment}: Melakukan studi longitudinal untuk mengukur dampak jangka panjang aplikasi terhadap kebiasaan pengelolaan keuangan pengguna dalam kondisi penggunaan sehari-hari.
\end{enumerate}

Pengembangan lebih lanjut diharapkan dapat meningkatkan akurasi ekstraksi, memperluas cakupan dokumen yang dapat diproses, dan memberikan solusi yang lebih komprehensif untuk kebutuhan pengelolaan keuangan digital Gen Z di Indonesia.