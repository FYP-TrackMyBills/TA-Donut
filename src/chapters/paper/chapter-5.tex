\section{Conclusion}

This research successfully developed TrackMyBills, a mobile expense tracking system addressing Generation Z's challenges in manual expense recording despite widespread QRIS adoption in Indonesia. The system employs OCR-free document processing using fine-tuned Donut transformer models to automatically extract financial information from payment documents.

The solution comprises a Flutter mobile application and FastAPI backend service with two specialized models: a base model (CORD-v2 dataset) achieving 84.68\% F1-score for paper receipts, and a custom model (QRIS-TF dataset) achieving 81.50\% F1-score with perfect precision for digital payments. Comprehensive evaluation demonstrates 100\% functional test success rate and 71.83 SUS score, exceeding industry usability standards with strong user acceptance among Generation Z participants.

Key contributions include demonstrating OCR-free document processing viability for financial applications, developing dual-model architecture optimized for different document types, and creating a mobile-first solution specifically designed for Generation Z preferences. Results indicate 80\% user adoption willingness and 60\% belief in improved expense tracking discipline, validating the system's effectiveness in addressing real-world financial management challenges.

\subsection{Future Work}
Based on evaluation findings, future research directions include developing a unified model capable of handling all document types using task prompts to reduce system complexity, creating comprehensive datasets without information removal for holistic document understanding, and eliminating manual image cropping requirements. System enhancements should focus on banking service integration for automatic transaction import, historical transaction editing capabilities, and cloud database integration for multi-device access and data security. Additional improvements include enhanced model training with diverse payment document data and expanded platform support for broader user accessibility.