\clearpage
\chapter*{ABSTRAK}
\addcontentsline{toc}{chapter}{ABSTRAK}
\begin{center}
	\center
	\begin{singlespace}
		\large\bfseries{\thetitle}

		\normalfont\normalsize
		Oleh:

		\bfseries \theauthor
	\end{singlespace}
\end{center}

\begin{singlespace}
	% \small
	QRIS menjadi metode pembayaran yang semakin populer di Indonesia, terutama di kalangan Gen Z. Namun, banyak Gen Z yang masih kesulitan dalam mencatat pengeluaran mereka secara manual. Tugas akhir ini bertujuan untuk mengembangkan sistem pencatatan pengeluaran berbasis \emph{mobile} yang dapat membantu pengguna, yaitu Gen Z, dalam mencatat pengeluaran mereka dengan lebih mudah. Sistem ini menggunakan model \donut{} untuk mengekstrak informasi penting dari gambar bukti pembayaran dan menyimpannya dalam format yang terstruktur untuk kemudian dapat ditampilkan kepada pengguna. Model \donut{} adalah model SOTA \emph{end-to-end} yang dapat digunakan untuk mengekstrak informasi dari dokumen tanpa memerlukan OCR. Metodologi penelitian menggunakan metodologi \dsrm. Pengembangan dilakukan secara bertahap, dimulai dari mengumpulkan data, eksplorasi data, \emph{modelling}, dan evaluasi terhadap model yang dihasilkan. Model \donut{} di \emph{fine-tune} pada \dataset{} CORD-v2 untuk dokumen struk pembayaran kertas dan \dataset{} QRIS-TF untuk dokumen pembayaran QRIS dan transfer. Dengan dua jenis model tersebut, layanan \emph{backend} DonutAPI dikembangkan dengan FastAPI sebagai antarmuka REST API untuk inferensi model. Aplikasi \emph{mobile} TrackMyBills dikembangkan dengan Flutter sebagai antarmuka pengguna. Pengujian pengalaman pengguna menunjukkan bahwa aplikasi memiliki tingkat kepuasan pengguna yang baik, dengan nilai SUS di angka \textbf{71,83} rata-rata di atas ambang batas nilai SUS, yaitu pada \textbf{68}.  \emph{Base model} menunjuk \fscore{} 84,68\%, dan \mcer{} 18,85\%. \emph{Custom model} menunjukkan  \fscore{} 81,50\%, dan \mcer{} 17,20\%. Hasil evaluasi menunjukkan bahwa aplikasi TrackMyBills memiliki tingkat kepuasan pengguna yang baik dan dinilai \emph{usable} dan tidak membingungkan oleh mayoritas responden. Aplikasi ini dapat membantu Gen Z dalam mencatat pengeluaran mereka dengan lebih mudah dan efisien.

	\textbf{Kata kunci: QRIS, Donut, Sistem Pencatatan Pengeluaran, Gen Z, Tanpa OCR}

\end{singlespace}
\clearpage