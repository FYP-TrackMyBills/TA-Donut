\chapter{Pendahuluan}
\label{chapter:pendahuluan}

\section{Latar Belakang}
\label{sec:latar-belakang}

Transaksi pembayaran merupakan bagian penting dari aktivitas ekonomi, baik dalam lingkup individu maupun organisasi. Proses ini mencakup pengalihan dana antara pihak yang terlibat untuk memenuhi kewajiban finansial. Dalam perkembangannya, metode pembayaran telah bertransformasi dari yang awalnya menggunakan uang tunai dan cek menjadi transfer bank dan sistem yang lebih modern berbasis digital. Transformasi ini bertujuan untuk meningkatkan efisiensi, keamanan, dan kenyamanan dalam melakukan transaksi.

\qrisfull{} menjadi salah satu faktor pesatnya perkembangan teknologi pembayaran digital di Indonesia. \qris{} diluncurkan oleh Bank Indonesia pada tahun 2019 dan menjadi metode pembayaran yang populer dalam waktu kurang dari lima tahun karena kemudahannya \parencite{qris}. Generasi Z, sebagai pengguna utama metode pembayaran ini, merasakan kemudahan dalam melakukan pembayaran atau pengeluaran uang. Goodstats menunjukkan bahwa 38\% Gen Z menggunakan \qris{} dalam kehidupan sehari-hari \parencite{qris2023goodstats}.

% , sedangkan di kalangan milenial angkanya mencapai 25\% .

Pada April 2024, Bank Indonesia melaporkan jumlah pengguna \qris{} mencapai angka 48,12 juta dan jumlah \emph{merchant} yang menggunakan mencapai 31,61 juta \parencite{CNNqris2024}. Total nilai transaksi dengan menggunakan metode pembayaran \qris{} telah mencapai Rp 31,65 triliun, yaitu meningkat sebesar 149,46\% secara tahunan pada bulan Februari 2024 \parencite{Tempo2024BIQRIS}. Peningkatan ini mencerminkan adopsi yang signifikan di kalangan masyarakat, khususnya Gen Z, yang cenderung memilih transaksi nontunai. Survei dari Jawa Pos Radar Lawu mengungkapkan bahwa 57\% dari kedua generasi ini lebih memilih transaksi nontunai, dengan \qris{} menjadi salah satu metode yang paling efektif \parencite{jawapos2024qris}.

Studi yang dilakukan oleh \cite{beck2019managing} tentang mengatur keuangan pribadi menunjukkan bahwa Gen Z memiliki pengetahuan akan finansial yang cukup terbatas, tetapi mereka paham akan pentingnya pengelolaan keuangan yang baik. \cite{johan2021effect} melakukan studi terhadap 521 mahasiswa Indonesia di IPB tentang hubungan pengetahuan finansial terhadap perilaku keuangan. Studi tersebut mengungkapkan bahwa 84\% dari mahasiswa yang belum mendapatkan pengetahuan finansial tidak melakukan pencatatan pengeluaran mereka. Hal yang serupa juga terjadi walaupun pengetahuan finansial telah diajarkan. Sebanyak 78\% dari mahasiswa yang telah mendapatkan pengetahuan finansial masih belum melakukan pencatatan pengeluaran. 

Peristiwa serupa juga terjadi pada Gen Z di Amerika Serikat. Berdasarkan studi yang dilakukan \cite{lewis2019follow}, salah satu responden merasa frustrasi dan kesulitan untuk mengontrol pengeluaran digital yang "hanya beberapa klik" dalam wawancara terstrukturnya. Responden lainnya, dengan pemasukan dari uang saku, mengungkapkan bahwa ia tidak menyadari bahwa transaksi digital bisa tertumpuk dengan sangat cepat \parencite{lewis2019follow}. Studi tentang \emph{financial tracking} oleh \cite{kaye2014money} menunjukkan bahwa salah satu alasan kesulitan pencatatan pengeluaran karena sulitnya klasifikasi informasi pada bukti dan struk pembayaran yang menyebabkan pencatatan pada \emph{platform} secara manual menjadi tidak terstruktur. Walaupun Gen Z telah mengadopsi teknologi pembayaran digital, mereka masih merasa malas untuk mengelola dan mencatat pengeluaran mereka secara efektif. Banyak pengguna masih harus melakukan pencatatan manual untuk memantau pengeluaran mereka yang rentan terhadap kesalahan dan memakan waktu.

% Kemudahan ini mendorong peningkatan frekuensi transaksi digital di kalangan pengguna yang mengakibatkan kenaikan pada angka volume transaksi. Namun, seiring dengan meningkatnya volume transaksi, belum terdapat sistem dengan mekanisme pencatatan transaksi. Banyak pengguna masih harus melakukan pencatatan manual untuk memantau pengeluaran mereka yang rentan terhadap kesalahan dan memakan waktu. 

Salah satu solusi potensial untuk masalah ini adalah penerapan teknologi berbasis \cvfull{} dan \dl{}. Implementasi teknologi tersebut memiliki kapabilitas untuk pemindaian dan ekstraksi data dari dokumen gambar secara akurat, seperti \ocrfull, \cnnfull, \transformer, dan lainnya. Hasil ekstraksi tersebut dapat dipetakan pada atribut terkait tanpa perlu adanya upaya manual untuk melakukan klasifikasi atribut.

% \layoutlm{} dan \bert{} merupakan model \transformer{} yang menggabungkan informasi visual dan tekstual untuk memahami dokumen. Oleh karena itu, sistem yang dapat mengonversi dan mengekstrak data dari bukti pembayaran berbasis kertas menjadi dokumen digital.

\section{Rumusan Masalah}
\label{sec:rumusan-masalah}

Berdasarkan latar belakang dan masalah yang dihadapi oleh individu untuk melakukan mencatat pengeluaran, dapat dirumuskan masalah yang akan diselesaikan adalah sebagai berikut.

\begin{center}
	% "Bagaimana sistem yang dapat mengekstraksi dan mengklasifikasikan hasi ekstraksi dari bukti pembayaran QRIS, transfer, dan struk pembayaran berbasis kertas dengan format bervariasi?"
	"Bagaimana sistem yang dapat mempermudah Gen Z untuk melakukan pencatatan transaksi? Transaksi berupa pembayaran dari aplikasi dan cetak (QRIS, transfer, dan struk pembayaran)"
\end{center}

\section{Tujuan}
\label{sec:tujuan}

Berdasarkan rumusan masalah yang telah didefinisikan pada \autoref{sec:rumusan-masalah}, didapatkan jawaban dari rumusan masalah dan hasil akhir yang ingin diperoleh, yaitu sistem yang dapat mengekstrak dan memetakan data dari bukti pembayaran QRIS, transfer, dan struk pembayaran cetak berbasis kertas dan bervariasi untuk mempermudah proses pencatatan pengeluaran.

\section{Batasan Masalah}
\label{sec:batasan-masalah}

Dalam pemenuhan solusi untuk menjawab masalah yang ada, terdapat berbagai jenis batasan yang perlu dipertimbangkan. Batasan-batasan tersebut ditentukan berdasarkan jangka waktu yang tersedia dan lingkup masalah yang akan diselesaikan. Berikut adalah batasan selama pengerjaan tugas akhir ini:
\begin{enumerate}
	\item Sistem hanya dapat memproses dokumen keuangan, seperti bukti pembayaran QRIS, transfer, serta struk yang dicetak dan tidak ditulis
	\item Dokumen keuangan yang diterima hanya dalam format gambar digital dengan format JPG, JPEG, atau PNG dengan kualitas gambar yang jelas dan tidak blur.
	\item  Pengembangan sistem dibatasi untuk memproses dokumen dari bank-bank digital ternama, yaitu SeaBank dan Neobank, bank konvensional ternama, yaitu BCA, dan aplikasi pembayaran digital populer, yaitu \gopay{}.
	\item Data yang diekstrak dari bukti pembayaran QRIS dan transfer dibatasi pada informasi penting transaksi yaitu:
	      \begin{enumerate}
		      \item Nominal pembayaran
		      \item ID transaksi
		      \item Aplikasi pembayaran
		      \item Penerima pembayaran
		      \item Tanggal dan waktu transaksi
		      \item Tipe transaksi
	      \end{enumerate}
	\item Data yang diekstrak dari struk pembayaran dibatasi pada informasi penting transaksi yaitu:
	      \begin{enumerate}
		      \item Nominal pembayaran
		      \item Barang-barang transaksi
		      \item Jumlah tiap barang 
		      \item Harga tiap barang
	      \end{enumerate}
	\item Sistem dikembangkan untuk memproses satu dokumen bukti pembayaran dalam satu waktu.
	\item Sistem dirancang untuk beroperasi dalam \emph{platform mobile} Android, tidak mencakup pengembangan untuk \emph{platform desktop} dan iOS.
	\item Implementasi tidak mencakup integrasi dengan sistem akuntansi atau sistem pencatatan keuangan pihak ketiga.
	\item Pengembangan antarmuka pengguna dibatasi pada fitur-fitur dasar yang diperlukan untuk mengunggah gambar, memproses, dan menampilkan hasil ekstraksi data.
	\item Pengembangan sistem tidak mencakup fitur keamanan tingkat lanjut, seperti enkripsi data atau otentikasi pengguna.
	\item Sistem tidak akan menggunakan \emph{on-device inference} untuk model \dl{} yang digunakan, sehingga memerlukan koneksi internet untuk melakukan ekstraksi data.
	\item Pengembangan sistem tidak mencakup fitur untuk mengelola atau menyimpan data pengguna pada server, sehingga data hanya akan disimpan secara lokal.
\end{enumerate}

\section{Metodologi}
\label{sec:metodologi}

Metodologi yang akan digunakan pada tugas akhir ini adalah \dsrmfull. Metodologi \dsrm{} menjadi metodologi yang cocok untuk digunakan karena menggabungkan prinsip, praktik, dan prosedur yang diperlukan, serta memenuhi tujuan menghasilkan artefak (sistem) yang dapat digunakan untuk mempresentasikan dan mengevaluasi penelitian dalam bidang Sistem Informasi \parencite{peffers2007dsrm}. Tahapan \dsrm{} tergambar pada \autoref{fig:dsrm}.

\begin{figure}[htbp]
	\centering
	\includegraphics[width=.8\textwidth]{images/dsrm.png}
	\caption{Proses dalam Metodologi \dsrm{} \parencite{peffers2007dsrm}.}
	\label{fig:dsrm}
\end{figure}

\newpage

\dsrm{} mencakup beberapa tahapan, yakni sebagai berikut:
\begin{enumerate}
	\item \emph{Problem Identification and Motivation}~\\
	      Tahapan \emph{problem identification and motivation} bertujuan untuk mengidentifikasi masalah yang ada dan ingin diselesaikan. Pada tahap ini, peneliti melakukan analisis terhadap permasalahan yang ada, yaitu mencari cara untuk mengembangkan sistem yang dapat mengonversi dan mengekstrak data dari bukti-bukti pembayaran digital dan struk pembayaran kertas yang dicetak digital dengan format yang bervariatif menjadi dokumen digital seperti yang telah dijelaskan pada \autoref{sec:latar-belakang} dan \autoref{sec:rumusan-masalah}.
	\item \emph{Define the Objectives for a Solution}~\\
	      Pada tahapan ini, peneliti akan mendefinisikan tujuan dari solusi yang akan dikembangkan. Tujuan dari penelitian ini adalah mengembangkan sistem berbasis \emph{computer vision} dan \dl{} untuk memindai, mengonversi, dan mengekstrak data dari bukti pembayaran dengan format yang berbeda-beda secara akurat dan cepat untuk mendukung pencatatan dan pengelolaan transaksi keuangan secara digital seperti yang telah dijelaskan pada \autoref{sec:tujuan}.
	\item \emph{Design and Development}~\\
	      Tahapan \emph{design and development} merupakan tahapan untuk merancang dan mengembangkan sistem yang akan digunakan untuk menyelesaikan masalah yang telah diidentifikasi. Pada tahap ini, peneliti akan merancang arsitektur sistem, membuat rancangan tahapan desain dari arsitektur sistem, dan mengembangkan sistem sesuai dengan spesifikasi yang telah ditentukan.	Peneliti juga akan mengumpulkan dan melakukan eksplorasi pada data yang akan digunakan untuk melatih model \dl{} yang akan digunakan dalam sistem. Data yang digunakan adalah data bukti pembayaran digital dan struk pembayaran kertas yang dicetak digital dengan format yang bervariatif. Eksplorasi data dilakukan untuk mengidentifikasi variasi data yang diperlukan untuk mengembangkan sistem sesuai dengan spesifikasi yang diinginkan.
	\item \emph{Demonstration}~\\
	      Tahapan \emph{demonstration} merupakan tahapan untuk mendemonstrasikan sistem yang telah dikembangkan. Pada tahap ini, peneliti akan melakukan pengujian terhadap sistem yang telah dikembangkan untuk memastikan bahwa sistem dapat berfungsi sesuai dengan spesifikasi yang telah ditentukan. Pengujian dilakukan dengan menggunakan \dataset{} asli yang telah dikumpulkan.
	\item \emph{Evaluation}~\\
	      Tahapan \emph{evaluation} merupakan tahapan untuk mengevaluasi sistem yang telah dikembangkan. Pada tahap ini, peneliti akan melakukan evaluasi terhadap sistem yang telah dikembangkan dengan metrik yang telah didefinisikan untuk setiap fitur yang ada pada sistem. Evaluasi akan dibagi menjadi dua jenis, yaitu evaluasi terhadap fungsionalitas sistem dan evaluasi terhadap kinerja model.
\item \emph{Communication}~\\
	      Tahapan \emph{communication} merupakan tahapan untuk mengkomunikasikan hasil penelitian yang telah dilakukan. Pada tahap ini, peneliti akan menyusun laporan penelitian yang akan digunakan sebagai media untuk mengkomunikasikan hasil penelitian yang telah dilakukan. Laporan penelitian akan mencakup pendahuluan, studi literatur, analisis masalah, desain solusi, evaluasi, dan kesimpulan dan saran yang perlu disampaikan dan diketahui oleh pembaca.
\end{enumerate}
