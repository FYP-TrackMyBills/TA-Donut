\chapter{Pendahuluan}

\section{Latar Belakang}
\label{sec:latarbelakang}

Transaksi pembayaran merupakan bagian penting dari aktivitas ekonomi, baik dalam lingkup individu maupun organisasi. Proses ini mencakup pengalihan dana antara pihak yang terlibat untuk memenuhi kewajiban finansial. Dalam perkembangannya, metode pembayaran telah bertransformasi dari yang awalnya menggunakan uang tunai dan cek menjadi transfer bank dan sistem yang lebih modern berbasis digital. Transformasi ini bertujuan untuk meningkatkan efisiensi, keamanan, dan kenyamanan dalam melakukan transaksi.

\qrisfull{} menjadi salah satu faktor pesatnya perkembangan teknologi pembayaran digital di Indonesia. \qris{} diluncurkan oleh Bank Indonesia pada tahun 2019 dan menjadi metode pembayaran yang populer dalam waktu kurang dari lima tahun karena kemudahan dan efisiensinya. Generasi Z dan millenial, sebagai pengguna utama metode pembayaran ini, merasakan kemudahan dalam melakukan pembayaran atau pengeluaran uang. Goodstats menunjukkan bahwa 38\% Gen Z menggunakan \qris{} dalam kehidupan sehari-hari, sementara di kalangan millenial angkanya mencapai 
25\%. 

Pada April 2024, Bank Indonesia melaporkan jumlah pengguna \qris{} mencapai angka 48,12 juta dan jumlah merchant yang menggunakan mencapai 31,61 juta. Total nilai transaksi dengan menggunakan metode pembayaran \qris{} telah mencapai Rp 31,65 triliun, yaitu meningkat sebesar 149,46\% secara tahunan 
pada bulan Februari 2024. Peningkatan ini mencerminkan adopsi yang signifikan di kalangan masyarakat, khususnya Gen Z dan millenial, yang cenderung memilih 
transaksi non-tunai. Survei dari Jawa Pos Radar Lawu mengungkapkan bahwa 57\% dari kedua generasi ini lebih memilih transaksi non-tunai, dengan \qris{} menjadi salah satu metode yang paling efektif. 

\newpage

Kemudahan ini mendorong peningkatan frekuensi transaksi digital di kalangan pengguna yang mengakibatkan kenaikan pada angka volume transaksi. Namun, seiring dengan meningkatnya volume transaksi, belum terdapat sistem dengan mekanisme pencatatan transaksi yang akurat dan cepat. Banyak pengguna masih harus melakukan pencatatan manual untuk memantau pengeluaran mereka yang rentan terhadap kesalahan dan memakan waktu. Salah satu solusi potensial untuk masalah ini adalah penerapan teknologi berbasis \cv{} dan \dl{} yang dapat melakukan pemindaian dan ekstraksi data dari dokumen gambar secara akurat, seperti \ocrfull, \cnnfull, \transformer, dan lainnya.

\layoutlm{} dan \bert{} merupakan model transformer yang menggabungkan informasi visual dan tekstual untuk memahami dokumen. Namun, kedua model ini masih bergantung pada \ocr. Ketergantungan ini membuat kedua model tersebut kurang efisien. Oleh karena itu, sistem yang dapat mengonversi dan mengekstrasi data dari bukti pembayaran berbasis kertas menjadi dokumen digital yang tidak bergantung pada OCR menjadi sistem yang diperlukan. 

\section{Rumusan Masalah}

Rumusan Masalah berisi masalah utama yang dibahas dalam tugas akhir. Rumusan masalah yang baik memiliki struktur sebagai berikut:

\begin{enumerate}
	\item Penjelasan ringkas tentang kondisi/situasi yang ada sekarang terkait dengan topik utama yang dibahas Tugas Akhir.
	\item Pokok persoalan dari kondisi/situasi yang ada, dapat dilihat dari kelemahan atau kekurangannya. \textbf{Bagian ini merupakan inti dari rumusan masalah}.
	\item Elaborasi lebih lanjut yang menekankan pentingnya untuk menyelesaikan pokok persoalan tersebut.
	\item Usulan singkat terkait dengan solusi yang ditawarkan untuk menyelesaikan persoalan.
\end{enumerate}

Penting untuk diperhatikan bahwa persoalan yang dideskripsikan pada subbab ini akan dipertanggungjawabkan di bab Evaluasi apakah terselesaikan atau tidak.

\section{Tujuan}

Tuliskan tujuan utama dan/atau tujuan detil yang akan dicapai dalam pelaksanaan tugas akhir. Fokuskan pada hasil akhir yang ingin diperoleh setelah tugas akhir diselesaikan, terkait dengan penyelesaian persoalan pada rumusan masalah. Penting untuk diperhatikan bahwa tujuan yang dideskripsikan pada subbab ini akan dipertanggungjawabkan di akhir pelaksanaan tugas akhir apakah tercapai atau tidak.

\section{Batasan Masalah}

Tuliskan batasan-batasan yang diambil dalam pelaksanaan tugas akhir. Batasan ini dapat dihindari (tidak perlu ada) jika topik/judul tugas akhir dibuat cukup spesifik.

\section{Metodologi}

Tuliskan semua tahapan yang akan dilalui selama pelaksanaan tugas akhir. Tahapan ini spesifik untuk menyelesaikan persoalan tugas akhir. Tahapan studi literatur tidak perlu dituliskan karena ini adalah pekerjaan yang harus Anda lakukan selama proses pelaksanaan tugas akhir.

\section{Jadwal Pelaksanaan Tugas Akhir}

Tuliskan rencana kegiatan dan jadwal (dirinci sampai per minggu) mulai dari awal pelaksanaan Tugas Akhir I s.d. sidang tugas akhir berikut milestones dan deliverables yang harus diberikan. Jadwal ini dapat dibantu dengan membuat sebuah tabel timeline.