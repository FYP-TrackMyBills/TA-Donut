\section{Dasar Teori}
\label{sec:dasar-teori}
Subbab dasar teori merupakan subbab yang akan memberikan pengenalan
terhadap permasalahan yang akan dihadapi. Subbab ini juga akan memberikan konteks dan keterangan lebih lanjut mengenai teknologi-teknologi yang akan digunakan atau dipertimbangkan sebagai alternatif solusi dari permasalahan yang diangkat. Berikut merupakan landasan-landasan teori yang akan digunakan dalam penelitian ini.

\subsection{Pemrosesan Dokumen Finansial}
\label{subsec:pemrosesan-dokumen-finansial}

Dokumen struk pembayaran memegang peranan penting dalam pengelolaan
keuangan baik individu maupun institusi. Proses ekstraksi data dari dokumen-dokumen ini sering kali menghadapi tantangan yang signifikan, termasuk keanekaragaman tata letak, kualitas gambar yang bervariasi, dan kebutuhan untuk
mengekstrak informasi dengan akurasi tinggi. Sebagai contoh, elemen-elemen penting seperti total harga, tanggal transaksi, dan nomor rekening sering kali tersebar di berbagai posisi dalam dokumen yang berbeda, sehingga membuat ekstraksi data secara manual menjadi lambat dan rentan terhadap kesalahan.

Teknologi yang umumnya dapat digunakan untuk menanggulangi masalah ini adalah \cv{}, salah satunya \ocr. Secara tradisional, \ocr{} digunakan untuk mengonversi teks dalam gambar menjadi format digital yang dapat diproses lebih lanjut. Namun, OCR memiliki
keterbatasan, terutama dalam menangani dokumen dengan kualitas gambar rendah atau format teks yang tidak standar. Selain itu, ketergantungan pada OCR dapat meningkatkan kompleksitas dan biaya pemrosesan \parencite{kim2021donut}.

\dlfl{} memiliki berbagai jenis algoritma dan model yang mendukung pengembangan sistem tersebut. \dl{} merupakan cabang \ml{} yang memanfaatkan jaringan saraf tiruan berlapis untuk mengenali pola kompleks dalam data, termasuk dokumen finansial. Salah satu model yang umum digunakan adalah \cnn. \cnn{} dirancang untuk memproses data berbentuk gambar dengan mendeteksi fitur lokal seperti teks, garis, atau elemen visual lainnya sehingga cocok untuk ekstraksi data dari dokumen. Namun, \cnn{} terbatas dalam memahami hubungan global antar
elemen dalam dokumen sehingga \transformer{} hadir sebagai solusi yang lebih canggih \parencite{alzubaidi2021review}.

Dengan mekanisme \attention-nya, \transformer{}, seperti LayoutLM dan
Donut, mampu menangkap hubungan kontekstual antar elemen secara global. Hal ini membuat penggunaan \transformer{} dengan struktur kompleks dan informasi
tersebar. \transformer{} memberikan pendekatan yang kuat untuk memastikan akurasi dan efisiensi dalam pemrosesan dokumen finansial.

\subsection{\emph{Deep Learning}}
\label{subsec:dl}

\dlfl{} adalah cabang dari \ml{} yang menggunakan \annfull{} dengan banyak lapisan (\emph{deep}) untuk mempelajari representasi data yang kompleks. Deep Learning memungkinkan komputer untuk menganalisis data tidak terstruktur, seperti teks, gambar, audio, dan video dengan tingkat akurasi yang sangat tinggi. Model-model \dl{} belajar dengan cara memproses data melalui lapisan-lapisan neuron atau saraf yang dirancang untuk mengekstrak fitur-fitur penting, baik yang eksplisit maupun tersembunyi \parencite{Goodfellow-et-al-2016}. 

Terdapat beberapa model dan arsitektur yang dapat digunakan dalam \dl. Setiap model tersebut ditujukan untuk memenuhi kasus yang berbeda-beda. Berikut adalah arsitektur dan model yang relevan dengan kasus ekstraksi data dari dokumen finansial:

\begin{enumerate}
	\item \cnn\--\transformer{} (Kombinasi \cnn{} dan \transformer)~\\
	      \cnn{} dan \transformer{} adalah dua metode yang dapat digunakan secara terpisah atau bersama untuk menangani data visual. \cnn{} adalah salah satu jenis \MakeLowercase{{\nn{}}} yang paling sering digunakan pada data gambar. CNN bisa digunakan untuk mendeteksi dan mengenali fitur signifikan tanpa supervisi manusia pada sebuah gambar \parencite{alzubaidi2021review}. \transformer{} adalah sebuah arsitektur \MakeLowercase{\nn{}} yang meng-\emph{encode} data menjadi fitur-fitur melalui mekanisme \attention. Secara garis besar, \transformer{} akan membagi gambar menjadi beberapa \patch{} dan mengkalkulasi representasinya dan hubungan antar \patch{} tersebut \parencite{han2021transformer}. 
	\item \crnnfull~\\
	      \crnn{} adalah kombinasi dari \cnn{} dan \rnn{} yang merupakan dua model yang paling sering digunakan. \cnn{} digunakan pada tahap awal untuk mengekstrak fitur visual dan gambar. Hasil dari implementasi \cnn{} akan digunakan oleh model \rnn{} sebagai data input untuk untuk menangkap hubungan antar elemen teks, seperti urutan huruf dalam kata atau angka dalam bilangan \parencite{wang2019convolutional}. \crnn{} telah diaplikasikan pada klasifikasi musik, audio, dan klasifikasi data dan teks yang bersifat \emph{hyperspectral}.
	\item Model \objectdetection~\\
	      Model \objectdetection{} adalah salah satu model dalam \dl{} yang digunakan untuk mendeteksi dan mengenali objek-objek tertentu dalam sebuah gambar atau dokumen, termasuk elemen-elemen seperti teks, tabel, atau simbol pada dokumen finansial. Dua jenis utama model \objectdetection{} yang sering digunakan adalah \yolofull{} dan \rcnnfull{}.
	      \begin{enumerate}
		      \item \yolo~\\
		            \yolo{} adalah model \objectdetection{} yang dirancang untuk kecepatan tinggi dan efisiensi. \yolo{} memproses seluruh gambar dalam satu tahap dengan membagi gambar menjadi \grid{}, kemudian memprediksi bounding box dan kelas objek secara bersamaan di setiap \grid{} \parencite{diwan2023object}. Model ini terkenal karena kecepatannya, membuatnya cocok untuk aplikasi \emph{real-time}, meskipun memiliki  kompromi dalam akurasi untuk objek kecil atau yang saling berdekatan.
		      \item \rcnn~\\
		            \rcnn{} menggunakan pendekatan dua tahap. \rcnn{} menghasilkan proposal wilayah yang berpotensi mengandung objek pada tahap pertama. Pada tahap kedua, setiap proposal dianalisis lebih mendalam menggunakan \cnn{} untuk mengklasifikasikan objek dan memperbaiki bounding box. \rcnn{} dan variannya (Fast \rcnn{} dan Faster \rcnn{}) menawarkan akurasi tinggi, tetapi memerlukan waktu komputasi lebih lama sehingga lebih cocok untuk aplikasi yang tidak membutuhkan deteksi \emph{real-time} \parencite{xie2021oriented}. 
	      \end{enumerate}
\end{enumerate}

\subsection{Transformer}
\label{subsec:transformer}

\transformer{} adalah suatu jenis arsitektur jarigan yang baru dalam \dl{} yang digunakan untuk mentransformasi sebuah deretan data menjadi sesuatu dengan karakteristik, seperti panjang atau format, yang berbeda. Transformer tidak menggunakan lapisan rekursif (\rnn) atau konvolusi (\cnn).  Transformer menggunakan mekanisme \selfattention{} yang memungkinkannya untuk memahami hubungan antar elemen dalam sebuah deretan atau urutan
(\sequence). Tidak seperti pada \rnn{}, mekanisme \selfattention tidak perlu memperhatikan jarak antar elemen. Hal ini membuat \transformer{} menjadi sangat efisien untuk memproses data sekuensial. \transformer{} banyak digunakan pada proses penerjemahan bahasa, rangkuman teks, dan pengenalan suara.

\transformer{} terdiri atas dua komponen utama, yaitu \encoder{} dan \decoder. \encoderfl{} bertugas untuk membaca dan memproses sequence masukan, sementara \decoder{} menghasilkan \sequence{} keluaran berdasarkan informasi dari \encoder.

\encoderfl{} dan \decoder{} terdiri dari beberapa lapisan yang identik, dengan masing-masing memiliki dua sub-lapisan utama, yaitu:

\begin{enumerate}
	\item \mha~\\
	      Sublapisan ini menggunakan mekanisme \selfattention{} untuk memahami hubungan antar elemen dalam sebuah \sequence{}. Misalnya, dalam sebuah kalimat, mekanisme ini dapat mengenali bahwa kata "dia"
	      merujuk pada "ibu" meskipun terdapat kata-kata lain di antaranya.
	      \mha{} merupakan gabungan beberapa lapisan (\layer)
	      \emph{Scaled Dot-Product Attention} atau akan disingkat menjadi \attention. Perbandingan proses pada \mha{} dan \emph{Scaled Dot-Product Attention} dapat dilihat pada \autoref{fig:attention}

	      \begin{figure}[htbp]
		      \centering
		      \includegraphics[width=.8\textwidth]{images/attentionmha.png}
		      \caption{\emph{Scaled Dot-Product Attention} (kiri) dan \mha{} (kanan) yang merupakan beberapa \layer{} \attention{} berjalan paralel \parencite{vaswani2017attention}}
		      \label{fig:attention}
	      \end{figure}

	      $Q$ (\emph{Query}), $K$ (\emph{Key}), dan $V$ (\emph{Value}) adalah representasi vektor dari setiap elemen dalam sebuah \sequence. Simbol $d_k$ adalah dimensi dari vektor $K$, yang digunakan untuk menskalakan hasil perkalian \emph{dot-product} $QK^\mathsf{T}$ agar proses pelatihan tetap stabil. Fungsi Softmax kemudian mengubah skor tersebut menjadi bobot probabilistik untuk menentukan elemen mana yang paling relevan untuk diperhatikan.

	      \pagebreak

	      Perhitungan \attention{} dilakukan dengan formula seperti yang ditunjukkan pada persamaan \eqref{eq:attention-softmax} \parencite{vaswani2017attention} yang didefinisikan sebagai berikut:

	      \begin{equation}
		      \label{eq:attention-softmax}
		      \operatorname{Attention}(Q, K, V) = \operatorname{SoftMax}\left(\frac{QK^\mathsf{T}}{\sqrt{d_k}}\right)V
	      \end{equation}
		  \addcontentsline{loe}{myequations}{\protect\numberline{\theequation}Persamaan Softmax}

	\item \ffnfull~\\
	      Setelah \attention{} dihitung, informasi dari setiap elemen diproses melalui jaringan \emph{feed-forward} yang sama untuk setiap posisi dalam urutan. Jaringan ini terdiri dari dua lapisan linear dengan fungsi aktivasi ReLU pada persamaan \eqref{eq:ffn} \parencite{vaswani2017attention}:

	      \begin{equation}
		      \label{eq:ffn}
		      \operatorname{FFN}(x) = \max(0, xW_1 + b_1)W_2 + b_2
	      \end{equation}
		  \addcontentsline{loe}{myequations}{\protect\numberline{\theequation}Fungsi Aktivasi ReLU pada \ffn}

	      \decoderfl{} memiliki perbedaan signifikan dibandingkan dengan \encoder.
	      \encoderfl{} terdiri dari beberapa lapisan yang masing-masing memiliki mekanisme \selfattention{} dan FFN. Setiap elemen dalam \sequence{} dapat saling “memperhatikan”. \decoderfl{} memiliki lapisan tambahan yang \encoder{}\--\decoder{} \attention. Lapisan ini memungkinkan \decoder{} untuk "memperhatikan" keluaran dari \encoder. \decoderfl{} memiliki mekanisme \emph{masking} untuk memastikan bahwa posisi saat ini hanya bergantung pada posisi sebelumnya.

	      \transformer{} memiliki keunggulan dibandingkan dengan metode konvensional pemrosesan sekuensial, yaitu \rnn. \transformer{} tidak bergantung
	      pada pemrosesan berurutan. \transformer{} dapat memproses seluruh urutan secara bersamaan, membuat pelatihan jauh lebih cepat. Implementasi \transformer{} menunjukkan hasil terbaik di berbagai kasus, seperti penerjemahan bahasa,
	      dibandingkan dengan arsitektur lainnya.

\end{enumerate}


\subsection{Transformer}
\label{subsec:transformer}

Transformer adalah suatu jenis arsitektur jarigan yang baru dalam deep learning yang digunakan untuk mentransformasi sebuah deretan data menjadi sesuatu dengan karakteristik, seperti panjang atau format, yang berbeda. Transformer tidak menggunakan lapisan rekursif (RNN) atau konvolusi (CNN).  Transformer menggunakan mekanisme self-attention yang memungkinkannya 
untuk memahami hubungan antar elemen dalam sebuah deretan atau urutan 
(sequence). Tidak seperti pada RNN, mekanisme self-attention tidak perlu 
memperhatikan jarak antar elemen. Hal ini membuat transformer menjadi sangat efisien untuk memproses data sekuensial. Transformer banyak digunakan pada proses penerjemahan bahasa, rangkuman teks, dan pengenalan suara.  

Transformer terdiri atas dua komponen utama, yaitu encoder dan decoder. Encoder bertugas untuk membaca dan memproses sequence masukan, sementara decoder menghasilkan sequence keluaran berdasarkan informasi dari encoder.  

Encoder dan decoder terdiri dari beberapa lapisan yang identik, dengan 
masing-masing memiliki dua sublapisan utama, yaitu:


\subsection{\swin}
\label{subsec:swin}

Swin Transformer adalah sebuah arsitektur Vision Transformer yang 
dirancang untuk backbone dalam berbagai tugas computer vision seperti klasifikasi 
gambar, deteksi objek, dan segmentasi semantik. Nama "Swin" berasal dari konsep 
"Shifted Window" yang menjadi elemen utama dalam desainnya. Tidak seperti 
Vision Transformer (ViT) yang menggunakan metode self-attention secara global 
pada seluruh gambar, Swin Transformer memperkenalkan self-attention berbasis 
local window yang secara signifikan mengurangi kompleksitas komputasi (Liu dkk., 2021).

\subsection{\bartfull}
\label{subsec:bart}

\bartfull adalah adalah model \ml{} yang dirancang untuk memahami, menghasilkan, dan merekonstruksi teks dalam bahasa alami (\emph{natural language}). \bart{} merupakan \emph{denoising autoencoder} yang dilatih untuk memperbaiki teks yang telah dirusak oleh berbagai transformasi sehingga mampu mengembalikan teks ke bentuk aslinya \parencite{lewis2019bart}. Arsitektur \bart{} menggabungkan kelebihan model \bert{} yang memiliki \emph{encoder bidirectional} dan GPT yang menggunakan \emph{decoder autoregressive}. Hal ini menjadikannya sangat fleksibel untuk berbagai tugas \nlpfull.

\bart{} dibangun di atas arsitektur transformer yang terdiri atas dua 
komponen utama, yaitu \emph{bidirectional encoder} dan \textit{autoregressive decoder}. 
\emph{Bidirectional encoder} akan mengolah teks input dengan cara memahami hubungan antar token secara dua arah. \emph{Autoregressive decoder} akan menghasilkan teks secara berurutan, token demi token, dengan mempertimbangkan \emph{sequence} yang sudah dihasilkan.

\autoref{fig:bart} menunjukkan cara \bart{} bekerja. Teks input akan dirusak terlebih dahulu, kemudian dibaca oleh \encoder. Hasil bacaan \encoder{} akan diteruskan ke \decoder{} untuk mengembalikan bagian teks yang “dirusak”. Dengan kemampuannya untuk merekonstruksi teks dari masukan yang rusak, \bart{} menjadi suatu \transformer{} yang sangat baik untuk memahami struktur bahasa dan menghasilkan teks yang koheren walaupun masukan dinilai 
rusak. Penggunaan \bart{} mirip dengan \bert{} dan GPT, seperti klasifikasi teks, generasi teks, dan penejermahan teks. 

\begin{figure}
\centering
\includegraphics[width=0.8\textwidth]{images/bart.png}
\caption{Cara kerja \bart{} \parencite{lewis2019bart}.}
\label{fig:bart}
\end{figure}




% \subsection{\onnx}
\label{subsec:onnx}

\emph{Open Neural Network Exchange} \onnx{} adalah format standar \emph{open-source} untuk merepresentasikan model \ml{} yang memungkinkan interoperabilitas antara berbagai \emph{framework} \dl. Dikembangkan oleh Facebook dan Microsoft pada tahun 2017, \onnx{} telah berkembang menjadi proyek yang lulus dari Linux Foundation AI dengan dukungan dari perusahaan teknologi besar termasuk IBM, Intel, AMD, ARM, Qualcomm, dan NVIDIA \parencite{onnxgithub2019}.

\onnx{} berfungsi sebagai representasi universal struktur komputasi dari \nn. Komponen arsitektur inti \onnx{} terdiri dari beberapa elemen penting. \emph{Node} merepresentasikan operasi matematika. \emph{Edge} merepresentasikan tensor yang mengalir antar operasi. \emph{Initializer} berfungsi untuk menyimpan bobot model dan konstanta yang diperlukan.

% Konsep teknis utama \onnx{} meliputi representasi menengah (\emph{Intermediate Representation}) yang berfungsi sebagai representasi universal untuk menangkap struktur graf komputasi dari jaringan neural. \onnx{} dibangun menggunakan format serialisasi \emph{Protocol Buffers} (protobuf) Google untuk penyimpanan dan transmisi model yang efisien. Format ini juga mendefinisikan \emph{Operator Set} (Opset) yang merupakan koleksi operator berversi untuk mempertahankan kompatibilitas mundur. Arsitektur \onnx{} merepresentasikan jaringan neural sebagai graf terarah asiklik (DAG) dimana \emph{node} merepresentasikan operasi dan \emph{edge} merepresentasikan aliran data.

% \subsubsection{Interoperabilitas dan Optimasi}

\onnx{} menyediakan interoperabilitas yang memungkinkan konversi model ke berbagai format, seperti \pytorch, \tensorflow, Keras, Scikit-learn, dan \emph{framework} lainnya. Hal ini memungkinkan \emph{deployment} tunggal yang memungkinkan pengembang untuk melakukan pelatihan model pada \emph{framework} apapun dan di-\emph{deploy} menggunakan \onnx Runtime. Format ini bersifat agnostik terhadap \emph{hardware} dan memungkinkan model berjalan di berbagai jenis \emph{hardware} seperti CPU, GPU, dan akselerator khusus seperti TPU.

% Dari aspek optimasi, \onnx{} menyediakan optimasi graf otomatis yang mencakup \emph{constant folding}, \emph{operator fusion}, dan eliminasi \emph{node} redundan. Format ini juga menyediakan optimasi spesifik \emph{hardware} yang memanfaatkan \emph{kernel} khusus untuk platform \emph{hardware} berbeda. Efisiensi memori juga menjadi keunggulan dengan alokasi memori yang dioptimalkan dan manajemen \emph{lifecycle} tensor yang efisien.

% \subsubsection{Aplikasi dalam \ml dan \cv}

% \onnx memiliki aplikasi luas dalam berbagai tugas \ml dan \cv. Dalam klasifikasi gambar, \onnx mendukung model seperti ResNet, EfficientNet, dan MobileNet untuk klasifikasi \emph{real-time}. Untuk deteksi objek, format ini kompatibel dengan model \yolo, SSD, dan \rcnn untuk berbagai aplikasi. Aplikasi lainnya mencakup segmentasi semantik dengan model U-Net dan DeepLab, sistem \ocr dan pemahaman dokumen untuk deteksi dan pengenalan teks, serta pengenalan wajah untuk sistem \emph{real-time} dan biometrik.

% Implementasi industri \onnx mencakup berbagai layanan besar seperti layanan Microsoft yang meliputi pencarian Bing, aplikasi Office, dan Azure Cognitive Services. Dalam industri otomotif, \onnx digunakan dalam sistem mengemudi otonom. Sektor kesehatan memanfaatkan \onnx untuk analisis pencitraan medis dan sistem diagnostik, sementara manufaktur menggunakannya untuk sistem kontrol kualitas dan deteksi cacat \parencite{onnxruntime2020}.


\subsection{\flutter}
\label{subsec:flutter}

\flutter{} adalah SDK (\emph{Software Development Kit}) UI \emph{open-source} dari Google yang diluncurkan pada tahun 2017, dirancang sebagai \emph{framework cross-platform} yang memungkinkan pengembang membuat aplikasi yang dikompilasi secara \emph{native} untuk mobile, web, desktop, dan perangkat \emph{embedded} dari satu basis kode. \emph{Framework} ini menggunakan bahasa pemrograman Dart dan mengimplementasikan paradigma UI reaktif dan deklaratif \parencite{flutter2021}.

\flutter memiliki beberapa karakteristik teknis utama yang membedakannya dari \emph{framework} pengembangan aplikasi lainnya, yaitu arsitektur \emph{single codebase} yang memungkinkan . Arsitektur basis kode tunggal memungkinkan pendekatan tulis sekali dan \emph{deploy} di mana saja (\emph{write once, deploy anywhere}), memberikan efisiensi tinggi dalam pengembangan. Model pemrograman reaktif mengimplementasikan paradigma UI = f(state) dimana antarmuka pengguna bereaksi terhadap perubahan \emph{state} secara otomatis. Kompilasi \emph{native} memungkinkan \flutter mengkompilasi ke kode mesin ARM/Intel untuk kinerja optimal yang mendekati aplikasi \emph{native} asli.

Arsitektur berbasis \emph{widget} menjadi ciri khas \flutter dimana segala sesuatu adalah \emph{widget} yang merupakan deskripsi \emph{immutable} dari komponen UI. Fitur \emph{hot reload} memungkinkan \emph{stateful hot reload} selama development untuk iterasi cepat tanpa kehilangan \emph{state} aplikasi, meningkatkan produktivitas pengembang secara signifikan.

\subsubsection{Arsitektur Berlapis}

\flutter terdiri dari empat \layer utama yang saling terintegrasi. \emph{Layer Framework} yang ditulis dalam Dart mencakup library \emph{widget} Material Design dan Cupertino untuk implementasi desain platform-specific. \emph{Layer widget} menyediakan abstraksi komposisi untuk membangun UI yang kompleks, sementara \emph{layer rendering} menangani manajemen \emph{layout} dan perhitungan posisi elemen. \emph{Layer foundation} menyediakan layanan inti seperti animasi, \emph{painting}, dan \emph{gestures} yang fundamental untuk interaksi pengguna.

\emph{Layer Engine} yang ditulis dalam C++ berisi mesin grafis Skia untuk \emph{rendering} dengan transisi menuju Impeller untuk performa yang lebih baik. Runtime Dart dan \emph{virtual machine} berjalan pada layer ini untuk eksekusi kode aplikasi. \emph{Layout} teks dan operasi \emph{file} I/O juga dikelola pada layer ini, bersama dengan implementasi tingkat rendah dari \api inti \flutter.

\emph{Layer Embedder} bersifat spesifik platform dan menangani integrasi dengan sistem operasi yang mendasarinya. Untuk Android menggunakan Java/C++, sedangkan iOS menggunakan Swift/Objective-C. Layer ini bertanggung jawab untuk koordinasi layanan sistem operasi, manajemen \emph{event loop}, dan eksposur \api spesifik platform untuk fungsionalitas yang tidak tersedia secara cross-platform.

\subsubsection{\emph{Cross-Platform Development}}

\flutter menyediakan efisiensi pengembangan yang signifikan dengan pengurangan waktu pengembangan hingga 50\% dibandingkan pengembangan \emph{native} tradisional. Satu tim pengembang dapat menargetkan \emph{multiple platform} secara bersamaan, mengurangi kebutuhan tenaga kerja dan kompleksitas manajemen proyek. Lingkungan pengembangan dan \emph{tooling} yang terpadu memungkinkan konsistensi dalam proses development, sementara logika bisnis dan komponen UI dapat dibagi antar platform.

Konsistensi menjadi aspek penting dengan UI yang identik di semua platform, memberikan pengalaman pengguna yang seragam. Perilaku dan kinerja yang konsisten mengurangi bug spesifik platform dan memberikan pengalaman \emph{brand} yang terpadu. Hal ini sangat penting untuk aplikasi komersial yang memerlukan identitas visual yang kuat di berbagai platform.

\subsubsection{Integrasi dengan \ml dan \cv}

\flutter menyediakan beberapa jalur untuk integrasi \ml yang relevan untuk aplikasi \cv. Integrasi \tensorflow Lite dimungkinkan melalui paket \texttt{tflite\_flutter} yang memungkinkan inferensi model langsung pada perangkat. Dukungan tersedia untuk model kustom dan \emph{pre-trained}, dengan kompatibilitas \emph{multi-platform} untuk Android, iOS, dan Desktop. Akselerasi \emph{hardware} juga didukung melalui \gpu dan \emph{Neural Processing Units} untuk inferensi yang lebih cepat.

Firebase ML Kit menyediakan \api ML \emph{on-device} untuk tugas umum seperti pengenalan teks, deteksi wajah, dan \emph{scanning barcode}. \emph{Labeling} gambar dan deteksi \emph{landmark} juga tersedia dengan kemampuan pemrosesan \emph{real-time} yang memadai untuk aplikasi produksi. Kemampuan \cv dalam \flutter mencakup pemrosesan \emph{stream} kamera \emph{real-time} untuk aplikasi yang memerlukan analisis video langsung. Klasifikasi gambar dan deteksi objek dapat diimplementasikan dengan mudah, begitu juga dengan implementasi \ocr untuk pengenalan teks. Pengenalan wajah dan autentikasi biometrik juga didukung untuk aplikasi keamanan \parencite{flutteronnx2022}.


\subsection{FastAPI}
\label{subsec:fastapi}

FastAPI adalah \emph{framework} \emph{web} Python modern dan berperforma tinggi untuk membangun \api{} dengan Python 3.7+ berdasarkan \emph{type hints} standar Python. Dibuat oleh Sebastián Ramírez, \emph{framework} ini merepresentasikan evolusi signifikan dalam pengembangan \emph{web} Python. FastAPI menggabungkan kesederhanaan Flask dengan kekuatan pemrograman \emph{asynchronous} modern \parencite{ramirez2020fastapi}.

FastAPI memiliki integrasi sistem \emph{type} sebagai fitur unggulan yang memanfaatkan sistem \emph{type hint} Python untuk validasi data otomatis, serialisasi, dan pembuatan dokumentasi secara otomatis. FastAPI memiliki \emph{tools} dokumentasi yang sudah terintegrasi, yaitu OpenAPI (sebelumnya Swagger). \emph{Framework} ini juga menggunakan skema \json yang memberikan kompatibilitas yang luas dengan ekosistem pengembangan modern. FastAPI dibangun di atas \emph{Asynchronous Server Gateway Interface} (ASGI) yang mewakili kemajuan arsitektur yang signifikan dibandingkan \emph{framework} berbasis \emph{Web Server Gateway Interface} (WSGI) tradisional \parencite{ramirez2020fastapi}. 

% Komponen arsitektur ASGI meliputi \emph{scope} yang berisi informasi \emph{scope} koneksi dengan tipe protokol dan metadata yang diperlukan. \emph{Receive} berfungsi sebagai \emph{callable awaitable} untuk menerima \emph{event} dari \emph{client}, sementara \emph{send} bertindak sebagai \emph{callable awaitable} untuk mengirim respons ke \emph{client}.

% Arsitektur berlapis FastAPI terdiri dari beberapa \layer terintegrasi yang bekerja secara harmonis. \emph{Application Layer} merupakan aplikasi inti FastAPI yang menangani \emph{routing} dan \emph{dependency injection} dengan sistem yang sophisticated. \emph{Framework Layer} menggunakan Starlette yang menyediakan kompatibilitas ASGI dan fungsionalitas \emph{framework} web yang diperlukan. \emph{Validation Layer} memanfaatkan Pydantic untuk validasi dan serialisasi data dengan performa tinggi, sementara \emph{Server Layer} menggunakan \emph{server} ASGI Uvicorn untuk penanganan \emph{request} HTTP yang efisien.

% \subsubsection{\emph{Async Programming} dan Kinerja}

% Model \emph{async processing} membedakan FastAPI dari \emph{framework synchronous} tradisional dengan memproses \emph{request} secara asinkron. \emph{Event loop single-threaded} menangani berbagai koneksi bersamaan tanpa overhead context switching yang tinggi. \emph{Coroutines} memungkinkan fungsi \emph{async} yang dapat dihentikan dan dilanjutkan sesuai kebutuhan, sementara \emph{non-blocking} I/O memberikan penanganan operasi I/O yang efisien tanpa memblokir \emph{thread} utama.

% Analisis kinerja berdasarkan \emph{benchmark} TechEmpower independen menunjukkan kinerja superior FastAPI dalam berbagai aspek. \emph{Throughput} FastAPI mencapai sekitar 3 kali lebih tinggi dari Flask dalam kondisi beban tinggi. P99 \emph{latency} yang lebih rendah dicapai karena pemrosesan \emph{async} yang efisien, sementara efisiensi memori menunjukkan \emph{footprint} memori yang berkurang per \emph{request}. Karakteristik \emph{scaling} linear dengan \emph{concurrent requests} memungkinkan aplikasi menangani beban yang meningkat dengan graceful degradation.

% \subsubsection{Integrasi dengan Sistem \cv dan \ml}

% FastAPI menyediakan arsitektur optimal untuk \emph{deployment} model \ml dalam konteks \cv dengan berbagai kemampuan teknis. Pemrosesan \emph{async} memungkinkan pemrosesan gambar \emph{non-blocking} untuk \emph{throughput} tinggi, sangat penting untuk aplikasi yang menangani volume gambar besar. Manajemen memori yang efisien memungkinkan penanganan data gambar besar tanpa degradasi performa yang signifikan. Dukungan \emph{streaming} memungkinkan kemampuan pemrosesan video \emph{real-time} untuk aplikasi yang memerlukan analisis kontinyu, sementara pemrosesan bersamaan memungkinkan \emph{multiple pipeline} pemrosesan gambar berjalan secara paralel.

% Arsitektur \emph{model serving} yang disediakan FastAPI mencakup tahapan yang komprehensif. \emph{Model loading} terjadi saat \emph{startup} aplikasi untuk memuat model \ml ke memori dengan efisien. \emph{Preprocessing pipeline} menangani tahap preprocessing gambar sebelum inferensi untuk memastikan format data yang sesuai. Inferensi model dilakukan dengan eksekusi model \ml untuk prediksi yang akurat dan cepat. \emph{Postprocessing} memproses hasil prediksi untuk format yang diinginkan sesuai kebutuhan aplikasi, dan \emph{response serialization} melakukan serialisasi hasil ke format \json atau format lainnya sesuai spesifikasi \api.

% FastAPI mendukung integrasi dengan berbagai \emph{library} \ml Python seperti \pytorch, \tensorflow, dan \onnx Runtime, menjadikannya pilihan yang sesuai untuk \emph{deployment} model \cv dalam lingkungan produksi \parencite{techempowerbenchmark2023}.


\subsection{Metrik Evaluasi}
\label{subsec:metrik-evaluasi}

Evaluasi kinerja sistem \ml{} memerlukan metrik yang komprehensif dan sesuai dengan domain aplikasi. Sistem pemahaman dokumen dan pengenalan teks memerlukan kombinasi metrik klasifikasi standar dan metrik khusus untuk evaluasi pemahaman dokumen. Metrik evaluasi yang digunakan dalam penelitian ini mencakup \accuracy, \precision, \recall, \fscore, \coverage, dan \mcer{} (\emph{Mean Character Error Rate}).

\subsubsection{Metrik Klasifikasi Standar}

\accuracyfl{} mengukur proporsi prediksi yang benar baik positif maupun negatif di seluruh \emph{instance} dalam \dataset. Formula \accuracy{} didefinisikan sebagai perbandingan antara jumlah prediksi benar dengan total prediksi yang dibuat. Dalam konteks klasifikasi biner, \accuracy{} dihitung menggunakan matriks yang terdiri dari \emph{True Positive} (TP), \emph{True Negative} (TN), \emph{False Positive} (FP), dan \emph{False Negative} (FN). Persamaan \eqref{eq:accuracy} menunjukkan perhitungan \accuracy{} \parencite{jayaswal2020evalmetrics}. 

\begin{equation}
    \label{eq:accuracy}
\text{Accuracy} = \frac{TP + TN}{TP + TN + FP + FN}
\end{equation}
\addcontentsline{loe}{myequations}{\protect\numberline{\theequation}Persamaan \accuracy}

\precisionfl{} mengukur proporsi prediksi positif yang benar di antara semua prediksi positif yang dibuat model. Metrik ini sangat penting ketika biaya \emph{false positive} tinggi dalam aplikasi tertentu. Persamaan \eqref{eq:precision} menunjukkan perhitungan \precision. Perhitungan \precision{} didefinisikan sebagai rasio antara \emph{True Positive} (TP) dengan jumlah total prediksi positif yang dibuat, yaitu jumlah \emph{True Positive} ditambah \emph{False Positive} \parencite{jayaswal2020evalmetrics}. \precisionfl{} tinggi menunjukkan bahwa model meminimalkan \emph{false positive}, memastikan prediksi positif kemungkinan besar benar dan dapat diandalkan untuk pengambilan keputusan.

\begin{equation}
    \label{eq:precision}
\text{Precision} = \frac{TP}{TP + FP}
\end{equation}
\addcontentsline{loe}{myequations}{\protect\numberline{\theequation}Persamaan \precision}

\recallfl{} mengukur proporsi \emph{instance} positif aktual yang berhasil diidentifikasi dengan benar oleh model. Metrik ini kritikal ketika biaya \emph{false negative} tinggi dalam sistem yang memerlukan deteksi lengkap. Persamaan \eqref{eq:recall} menunjukkan rumus yang dapat digunakan untuk menghitung \recall{} \parencite{jayaswal2020evalmetrics}.\recallfl{} tinggi berarti model menangkap sebagian besar \emph{instance} positif, meminimalkan kemungkinan melewatkan data penting yang harus dideteksi.

\begin{equation}
    \label{eq:recall}
\text{Recall} = \frac{TP}{TP + FN}
\end{equation}
\addcontentsline{loe}{myequations}{\protect\numberline{\theequation}Persamaan \recall}

\fscore{} adalah rata-rata harmonik dari \precision{} dan \recall{}. \fscore{} memberikan ukuran seimbang yang mempertimbangkan kedua metrik secara setara. Persamaan \eqref{eq:fscore} menunjukkan rumus yang dapat digunakan untuk menghitung \fscore{} \parencite{jayaswal2020evalmetrics}. \fscore{} tinggi menunjukkan bahwa model tidak hanya akurat dalam prediksi positif tetapi juga menangkap sebagian besar \emph{instance} positif yang relevan. \fscore{} memerlukan \precision{} dan \recall{} tinggi untuk mencapai skor tinggi. \fscore{} akan menjadi nol jika salah satu dari \precision{} atau \recall{} adalah nol.

\begin{equation}
    \label{eq:fscore}
\text{F1-Score} = 2 \times \frac{\text{Precision} \times \text{Recall}}{\text{Precision} + \text{Recall}}
\end{equation}
\addcontentsline{loe}{myequations}{\protect\numberline{\theequation}Persamaan \fscore}

% \subsubsection{\ted (\emph{Tree Edit Distance})}

\ted adalah \emph{sequence minimum-cost} dari operasi edit \emph{node} yang diperlukan untuk mentransformasi satu \emph{tree} menjadi \emph{tree} lain \parencite{zhang1989tree}. Operasi edit yang dapat dilakukan meliputi \emph{delete} untuk menghapus \emph{node} dan menghubungkan \emph{children}-nya ke \emph{parent}, \emph{insert} untuk menambahkan \emph{node} antara \emph{node} yang ada dan \emph{children}-nya, dan \emph{relabel} untuk mengubah \emph{label} dari \emph{node}.

Formulasi matematika \ted didefinisikan sebagai optimasi biaya minimum untuk transformasi struktur \emph{tree}.

\begin{equation}
\text{TED}(T_1, T_2) = \min\{\text{cost}(\text{sequence}) : \text{sequence transforms } T_1 \text{ to } T_2\}
\end{equation}

Definisi rekursif \ted memungkinkan perhitungan efisien melalui pemrograman dinamis.

\begin{equation}
\text{TED}(F, G) = \min \begin{cases}
\text{TED}(F-v, G) + \text{cost}(\text{delete } v) \\
\text{TED}(F, G-w) + \text{cost}(\text{insert } w) \\
\text{TED}(F-v, G-w) + \text{cost}(\text{relabel } v \rightarrow w)
\end{cases}
\end{equation}

\ted memiliki aplikasi luas dalam pemahaman dokumen terstruktur, mencakup analisis similaritas dokumen XML/HTML, analisis dan perbandingan \emph{layout} dokumen, pengenalan struktur dokumen yang kompleks, dan perbandingan data hierarkis. Dalam konteks pemahaman dokumen, \ted berguna untuk mengevaluasi seberapa baik model memahami struktur hierarkis informasi dalam dokumen terstruktur.

\subsubsection{\mcer{} (\emph{Mean Character Error Rate})}

\mcer{} adalah rata-rata \emph{Character Error Rate} (CER) di beberapa dokumen atau segmen teks yang memberikan ukuran komprehensif akurasi deteksi teks \parencite{neudecker2021survey}. Metrik ini sangat relevan untuk sistem pengenalan teks yang bergantung pada akurasi pengenalan karakter. Persamaan \eqref{eq:mcer} menunjukkan rumus yang digunakan untuk menghitung \mcer{} dengan $\text{CER}_i$ adalah \emph{Character Error Rate} untuk dokumen $i$. Metrik ini mengukur kesalahan pengenalan karakter pada tingkat granular dan memberikan informasi tentang akurasi pengenalan teks pada level karakter.

\begin{equation}
    \label{eq:mcer}
\text{mCER} = \frac{\sum{S_i + D_i + I_i}}{\sum{N_i}}
\end{equation}
\addcontentsline{loe}{myequations}{\protect\numberline{\theequation}Persamaan \emph{Mean Character Error Rate} (mCER)}

Persamaan \eqref{eq:cer} menunjukkan cara perhitungan CER dengan parameter yang meliputi $S$ yang merepresentasikan jumlah substitusi karakter, $D$ untuk jumlah \emph{deletion} atau penghapusan karakter, $I$ untuk jumlah \emph{insertion} atau penambahan karakter, dan $N$ sebagai total jumlah karakter dalam \emph{Ground Truth}.

\begin{equation}
    \label{eq:cer}
\text{CER}_i = \frac{S_i + D_i + I_i}{N_i}
\end{equation}
\addcontentsline{loe}{myequations}{\protect\numberline{\theequation}Persamaan \emph{Character Error Rate} (CER)}

% \mcer{} digunakan secara luas dalam berbagai aplikasi evaluasi \ocr. \emph{Benchmark} mesin \ocr{} menggunakan \mcer{} untuk membandingkan kinerja algoritma yang berbeda dan memantau perkembangan dari waktu ke waktu. Proyek digitalisasi menggunakan metrik ini untuk menilai kualitas hasil konversi dokumen fisik ke digital.

% \subsubsection{Integrasi Metrik untuk Evaluasi Sistem}

% Evaluasi sistem pemahaman dokumen yang komprehensif memerlukan pendekatan \emph{multi-level} yang mengintegrasikan berbagai metrik. Tingkat karakter menggunakan \mcer untuk mengukur akurasi pengenalan teks individual dan memastikan setiap karakter dapat dibaca dengan benar. Tingkat \emph{field} memanfaatkan \accuracy, \precision, \recall, dan F1-\emph{score} untuk evaluasi ekstraksi \emph{field} spesifik dalam dokumen terstruktur. Tingkat struktur menggunakan \ted untuk mengevaluasi pemahaman struktur dokumen keseluruhan dan hubungan antar elemen. Tingkat sistem mengkombinasikan semua metrik untuk evaluasi kinerja sistem secara holistik.

% Interpretasi dan \emph{trade-off} antar metrik memberikan perspektif berbeda tentang kinerja sistem. \accuracy memberikan gambaran umum kinerja tetapi dapat menyesatkan pada data tidak seimbang sehingga perlu dikombinasikan dengan metrik lain. \precision menjadi penting ketika biaya \emph{false positive} tinggi dalam aplikasi yang memerlukan presisi tinggi. \recall kritikal ketika biaya \emph{false negative} tinggi dalam sistem yang memerlukan deteksi lengkap. F1-\emph{score} memberikan keseimbangan antara \precision dan \recall untuk evaluasi menyeluruh. \ted mengukur pemahaman struktur dokumen yang kompleks dan hubungan hierarkis antar elemen, sementara \mcer memberikan ukuran granular akurasi pengenalan teks pada level karakter.

% Penggunaan metrik evaluasi yang komprehensif memastikan bahwa sistem pemahaman dokumen tidak hanya akurat dalam mengenali teks, tetapi juga mampu memahami struktur dokumen dan mengekstrak informasi relevan dengan presisi tinggi yang diperlukan untuk berbagai aplikasi \parencite{bille2005tree}.

\subsubsection{\emph{System Usability Scale} (SUS)}
\label{subsubsec:sus}

\emph{System Usability Scale} (SUS) adalah instrumen evaluasi usabilitas yang dikembangkan oleh Brooke pada tahun 1996 sebagai alat ukur cepat dan andal untuk mengevaluasi usabilitas sistem \parencite{brooke1996sus}. SUS terdiri dari sepuluh pertanyaan dengan skala \emph{Likert} lima poin yang dirancang untuk memberikan skor tunggal yang merepresentasikan penilaian subjektif pengguna terhadap usabilitas sistem. Instrumen ini telah menjadi salah satu metode evaluasi usabilitas yang paling banyak digunakan dalam interaksi manusia komputer karena kesederhanaan, reliabilitas, dan validitasnya \parencite{bangor2008empirical}.

SUS menggunakan sepuluh pertanyaan standar yang mencakup aspek-aspek fundamental usabilitas, termasuk efektivitas, efisiensi, dan kepuasan pengguna. Lima pertanyaan dinyatakan secara positif dan lima lainnya secara negatif untuk mengurangi kemungkinan respons bias dan meningkatkan keandalan pengukuran. Setiap pertanyaan dinilai pada skala 1 hingga 5, dengan 1 menunjukkan "sangat tidak setuju" dan 5 menunjukkan "sangat setuju". Pertanyaan SUS meliputi penilaian terhadap keinginan menggunakan sistem secara teratur, kompleksitas sistem yang dirasakan, kemudahan penggunaan, kebutuhan dukungan teknis, integrasi fungsi sistem, inkonsistensi, kemudahan pembelajaran, dan kepercayaan diri pengguna dalam menggunakan sistem \parencite{brooke1996sus}.

Perhitungan skor SUS mengikuti metodologi khusus untuk menghasilkan skor akhir antara 0 hingga 100. Untuk pertanyaan bernomor ganjil (pernyataan positif), skor dihitung dengan mengurangi 1 dari respons pengguna. Untuk pertanyaan bernomor genap (pernyataan negatif), skor dihitung dengan mengurangi respons pengguna dari 5. Jumlah total skor dari sepuluh pertanyaan kemudian dikalikan dengan 2,5 untuk menghasilkan skor SUS akhir. Persamaan \eqref{eq:sus-score} menunjukkan formula perhitungan skor SUS dengan $R_i$ adalah respons untuk pertanyaan $i$ \parencite{tullis2013measuring}.

\begin{equation}
    \label{eq:sus-score}
\text{SUS Score} = \left(\sum_{i=1,3,5,7,9} (R_i - 1) + \sum_{j=2,4,6,8,10} (5 - R_j)\right) \times 2.5
\end{equation}
\addcontentsline{loe}{myequations}{\protect\numberline{\theequation}Persamaan Skor SUS}

Interpretasi skor SUS menggunakan \emph{benchmark} yang telah ditetapkan berdasarkan penelitian ekstensif terhadap ribuan evaluasi usabilitas. Skor SUS di atas 68 dianggap diatas rata-rata, skor antara 68-80 menunjukkan usabilitas yang baik, skor 80-90 menunjukkan usabilitas yang sangat baik, dan skor di atas 90 menunjukkan usabilitas yang luar biasa. Skor di bawah 68 menunjukkan bahwa sistem memerlukan perbaikan usabilitas yang signifikan \parencite{bangor2009determining}. Penelitian oleh Sauro pada tahun 2011 menunjukkan bahwa skor SUS memiliki korelasi yang kuat dengan metrik usabilitas objektif seperti tingkat penyelesaian tugas dan efisiensi \parencite{sauro2011measuring}.

Keunggulan SUS sebagai instrumen evaluasi meliputi kemudahan administrasi, waktu pengisian yang singkat (2-5 menit), skor tunggal yang mudah dipahami, dan validitas yang telah terbukti melalui berbagai domain aplikasi \parencite{lewis2018sus}. SUS juga memungkinkan perbandingan usabilitas antar sistem yang berbeda dan \emph{benchmarking} terhadap standar industri.

Dalam konteks evaluasi sistem \emph{mobile} untuk ekstraksi data pembayaran, SUS menjadi instrumen yang relevan untuk mengukur sejauh mana pengguna Gen Z dapat menggunakan aplikasi dengan mudah dan efektif. Evaluasi usabilitas dengan SUS dapat mengidentifikasi apakah antarmuka aplikasi memenuhi ekspektasi pengguna dalam hal kemudahan penggunaan, efisiensi proses ekstraksi data, dan kepuasan pengguna secara keseluruhan terhadap sistem yang dikembangkan.


