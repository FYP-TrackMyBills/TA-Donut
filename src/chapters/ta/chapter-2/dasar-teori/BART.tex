\subsection{Transformer}
\label{subsec:transformer}

Transformer adalah suatu jenis arsitektur jarigan yang baru dalam deep learning yang digunakan untuk mentransformasi sebuah deretan data menjadi sesuatu dengan karakteristik, seperti panjang atau format, yang berbeda. Transformer tidak menggunakan lapisan rekursif (RNN) atau konvolusi (CNN).  Transformer menggunakan mekanisme self-attention yang memungkinkannya 
untuk memahami hubungan antar elemen dalam sebuah deretan atau urutan 
(sequence). Tidak seperti pada RNN, mekanisme self-attention tidak perlu 
memperhatikan jarak antar elemen. Hal ini membuat transformer menjadi sangat efisien untuk memproses data sekuensial. Transformer banyak digunakan pada proses penerjemahan bahasa, rangkuman teks, dan pengenalan suara.  

Transformer terdiri atas dua komponen utama, yaitu encoder dan decoder. Encoder bertugas untuk membaca dan memproses sequence masukan, sementara decoder menghasilkan sequence keluaran berdasarkan informasi dari encoder.  

Encoder dan decoder terdiri dari beberapa lapisan yang identik, dengan 
masing-masing memiliki dua sublapisan utama, yaitu:
