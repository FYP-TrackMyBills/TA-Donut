\subsection{\emph{Deep Learning}}
\label{subsec:dl}

\dlfl{} adalah cabang dari \ml{} yang menggunakan \annfull{} dengan banyak lapisan (\emph{deep}) untuk mempelajari representasi data yang kompleks. Deep Learning memungkinkan komputer untuk menganalisis data tidak terstruktur, seperti teks, gambar, audio, dan video dengan tingkat akurasi yang sangat tinggi. Model-model \dl{} belajar dengan cara memproses data melalui lapisan-lapisan neuron atau saraf yang dirancang untuk mengekstrak fitur-fitur penting, baik yang eksplisit maupun tersembunyi. \parencite{Goodfellow-et-al-2016}

Terdapat beberapa model dan arsitektur yang dapat digunakan dalam \dl. Setiap model tersebut ditujukan untuk memenuhi kasus yang berbeda-beda. Berikut adalah arsitektur dan model yang relevan dengan kasus ekstraksi data dari dokumen finansial:

\begin{enumerate}
	\item \cnn\--\transformer{} (Kombinasi \cnn{} dan \transformer)~\\
	      \cnn{} dan \transformer{} adalah dua metode yang dapat digunakan secara terpisah atau bersama untuk menangani data visual. \cnn{} adalah salah satu jenis \MakeLowercase{{\nn{}}} yang paling sering digunakan pada data gambar. CNN bisa digunakan untuk mendeteksi dan mengenali fitur signifikan tanpa supervisi manusia pada sebuah gambar \parencite{alzubaidi2021review}. \transformer{} adalah sebuah arsitektur \MakeLowercase{\nn{}} yang meng-\emph{encode} data menjadi fitur-fitur melalui mekanisme \attention. Secara garis besar, \transformer{} akan membagi gambar menjadi beberapa \patch{}, mengkalkulasi representasinya dan hubungan antar \patch{} tersebut. \parencite{han2021transformer}
	\item \crnnfull~\\
	      \crnn{} adalah kombinasi dari \cnn{} dan \rnn{} yang merupakan dua model yang paling sering digunakan. \cnn{} digunakan pada tahap awal untuk mengekstrak fitur visual dan gambar. Hasil dari implementasi \cnn{} akan digunakan oleh model \rnn{} sebagai data input untuk untuk menangkap hubungan antar elemen teks, seperti urutan huruf dalam kata atau angka dalam bilangan \parencite{wang2019convolutional}. \crnn{} telah diaplikasikan pada klasifikasi musik, audio, dan klasifikasi data dan teks yang bersifat \emph{hyperspectral}.
	\item Model \objectdetection~\\
	      Model \objectdetection{} adalah salah satu model dalam \dl{} yang digunakan untuk mendeteksi dan mengenali objek-objek tertentu dalam sebuah gambar atau dokumen, termasuk elemen-elemen seperti teks, tabel, atau simbol pada dokumen finansial. Dua jenis utama model \objectdetection{} yang sering digunakan adalah \yolofull{} dan \rcnnfull{}.
	      \begin{enumerate}
		      \item \yolo~\\
		            \yolo{} adalah model \objectdetection{} yang dirancang untuk kecepatan tinggi dan efisiensi. \yolo{} memproses seluruh gambar dalam satu tahap dengan membagi gambar menjadi \grid{}, kemudian memprediksi bounding box dan kelas objek secara bersamaan di setiap \grid{} \parencite{diwan2023object}. Model ini terkenal karena kecepatannya, membuatnya cocok untuk aplikasi \emph{real-time}, meskipun memiliki  kompromi dalam akurasi untuk objek kecil atau yang saling berdekatan.
		      \item \rcnn~\\
		            \rcnn{} menggunakan pendekatan dua tahap. \rcnn{} menghasilkan proposal wilayah yang berpotensi mengandung objek pada tahap pertama. Pada tahap kedua, setiap proposal dianalisis lebih mendalam menggunakan \cnn{} untuk mengklasifikasikan objek dan memperbaiki bounding box. \rcnn{} dan variannya (Fast \rcnn{} dan Faster \rcnn{}) menawarkan akurasi tinggi, tetapi memerlukan waktu komputasi lebih lama sehingga lebih cocok untuk aplikasi yang tidak membutuhkan deteksi \emph{real-time}. \parencite{xie2021oriented}
	      \end{enumerate}
\end{enumerate}