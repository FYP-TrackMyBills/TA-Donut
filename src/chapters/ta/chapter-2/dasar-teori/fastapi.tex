\subsection{FastAPI}
\label{subsec:fastapi}

FastAPI adalah \emph{framework} \emph{web} Python modern dan berkinerja tinggi untuk membangun \api{} dengan Python 3.7+ berdasarkan \emph{type hints} standar Python. Dibuat oleh Sebastián Ramírez, \emph{framework} ini merepresentasikan evolusi signifikan dalam pengembangan \emph{web} Python. FastAPI menggabungkan kesederhanaan Flask dengan kekuatan pemrograman asinkron modern \parencite{ramirez2020fastapi}.

FastAPI memiliki integrasi sistem \emph{type} sebagai fitur unggulan yang memanfaatkan sistem \emph{type hints} Python untuk validasi data otomatis, serialisasi, dan pembuatan dokumentasi secara otomatis. FastAPI memiliki \emph{tools} dokumentasi yang sudah terintegrasi, yaitu OpenAPI dan Swagger. \emph{Framework} ini juga menggunakan skema \json{} yang memberikan kompatibilitas yang luas dengan ekosistem pengembangan modern. FastAPI dibangun di atas \emph{Asynchronous Server Gateway Interface} (ASGI) yang mewakili kemajuan arsitektur yang signifikan di bandingkan \emph{framework} berbasis \emph{Web Server Gateway Interface} (WSGI) tradisional \parencite{ramirez2020fastapi}. 

% Komponen arsitektur ASGI meliputi \emph{scope} yang berisi informasi \emph{scope} koneksi dengan tipe protokol dan metadata yang diperlukan. \emph{Receive} berfungsi sebagai \emph{callable awaitable} untuk menerima \emph{event} dari \emph{client}, sementara \emph{send} bertindak sebagai \emph{callable awaitable} untuk mengirim respons ke \emph{client}.

% Arsitektur berlapis FastAPI terdiri dari beberapa \layer terintegrasi yang bekerja secara harmonis. \emph{Application Layer} merupakan aplikasi inti FastAPI yang menangani \emph{routing} dan \emph{dependency injection} dengan sistem yang sophisticated. \emph{Framework Layer} menggunakan Starlette yang menyediakan kompatibilitas ASGI dan fungsionalitas \emph{framework} web yang diperlukan. \emph{Validation Layer} memanfaatkan Pydantic untuk validasi dan serialisasi data dengan performa tinggi, sementara \emph{Server Layer} menggunakan \emph{server} ASGI Uvicorn untuk penanganan \emph{request} HTTP yang efisien.

% \subsubsection{\emph{Async Programming} dan Kinerja}

% Model \emph{async processing} membedakan FastAPI dari \emph{framework synchronous} tradisional dengan memproses \emph{request} secara asinkron. \emph{Event loop single-threaded} menangani berbagai koneksi bersamaan tanpa overhead context switching yang tinggi. \emph{Coroutines} memungkinkan fungsi \emph{async} yang dapat dihentikan dan dilanjutkan sesuai kebutuhan, sementara \emph{non-blocking} I/O memberikan penanganan operasi I/O yang efisien tanpa memblokir \emph{thread} utama.

% Analisis kinerja berdasarkan \emph{benchmark} TechEmpower independen menunjukkan kinerja superior FastAPI dalam berbagai aspek. \emph{Throughput} FastAPI mencapai sekitar 3 kali lebih tinggi dari Flask dalam kondisi beban tinggi. P99 \emph{latency} yang lebih rendah dicapai karena pemrosesan \emph{async} yang efisien, sementara efisiensi memori menunjukkan \emph{footprint} memori yang berkurang per \emph{request}. Karakteristik \emph{scaling} linear dengan \emph{concurrent requests} memungkinkan aplikasi menangani beban yang meningkat dengan graceful degradation.

% \subsubsection{Integrasi dengan Sistem \cv dan \ml}

% FastAPI menyediakan arsitektur optimal untuk \emph{deployment} model \ml dalam konteks \cv dengan berbagai kemampuan teknis. Pemrosesan \emph{async} memungkinkan pemrosesan gambar \emph{non-blocking} untuk \emph{throughput} tinggi, sangat penting untuk aplikasi yang menangani volume gambar besar. Manajemen memori yang efisien memungkinkan penanganan data gambar besar tanpa degradasi performa yang signifikan. Dukungan \emph{streaming} memungkinkan kemampuan pemrosesan video \emph{real-time} untuk aplikasi yang memerlukan analisis kontinyu, sementara pemrosesan bersamaan memungkinkan \emph{multiple pipeline} pemrosesan gambar berjalan secara paralel.

% Arsitektur \emph{model serving} yang disediakan FastAPI mencakup tahapan yang komprehensif. \emph{Model loading} terjadi saat \emph{startup} aplikasi untuk memuat model \ml ke memori dengan efisien. \emph{Preprocessing pipeline} menangani tahap preprocessing gambar sebelum inferensi untuk memastikan format data yang sesuai. Inferensi model dilakukan dengan eksekusi model \ml untuk prediksi yang akurat dan cepat. \emph{Postprocessing} memproses hasil prediksi untuk format yang diinginkan sesuai kebutuhan aplikasi, dan \emph{response serialization} melakukan serialisasi hasil ke format \json atau format lainnya sesuai spesifikasi \api.

% FastAPI mendukung integrasi dengan berbagai \emph{library} \ml Python seperti \pytorch, \tensorflow, dan \onnx Runtime, menjadikannya pilihan yang sesuai untuk \emph{deployment} model \cv dalam lingkungan produksi \parencite{techempowerbenchmark2023}.
