\subsection{\flutter}
\label{subsec:flutter}

\flutter{} adalah SDK (\emph{Software Development Kit}) UI \emph{open-source} dari Google yang diluncurkan pada tahun 2017. \flutter{} dirancang sebagai \emph{framework cross-platform} yang memungkinkan pengembang untuk membuat aplikasi yang dapat dijalankan secara \emph{native} pada berbagai \emph{platform} pada \emph{mobile}, \emph{web}, \emph{desktop}, dan perangkat \emph{embedded} dari satu \emph{codebase} \parencite{flutter2021}.

\flutter{} memiliki beberapa karakteristik utama yang membedakannya dari \emph{framework} pengembangan aplikasi lainnya, yaitu arsitektur \emph{single codebase} yang memungkinkan pengembangan \emph{cross-platform} dengan penulisan kode satu kali. Flutter juga menggunakan arsitektur berbasis \emph{widget}. Aristektur ini membuat segala sesuatu menjadi \emph{widget} yang merupakan komponen UI yang bersifat \emph{immutable}. 

% \flutter{} terdiri dari tiga \layer utama yang saling terintegrasi. \emph{Layer Framework} yang ditulis dalam Dart untuk implementasi desai \emph{platform-specific}. \emph{Layer widget} menyediakan abstraksi komposisi untuk membangun UI yang kompleks. \emph{layer rendering} menangani manajemen \emph{layout} dan posisi elemen. \emph{Layer foundation} menyediakan layanan inti seperti animasi, \emph{painting}, dan \emph{gestures} yang fundamental untuk interaksi pengguna.

% \emph{Layer Engine} yang ditulis dalam C++ berisi mesin grafis Skia untuk \emph{rendering} dengan transisi menuju Impeller untuk performa yang lebih baik. Runtime Dart dan \emph{virtual machine} berjalan pada layer ini untuk eksekusi kode aplikasi. \emph{Layout} teks dan operasi \emph{file} I/O juga dikelola pada layer ini, bersama dengan implementasi tingkat rendah dari \api inti \flutter.

% \emph{Layer Embedder} bersifat spesifik platform dan menangani integrasi dengan sistem operasi yang mendasarinya. Untuk Android menggunakan Java/C++, sedangkan iOS menggunakan Swift/Objective-C. Layer ini bertanggung jawab untuk koordinasi layanan sistem operasi, manajemen \emph{event loop}, dan eksposur \api spesifik platform untuk fungsionalitas yang tidak tersedia secara cross-platform.

% \subsubsection{\emph{Cross-Platform Development}}

\flutter{} menyediakan efisiensi pengembangan yang signifikan dibandingkan pengembangan tradisional. Pengembang dapat menargetkan \emph{multiple platform} secara bersamaan, mengurangi kebutuhan tenaga kerja dan kompleksitas manajemen proyek. Lingkungan pengembangan dan \emph{tools} yang terpadu memungkinkan konsistensi selama proses pengembangan.

% Konsistensi menjadi aspek penting dengan UI yang identik di semua platform, memberikan pengalaman pengguna yang seragam. Perilaku dan kinerja yang konsisten mengurangi bug spesifik platform dan memberikan pengalaman \emph{brand} yang terpadu. Hal ini sangat penting untuk aplikasi komersial yang memerlukan identitas visual yang kuat di berbagai platform.

% \subsubsection{Integrasi dengan \ml dan \cv}

\flutter{} menyediakan beberapa jalur untuk integrasi \ml{} yang relevan untuk aplikasi \cv{}. Integrasi \tensorflow Lite dengan \flutter{} didukung melalui paket \texttt{tflite\_flutter} yang memungkinkan inferensi model langsung pada perangkat. Implementasi dan inferensi dari model kustom dan \emph{pre-trained} juga didukung untuk kompatibilitas \emph{multi-platform}.

% Firebase ML Kit menyediakan \api ML \emph{on-device} untuk tugas umum seperti pengenalan teks, deteksi wajah, dan \emph{scanning barcode}. \emph{Labeling} gambar dan deteksi \emph{landmark} juga tersedia dengan kemampuan pemrosesan \emph{real-time} yang memadai untuk aplikasi produksi. Kemampuan \cv dalam \flutter mencakup pemrosesan \emph{stream} kamera \emph{real-time} untuk aplikasi yang memerlukan analisis video langsung. Klasifikasi gambar dan deteksi objek dapat diimplementasikan dengan mudah, begitu juga dengan implementasi \ocr untuk pengenalan teks. Pengenalan wajah dan autentikasi biometrik juga didukung untuk aplikasi keamanan \parencite{flutteronnx2022}.
