\subsection{Metrik Evaluasi}
\label{subsec:metrik-evaluasi}

Evaluasi kinerja sistem \cv dan \ml memerlukan metrik yang komprehensif dan sesuai dengan domain aplikasi. Untuk sistem pemahaman dokumen dan pengenalan teks, diperlukan kombinasi metrik klasifikasi standar dan metrik khusus untuk evaluasi pemahaman dokumen terstruktur dan akurasi pengenalan teks \parencite{rice1996ocr}.

\subsubsection{Metrik Klasifikasi Standar}

\accuracy mengukur proporsi prediksi yang benar baik positif maupun negatif di seluruh \emph{instance} dalam \dataset. Formula \accuracy didefinisikan sebagai perbandingan antara jumlah prediksi benar dengan total prediksi yang dibuat. Dalam konteks klasifikasi biner, \accuracy dihitung menggunakan matriks konfusi yang terdiri dari \emph{True Positive} (TP), \emph{True Negative} (TN), \emph{False Positive} (FP), dan \emph{False Negative} (FN).

\begin{equation}
\text{Accuracy} = \frac{TP + TN}{TP + TN + FP + FN}
\end{equation}

\precision mengukur proporsi prediksi positif yang benar di antara semua prediksi positif yang dibuat model. Metrik ini sangat penting ketika biaya \emph{false positive} tinggi dalam aplikasi tertentu.

\begin{equation}
\text{Precision} = \frac{TP}{TP + FP}
\end{equation}

\precision tinggi menunjukkan bahwa model meminimalkan \emph{false positive}, memastikan prediksi positif kemungkinan besar benar dan dapat diandalkan untuk pengambilan keputusan.

\recall atau sensitivitas mengukur proporsi \emph{instance} positif aktual yang berhasil diidentifikasi dengan benar oleh model. Metrik ini kritikal ketika biaya \emph{false negative} tinggi dalam sistem yang memerlukan deteksi lengkap.

\begin{equation}
\text{Recall} = \frac{TP}{TP + FN}
\end{equation}

\recall tinggi berarti model menangkap sebagian besar \emph{instance} positif, meminimalkan kemungkinan melewatkan data penting yang harus dideteksi.

F1-\emph{score} adalah rata-rata harmonik dari \precision dan \recall, memberikan ukuran seimbang yang mempertimbangkan kedua metrik secara setara. Penggunaan rata-rata harmonik memiliki signifikansi khusus karena menghukum nilai ekstrem lebih dari rata-rata aritmatika.

\begin{equation}
\text{F1-Score} = 2 \times \frac{\text{Precision} \times \text{Recall}}{\text{Precision} + \text{Recall}}
\end{equation}

F1-\emph{score} memerlukan \precision dan \recall tinggi untuk mencapai skor tinggi, menjadi nol jika salah satu dari \precision atau \recall adalah nol, dan lebih konservatif dari rata-rata aritmatika dengan menekankan keseimbangan antara kedua metrik.

% \subsubsection{\ted (\emph{Tree Edit Distance})}

\ted adalah \emph{sequence minimum-cost} dari operasi edit \emph{node} yang diperlukan untuk mentransformasi satu \emph{tree} menjadi \emph{tree} lain \parencite{zhang1989tree}. Operasi edit yang dapat dilakukan meliputi \emph{delete} untuk menghapus \emph{node} dan menghubungkan \emph{children}-nya ke \emph{parent}, \emph{insert} untuk menambahkan \emph{node} antara \emph{node} yang ada dan \emph{children}-nya, dan \emph{relabel} untuk mengubah \emph{label} dari \emph{node}.

Formulasi matematika \ted didefinisikan sebagai optimasi biaya minimum untuk transformasi struktur \emph{tree}.

\begin{equation}
\text{TED}(T_1, T_2) = \min\{\text{cost}(\text{sequence}) : \text{sequence transforms } T_1 \text{ to } T_2\}
\end{equation}

Definisi rekursif \ted memungkinkan perhitungan efisien melalui pemrograman dinamis.

\begin{equation}
\text{TED}(F, G) = \min \begin{cases}
\text{TED}(F-v, G) + \text{cost}(\text{delete } v) \\
\text{TED}(F, G-w) + \text{cost}(\text{insert } w) \\
\text{TED}(F-v, G-w) + \text{cost}(\text{relabel } v \rightarrow w)
\end{cases}
\end{equation}

\ted memiliki aplikasi luas dalam pemahaman dokumen terstruktur, mencakup analisis similaritas dokumen XML/HTML, analisis dan perbandingan \emph{layout} dokumen, pengenalan struktur dokumen yang kompleks, dan perbandingan data hierarkis. Dalam konteks pemahaman dokumen, \ted berguna untuk mengevaluasi seberapa baik model memahami struktur hierarkis informasi dalam dokumen terstruktur.

\subsubsection{\mcer (\emph{Mean Character Error Rate})}

\mcer adalah rata-rata \emph{Character Error Rate} di beberapa dokumen atau segmen teks, memberikan ukuran komprehensif akurasi \ocr \parencite{holley2009ocr}. Metrik ini sangat relevan untuk sistem pengenalan teks yang bergantung pada akurasi pengenalan karakter.

\begin{equation}
\text{MCER} = \frac{1}{N} \times \sum_{i=1}^{N} \text{CER}_i
\end{equation}

dimana $\text{CER}_i$ adalah \emph{Character Error Rate} untuk dokumen $i$ yang dihitung berdasarkan \emph{Levenshtein Distance}.

\begin{equation}
\text{CER}_i = \frac{S_i + D_i + I_i}{N_i}
\end{equation}

Parameter dalam perhitungan ini meliputi $S$ yang merepresentasikan jumlah substitusi karakter, $D$ untuk jumlah \emph{deletion} atau penghapusan karakter, $I$ untuk jumlah \emph{insertion} atau penambahan karakter, dan $N$ sebagai total jumlah karakter dalam \emph{ground truth}.

\mcer digunakan secara luas dalam berbagai aplikasi evaluasi \ocr. \emph{Benchmark} mesin \ocr menggunakan \mcer untuk membandingkan kinerja algoritma yang berbeda dan melacak \emph{improvement} dari waktu ke waktu. Proyek digitalisasi menggunakan metrik ini untuk penilaian kualitas hasil konversi dokumen fisik ke digital.

\subsubsection{Integrasi Metrik untuk Evaluasi Sistem}

Evaluasi sistem pemahaman dokumen yang komprehensif memerlukan pendekatan \emph{multi-level} yang mengintegrasikan berbagai metrik. Tingkat karakter menggunakan \mcer untuk mengukur akurasi pengenalan teks individual dan memastikan setiap karakter dapat dibaca dengan benar. Tingkat \emph{field} memanfaatkan \accuracy, \precision, \recall, dan F1-\emph{score} untuk evaluasi ekstraksi \emph{field} spesifik dalam dokumen terstruktur. Tingkat struktur menggunakan \ted untuk mengevaluasi pemahaman struktur dokumen keseluruhan dan hubungan antar elemen. Tingkat sistem mengkombinasikan semua metrik untuk evaluasi kinerja sistem secara holistik.

Interpretasi dan \emph{trade-off} antar metrik memberikan perspektif berbeda tentang kinerja sistem. \accuracy memberikan gambaran umum kinerja tetapi dapat menyesatkan pada data tidak seimbang, sehingga perlu dikombinasikan dengan metrik lain. \precision menjadi penting ketika biaya \emph{false positive} tinggi dalam aplikasi yang memerlukan presisi tinggi. \recall kritikal ketika biaya \emph{false negative} tinggi dalam sistem yang memerlukan deteksi lengkap. F1-\emph{score} memberikan keseimbangan antara \precision dan \recall untuk evaluasi menyeluruh. \ted mengukur pemahaman struktur dokumen yang kompleks dan hubungan hierarkis antar elemen, sementara \mcer memberikan ukuran granular akurasi pengenalan teks pada level karakter.

Penggunaan metrik evaluasi yang komprehensif memastikan bahwa sistem pemahaman dokumen tidak hanya akurat dalam mengenali teks, tetapi juga mampu memahami struktur dokumen dan mengekstrak informasi relevan dengan presisi tinggi yang diperlukan untuk berbagai aplikasi \parencite{bille2005tree}.
