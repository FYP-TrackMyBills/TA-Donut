\subsection{Metrik Evaluasi}
\label{subsec:metrik-evaluasi}

Evaluasi kinerja sistem \ml{} memerlukan metrik yang komprehensif dan sesuai dengan domain model. Sistem pemahaman dokumen dan pengenalan teks memerlukan kombinasi metrik klasifikasi standar dan metrik khusus untuk evaluasi pemahaman dokumen. Metrik evaluasi yang digunakan dalam penelitian ini mencakup \accuracy, \precision, \recall, \fscore, \coverage, dan \mcer{} (\emph{Mean Character Error Rate}).

\subsubsection{Metrik Klasifikasi Standar}

\accuracyfl{} mengukur proporsi prediksi yang benar, baik positif maupun negatif, di seluruh \emph{instance} dalam \dataset. Formula \accuracy{} didefinisikan sebagai perbandingan antara jumlah prediksi benar dengan total prediksi yang dibuat. Dalam konteks klasifikasi biner, \accuracy{} dihitung menggunakan matriks yang terdiri dari \emph{True Positive} (TP), \emph{True Negative} (TN), \emph{False Positive} (FP), dan \emph{False Negative} (FN). Persamaan \eqref{eq:accuracy} menunjukkan perhitungan \accuracy{} \parencite{jayaswal2020evalmetrics}. 

\begin{equation}
    \label{eq:accuracy}
\text{Accuracy} = \frac{TP + TN}{TP + TN + FP + FN}
\end{equation}
\addcontentsline{loe}{myequations}{\protect\numberline{\theequation}Persamaan \accuracy}

\precisionfl{} mengukur proporsi prediksi positif yang benar di antara semua prediksi positif yang dibuat model. Metrik ini sangat penting ketika biaya \emph{false positive} tinggi dalam aplikasi tertentu. Persamaan \eqref{eq:precision} menunjukkan perhitungan \precision. \precisionfl{} tinggi menunjukkan bahwa model meminimalkan \emph{False Positive}. Dengan demikian, model dapat meningkatkan kemungkinan kebenaran prediksi positif sehingga hasil prediksi lebih dapat diandalkan untuk pengambilan keputusan.

\begin{equation}
    \label{eq:precision}
\text{Precision} = \frac{TP}{TP + FP}
\end{equation}
\addcontentsline{loe}{myequations}{\protect\numberline{\theequation}Persamaan \precision}

\recallfl{} mengukur proporsi prediksi positif aktual yang berhasil diidentifikasi dengan benar oleh model. Metrik ini penting ketika biaya \emph{False Negative} tinggi dalam sistem yang memerlukan deteksi lengkap. Persamaan \eqref{eq:recall} menunjukkan rumus yang dapat digunakan untuk menghitung \recall{} \parencite{jayaswal2020evalmetrics}. Angka \recall{} tinggi menunjukkan bahwa model dapat menangkap sebagian besar prediksi positif. Dengan demikian, model meminimalkan kemungkinan adanya data penting yang terlewat dalam upaya prediksi.

\begin{equation}
    \label{eq:recall}
\text{Recall} = \frac{TP}{TP + FN}
\end{equation}
\addcontentsline{loe}{myequations}{\protect\numberline{\theequation}Persamaan \recall}

\fscore{} adalah rata-rata harmonik dari \precision{} dan \recall{}. \fscore{} memberikan ukuran seimbang yang mempertimbangkan kedua metrik secara setara. Persamaan \eqref{eq:fscore} menunjukkan rumus yang dapat digunakan untuk menghitung \fscore{} \parencite{jayaswal2020evalmetrics}. \fscore{} tinggi menunjukkan bahwa model tidak hanya akurat dalam prediksi positif tetapi juga menangkap sebagian besar prediksi positif yang relevan. \fscore{} memerlukan \precision{} dan \recall{} tinggi untuk mencapai skor tinggi. \fscore{} akan menjadi nol jika salah satu dari \precision{} atau \recall{} adalah nol.

\begin{equation}
    \label{eq:fscore}
\text{F1-Score} = 2 \times \frac{\text{Precision} \times \text{Recall}}{\text{Precision} + \text{Recall}}
\end{equation}
\addcontentsline{loe}{myequations}{\protect\numberline{\theequation}Persamaan \fscore}

% \subsubsection{\ted (\emph{Tree Edit Distance})}

\ted adalah \emph{sequence minimum-cost} dari operasi edit \emph{node} yang diperlukan untuk mentransformasi satu \emph{tree} menjadi \emph{tree} lain \parencite{zhang1989tree}. Operasi edit yang dapat dilakukan meliputi \emph{delete} untuk menghapus \emph{node} dan menghubungkan \emph{children}-nya ke \emph{parent}, \emph{insert} untuk menambahkan \emph{node} antara \emph{node} yang ada dan \emph{children}-nya, dan \emph{relabel} untuk mengubah \emph{label} dari \emph{node}.

Formulasi matematika \ted didefinisikan sebagai optimasi biaya minimum untuk transformasi struktur \emph{tree}.

\begin{equation}
\text{TED}(T_1, T_2) = \min\{\text{cost}(\text{sequence}) : \text{sequence transforms } T_1 \text{ to } T_2\}
\end{equation}

Definisi rekursif \ted memungkinkan perhitungan efisien melalui pemrograman dinamis.

\begin{equation}
\text{TED}(F, G) = \min \begin{cases}
\text{TED}(F-v, G) + \text{cost}(\text{delete } v) \\
\text{TED}(F, G-w) + \text{cost}(\text{insert } w) \\
\text{TED}(F-v, G-w) + \text{cost}(\text{relabel } v \rightarrow w)
\end{cases}
\end{equation}

\ted memiliki aplikasi luas dalam pemahaman dokumen terstruktur, mencakup analisis similaritas dokumen XML/HTML, analisis dan perbandingan \emph{layout} dokumen, pengenalan struktur dokumen yang kompleks, dan perbandingan data hierarkis. Dalam konteks pemahaman dokumen, \ted berguna untuk mengevaluasi seberapa baik model memahami struktur hierarkis informasi dalam dokumen terstruktur.

\subsubsection{\mcer{} (\emph{Mean Character Error Rate})}

\mcer{} adalah rata-rata \emph{Character Error Rate} (CER) di beberapa dokumen atau segmen teks yang memberikan ukuran komprehensif akurasi deteksi teks \parencite{neudecker2021survey}. Metrik ini sangat relevan untuk sistem pengenalan teks yang bergantung pada akurasi pengenalan karakter. Persamaan \eqref{eq:mcer} menunjukkan rumus yang digunakan untuk menghitung \mcer{}. Metrik ini mengukur kesalahan pengenalan karakter pada tingkat granular dan memberikan informasi tentang akurasi pengenalan teks pada level karakter.

\begin{equation}
    \label{eq:mcer}
\text{mCER} = \frac{\sum{S_i + D_i + I_i}}{\sum{N_i}}
\end{equation}
\addcontentsline{loe}{myequations}{\protect\numberline{\theequation}Persamaan \emph{Mean Character Error Rate} (mCER)}

Persamaan \eqref{eq:cer} menunjukkan cara perhitungan CER dengan parameter yang meliputi $S$ yang merepresentasikan jumlah substitusi karakter, $D$ untuk jumlah \emph{deletion} atau penghapusan karakter, $I$ untuk jumlah \emph{insertion} atau penambahan karakter, dan $N$ sebagai total jumlah karakter dalam \emph{Ground Truth}.

\begin{equation}
    \label{eq:cer}
\text{CER}_i = \frac{S_i + D_i + I_i}{N_i}
\end{equation}
\addcontentsline{loe}{myequations}{\protect\numberline{\theequation}Persamaan \emph{Character Error Rate} (CER)}

% \mcer{} digunakan secara luas dalam berbagai aplikasi evaluasi \ocr. \emph{Benchmark} mesin \ocr{} menggunakan \mcer{} untuk membandingkan kinerja algoritma yang berbeda dan memantau perkembangan dari waktu ke waktu. Proyek digitalisasi menggunakan metrik ini untuk menilai kualitas hasil konversi dokumen fisik ke digital.

% \subsubsection{Integrasi Metrik untuk Evaluasi Sistem}

% Evaluasi sistem pemahaman dokumen yang komprehensif memerlukan pendekatan \emph{multi-level} yang mengintegrasikan berbagai metrik. Tingkat karakter menggunakan \mcer untuk mengukur akurasi pengenalan teks individual dan memastikan setiap karakter dapat dibaca dengan benar. Tingkat \emph{field} memanfaatkan \accuracy, \precision, \recall, dan F1-\emph{score} untuk evaluasi ekstraksi \emph{field} spesifik dalam dokumen terstruktur. Tingkat struktur menggunakan \ted untuk mengevaluasi pemahaman struktur dokumen keseluruhan dan hubungan antar elemen. Tingkat sistem mengkombinasikan semua metrik untuk evaluasi kinerja sistem secara holistik.

% Interpretasi dan \emph{trade-off} antar metrik memberikan perspektif berbeda tentang kinerja sistem. \accuracy memberikan gambaran umum kinerja tetapi dapat menyesatkan pada data tidak seimbang sehingga perlu dikombinasikan dengan metrik lain. \precision menjadi penting ketika biaya \emph{false positive} tinggi dalam aplikasi yang memerlukan presisi tinggi. \recall kritikal ketika biaya \emph{false negative} tinggi dalam sistem yang memerlukan deteksi lengkap. F1-\emph{score} memberikan keseimbangan antara \precision dan \recall untuk evaluasi menyeluruh. \ted mengukur pemahaman struktur dokumen yang kompleks dan hubungan hierarkis antar elemen, sementara \mcer memberikan ukuran granular akurasi pengenalan teks pada level karakter.

% Penggunaan metrik evaluasi yang komprehensif memastikan bahwa sistem pemahaman dokumen tidak hanya akurat dalam mengenali teks, tetapi juga mampu memahami struktur dokumen dan mengekstrak informasi relevan dengan presisi tinggi yang diperlukan untuk berbagai aplikasi \parencite{bille2005tree}.

\subsubsection{\emph{System Usability Scale} (SUS)}
\label{subsubsec:sus}

\emph{System Usability Scale} (SUS) adalah instrumen evaluasi usabilitas yang dikembangkan oleh Brooke pada tahun 1996 sebagai alat ukur cepat dan andal untuk mengevaluasi usabilitas sistem \parencite{brooke1996sus}. SUS terdiri dari sepuluh pertanyaan dengan skala \emph{Likert} lima poin yang dirancang untuk memberikan skor tunggal yang merepresentasikan penilaian subjektif pengguna terhadap usabilitas sistem. Instrumen ini telah menjadi salah satu metode evaluasi usabilitas yang paling banyak digunakan dalam interaksi manusia komputer karena kesederhanaan, reliabilitas, dan validitasnya \parencite{bangor2008empirical}.

SUS menggunakan sepuluh pertanyaan standar yang mencakup aspek-aspek fundamental usabilitas, yaitu efektivitas, efisiensi, dan kepuasan pengguna. Lima pertanyaan dinyatakan secara positif dan lima lainnya secara negatif untuk mengurangi kemungkinan respons bias dan meningkatkan keandalan pengukuran. Setiap pertanyaan dinilai pada skala 1 hingga 5, dengan 1 menunjukkan "sangat tidak setuju" dan 5 menunjukkan "sangat setuju". Pertanyaan SUS meliputi penilaian terhadap keinginan menggunakan sistem secara teratur, kompleksitas sistem yang dirasakan, kemudahan penggunaan, kebutuhan dukungan teknis, integrasi fungsi sistem, inkonsistensi, kemudahan pembelajaran, dan kepercayaan diri pengguna dalam menggunakan sistem \parencite{brooke1996sus}.

Perhitungan skor SUS mengikuti metodologi khusus untuk menghasilkan skor akhir antara 0 hingga 100. Untuk pertanyaan bernomor ganjil (pernyataan positif), skor dihitung dengan mengurangi 1 dari respons pengguna. Untuk pertanyaan bernomor genap (pernyataan negatif), skor dihitung dengan mengurangi respons pengguna dari 5. Jumlah total skor dari sepuluh pertanyaan kemudian dikalikan dengan 2,5 untuk menghasilkan skor SUS akhir. Persamaan \eqref{eq:sus-score} menunjukkan formula perhitungan skor SUS dengan $R_i$ adalah respons untuk pertanyaan $i$ \parencite{tullis2013measuring}.

\begin{equation}
    \label{eq:sus-score}
\text{SUS Score} = \left(\sum_{i=1,3,5,7,9} (R_i - 1) + \sum_{j=2,4,6,8,10} (5 - R_j)\right) \times 2.5
\end{equation}
\addcontentsline{loe}{myequations}{\protect\numberline{\theequation}Persamaan Skor SUS}

Interpretasi skor SUS menggunakan \emph{benchmark} yang telah ditetapkan berdasarkan penelitian ekstensif terhadap ribuan evaluasi usabilitas. Skor SUS di atas 68 dianggap diatas rata-rata, skor antara 68-80 menunjukkan usabilitas yang baik, skor 80-90 menunjukkan usabilitas yang sangat baik, dan skor di atas 90 menunjukkan usabilitas yang luar biasa. Skor di bawah 68 menunjukkan bahwa sistem memerlukan perbaikan usabilitas yang signifikan \parencite{bangor2009determining}. Penelitian oleh Sauro pada tahun 2011 menunjukkan bahwa skor SUS memiliki korelasi yang kuat dengan metrik usabilitas objektif seperti tingkat penyelesaian tugas dan efisiensi \parencite{sauro2011measuring}.

Keunggulan SUS sebagai instrumen evaluasi meliputi kemudahan administrasi, waktu pengisian yang singkat (2-5 menit), skor tunggal yang mudah dipahami, dan validitas yang telah terbukti melalui berbagai domain aplikasi \parencite{lewis2018sus}. SUS juga memungkinkan perbandingan usabilitas antar sistem yang berbeda dan \emph{benchmarking} terhadap standar industri.

Dalam konteks evaluasi sistem \emph{mobile} untuk ekstraksi data pembayaran, SUS menjadi instrumen yang relevan untuk mengukur sejauh mana pengguna Gen Z dapat menggunakan aplikasi dengan mudah dan efektif. Evaluasi usabilitas dengan SUS dapat mengidentifikasi apakah antarmuka aplikasi memenuhi ekspektasi pengguna dalam hal kemudahan penggunaan, efisiensi proses ekstraksi data, dan kepuasan pengguna secara keseluruhan terhadap sistem yang dibangun.
