\subsection{Pemrosesan Dokumen Finansial}
\label{subsec:pemrosesan-dokumen-finansial}

Dokumen struk pembayaran memegang peranan penting dalam pengelolaan
keuangan baik individu maupun institusi. Proses ekstraksi data dari dokumen-dokumen ini sering kali menghadapi tantangan yang signifikan, termasuk keanekaragaman tata letak, kualitas gambar yang bervariasi, dan kebutuhan untuk
mengekstrak informasi dengan akurasi tinggi. Sebagai contoh, elemen-elemen penting seperti total harga, tanggal transaksi, dan nomor rekening sering kali tersebar di berbagai posisi dalam dokumen yang berbeda, sehingga membuat ekstraksi data secara manual menjadi lambat dan rentan terhadap kesalahan.

Teknologi yang umumnya dapat digunakan untuk menanggulangi masalah ini adalah \cv{}, salah satunya \ocr. Secara tradisional, \ocr{} digunakan untuk mengonversi teks dalam gambar menjadi format digital yang dapat diproses lebih lanjut. Namun, OCR memiliki
keterbatasan, terutama dalam menangani dokumen dengan kualitas gambar rendah atau format teks yang tidak standar. Selain itu, ketergantungan pada OCR dapat meningkatkan kompleksitas dan biaya pemrosesan \parencite{kim2021donut}.

\dlfl{} memiliki berbagai jenis algoritma dan model yang mendukung pengembangan sistem tersebut. \dlfl{} merupakan cabang \ml{} yang memanfaatkan jaringan saraf tiruan berlapis untuk mengenali pola kompleks dalam data, termasuk dokumen finansial. Salah satu model yang umum digunakan adalah \cnn. \cnn{} dirancang untuk memproses data berbentuk gambar dengan mendeteksi fitur lokal seperti teks, garis, atau elemen visual lainnya sehingga \linebreak cocok untuk ekstraksi data dari dokumen. Namun, \cnn{} terbatas dalam memahami hubungan global antar
elemen dalam dokumen sehingga \transformer{} hadir sebagai solusi yang lebih canggih \parencite{alzubaidi2021review}.

Dengan mekanisme \attention-nya, \transformer{}, seperti LayoutLM dan
Donut, mampu menangkap hubungan kontekstual antar elemen secara global. Hal ini membuat penggunaan \transformer{} dengan struktur kompleks dan informasi
tersebar. \transformer{} memberikan pendekatan yang kuat untuk memastikan akurasi dan efisiensi dalam pemrosesan dokumen finansial.