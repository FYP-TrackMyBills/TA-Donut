\section{Penelitian Terkait}
\label{sec:penelitianterkait}

Subbab penelitian terkait akan memberikan rangkuman dari penelitian-penelitian yang pernah dilakukan pada kasus yang serupa atau menggunakan alternatif solusi yang serupa. Subbab ini ditujukan sebagai referensi untuk mendapatkan penggunaan alternatif solusi atau pendekatan yang telah dilakukan  terhadap masalah serupa. Berikut adalah penelitian-penelitian terkait tersebut. 

\subsection{\textit{A Comprehensive Analysis of LayoutLM and Donut for Document Classification}}
\label{sec:penelitian-1}
Penelitian oleh \textciteyear{bajrami2023comprehensive}  akan melakukan analisis komparasi dari dua buah \textit{pre-trained model} berbasis \transformer{}, yaitu \layoutlm{} dan \donut, untuk klasifikasi dokumen. Penelitian ini melakukan perbandingan kinerja kedua model pada dua \dataset{} dengan karakteristik berbeda dengan melakukan evaluasi 
terhadap nilai \accuracy, \precision, dan \fscore{} dari kedua model. 

\subsubsection{Gambaran Umum}
Penelitian ini bertujuan untuk membandingkan kinerja \donut{} dan \layoutlm{} dalam klasifikasi dokumen. Studi ini membandingkan kinerja keduanya pada data yang tidak terstuktur (\textit{unstructured data}), seperti struk, \textit{invoice}, dan catatan tulis tangan.

\subsubsection{Analisis dan Metode}
\layoutlm{} adalah model yang mengombinasikan deteksi objek, pengenalan teks, dan analisis \textit{layout}. \layoutlm{} memiliki dependensi terhadap teknologi \ocr{} untuk melakukan ekstraksi teks dari dokumen berbentuk gambar. \donut{} adalah sebuah model \transformer{} untuk melakukan pemahaman dokumen yang telah bersifat \sotafull{} dan tidak memerlukan integrasi \ocr{}.

Analisis komparasi kedua model ini dilakukan pada dua jenis \dataset. \datasetfl{} pertama adalah sebuah \dataset{} dengan jumlah 10.000 sampel data yang terdiri dari dokumen perusahaan dengan kompleksitas yang lebih tinggi. Dataset 
kedua terdiri dari 50.000 sampel yang lebih terdiversifikasi. Kedua model di-\textit{fine-tuned} dengan menggunakan \textit{transfer learning} dengan \textit{pre-trained weights} dan evaluasi dilakukan dengan metrik \accuracy, \precision, dan \fscore.  

Pada \dataset{} 1, hasil yang didapatkan dari model \layoutlm{} memiliki angka yang lebih tinggi, yaitu pada angka 0,88 dibandingkan dengan penggunaan model \donut{} pada angka 0,74. Hal ini mengindikasikan bahwa \layoutlm{} lebih unggul untuk menangani dokumen dengan struktur yang lebih kompleks dibandingkan \donut{}. Pada \dataset{} 2, \donut{} memiliki angka akurasi yang lebih tinggi, yaitu 0,91 sedangkan \layoutlm{} dengan mendapatkan angka 0,82 \parencite{bajrami2023comprehensive}.

\subsubsection{Kesimpulan}

\layoutlm{} dan \donut{} merupakan pilihan yang efektif untuk melakukan klasifikasi dokumen, tetapi dengan kebutuhan yang berbeda. \layoutlm{} lebih 
cocok untuk digunakan pada dokumen dengan kompleksitas yang lebih tinggi, namun hasil yang didapatkan lebih lama karena memiliki \textit{pipeline} yang lebih 
panjang akibat adanya proses \ocr{} terlebih dahulu. \donut{} lebih cocok untuk digunakan pada \dataset{} dengan diversifikasi dokumen yang lebih banyak dan dapat menghasilkan dengan lebih cepat akibat \emph{pipeline} yang lebih pendek akibat tidak perlu integrasi dengan \ocr. Dokumen yang memiliki \textit{noise} atau resolusi yang buruk akan mengurangi akurasi secara signifikan untuk \layoutlm{} dan \donut.

\begin{table}[h!]
    \centering % Centers the table on the page
    \caption{Hasil perbandingan implementasi Donut sebelum dan setelah \textit{fine-tuning} \parencite{carta2024end}}
    \label{tab:donut-comparison-on-id-documents}
    \begin{tabularx}{\textwidth}{|l|C|C|C|C|C|C|}
        \hline
        % --- First Header Row ---
        % The "Dokumen" cell spans 2 rows vertically.
        % The "Donut Base" and "Donut Synth" cells each span 3 columns horizontally.
        \multirow{2}{*}{Dokumen} & \multicolumn{3}{c|}{Donut Base} & \multicolumn{3}{c|}{Donut Synth} \\
        \cline{2-7} % Creates a horizontal line from column 2 to 7
        
        % --- Second Header Row ---
        % The first column is left blank because the multirow cell is occupying it.
        & TED & F1 & mCER & TED & F1 & mCER \\
        \hline% Double line to separate header from body
        
        % --- Table Body ---
        EIC & 0.821 & 0.441 & 0.136 & 0.893 & 0.554 & 0.084 \\ \hline
        IC  & 0.769 & 0.594 & 0.171 & 0.885 & 0.712 & 0.090 \\ \hline
        HC  & 0.959 & 0.891 & 0.039 & 0.986 & 0.943 & 0.014 \\ \hline
        DL  & 0.929 & 0.813 & 0.057 & 0.973 & 0.866 & 0.025 \\ \hline
    \end{tabularx}
\end{table}

% \begin{table}[h!]
%     \caption{Tabel dengan empat kolom lebar merata}
%     \label{tab:random-even}
    
%     % 3. Use the tabularx environment and the new 'C' column type
%     \begin{tabularx}{\textwidth}{|X|X|X|X|}
%         \hline
%         Title1 & Title2 & Title3 & Title4 \\
%         \hline
%         dafdafdsfdsdfsfdsadfadsfdasfdfdasfd   & dafdafdsfdsdfsfdsadfadsfdasfdfdasfd   & 0.68   & 1.90   \\ \hline
%         dafdafdsfdsdfsfdsadfadsfdasfdfdasfd   & 2.92   & 1.06   & 2.75   \\ \hline
%         dafdafdsfdsdfsfdsadfadsfdasfdfdasfd   & 3.23   & 1.16   & 3.78   \\ \hline
%         3791   & 4.39   & 1.40   & 4.14   \\ \hline
%         4625   & 6.72   & 1.87   & 5.59   \\
%         \hline
%     \end{tabularx}
% \end{table}