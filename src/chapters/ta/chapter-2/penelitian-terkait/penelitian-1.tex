\subsection{\textit{A Comprehensive Analysis of LayoutLM and Donut for Document Classification}}
\label{sec:penelitian-1}
Penelitian oleh \textciteyear{bajrami2023comprehensive}  akan melakukan analisis komparasi dari dua buah \textit{pre-trained model} berbasis \transformer{}, yaitu \layoutlm{} dan \donut, untuk klasifikasi dokumen. Penelitian ini melakukan perbandingan kinerja kedua model pada dua \dataset{} dengan karakteristik berbeda dengan melakukan evaluasi 
terhadap nilai \accuracy, \precision, dan \fscore{} dari kedua model. 

\subsubsection{Gambaran Umum}
Penelitian ini bertujuan untuk membandingkan kinerja \donut{} dan \layoutlm{} dalam klasifikasi dokumen. Studi ini membandingkan kinerja keduanya pada data yang tidak terstuktur (\textit{unstructured data}), seperti struk, \textit{invoice}, dan catatan tulis tangan.

\subsubsection{Analisis dan Metode}
\layoutlm{} adalah model yang mengombinasikan deteksi objek, pengenalan teks, dan analisis \textit{layout}. \layoutlm{} memiliki dependensi terhadap teknologi \ocr{} untuk melakukan ekstraksi teks dari dokumen berbentuk gambar. \donut{} adalah sebuah model \transformer{} untuk melakukan pemahaman dokumen yang telah bersifat \sotafull{} dan tidak memerlukan integrasi \ocr{}.

Analisis komparasi kedua model ini dilakukan pada dua jenis \dataset. \datasetfl{} pertama adalah sebuah \dataset{} dengan jumlah 10.000 sampel data yang terdiri dari dokumen perusahaan dengan kompleksitas yang lebih tinggi. Dataset 
kedua terdiri dari 50.000 sampel yang lebih terdiversifikasi. Kedua model di-\textit{fine-tuned} dengan menggunakan \textit{transfer learning} dengan \textit{pre-trained weights} dan evaluasi dilakukan dengan metrik \accuracy, \precision, dan \fscore.  

Pada \dataset{} 1, hasil yang didapatkan dari model \layoutlm{} memiliki angka yang lebih tinggi, yaitu pada angka 0,88 dibandingkan dengan penggunaan model \donut{} pada angka 0,74. Hal ini mengindikasikan bahwa \layoutlm{} lebih unggul untuk menangani dokumen dengan struktur yang lebih kompleks dibandingkan \donut{}. Pada \dataset{} 2, \donut{} memiliki angka akurasi yang lebih tinggi, yaitu 0,91, sedangkan \layoutlm{} dengan mendapatkan angka 0,82 \parencite{bajrami2023comprehensive}.

\subsubsection{Kesimpulan}

\layoutlm{} dan \donut{} merupakan pilihan yang efektif untuk melakukan klasifikasi dokumen, tetapi dengan kebutuhan yang berbeda. \layoutlm{} lebih 
cocok untuk digunakan pada dokumen dengan kompleksitas yang lebih tinggi, namun hasil yang didapatkan lebih lama karena memiliki \textit{pipeline} yang lebih 
panjang akibat adanya proses \ocr{} terlebih dahulu. \donut{} lebih cocok untuk digunakan pada \dataset{} dengan diversifikasi dokumen yang lebih banyak dan dapat menghasilkan dengan lebih cepat akibat \emph{pipeline} yang lebih pendek akibat tidak perlu integrasi dengan \ocr. Dokumen yang memiliki \textit{noise} atau resolusi yang buruk akan mengurangi akurasi secara signifikan untuk \layoutlm{} dan \donut.