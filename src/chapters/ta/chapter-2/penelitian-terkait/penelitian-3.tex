\subsection{\textit{The Future of Document Indexing: GPT and Donut Revolutionize Table of Content Processing}}
\label{sec:penelitian-3}
Penelitian yang dilakukan oleh \cite{feyisa2024future} ini mempresentasikan bagaimana model dengan arsitektur \donut{} dan GPT-3.5 Turbo dapat digunakan untuk melakukan ekstraksi informasi dari dokumen tanpa perlu adanya pendekatan manual yang dilakukan. Dokumen yang digunakan merupakan data yang terstruktur dan bersifat kompleks. Hasil dari model yang digunakan disajikan dalam bentuk 
\textit{dashboard} dan dievaluasi dengan menggunakan akurasi setiap model yang digunakan.

\subsubsection{Gambaran Umum}
Penelitian ini menjelaskan mengenai bagaimana teknologi AI seperti \donut{} dan GPT-3.5 Turbo membuat gebrakan baru pada pemrosesan \emph{document indexing} dan daftar isi untuk dokumen-dokumen kompleks. Permasalahan yang ditemukan adalah ekstraksi informasi secara manual dari dokumen yang kompleks dan panjang sangat memakan waktu dan rawan kesalahan. 

\subsubsection{Analisis dan Metode}
Solusi yang digunakan adalah solusi berbasis AI dengan menggunakan model dengan arsitektur \donut{} untuk melakukan ekstraksi data dari gambar dan dokumen dan GPT-3.5 Turbo untuk melakukan retrukturisasi daftar isi. Sistem yang dirancang akan mengidentifikasi halaman daftar isi dan melakukan ekstraksi \emph{headings} dan \emph{subheadings}-nya dan mengonversinya menjadi data dengan format \json{} yang terstruktur untuk digunakan pada \emph{dashboard} atau \emph{database}.  

Data atau dokumen yang digunakan akan dikonversi menjadi gambar untuk 
\donut{} dan teks untuk GPT untuk kemudian digunakan untuk melakukan \emph{fine-tuning} model pada fase \emph{training model} \parencite{feyisa2024future}. Model yang telah di-\emph{fine-tune} ini akan melakukan identifikasi halaman daftar isi dan melakukan ekstraksi dan merestrukturisasi \emph{headings} dan \emph{subheadings} dalam format \json. Hasil yang telah diekstrak dalam format \json{} akan diintegrasikan ke dalam \emph{dashboard} yang telah dibuat dalam bentuk \emph{website}. Evaluasi terhadap model dilakukan dengan 
menggunakan \accuracy{}. Model dengan arsitektur Donut yang digunakan mencapai nilai \accuracy{} hingga 85\% dan GPT-3.5 Turbo mencapai \accuracy{} 89\%. 

\subsubsection{Kesimpulan}

Penggunaan \emph{Large Language Model} (LLM) dan teknologi \cv, seperti OpenAI GPT-3.5 Turbo dan \donut, memberikan solusi yang efisien untuk menyusun informasi dari dokumen besar secara otomatis. Dengan kemampuan untuk mengorganisasi data, seperti daftar isi dalam dokumen spesifikasi, teknologi ini mampu mengurangi risiko kesalahan, waktu, dan biaya yang berlebihan. Pendekatan ini menjadi langkah inovatif yang mendukung pengelolaan dokumen teknis yang kompleks dan meningkatkan efisiensi dan akurasi dalam berbagai sektor industri. 