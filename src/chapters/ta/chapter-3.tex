\chapter{Analisis Masalah}
\label{chap:analisis-masalah}
Bab ini membahas proses analisis yang dilakukan terhadap permasalahan sistem yang ada saat ini dan kebutuhan sistem yang akan dikembangkan. \autoref{fig:current-state} menunjukkan kondisi saat ini. Penulis akan mengidentifikasi masalah sekarang dan kebutuhan sistem yang akan dikembangkan untuk menghasilkan kebutuhan fungsional dan non-fungsional yang akan digunakan untuk membangun sistem. 

Setelah kebutuhan sistem teridentifikasi, eksplorasi akan dilakukan terhadap berbagai alternatif solusi yang dapat digunakan untuk memenuhi kebutuhan tersebut. Setiap alternatif solusi akan dianalisis berdasarkan kriteria yang telah ditentukan sesuai dengan kebutuhan sistem dan tugas akhir ini. Hasil analisis ini akan membantu dalam memilih solusi yang paling sesuai untuk perancangan dan pengembangan sistem pada bab selanjutnya.

\section{Analisis Kondisi Saat Ini}
\label{sec:analisis-kondisi-saat-ini}


\section{Analisis Kebutuhan}
\label{sec:analisis-kebutuhan}

\subsection{Identifikasi Masalah Pengguna}
\label{subsec:identifikasi-masalah-pengguna}
\subsubsection{Analisis Kebutuhan Pengguna}
\subsubsubsection{Analisis Kebutuhan Sistem}

\subsection{Kebutuhan Fungsional}
\label{subsec:kebutuhan-fungsional}

\subsection{Kebutuhan Non-Fungsional}
\label{subsec:kebutuhan-non-fungsional}

\section{Analisis Pemilihan Solusi}
\label{sec:analisis-pemilihan-solusi}

Subbab ini akan menjelaskan mengenai desain konsep solusi yang akan digunakan pada sistem yang akan dikembangkan. Penulis akan mengeksplorasi berbagai alternatif solusi yang dapat digunakan untuk memenuhi kebutuhan sistem yang telah diidentifikasi pada \autoref{sec:analisis-kebutuhan}. Masing-masing alternatif solusi akan dianalisis berdasarkan kriteria yang telah ditentukan sesuai dengan kebutuhan sistem dan tugas akhir ini. Hasil analisis ini akan membantu penulis dalam memilih solusi yang paling sesuai untuk perancangan dan pengembangan sistem pada \autoref{chapter:desain-implementasi}.

\subsection{Alternatif Solusi}
\label{subsec:alternatif-solusi}
Setiap alternatif solusi yang akan dijelaskan pada subbab ini merupakan alternatif yang dipertimbangkan untuk memenuhi kebutuhan ekstraksi data dokumen. Metode ekstraksi data dokumen adalah metode yang digunakan untuk mengekstrak data dari dokumen digital yang digunakan, yaitu struk dan bukti pembayaran. Alternatif solusi yang dapat dipertimbangkan dapat dibagi menjadi alternatif solusi yang memerlukan \ocr{} dan alternatif solusi yang tidak memerlukan \ocr. 

\subsubsection{Alternatif Solusi yang Memerlukan OCR}
\label{subsubsec:alternatif-solusi-memerlukan-ocr}
Alternatif solusi yang memerlukan \ocr{} adalah alternatif solusi yang memerlukan proses ekstraksi teks dari gambar dokumen terlebih dahulu sebelum melakukan ekstraksi data. Proses ekstraksi teks ini biasanya dilakukan dengan menggunakan model \ocr{} yang telah dilatih sebelumnya. Beberapa alternatif solusi yang memerlukan \ocr{} antara lain:
\begin{enumerate}
    \item \layoutlm ~\\ Model ini adalah salah satu pelopor dalam menggabungkan informasi teks, tata letak, dan visual dari sebuah dokumen. Berbeda dengan model bahasa seperti \bert{} yang hanya memahami teks, \layoutlm{} memperkaya representasi token teks dengan menyertakan \textit{embedding} posisi 2D (koordinat x,y dari \emph{bounding box}) dan \textit{embedding} visual. Dengan demikian, model ini dapat memahami konteks yang berasal dari struktur dokumen. \layoutlm{} menjadi model yang sangat efektif untuk tugas ekstraksi informasi dari dokumen semi-terstruktur seperti formulir dan faktur \parencite{xu2020layoutlm}. 
    \item \bert{} ~\\
    \bert{} adalah model bahasa murni dan tidak memiliki kemampuan untuk memproses gambar atau tata letak secara inheren. Dalam alur kerja ekstraksi data dokumen, peran BERT berada di tahap \textit{downstream} setelah teks diekstraksi oleh sistem \ocr. Teks mentah dari \ocr{} kemudian dimasukkan ke \bert{} untuk tugas-tugas \nlp{} seperti Klasifikasi Teks atau \emph{Named Entity Recognition} (NER). Jadi, \bert{} tidak mengekstraksi data dari gambar, melainkan dari teks yang sudah diekstraksi \parencite{koroteev2021bert}. 
\end{enumerate}

\subsubsection{Alternatif Solusi yang Tidak Memerlukan OCR}
\label{subsubsec:alternatif-solusi-tidak-memerlukan-ocr}
Alternatif solusi yang tidak memerlukan \ocr{} adalah alternatif solusi yang dapat langsung mengekstrak data dari gambar dokumen tanpa perlu melakukan ekstraksi teks terlebih dahulu. Beberapa alternatif solusi yang tidak memerlukan \ocr{} antara lain:

\begin{enumerate}
    \item \donut{} ~\\ Model ini memperkenalkan pendekatan yang revolusioner karena tidak memerlukan modul \ocr{} yang terpisah. Secara teknis, Donut menggunakan arsitektur \textit{encoder-decoder} berbasis \transformer. \textit{Encoder} (biasanya \swin) memproses gambar dokumen secara langsung sebagai piksel, lalu \textit{decoder} (seperti \bart) menerjemahkan representasi visual tersebut langsung menjadi keluaran terstruktur (misalnya, format \json) sesuai dengan instruksi yang diberikan. Kemampuannya untuk belajar dari gambar ke teks secara \textit{end-to-end} menjadikannya sangat fleksibel dan kuat untuk berbagai jenis dokumen tanpa bergantung pada kualitas hasil \ocr{} \parencite{kim2021donut}.

    \item \templatematching{} ~\\ \templatematching{} adalah salah satu metode \cv yang menggeser sebuah gambar "templat" kecil di atas gambar sumber yang lebih besar. Pada setiap posisi, sebuah metrik kemiripan (seperti korelasi silang) dihitung. Lokasi dengan skor kemiripan tertinggi dianggap sebagai posisi dari objek yang dicari. Metode ini cepat dan sederhana, namun kurang andal terhadap perubahan skala, rotasi, pencahayaan, atau deformasi pada objek \parencite{bradski2008learning}.

    \item \crnn ~\\ 
    \crnn{} adalah arsitektur \dl{} yang menjadi standar de facto untuk pengenalan teks atau \ocr{} dan merupakan gabungan dari \cnn{} dan \rnn. \cnn{} berfungsi sebagai ekstraktor fitur visual yang kuat dari gambar teks. Fitur-fitur ini kemudian diolah sebagai sebuah urutan (sekuens) oleh \rnn, biasanya jenis LSTM, yang dapat memprediksi urutan karakter. Kombinasi ini sangat efektif untuk mengenali baris teks dengan berbagai gaya dan variasi \parencite{shi2016end}.

    \item \yolo ~\\ \yolo{} adalah standar model yang digunakan dalam deteksi objek \textit{real-time}. Dalam konteks dokumen, \yolo{} tidak digunakan untuk membaca teks, melainkan untuk mendeteksi lokasi (\emph{bounding box}) dari elemen-elemen visual secara cepat, seperti tabel, gambar, logo, atau blok tanda tangan. \yolo{} cenderung kurang akurat untuk objek kecil atau tumpang tindih karena mekanismenya yang hanya memiliki satu kali proses \parencite{diwan2023object}. 
    
    \item \rcnn{} ~\\ \rcnn{} dan keluarganya (Fast R-CNN, Faster R-CNN) adalah arsitektur deteksi objek dua tahap (\textit{two-stage detector}). Tahap pertama adalah menghasilkan "proposal wilayah" (\textit{region proposals}), yaitu area-area potensial di mana objek mungkin berada. Tahap kedua adalah mengklasifikasikan setiap proposal ini menggunakan \cnn. Pendekatan dua tahap ini membuat \rcnn{} secara umum lebih akurat daripada \yolo, terutama untuk deteksi objek yang kompleks, namun dengan kecepatan yang lebih lambat. Dalam analisis dokumen, \rcnn{} digunakan untuk lokalisasi elemen visual yang memerlukan presisi tinggi \parencite{xie2021oriented}.
\end{enumerate}

\subsection{Analisis Penentuan Solusi}
\label{subsec:analisis-penentuan-solusi}

Subbab ini akan menjelaskan mengenai analisis yang digunakan untuk menentukan solusi yang akan digunakan pada sistem yang akan dikembangkan. Penulis akan menganalisis berbagai alternatif solusi yang telah diidentifikasi pada \autoref{subsec:alternatif-solusi} berdasarkan kriteria yang telah ditentukan sesuai dengan kebutuhan sistem. Hasil analisis akan digunakan untuk menentukan solusi yang paling sesuai untuk perancangan dan pengembangan sistem pada \autoref{chapter:desain-implementasi}.

Penulis akan melakukan analisis dengan mempertimbangkan kelebihan dan kekurangan yang digunakan dari masing-masing alternatif solusi dari subbab sebelumnya. Kelebihan dan kekurangan setiap alternatif solusi akan disajikan dalam bentuk tabel pada \autoref{tab:model-comparison}. Tabel ini akan memberikan gambaran yang jelas mengenai kelebihan dan kekurangan dari masing-masing alternatif solusi sesuai dengan kebutuhan sistem.
\pagebreak

% --- TABEL PERTAMA ---
\begin{table}[h!]
\caption{Perbandingan Kelebihan dan Kekurangan Model}
\label{tab:model-comparison}
% Mengatur agar list tidak memakan banyak ruang vertikal
\setlist[enumerate]{itemsep=0mm, topsep=1mm, nosep}
\begin{tabularx}{\linewidth}{|p{1.75cm}|X|X|}
\hline
\textbf{Model} & \textbf{Kelebihan} & \textbf{Kekurangan} \\
\hline

% --- Donut Model ---
\textbf{Donut} &
\begin{enumerate}
    \item Tidak memiliki ketergantungan pada OCR (\textit{end-to-end processing})
    \item Memiliki \textit{pipeline} yang lebih sederhana
    \item Cocok untuk variasi dokumen yang beragam
    \item Merupakan SOTA untuk VDU (\textit{Visual Document Understanding})
    \item Merupakan \textit{pre-trained model}
\end{enumerate}
&
\begin{enumerate}
    \item Memerlukan data pelatihan dalam jumlah besar
    \item Membutuhkan kebutuhan komputasi yang tinggi untuk pelatihan
    \item Kurang optimal untuk dokumen dengan kualitas rendah
\end{enumerate}
\\ \hline
\pagebreak

% --- LayoutLM Model ---
\textbf{LayoutLM} &
\begin{enumerate}
    \item Cocok untuk dokumen yang lebih kompleks dengan elemen tabel dan formulir
    \item Dapat digunakan untuk dokumen terstruktur dan semi-terstruktur
    \item Merupakan \textit{pre-trained model}
\end{enumerate}
&
\begin{enumerate}
    \item Memiliki ketergantungan pada OCR
    \item Kurang optimal untuk dokumen dengan kualitas rendah
\end{enumerate}
\\ \hline

% --- BERT Model ---
\textbf{BERT} &
\begin{enumerate}
    \item Cocok untuk dokumen yang lebih kompleks dengan elemen tabel dan formulir
    \item Dapat digunakan untuk dokumen terstruktur dan semi-terstruktur
    \item Merupakan \textit{pre-trained model}
\end{enumerate}
&
\begin{enumerate}
    \item Kurang cocok untuk dokumen dengan elemen visual atau spasial
    \item Memiliki ketergantungan pada OCR
\end{enumerate}
\\ \hline

% --- Template Matching ---
\textbf{Template Matching} &
\begin{enumerate}
    \item Cepat dan ringan untuk dokumen dengan pola seragam
    \item Kemudahan implementasi
    \item Tidak memerlukan data pelatihan yang beragam
\end{enumerate}
&
\begin{enumerate}
    \item Kurang fleksibel untuk berbagai variasi format dokumen
    \item Variasi posisi dan kualitas dokumen menjadi tantangan
    \item Metode yang tergolong konvensional
\end{enumerate}
\\ \hline
\end{tabularx}
\end{table}


% --- TABEL KEDUA (LANJUTAN) ---
\begin{table}[h!]
\ContinuedFloat % Perintah ini mencegah nomor tabel bertambah
\caption{Perbandingan Kelebihan dan Kekurangan Model (lanjutan)}
% Mengatur agar list tidak memakan banyak ruang vertikal
\setlist[enumerate]{itemsep=0mm, topsep=1mm, nosep}
\begin{tabularx}{\linewidth}{|p{1.75cm}|X|X|}
\hline
\textbf{Model} & \textbf{Kelebihan} & \textbf{Kekurangan} \\
\hline

% --- CRNN ---
\textbf{CRNN} &
\begin{enumerate}
    \item Cocok untuk pengenalan teks dalam gambar
    \item Sangat cocok untuk menangani teks dengan font dan orientasi yang bervariasi
\end{enumerate}
&
\begin{enumerate}
    \item Tidak optimal untuk menangani elemen visual dan spasial dalam gambar
    \item Hanya berfokus pada teks
\end{enumerate}
\\ \hline

% --- YOLO ---
\textbf{YOLO} &
\begin{enumerate}
    \item Model yang cepat untuk melakukan pengenalan objek secara \textit{real-time}
    \item Cocok untuk pengenalan elemen visual dengan ukuran signifikan dengan cepat
\end{enumerate}
&
\begin{enumerate}
    \item Kurang efektif untuk tugas berbasis teks atau dokumen
    \item Tidak dirancang untuk dokumen dengan berbagai elemen teks
\end{enumerate}
\\ \hline

% --- R-CNN ---
\textbf{R-CNN} &
\begin{enumerate}
    \item Lebih akurat untuk melakukan deteksi objek pada gambar yang kompleks dibandingkan dengan \newline YOLO
    \item Cocok untuk dokumen dengan elemen visual yang signifikan
\end{enumerate}
&
\begin{enumerate}
    \item Lebih lambat untuk melakukan pengenalan objek dibandingkan dengan \newline YOLO
    \item Membutuhkan pelatihan yang intens
    \item Kurang cocok untuk melakukan pengenalan teks yang bervariasi
\end{enumerate}
\\ \hline
\end{tabularx}
\end{table}

\pagebreak

Berdasarkan analisis pada \autoref{tab:model-comparison}, penulis akan memilih \donut{} sebagai solusi yang akan digunakan pada sistem yang akan dikembangkan. Hal ini dikarenakan \donut{} memiliki kelebihan yang signifikan dibandingkan dengan alternatif solusi lainnya, yaitu tidak memerlukan ketergantungan pada \ocr{} dan dapat melakukan ekstraksi data secara langsung dari gambar dokumen. Selain itu, \donut{} juga merupakan model yang telah terbukti efektif dalam tugas ekstraksi informasi dari dokumen semi-terstruktur seperti formulir dan faktur. Meskipun \donut{} memerlukan data pelatihan dalam jumlah besar dan kebutuhan komputasi yang tinggi, penulis akan menggunakan model \donut{} yang telah dilatih sebelumnya untuk mengatasi masalah ini. Dengan demikian, penulis akan melanjutkan ke tahap perancangan dan pengembangan sistem pada \autoref{chapter:desain-implementasi} dengan menggunakan \donut{} sebagai solusi utama untuk ekstraksi data dokumen.