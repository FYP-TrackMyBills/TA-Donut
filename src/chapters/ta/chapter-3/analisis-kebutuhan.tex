\section{Analisis Kebutuhan}
\label{sec:analisis-kebutuhan}

Subbab ini menjelaskan mengenai analisis kebutuhan sistem yang akan dibangun. Masalah yang ada pada sistem saat ini diidentifikasikan untuk kemudian menjadi pertimbangan penentuan kebutuhan fungsional dan non-fungsional sistem. Analisis kebutuhan tersebut kemudian digunakan untuk menentukan alternatif solusi yang akan diimplementasikan pada \autoref{subsec:alternatif-solusi}.

\subsection{Identifikasi Masalah Pengguna}
\label{subsec:identifikasi-masalah-pengguna}
\autoref{sec:latar-belakang} tentang latar belakang telah menjelaskan bahwa banyak pengguna, yang terdiri dari kalangan Gen Z, memiliki kesulitan untuk memenuhi seluruh proses pada \autoref{fig:current-state}. \autoref{tab:masalah-pengguna} menunjukkan masalah yang dihadapi pengguna dalam proses pencatatan pengeluaran mereka.

\begin{table}[h!]
\caption{Identifikasi Masalah Pengguna}
\label{tab:masalah-pengguna}
\begin{tabularx}{\linewidth}{|l|X|}
\hline
\textbf{Kode} & \textbf{Deskripsi Masalah} \\
\hline
P-01 & Pengguna merasa kesulitan untuk menyadari seberapa besar total pengeluaran mereka yang dapat terakumulasi dengan cepat dari transaksi mereka. \\
\hline
P-02 & Pengguna merasa kesulitan untuk mencatat pengeluaran secara manual dari berbagai bukti transaksi. Proses ini memakan waktu, rentan terhadap kesalahan input. \\
\hline
P-03 & Pengguna merasa kesulitan untuk memetakan informasi dari bukti pembayaran yang tidak terstruktur sehingga data tidak masuk dengan tepat pada \emph{platform} keuangan. \\
\hline
\end{tabularx}
\end{table}
% \begin{enumerate}
%     \item Pengguna merasa kesulitan untuk menyadari seberapa besar total pengeluaran mereka yang dapat terakumulasi dengan cepat dari transaksi mereka.
%     \item Pengguna merasa kesulitan untuk mencatat pengeluaran secara manual dari berbagai bukti transaksi. Proses ini memakan waktu, rentan terhadap kesalahan input
%     \item Pengguna merasa kesulitan untuk mengklasifikasikan dan mengelompokkan informasi dari bukti pembayaran yang tidak terstruktur sehingga pengguna kesulitan untuk memasukkan data ke dalam \emph{platform} keuangan.
% \end{enumerate}

\subsection{Kebutuhan Fungsional}
\label{subsec:kebutuhan-fungsional}
Berdasarkan identifikasi masalah pengguna pada \autoref{subsec:identifikasi-masalah-pengguna} dan \autoref{sec:latar-belakang}, kebutuhan fungsional atau \emph{Functional Requirement} (FR) akan ditentukan dan dipenuhi oleh sistem yang dibangun. Kebutuhan fungsional adalah kebutuhan mengenai fitur yang akan disediakan oleh sistem. Kebutuhan fungsional  sistem ini akan disajikan dalam bentuk tabel pada \autoref{tab:kebutuhan-fungsional}.

\begin{table}[h!]
\caption{Kebutuhan Fungsional Sistem}
\label{tab:kebutuhan-fungsional}
\begin{tabularx}{\linewidth}{|l|X|l|}
\hline
\textbf{Kode} & \textbf{Kebutuhan Fungsional} & \textbf{Kode P}\\
\hline
FR-01 & Sistem harus dapat menerima masukan berupa gambar bukti transaksi, yaitu struk fisik cetakan dan tangkapan layar bukti pembayaran digital, yaitu QRIS dan transfer. & P-02\\
\hline
FR-02 & Sistem harus dapat memotong area relevan transaksi berdasarkan keinginan pengguna dari gambar yang diunggah. & P-02\\
\hline
FR-03 & Sistem harus dapat mengekstrak dan memetakan informasi kunci setiap bukti transaksi, yaitu total pembayaran. & P-02, P-03\\
\hline
FR-04 & Sistem harus menampilkan hasil ekstraksi kepada pengguna dan menyediakan fungsi mengoreksi data jika terdapat ketidakakuratan. & P-02\\
\hline
FR-05 & Sistem harus dapat menunjukkan total pengeluaran pengguna dan menampilkan total pengeluaran per kategori. & P-01 \\
\hline
\end{tabularx}
\end{table}

\subsection{Kebutuhan Non-Fungsional}
\label{subsec:kebutuhan-non-fungsional}
Kebutuhan non-fungsional atau \emph{Non-functional Requirement} (NF) adalah kebutuhan kualitas sistem yang akan dibangun. Berdasarkan identifikasi masalah pada \autoref{subsec:identifikasi-masalah-pengguna} dan \autoref{sec:latar-belakang}, kebutuhan non-fungsional dari sistem yang akan dikembangkan akan diidentifikasikan. Kebutuhan non-fungsional sistem ini akan disajikan dalam bentuk tabel pada \autoref{tab:kebutuhan-non-fungsional}.

\begin{table}[h!]
\caption{Kebutuhan Non-Fungsional Sistem}
\label{tab:kebutuhan-non-fungsional}
\begin{tabularx}{\linewidth}{|l|l|X|}
\hline
\textbf{Kode} & \textbf{Kategori} & \textbf{Kebutuhan Non-fungsional} \\
\hline
NF-01 & \emph{Usability} & Antarmuka pengguna (UI) minimalis dan sederhana untuk kemalasan mencatat manual. \\
\hline
NF-02 & \emph{Usability} & Hasil evaluasi SUS Score lebih besar dari 68. \\ 
\hline
NF-03 & \emph{Compatibility} & Sistem dapat menerima masukan gambar dengan \linebreak kualitas yang jelas dan tidak \emph{blur}. \\
\hline
% NF-04 & \emph{Compatibility} & Sistem dapat menerima dokumen bukti pembayaran QRIS dan transfer dari BCA, Seabank, Neobank, dan Gopay \\
% \hline
NF-04 & \emph{Compatibility} & Sistem dapat beroperasi pada sistem operasi Android. \\
\hline
% NF-06 & \emph{Supportability} & Implementasi sistem tidak mencakup integrasi  pihak ketiga \\
% \hline
NF-05 & \emph{Performance} & Hasil evaluasi \accuracy{}, \precision, \recall, dan \fscore{} lebih besar daripada 65\%, dan \mcer{} lebih kecil dari 20\%.\\
\hline
\end{tabularx}
\end{table}