\section{Analisis Kebutuhan}
\label{sec:analisis-kebutuhan}

Subbab ini akan menjelaskan mengenai analisis kebutuhan sistem yang akan dikembangkan. Penulis akan mengidentifikasi masalah yang ada pada sistem saat ini, kemudian menentukan kebutuhan fungsional dan non-fungsional. Analisis kebutuhan tersebut akan digunakan untuk menentukan alternatif solusi yang akan diimplementasikan pada \autoref{subsec:alternatif-solusi}.

\subsection{Identifikasi Masalah Pengguna}
\label{subsec:identifikasi-masalah-pengguna}
Pengguna sistem ini adalah Gen Z dan milenial yang memiliki kebiasaan berbelanja dengan cepat dan efisien menggunakan QRIS dan metode metode pembayaran elektronik. Namun, mereka seringkali mengalami kesulitan dalam mengelola dan menyimpan struk dan bukti pembayaran yang mereka miliki.Masalah utama yang dihadapi pengguna adalah tidak adanya sistem yang dapat langsung melakukan ekstraksi data dari struk dan bukti pembayaran yang telah diterima secara langsung. Masalah-masalah yang akan dihadapi pengguna kemudian adalah:
\begin{enumerate}
    \item Pengguna harus menyimpan struk dan bukti pembayaran secara manual, yang memakan waktu dan rentan terhadap kerusakan fisik.
    \item Pengguna harus memasukkan data dari struk dan bukti pembayaran ke dalam sistem atau aplikasi keuangan mereka secara manual untuk melakukan pencatatan pengeluaran mereka.
\end{enumerate}

\subsection{Kebutuhan Fungsional}
\label{subsec:kebutuhan-fungsional}
Berdasarkan identifikasi masalah pengguna pada \autoref{subsec:identifikasi-masalah-pengguna}, penulis mengidentifikasi kebutuhan fungsional yang harus dipenuhi oleh sistem yang akan dikembangkan. Kebutuhan fungsional adalah kebutuhan yang berisi pernyataan mengenai 
penjelasan layanan-layanan yang akan disediakan oleh sistem. Kebutuhan fungsional sistem ini akan disajikan dalam bentuk tabel pada \autoref{tab:kebutuhan-fungsional}.

\subsection{Kebutuhan Non-Fungsional}
\label{subsec:kebutuhan-non-fungsional}
Kebutuhan non-fungsional adalah kebutuhan yang berisi pernyataan mengenai kualitas sistem yang akan dikembangkan. Berdasarkan identifikasi masalah pada \autoref{subsec:identifikasi-masalah-pengguna}, penulis akan mengidentifikasikan kebutuhan non-fungsional dari sistem yang akan dikembangkan. Kebutuhan non-fungsional sistem ini akan disajikan dalam bentuk tabel pada \autoref{tab:kebutuhan-non-fungsional}.