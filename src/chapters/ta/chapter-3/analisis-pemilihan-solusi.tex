\section{Analisis Pemilihan Solusi}
\label{sec:analisis-pemilihan-solusi}

Subbab ini akan menjelaskan mengenai desain konsep solusi yang akan digunakan pada sistem yang akan dikembangkan. Penulis akan mengeksplorasi berbagai alternatif solusi yang dapat digunakan untuk memenuhi kebutuhan sistem yang telah diidentifikasi pada \autoref{sec:analisis-kebutuhan}. Masing-masing alternatif solusi akan dianalisis berdasarkan kriteria yang telah ditentukan sesuai dengan kebutuhan sistem dan tugas akhir ini. Hasil analisis ini akan membantu penulis dalam memilih solusi yang paling sesuai untuk perancangan dan pengembangan sistem pada \autoref{chapter:desain-implementasi}.

\subsection{Alternatif Solusi}
\label{subsec:alternatif-solusi}
Subbab ini akan membahas tentang alternatif solusi yang dapat digunakan untuk memenuhi kebutuhan sistem yang telah diidentifikasi pada \autoref{sec:analisis-kebutuhan}. Penulis akan mengeksplorasi berbagai alternatif solusi yang dapat digunakan untuk memenuhi kebutuhan sistem yang telah diidentifikasi pada \autoref{sec:analisis-kebutuhan}. Penulis akan mempertimbangkan alternatif solusi untuk dua aspek utama pada sistem, yaitu metode ekstraksi data dokumen dan \emph{platform} pengembangan sistem. 

\subsubsection{Metode Ekstraksi Data Dokumen}
\label{subsubsec:model-ekstraksi-data-dokumen}
Metode ekstraksi data dokumen adalah metode yang digunakan untuk mengekstrak data dari dokumen yang telah diterima. Dengan mempertimbangkan kebutuhan sistem yang telah diidentifikasi pada \autoref{sec:analisis-kebutuhan}, penulis akan mengeksplorasi beberapa alternatif solusi untuk metode ekstraksi data dokumen. Alternatif solusi yang dapat digunakan sebagai metode ekstraksi data dokumen dapat dilihat pada \autoref{tab:alternatif-solusi-ekstraksi-data-dokumen}

\begin{table}[h!]
\caption{Perbandingan Kelebihan dan Kekurangan Model (Versi Perbaikan)}
\label{tab:model-comparison-fixed}
% We now use >{\RaggedRight}X to make the text left-aligned, which is better for lists.
\begin{tabularx}{\textwidth}{| l | >{\RaggedRight}X | >{\RaggedRight}X |}
\hline
\textbf{Model} & \textbf{Kelebihan} & \textbf{Kekurangan} \\
\hline

% --- Donut Model ---
\donut{} &
- Tidak memiliki ketergantungan pada OCR (\textit{end-to-end processing}) \newline
- Memiliki \textit{pipeline} yang lebih sederhana \newline
- Cocok untuk variasi dokumen yang beragam \newline
- Merupakan SOTA untuk VDU (\textit{Visual Document Understanding}) \newline
- Merupakan \textit{pre-trained model}
&
- Memerlukan data pelatihan dalam jumlah besar \newline
- Membutuhkan kebutuhan komputasi yang tinggi untuk pelatihan \newline
- Kurang optimal untuk dokumen dengan kualitas rendah
\\ \hline

% --- LayoutLM Model ---
LayoutLM &
- Cocok untuk dokumen yang lebih kompleks dengan elemen tabel dan formulir \newline
- Dapat digunakan untuk dokumen terstruktur dan semi-terstruktur \newline
- Merupakan \textit{pre-trained model}
&
- Memiliki ketergantungan pada OCR \newline
- Kurang optimal untuk dokumen dengan kualitas rendah
\\ \hline

% --- BERT Model ---
\bert{} &
- Cocok untuk dokumen yang lebih kompleks dengan elemen tabel dan formulir \newline
- Dapat digunakan untuk dokumen terstruktur dan semi-terstruktur \newline
- Merupakan \textit{pre-trained model}
&
- Kurang cocok untuk dokumen dengan elemen visual atau spasial \newline
- Memiliki ketergantungan pada OCR
\\ \hline

% --- Template Matching ---
Template Matching &
- Cepat dan ringan untuk dokumen dengan pola seragam \newline
- Kemudahan implementasi \newline
- Tidak memerlukan data pelatihan yang beragam
&
- Kurang fleksibel untuk berbagai variasi format dokumen \newline
- Variasi posisi dan kualitas dokumen menjadi tantangan \newline
- Metode yang tergolong konvensional
\\ \hline

% --- CRNN ---
\crnn{} &
- Cocok untuk pengenalan teks dalam gambar \newline
- Sangat cocok untuk menangani teks dengan font dan orientasi yang bervariasi
&
- Tidak optimal untuk menangani elemen visual dan spasial dalam gambar \newline
- Hanya berfokus pada teks
\\ \hline

% --- YOLO ---
\yolo &
- Model yang cepat untuk melakukan pengenalan objek secara \textit{real-time} \newline
- Cocok untuk pengenalan elemen visual dengan ukuran signifikan dengan cepat
&
- Kurang efektif untuk tugas berbasis teks atau dokumen \newline
- Tidak dirancang untuk dokumen dengan berbagai elemen teks
\\ \hline

% --- R-CNN ---
\rcnn{} &
- Lebih akurat untuk melakukan deteksi objek pada gambar yang kompleks dibandingkan dengan YOLO \newline
- Cocok untuk dokumen dengan elemen visual yang signifikan
&
- Lebih lambat untuk melakukan pengenalan objek dibandingkan dengan YOLO \newline
- Membutuhkan pelatihan yang intens \newline
- Kurang cocok untuk melakukan pengenalan teks yang bervariasi
\\ \hline
\end{tabularx}
\end{table}


\subsubsection{Platform Pengembangan Sistem}
\label{subsubsec:platform-pengembangan-sistem}

\subsection{Analisis Penentuan Solusi}
\label{subsec:analisis-penentuan-solusi}