\section{Hasil Desain}
\label{sec:hasil-desain}

Hasil desain merupakan implementasi konkret dari tahapan desain yang telah dirancang pada \autoref{sec:tahapan-desain}. Bagian ini menyajikan artefak-artefak yang dihasilkan dari proses pengembangan sistem, mencakup implementasi script pelatihan, model yang telah di-\emph{fine-tune}, framework evaluasi, layanan \emph{backend}, aplikasi \emph{mobile}, dan integrasi sistem secara keseluruhan.

Setiap artefak dikembangkan dengan mempertimbangkan kebutuhan fungsional dan non-fungsional yang telah didefinisikan, serta disesuaikan dengan ekspektasi pengguna Gen Z terhadap sistem mobile yang modern dan efisien. Implementasi menggunakan teknologi terkini dan best practices dalam pengembangan aplikasi mobile dan sistem berbasis \emph{machine learning}.

% \begin{figure}[htbp]
%     \centering
%     \includegraphics[width=\textwidth]{images/system-artifacts-overview.png}
%     \caption{Overview artefak sistem yang dihasilkan}
%     \label{fig:system-artifacts-overview}
% \end{figure}

Hasil desain menunjukkan implementasi sistem yang lengkap dan siap untuk deployment, dengan setiap komponen telah melalui proses testing dan validasi. Integrasi antar komponen dirancang untuk memberikan pengalaman pengguna yang seamless sambil mempertahankan akurasi dan reliabilitas sistem.

\subsection{Script Fine-tuning}
\label{subsec:script-fine-tuning}

Script \emph{fine-tuning} merupakan implementasi konkret dari strategi pelatihan yang telah dirancang pada \autoref{subsec:perancangan-fine-tuning}. Script ini dikembangkan dengan pendekatan modular yang memungkinkan fleksibilitas dalam konfigurasi pelatihan dan monitoring proses \emph{fine-tuning} secara real-time.

% \begin{figure}[htbp]
%     \centering
%     \includegraphics[width=0.8\textwidth]{images/fine-tuning-script-architecture.png}
%     \caption{Arsitektur script fine-tuning dengan komponen utama}
%     \label{fig:fine-tuning-script-architecture}
% \end{figure}

Implementasi script menggunakan framework PyTorch dan Transformers library dari HuggingFace untuk memastikan kompatibilitas dengan model \donut{} dan ekosistem \emph{machine learning} yang luas. Script dirancang dengan beberapa komponen utama: \emph{data loading}, \emph{model initialization}, \emph{training loop}, dan \emph{evaluation pipeline}.

Komponen \emph{data loading} mengimplementasikan \texttt{PaymentProofDataset} class yang secara khusus dirancang untuk menangani format data JSONL dengan struktur yang konsisten. Dataset loader ini memiliki kemampuan untuk memproses gambar dalam berbagai format (JPG, JPEG, PNG) dan melakukan preprocessing yang diperlukan seperti resizing dan normalisasi. Implementasi juga mencakup robust error handling untuk menangani file yang corrupted atau format anotasi yang tidak valid.

Proses inisialisasi model melakukan ekspansi vocabulary untuk mengakomodasi \emph{special tokens} yang spesifik untuk domain pembayaran Indonesia. Script secara otomatis mendeteksi dan menambahkan token baru seperti representasi mata uang rupiah, format tanggal Indonesia, dan terminologi pembayaran lokal. Konfigurasi model disesuaikan dengan kebutuhan pelatihan, termasuk pengaturan \emph{mixed precision} FP16 untuk optimasi memori dan kecepatan pelatihan.

\emph{Training loop} mengimplementasikan strategi yang telah dirancang dengan 30 epoch, \emph{batch size} 1, dan \emph{gradient accumulation} 8 steps. Script mencakup implementasi \emph{early stopping} dengan patience=3 untuk mencegah \emph{overfitting} dan mengoptimalkan penggunaan sumber daya komputasi. Monitoring progres pelatihan dilakukan melalui logging yang comprehensive, mencakup \emph{training loss}, \emph{validation loss}, dan metrik evaluasi khusus.

Script juga mengimplementasikan sistem \emph{checkpoint} otomatis yang menyimpan state model setiap 5 epoch, memungkinkan recovery dari interruption dan analisis evolusi kinerja model. Implementasi mencakup metadata lengkap untuk setiap checkpoint, termasuk konfigurasi pelatihan, metrik performance, dan timestamp untuk audit trail yang lengkap.

Untuk memastikan reproducibility, script mencakup seed management yang konsisten dan logging semua hyperparameter yang digunakan. Dokumentasi inline yang comprehensive memungkinkan maintainability dan adaptasi script untuk eksperimen future. Script dirancang dengan modularitas yang memungkinkan easy integration dengan berbagai \emph{monitoring tools} dan \emph{experiment tracking platforms}.


\input{chapters/ta/chapter-4/hasil-desain/model-fine-tuned.tex}

\subsection{Framework Evaluasi Terpadu}
\label{subsec:framework-evaluasi-terpadu}

Framework evaluasi terpadu merupakan implementasi konkret dari metodologi evaluasi yang dirancang untuk memberikan penilaian komprehensif terhadap kedua model yang telah dikembangkan. Framework ini mengintegrasikan berbagai metrik evaluasi dalam satu sistem yang kohesif dan dapat dieksekusi secara otomatis.

% \begin{figure}[htbp]
%     \centering
%     \includegraphics[width=0.8\textwidth]{images/evaluation-framework-implementation.png}
%     \caption{Implementasi framework evaluasi terpadu}
%     \label{fig:evaluation-framework-implementation}
% \end{figure}

Implementasi framework menggunakan pendekatan modular yang memungkinkan evaluasi terpisah untuk model QRIS-TF dan CORD-v2, namun tetap memberikan hasil yang dapat dibandingkan. Framework dirancang dengan fleksibilitas untuk menangani berbagai format output dan dapat dengan mudah diperluas untuk metrik evaluasi tambahan di masa depan.

Komponen utama framework mencakup \emph{data loader} yang dapat menangani format dataset yang berbeda, \emph{model inference engine} yang support multiple model types, \emph{metrics calculation module} yang mengimplementasikan semua metrik yang diperlukan, dan \emph{reporting system} yang menghasilkan output evaluasi dalam format yang mudah dipahami.

\emph{Metrics calculation module} mengimplementasikan lima metrik utama: Accuracy, Precision, Recall, F1-Score, dan mCER (Mean Character Error Rate). Untuk model QRIS-TF, framework menghitung metrik klasifikasi untuk mengevaluasi kemampuan membedakan antara transaksi QRIS dan transfer, serta metrik ekstraksi untuk mengevaluasi akurasi field extraction. Model CORD-v2 dievaluasi fokus pada metrik ekstraksi dengan emphasis pada field-field kritis seperti total pembayaran dan informasi merchant.

Framework mengimplementasikan sistem weighting yang memberikan prioritas berbeda untuk field-field yang berbeda. Field \texttt{total\_amount} diberikan bobot 3.0x karena kritikalitasnya dalam pencatatan keuangan, field \texttt{type} dan \texttt{target\_name} diberikan bobot 2.0x, sedangkan field lainnya menggunakan bobot standard 1.0x. Sistem weighting ini memastikan bahwa evaluasi mencerminkan importance relatif dari setiap field dalam konteks aplikasi praktis.

% \begin{figure}[htbp]
%     \centering
%     \includegraphics[width=0.9\textwidth]{images/evaluation-metrics-dashboard.png}
%     \caption{Dashboard hasil evaluasi dengan visualisasi metrik}
%     \label{fig:evaluation-metrics-dashboard}
% \end{figure}

Output framework mencakup detailed performance report yang menampilkan metrik aggregate dan per-field analysis. Report juga mencakup confusion matrix untuk komponen klasifikasi, character error analysis untuk detailed understanding dari kesalahan ekstraksi, dan confidence score distribution yang membantu dalam understanding model uncertainty.

Framework juga mengimplementasikan debugging capabilities yang memungkinkan analysis mendalam terhadap sample-sample yang mengalami kesalahan. Feature ini mencakup side-by-side comparison antara ground truth dan prediction, highlighting field-field yang bermasalah, dan analysis error patterns yang dapat membantu dalam improvement model di masa depan.

Untuk memastikan reproducibility, framework mencakup comprehensive logging yang mencatat semua parameter evaluasi, dataset yang digunakan, dan konfigurasi model. Seed management yang konsisten memastikan bahwa hasil evaluasi dapat direproduksi dengan kondisi yang identik. Framework juga support untuk batch evaluation yang memungkinkan comparison performa across different model checkpoints atau konfigurasi.

Implementasi framework menggunakan efficient memory management untuk menangani dataset yang besar tanpa mengalami memory overflow. Parallel processing capabilities memungkinkan evaluasi yang cepat even pada dataset yang substantial. Error handling yang robust memastikan bahwa evaluation process dapat continue bahkan ketika encounter sample yang bermasalah atau model prediction yang malformed.


\subsection{Backend Service (DonutAPI)}
\label{subsec:backend-service-donutapi}

DonutAPI merupakan implementasi \emph{backend service} yang menyediakan layanan ekstraksi data pembayaran melalui RESTful API. Service ini dirancang menggunakan framework FastAPI untuk memberikan performa tinggi, dokumentasi otomatis, dan type safety yang memudahkan integration dan maintenance.

% \begin{figure}[htbp]
%     \centering
%     \includegraphics[width=0.9\textwidth]{images/donutapi-architecture.png}
%     \caption{Arsitektur DonutAPI dengan komponen utama}
%     \label{fig:donutapi-architecture}
% \end{figure}

Arsitektur DonutAPI menggunakan pattern layered architecture yang memisahkan concern antara API layer, business logic layer, dan model inference layer. Pemisahan ini memungkinkan maintainability yang baik dan facilitates future enhancements tanpa mempengaruhi komponen lain.

API layer mengimplementasikan dua endpoint utama: \texttt{/predict/} untuk full prediction dengan detailed metadata, dan \texttt{/predict/simple} untuk simplified response yang focus pada extracted data saja. Kedua endpoint support parameter \texttt{model} yang memungkinkan client untuk specify model mana yang akan digunakan (base atau custom), memberikan flexibility dalam processing different document types.

\emph{Request handling} mengimplementasikan comprehensive validation yang mencakup file format validation (JPG, JPEG, PNG, WebP, BMP), file size limits (maksimum 10MB), dan image dimension constraints (maksimum 2048x2048 pixels). Validation ini memastikan bahwa hanya input yang valid yang diproses oleh model, reducing computational waste dan improving system reliability.

% \begin{figure}[htbp]
%     \centering
%     \includegraphics[width=0.8\textwidth]{images/api-request-flow.png}
%     \caption{Alur pemrosesan request dalam DonutAPI}
%     \label{fig:api-request-flow}
% \end{figure}

Model inference layer mengimplementasikan \emph{singleton pattern} untuk kedua model (QRIS-TF dan CORD-v2) untuk mencegah multiple instances yang dapat menyebabkan memory issues. Model loading menggunakan lazy initialization yang hanya memuat model ketika pertama kali dibutuhkan, optimizing startup time dan memory usage.

Service mengimplementasikan automatic model selection logic yang menganalisis karakteristik input image untuk menentukan model yang paling appropriate. Logic ini menggunakan simple heuristics berdasarkan image properties dan dapat with high accuracy menentukan apakah document adalah digital payment proof atau paper receipt.

Error handling dan response formatting mengikuti standard industri dengan HTTP status codes yang appropriate dan error messages yang informative. Service mengimplementasikan graceful degradation di mana jika primary model gagal, system secara otomatis fallback ke alternative processing atau memberikan informative error message kepada client.

% \begin{figure}[htbp]
%     \centering
%     \includegraphics[width=0.7\textwidth]{images/api-response-format.png}
%     \caption{Format respons API dengan metadata lengkap}
%     \label{fig:api-response-format}
% \end{figure}

Response formatting menggunakan Pydantic schemas untuk memastikan type safety dan consistent data structure. Full response format mencakup extracted data, confidence scores, processing metadata, dan model information. Simple response format hanya mengembalikan cleaned extracted data yang siap untuk consumption oleh mobile application.

Service juga mengimplementasikan health check endpoints (\texttt{/health} dan \texttt{/health/detailed}) yang memungkinkan monitoring system status dan model availability. Endpoints ini crucial untuk deployment dalam production environment dan integration dengan load balancers atau orchestration systems.

Logging system menggunakan structured JSON logging dengan request tracing yang memungkinkan debugging dan monitoring yang effective. Setiap request diberikan unique ID yang dapat ditrack across all log entries, facilitating troubleshooting dan performance analysis.

Security features mencakup CORS configuration untuk cross-origin requests, optional API key authentication, dan input sanitization untuk mencegah malicious uploads. File validation tidak hanya check format dan size, tetapi juga verify content integrity untuk memastikan bahwa uploaded files adalah valid images.

Deployment configuration menggunakan Docker containerization yang memungkinkan consistent deployment across different environments. Container includes all dependencies dan optimized untuk both CPU dan GPU inference, dengan automatic device detection yang menggunakan GPU jika available dan fallback ke CPU processing jika diperlukan.


\subsection{Aplikasi Mobile TrackMyBills}
\label{subsec:aplikasi-mobile-trackmybills}

TrackMyBills merupakan aplikasi \emph{mobile} berbasis Flutter. Aplikasi ini berfungsi sebagai antarmuka pengguna untuk sistem ekstraksi data pembayaran dengan integrasi fungsi kamera, akses galeri, \emph{sharing intent}, dan pemanggilan API. TrackMyBills dibangun dengan menggunakan pola \emph{Model-View-Controller} (MVC) yang memisahkan antarmuka, model yang digunakan, dan logika fitur. Flutter memiliki \emph{built-in state management} melalui \emph{widget}-nya, yaitu StatefulWidget untuk memudahkan pengelolaan \emph{state} dari sistem. 

% Design system aplikasi mengadopsi Material Design 3 principles dengan color scheme yang vibrant dan typography yang readable untuk mencerminkan aesthetic preferences Gen Z. Interface menggunakan card-based layouts dengan rounded corners dan subtle shadows yang memberikan modern appearance sambil maintaining functional clarity untuk financial data display.

% **Diagram yang disarankan**: App architecture diagram menunjukkan MVC pattern implementation, state management flow, dan component interactions dalam aplikasi mobile.

\subsubsection{Metode Masukan Gambar}
\label{subsubsec:metode-masukan-gambar}

TrackMyBills menyediakan tiga metode untuk menerima masukan gambar, yaitu melalui kamera, galeri, dan \emph{sharing intent}. 

Camera capture menggunakan native Android camera API dengan custom UI overlay yang memberikan guidance visual untuk optimal document framing. Feature ini mengimplementasikan automatic focus dan flash control dengan preview functionality yang memungkinkan pengguna untuk review foto sebelum processing. Camera interface juga menyediakan grid lines dan document detection hints untuk membantu pengguna mengambil foto dengan kualitas optimal.

Gallery selection mengintegrasikan dengan Android's native file picker untuk memilih existing images dari device storage. Feature ini support multiple file formats yang compatible dengan backend service dan mengimplementasikan automatic file validation sebelum upload. Image preview dengan zoom functionality memungkinkan pengguna untuk verify quality dan content sebelum submitting untuk processing.

Android sharing integration merupakan unique feature yang specifically designed untuk Gen Z behavior patterns. Aplikasi register sebagai sharing target untuk image MIME types, memungkinkan pengguna untuk directly share screenshots dari payment apps seperti Gopay, BCA Mobile, atau SeaBank ke TrackMyBills. Integration ini menggunakan Intent filters dan custom sharing UI yang seamlessly handle incoming shared content.

\subsubsection{Image Processing dan API Communication}
\label{subsubsec:image-processing-api}

Image processing workflow mengimplementasikan automatic optimization untuk efficient transmission ke backend API. Pre-upload processing mencakup image compression menggunakan adaptive quality settings yang balance antara file size dan visual quality. Compression algorithm mempertimbangkan original image dimensions dan content complexity untuk determine optimal compression ratio.

Image cropping functionality menggunakan UCrop library yang menyediakan intuitive cropping interface dengan predefined aspect ratios untuk different document types. Cropping feature particularly useful untuk paper receipts yang mungkin memerlukan framing adjustment untuk optimal OCR results. Real-time preview dengan zoom dan pan capabilities memudahkan precision cropping untuk various document sizes.

API communication menggunakan HTTP client dengan comprehensive retry mechanism dan exponential backoff untuk handling network issues. Request timeout configuration menggunakan adaptive values berdasarkan file size dan connection quality. Upload progress indicators menggunakan stream-based approach yang memberikan real-time feedback kepada pengguna selama transmission process.

Error handling mengimplementasikan fallback mechanism yang menampilkan mock data ketika backend API tidak tersedia. Strategi ini memastikan aplikasi tetap functional bahkan dalam kondisi network connectivity yang tidak optimal, memberikan user experience yang konsisten untuk pengguna Gen Z yang mengharapkan aplikasi selalu responsive.

\subsubsection{Data Management dan Storage}
\label{subsubsec:data-management}

Data persistence menggunakan SharedPreferences untuk lightweight local storage yang appropriate untuk financial transaction data. Storage architecture mengimplementasikan encrypted data containers untuk sensitive information seperti transaction amounts dan merchant names. Encryption menggunakan platform-native security features dengan automatic key management.

% Transaction management mengimplementasikan tiga method utama: addTransaction() untuk menambah transaksi baru dengan automatic state notification, loadTransactions() untuk loading data dari storage dengan comprehensive error handling, dan _saveTransactions() untuk persistence dalam format JSON. Receipt type preference management menggunakan loadReceiptTypePreference() dan saveReceiptTypePreference() untuk user customization.

Transaction history organization menggunakan structured data format yang enable efficient querying dan sorting berdasarkan various criteria seperti date, amount, atau transaction type. Data model support untuk both digital payment extractions dan receipt-based transactions dengan flexible schema yang dapat accommodate different data structures.

Automatic categorization menggunakan rule-based logic yang analyze extracted merchant names dan transaction patterns untuk suggest appropriate expense categories. Category suggestions dapat di-customize oleh pengguna dan learning mechanism dapat improve suggestion accuracy over time berdasarkan user preferences.

Local data backup dan restore functionality memungkinkan pengguna untuk export transaction history dalam standard formats seperti CSV atau JSON. Import capabilities mendukung data migration dari other expense tracking applications atau manual transaction logs.

\subsubsection{User Interface dan Experience Features}
\label{subsubsec:ui-ux-features}

Results display mengimplementasikan structured card layouts yang menampilkan extracted information dalam easily scannable format. Editable fields memungkinkan pengguna untuk correction atau addition information yang mungkin not captured perfectly oleh OCR process. Confidence indicators menggunakan visual cues seperti color coding atau icon badges untuk help users understand extraction reliability.

Expense tracking dashboard menyediakan overview dari spending patterns dengan basic analytics yang relevant untuk Gen Z financial awareness. Charts dan visualizations menggunakan simple bar charts dan pie charts yang easy to understand tanpa overwhelming complexity. Spending summaries available dalam different time periods seperti weekly, monthly, atau custom date ranges.

Search dan filter functionality memungkinkan quick access ke specific transactions berdasarkan various criteria. Search implementation menggunakan fuzzy matching untuk handle typos atau partial merchant names. Filter options mencakup date ranges, amount ranges, transaction types, dan custom categories.

Settings dan preferences screen menyediakan customization options untuk notification preferences, default categories, currency formatting, dan data retention policies. User profile management minimal sesuai dengan local-first approach yang tidak memerlukan user accounts atau cloud synchronization.

**Diagram yang disarankan**: User journey flowchart menunjukkan different paths dari image capture hingga transaction storage, termasuk error handling dan retry mechanisms.

Performance optimizations mencakup efficient image handling dengan automatic memory management, background processing untuk non-critical tasks, dan caching strategies untuk frequently accessed data. Aplikasi dioptimalkan untuk smooth performance pada mid-range Android devices yang commonly used oleh target demographic dengan memory footprint yang minimal dan battery usage yang efficient.


\subsection{Integrasi Sistem Keseluruhan}
\label{subsec:integrasi-sistem-keseluruhan}

Integrasi sistem keseluruhan merepresentasikan implementasi end-to-end dari semua komponen yang telah dikembangkan, menciptakan ecosystem yang cohesive untuk ekstraksi data pembayaran. Integrasi ini menggabungkan aplikasi mobile, backend service, dan model inference dalam satu workflow yang seamless.

% \begin{figure}[htbp]
%     \centering
%     \includegraphics[width=\textwidth]{images/complete-system-integration.png}
%     \caption{Integrasi sistem keseluruhan dengan alur data end-to-end}
%     \label{fig:complete-system-integration}
% \end{figure}

Alur integrasi dimulai dari user interaction pada aplikasi mobile dan berakhir dengan structured data yang siap untuk analysis atau storage. Setiap step dalam alur ini telah dioptimalkan untuk memberikan user experience yang optimal sambil maintaining data accuracy dan system reliability.

\emph{Data flow} mengikuti pattern yang consistent: image capture/sharing → preprocessing → API transmission → model selection → inference → response formatting → result display → local storage. Setiap transition antar komponen mengimplementasikan proper error handling dan fallback mechanisms yang memastikan system robustness.

\emph{Automatic model selection} merupakan salah satu achievement utama dari integrasi ini. System dapat secara intelligent menentukan model yang appropriate berdasarkan image characteristics tanpa requiring user input. Logic selection menganalisis visual patterns, image dimensions, dan content characteristics untuk routing ke model yang optimal.

% \begin{figure}[htbp]
%     \centering
%     \includegraphics[width=0.9\textwidth]{images/system-data-flow.png}
%     \caption{Alur data dan komunikasi antar komponen sistem}
%     \label{fig:system-data-flow}
% \end{figure}

\emph{Performance optimization} pada level system mencakup connection pooling untuk API communications, efficient memory management yang prevent memory leaks, dan caching strategies yang reduce redundant processing. System dirancang untuk handle concurrent requests dengan graceful degradation ketika resource constraints encountered.

Quality assurance mechanisms terintegrasi pada multiple levels. Image validation pada mobile app level memastikan quality minimal sebelum transmission. API validation memverifikasi data integrity dan format compliance. Model confidence scoring memberikan quality indicators yang dapat digunakan untuk automated quality control atau user feedback.

\emph{Error recovery strategies} mengimplementasikan multi-tier fallback approach. Jika primary model gagal, system automatically attempts dengan alternative model. Jika API communication fails, aplikasi mobile menyimpan request untuk retry ketika connection restored. Jika model inference menghasilkan low confidence results, system dapat prompt user untuk image retaking atau manual verification.

\emph{Monitoring dan logging} terintegrasi across all components memberikan comprehensive visibility into system performance. Request tracing menggunakan unique identifiers yang dapat ditrack dari mobile app hingga model inference dan back. Performance metrics collection memungkinkan identification dari bottlenecks dan optimization opportunities.

% \begin{figure}[htbp]
%     \centering
%     \includegraphics[width=0.8\textwidth]{images/system-monitoring-dashboard.png}
%     \caption{Dashboard monitoring sistem terintegrasi}
%     \label{fig:system-monitoring-dashboard}
% \end{figure}

\emph{Scalability considerations} dalam integrasi mencakup stateless design yang memungkinkan horizontal scaling, efficient resource utilization yang minimize computational overhead, dan modular architecture yang facilitate addition dari new models atau features tanpa major system changes.

Security implementation mencakup end-to-end encryption untuk data transmission, secure API authentication, dan proper data sanitization pada all entry points. Privacy protection ensures bahwa user data tidak retained longer than necessary dan proper anonymization untuk analytics purposes.

\emph{Deployment strategy} menggunakan containerized approach dengan Docker yang memungkinkan consistent deployment across different environments. Configuration management memungkinkan easy adjustment dari system parameters tanpa code changes. Health checking dan automatic restart capabilities ensure high availability dalam production deployment.

User feedback integration memungkinkan continuous improvement dari system accuracy. Feedback mechanism memungkinkan users untuk report incorrect extractions atau suggest improvements, creating feedback loop yang dapat digunakan untuk future model improvements atau system enhancements.

\emph{Backward compatibility} considerations memastikan bahwa future updates tidak break existing functionality. API versioning strategy memungkinkan smooth migration ketika new features added atau existing features enhanced. Database migration scripts memastikan data integrity ketika schema changes required.

System testing strategy mencakup unit testing untuk individual components, integration testing untuk component interactions, end-to-end testing untuk complete user workflows, dan performance testing untuk load handling capabilities. Automated testing pipeline memastikan bahwa system changes tidak introduce regressions atau performance degradations.

Hasil integrasi adalah system yang not only technically sound tetapi juga delivers real value kepada target users. Gen Z users dapat dengan mudah digitize payment information dengan minimal friction, enabling better financial tracking dan budgeting habits yang crucial untuk financial literacy development.
