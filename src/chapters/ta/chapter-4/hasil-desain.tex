\section{Hasil Desain}
\label{sec:hasil-desain}

Hasil desain adalah subbab yang menjelaskan mengenai artefak yang dihasilkan oleh sistem. Subbab ini sejalan dengan tahap \emph{Design and Development} pada metodologi \dsrm{} yang digunakan sebagai metodologi dalam tugas akhir ini. Artefak sistem yang dijelaskan pada subbab ini merupakan implementasi dari tahapan desain yang telah dirancang pada \autoref{sec:tahapan-desain}. 

Sistem ini memiliki tiga artefak utama yang saling terintegrasi untuk memberikan solusi \emph{end-to-end} dalam pengembangan sistem pencatatan pengeluaran berbasis \emph{mobile} yang lebih terotomasi. Artefak-artefak tersebut meliputi model \donut{} yang telah disesuaikan untuk domain pembayaran Indonesia, layanan \emph{backend} (DonutAPI) yang menyediakan inferensi model melalui REST API, dan aplikasi \emph{mobile} TrackMyBills yang menjadi antarmuka pengguna.

\subsection{Model yang Digunakan}
\label{subsec:model-yang-digunakan}

Sistem TrackMyBills menggunakan dua jenis model yang menggunakan arsitektur model \donut{} yang telah disesuaikan dengan karakteristik dokumen pembayaran Indonesia. Dua jenis model ini digunakan karena perbedaan signifikan dari segi \emph{layout}, struktur informasi, dan visual pada dokumen bukti pembayaran QRIS, transfer, dan struk pembayaran.

Kedua model yang diimplementasikan menggunakan arsitektur \donut{} yang menggabungkan \emph{vision encoder} dan \emph{text decoder} untuk memahami dokumen secara holistik. \donut{} menggunakan \swin{} sebagai \emph{vision encoder} untuk memproses informasi visual dokumen dan \bart{} sebagai \emph{text decoder} untuk menghasilkan keluaran terstruktur dalam format JSON. Kombinasi ini memungkinkan model untuk memahami tidak hanya teks yang terbaca, tetapi juga \emph{layout} dan konteks visual dokumen secara keseluruhan.

% **Diagram yang disarankan**: Arsitektur dual-model dengan visualisasi jalur inference untuk setiap jenis dokumen, menunjukkan preprocessing, vision encoding, text decoding, dan postprocessing untuk masing-masing model.

\subsubsection{Model DONUT QRIS-TF}
\label{subsubsec:donut-qris-tf}

Model \donut{} QRIS-TF, disebut \textbf{\emph{custom model}} ke depannya, merupakan hasil \emph{fine-tuning} dari model \donutcord{} pada \dataset{} QRIS dan transfer. Model ini dilatih menggunakan \dataset{} yang dikumpulkan dari berbagai \emph{platform} pembayaran populer di Indonesia untuk mencakup variasi \emph{layout} dan format. Jumlah data yang digunakan untuk melakukan pelatihan adalah 257 sampel, sedangkan 56 sampel digunakan untuk melakukan validasi dan 99 sampel digunakan sebagai data uji.

Konfigurasi \emph{custom model} menggunakan \emph{task prompt} khusus \texttt{<s\_payment\_proof>} yang memberikan konteks spesifik untuk jenis dokumen yang akan diproses. Model ini memiliki jumlah \emph{vocabulary} yang diperluas dengan 14 token spesial baru yang disesuaikan dengan informasi yang perlu diekstrak dari bukti pembayaran QRIS dan transfer, yang terdiri dari:
\begin{enumerate}
    \item \texttt{<s\_total\_amount>} dan \texttt{</s\_total\_amount>} untuk nominal pembayaran.
    \item \texttt{<s\_transaction\_time>} dan \texttt{</s\_transaction\_time>} untuk waktu transaksi.
    \item \texttt{<s\_transaction\_identifier>} dan \texttt{</s\_transaction\_identifier>} untuk ID transaksi.
    \item \texttt{<s\_type>} dan \texttt{</s\_type>} untuk jenis transaksi (QRIS atau transfer).
    \item \texttt{<s\_target\_name>} dan \texttt{</s\_target\_name>} untuk nama penerima.
    \item \texttt{<s\_application>} dan \texttt{</s\_application>} untuk aplikasi pembayaran yang digunakan.
    \item \texttt{<s\_payment\_proof>} dan \texttt{</s\_payment\_proof>} sebagai \emph{task prompt} khusus tugas ekstraksi dokumen QRIS dan transfer.
\end{enumerate}

Model dilatih dengan menambahkan 14 token spesial ke dalam \emph{vocabulary} model untuk memastikan bahwa model dapat mengenali dan menghasilkan keluaran yang sesuai. Pelatihan dilakukan dengan parameter berikut.
\begin{enumerate}
\item \emph{Learning rate}: 3e-5
\item \emph{Batch size}: 1
\item \emph{Epoch}: 30
\item \emph{Gradient accumulation steps}: 8
\item \emph{Weight decay}: 0,01
\end{enumerate}

Selama pelatihan, \emph{weight} dari setiap token spesial diatur untuk mengadaptasikan model untuk mengenali atribut relevan yang sesuai dengan tugas ekstraksi model. Model kemudian akan mencoba untuk memprediksi atribut dari masukan gambar yang diberikan dan mengkalkulasikan perbedaan antara prediksi dan \emph{ground truth} yang telah ditentukan. Hasil kalkulasi akan digunakan untuk mengubah \emph{weight} dari token spesial yang digunakan. Proses ini dilakukan berulang kali selama 30 \emph{epoch}. Pelatihan baru akan dihentikan saat \emph{validation loss} tidak berubah atau mengalami perubahan yang sangat sedikit selama 3 \emph{epoch} berturut-turut.

\emph{Training Loss}, \emph{loss} yang dikalkulasikan terhadap data pelatihan, berubah dari 3,7252 menjadi 0,0769 dan \emph{Validation Loss}, \emph{loss} yang dikalkulasikan terhadap data validasi, berubah dari 5,563 menjadi 0,5836. Kedua variabel ini perlu dikontrol untuk memastikan bahwa model tidak mengalami \emph{overfitting} pada data pelatihan. Jika \emph{Training Loss} terus berkurang dan \emph{Validation Loss} malah terus bertambah selama pelatihan, hal tersebut menjadi indikasi kuat terjadinya \emph{overfitting}. Hasil pelatihan menunjukkan bahwa model dapat belajar dengan baik dari data yang diberikan dan dapat menghasilkan keluaran yang sesuai dengan atribut yang diharapkan.

\subsubsection{Model DONUT CORD-v2}
\label{subsubsec:model-base}

PTM \donut{} \texttt{naver-clova-ix/donut-base-finetuned-cord-v2} digunakan sebagai model dasar, disebut \textbf{\emph{base model}} ke depannya. Karena \emph{base model} adalah sebuah PTM, model ini bukanlah artefak dari sistem. PTM ini telah di-\emph{fine-tune} pada \dataset{} CORD-v2 sehingga telah memiliki pemahaman yang baik terhadap struktur dan format struk pembayaran yang umum digunakan untuk dokumen struk pembayaran tanpa proses \emph{fine-tuning} tambahan. \datasetfl{} CORD-v2 memiliki kekurangan, yaitu \dataset{} yang digunakan memiliki bagian \emph{header} dan \emph{footer} yang di-\emph{blur} untuk memudahkan model memahami isi dokumen. \datasetfl{} CORD-v2 memiliki 800 sampel pelatihan, 100 sampel validasi, dan 100 sampel uji yang digunakan untuk melatih model \donut{} pada format struk pembayaran.

Konfigurasi \emph{base model} menggunakan \emph{task prompt} \texttt{<s\_cord-v2>} untuk melakukan tugas ekstraksi dari struk pembayaran. Informasi yang diekstrak dari struk pembayaran sangat terperinci hingga \emph{level bounding box} atribut. Informasi yang digunakan dari seluruh data ekstraksi tersebut terbatas pada atribut relevan sebagai berikut.
\begin{enumerate}
    \item \texttt{<s\_menu>} dan \texttt{</s\_menu>} untuk daftar menu dengan atribut berikut.
    \begin{enumerate}
        \item \texttt{<s\_nm>} dan \texttt{</s\_nm>} untuk nama menu
        \item \texttt{<s\_cnt>} dan \texttt{</s\_cnt>} untuk kuantitas
        \item \texttt{<s\_price>} dan \texttt{</s\_price>} untuk harga
    \end{enumerate}
    \item \texttt{<s\_total\_price>} dan \texttt{</s\_total\_price>} untuk total pembayaran
\end{enumerate}

% \emph{Base model} mengimplementasikan preprocessing yang robust untuk optimasi ukuran gambar dengan maximum dimension 2048 pixels dan automatic resizing menggunakan Lanczos resampling untuk mempertahankan kualitas visual. Generation configuration menggunakan single beam search dengan early stopping untuk optimasi kecepatan inference sambil mempertahankan akurasi yang tinggi.

Keluaran \emph{base model} berupa struktur JSON yang kompleks dan hierarkis, sesuai dengan format \emph{ground truth} pada \dataset{} CORD-v2 yang mencerminkan struktur struk. Model ini juga menggunakan token-to-JSON \emph{converter} untuk mengubah format yang dari deretan sekuensial token yang digunakan menjadi format JSON.

% **Diagram yang disarankan**: Comparison matrix antara kedua model menunjukkan input types, special tokens, output structure, dan use cases masing-masing model.

% Kedua model menggunakan automatic device detection yang dapat beradaptasi dengan hardware yang tersedia, baik CPU maupun GPU inference. Implementasi singleton pattern memastikan model hanya dimuat sekali dalam memory untuk mengoptimalkan resource usage dan mengurangi latency pada request subsequent.


\subsection{Backend Service DonutAPI}
\label{subsec:backend-service-donutapi}

DonutAPI merupakan implementasi REST API menggunakan framework FastAPI yang menyediakan layanan inferensi model DONUT untuk ekstraksi data dari dokumen pembayaran. Service ini dirancang dengan arsitektur modular yang memisahkan layer API, business logic, dan model inference untuk memudahkan maintenance dan pengembangan future features.

Arsitektur backend mengadopsi design patterns yang robust termasuk singleton pattern untuk model management, dependency injection untuk service components, dan comprehensive error handling untuk production reliability. Service mengimplementasikan automatic model loading saat startup dengan preloading mechanism yang memastikan kedua model siap digunakan sebelum menerima request dari client.

Application lifecycle management menggunakan FastAPI's lifespan context manager yang mengatur startup dan shutdown processes secara graceful. Startup sequence meliputi model preloading, device detection, dan system health verification. Shutdown sequence memastikan proper cleanup dari model resources dan GPU memory clearing jika menggunakan CUDA acceleration.

**Diagram yang disarankan**: Service architecture diagram menunjukkan layer separation, request flow, dan component interactions dalam DonutAPI.

\subsubsection{Endpoint \texttt{/predict/}}
\label{subsubsec:endpoint-predict-full}

Endpoint \texttt{/predict/} menyediakan full prediction service dengan detailed response yang mencakup metadata lengkap tentang inference process. Endpoint ini dirancang untuk use cases yang memerlukan informasi comprehensive tentang model performance dan processing details.

Request validation mengimplementasikan comprehensive checks termasuk file format validation (JPG, JPEG, PNG, WebP, BMP), file size limits maksimum 10MB, dan basic file integrity verification. Query parameter \texttt{model} bersifat required dan harus berupa \texttt{ModelType.BASE} atau \texttt{ModelType.CUSTOM} untuk menentukan model mana yang akan digunakan untuk inference.

Response structure untuk endpoint ini mencakup:
\begin{itemize}
    \item \texttt{success}: Boolean indicator untuk status inference
    \item \texttt{result}: Object containing extracted data dan metadata
    \item \texttt{model\_info}: Informasi detail tentang model yang digunakan
    \item \texttt{processing\_time\_seconds}: Waktu yang diperlukan untuk inference
    \item \texttt{device}: Hardware device yang digunakan untuk inference (CPU/GPU)
    \item \texttt{raw\_sequence}: Token sequence mentah sebelum postprocessing (khusus custom model)
    \item \texttt{confidence\_score}: Skor confidence untuk kualitas ekstraksi (khusus custom model)
\end{itemize}

Error handling untuk endpoint ini mengimplementasikan structured error responses dengan HTTP status codes yang appropriate dan detailed error messages untuk debugging purposes. Request tracing menggunakan unique request ID yang memudahkan correlation antara logs dan monitoring metrics.

\subsubsection{Endpoint \texttt{/predict/simple}}
\label{subsubsec:endpoint-predict-simple}

Endpoint \texttt{/predict/simple} menyediakan simplified response yang focus pada extracted data untuk mobile application consumption. Response format yang streamlined mengurangi bandwidth usage dan mempercepat parsing di sisi client application.

Response structure untuk simple endpoint hanya mencakup essential information:
\begin{itemize}
    \item \texttt{success}: Boolean indicator untuk status inference
    \item \texttt{data}: Object containing extracted data dalam raw format dan cleaned format
    \item \texttt{error}: Error message jika inference gagal
\end{itemize}

Data cleaning dan normalization dilakukan secara otomatis pada endpoint ini untuk memastikan client application menerima data yang ready-to-use. Proses cleaning mencakup currency formatting untuk amount fields, date parsing untuk timestamp fields, dan text normalization untuk name fields.

Custom model response pada endpoint simple mencakup automatic classification result yang menentukan apakah dokumen adalah QRIS atau transfer berdasarkan extracted type field. Base model response menyediakan structured receipt data dengan itemized information yang sudah dinormalisasi untuk mobile display.

\subsubsection{Model Selection dan Inference Service}
\label{subsubsec:model-selection-inference}

DonutAPI mengimplementasikan model selection logic berdasarkan query parameter 'model' yang menentukan apakah menggunakan Internal Model (QRIS-TF) atau Deployed Model (CORD-v2) berdasarkan document type yang akan diproses. Model selection ditentukan oleh receipt type yang dispecify dalam endpoint query parameters, memberikan control yang explicit kepada client application untuk memilih model yang sesuai dengan jenis dokumen.

Deployment architecture menggunakan three-tier approach dengan Android client, Docker container hosting FastAPI service dengan Internal Model, dan external HuggingFace service untuk Deployed Model. Arsitektur ini memungkinkan scalability dan maintainability yang optimal dengan separation between local inference dan cloud-based model serving.

Inference service menggunakan singleton pattern untuk kedua model instances yang memastikan memory efficiency dan menghindari multiple model loading. Model initialization menggunakan lazy loading approach yang hanya memuat model ketika pertama kali dibutuhkan, optimizing startup time dan memory usage.

Error recovery mechanism mengimplementasikan graceful degradation di mana jika primary model gagal, system secara otomatis fallback ke alternative processing atau memberikan informative error message kepada client. Fallback logic untuk custom model dapat menggunakan base model jika custom weights gagal dimuat.

Service juga mengimplementasikan resource monitoring yang track GPU memory usage, CPU utilization, dan inference latency untuk operational insights. Health check endpoints \texttt{/status} dan \texttt{/} menyediakan comprehensive system status termasuk model readiness, hardware information, dan performance metrics.

**Diagram yang disarankan**: API request/response flow diagram menunjukkan validation, model selection, inference, dan response formatting untuk kedua endpoints.

Deployment configuration menggunakan Docker containerization dengan multi-stage builds yang mengoptimalkan image size dan security. Container includes automatic device detection untuk CPU dan GPU environments dengan graceful fallback jika CUDA tidak tersedia. Production deployment menggunakan Uvicorn ASGI server dengan process management yang robust untuk high availability scenarios.


\subsection{Aplikasi Mobile TrackMyBills}
\label{subsec:aplikasi-mobile-trackmybills}

TrackMyBills merupakan aplikasi \emph{mobile} berbasis Flutter. Aplikasi ini berfungsi sebagai antarmuka pengguna untuk sistem ekstraksi data pembayaran dengan integrasi fungsi kamera, akses galeri, \emph{sharing intent}, dan pemanggilan API. TrackMyBills dibangun dengan menggunakan pola \emph{Model-View-Controller} (MVC) yang memisahkan antarmuka, model yang digunakan, dan logika fitur.

\subsubsection{Metode Masukan Gambar}
\label{subsubsec:metode-masukan-gambar}

TrackMyBills menyediakan tiga metode untuk menerima masukan gambar, yaitu melalui kamera, galeri, dan \emph{sharing intent}. TrackMyBills menggunakan \emph{package} \texttt{camera} untuk mengakses kamera perangkat, memungkinkan pengguna untuk mengambil foto bukti pembayaran secara langsung. Untuk bisa menggunakan fitur ini, pengguna harus memberikan izin akses kamera pada aplikasi. Selain menggunakan kamera, TrackMyBills juga menyediakan opsi untuk memilih gambar dari galeri perangkat. Pengguna dapat memilih gambar yang telah disimpan sebelumnya di galeri mereka. Fitur ini diimplementasikan dengan menggunakan \emph{package} \texttt{image\_picker} untuk memilih gambar dari galeri. Fitur ini memerlukan izin akses penyimpanan. TrackMyBills juga mendukung \emph{sharing intent} yang memungkinkan pengguna untuk langsung membagikan tangkapan layar atau gambar bukti pembayaran dari aplikasi lain ke TrackMyBills. Fitur ini memanfaatkan \emph{intent} Android untuk menerima gambar yang dibagikan oleh aplikasi lain. Fitur ini menggunakan \emph{package} \texttt{receive\_sharing\_intent} untuk menangani gambar yang dibagikan.

\subsubsection{Pemrosesan Gambar dan API \emph{Communication}}
\label{subsubsec:pemrosesan-gambar}

Pemrosesan gambar dilakukan dengan menggunakan \emph{package} \texttt{crop\_your\_image} yang menyediakan antarmuka pengguna untuk memotong gambar sebelum diunggah. Pengguna dapat memilih area yang relevan dari gambar bukti pembayaran untuk memastikan bahwa hanya informasi penting yang dikirim ke layanan \emph{backend}. 

Setelah gambar dipotong, TrackMyBills mengirimkan gambar tersebut ke layanan \emph{backend} menggunakan \emph{package} \texttt{http}. TrackMyBills melakukan \emph{request} HTTP POST ke \emph{endpoint} URL API layanan \emph{backend}. Gambar yang diunggah akan dikirim dalam format \emph{multipart/form-data} yang memungkinkan pengiriman \emph{file} bersama dengan data lainnya. Pemanggilan API mengimplementasikan mekanisme \emph{retry} untuk menangani kegagalan jaringan dengan \emph{package} \texttt{retry}. Jika terjadi kegagalan, TrackMyBills akan mencoba mengirim ulang permintaan hingga tiga kali sebelum akhirnya memunculkan \emph{dummy data} saat upaya gagal. Respons yang diterima dari layanan \emph{backend} berupa data terstruktur dalam format JSON yang berisi informasi yang diekstrak dari gambar bukti pembayaran.

\subsubsection{Manajemen Data dan Penyimpanan}
\label{subsubsec:manajemen-data-dan-penyimpanan}

TrackMyBills akan memetakan seluruh data yang diterima dari layanan \emph{backend} ke dalam model transaksi tergantung pada jenis bukti pembayaran terkait yang diunggah. Data yang sudah dipetakan ditampilkan kepada pengguna untuk dikonfirmasi. Pengguna memiliki kebebasan untuk mengubah data tersebut jika terdapat kesalahan atau ketidakakuratan. Pengguna dapat menekan tombol "Save Transaction" untuk mengonfirmasi dan menyimpan data transaksi yang telah dipetakan. Data transaksi yang telah tersimpan akan ditampilkan pada halaman pengeluaran dan diklasifikasikan berdasarkan kategori yang telah dipilih oleh pengguna. TrackMyBills menyediakan fitur untuk menampilkan total pengeluaran pengguna dan total pengeluaran per kategori, sehingga pengguna dapat dengan mudah melacak dan mengelola pengeluaran mereka.

TrackMyBills menyimpan data pengeluaran pengguna secara lokal menggunakan \emph{package} \texttt{shared\_preferences}. Data yang disimpan mencakup seluruh informasi terkait dalam bentuk JSON. Penyimpanan ini memungkinkan pengguna untuk mengakses data mereka secara luring tanpa perlu terhubung ke internet dan mengimplementasikan \emph{data persistence} secara lokal tanpa perlu khawatir kehilangan data. SharedPreference menyimpan data secara lokal dalam gawai pengguna dan pada lokasi yang hanya dapat diakses oleh aplikasi terkait. Hal ini memastikan bahwa data pengguna tetap aman dan tidak dapat diakses oleh aplikasi lain.



\subsection{Alur Kerja Sistem}
\label{subsec:alur-kerja-sistem}