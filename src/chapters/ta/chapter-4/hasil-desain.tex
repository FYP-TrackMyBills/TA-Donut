\section{Hasil Desain}
\label{sec:hasil-desain}

Hasil desain merupakan implementasi konkret dari tahapan desain yang telah dirancang pada \autoref{sec:tahapan-desain}. Bagian ini menyajikan artefak-artefak yang dihasilkan dari proses pengembangan sistem, mencakup implementasi script pelatihan, model yang telah di-\emph{fine-tune}, framework evaluasi, layanan \emph{backend}, aplikasi \emph{mobile}, dan integrasi sistem secara keseluruhan.

Setiap artefak dikembangkan dengan mempertimbangkan kebutuhan fungsional dan non-fungsional yang telah didefinisikan, serta disesuaikan dengan ekspektasi pengguna Gen Z terhadap sistem mobile yang modern dan efisien. Implementasi menggunakan teknologi terkini dan best practices dalam pengembangan aplikasi mobile dan sistem berbasis \emph{machine learning}.

% \begin{figure}[htbp]
%     \centering
%     \includegraphics[width=\textwidth]{images/system-artifacts-overview.png}
%     \caption{Overview artefak sistem yang dihasilkan}
%     \label{fig:system-artifacts-overview}
% \end{figure}

Hasil desain menunjukkan implementasi sistem yang lengkap dan siap untuk deployment, dengan setiap komponen telah melalui proses testing dan validasi. Integrasi antar komponen dirancang untuk memberikan pengalaman pengguna yang seamless sambil mempertahankan akurasi dan reliabilitas sistem.

\subsection{Script Fine-tuning}
\label{subsec:script-fine-tuning}

Script \emph{fine-tuning} merupakan implementasi konkret dari strategi pelatihan yang telah dirancang pada \autoref{subsec:perancangan-fine-tuning}. Script ini dikembangkan dengan pendekatan modular yang memungkinkan fleksibilitas dalam konfigurasi pelatihan dan monitoring proses \emph{fine-tuning} secara real-time.

% \begin{figure}[htbp]
%     \centering
%     \includegraphics[width=0.8\textwidth]{images/fine-tuning-script-architecture.png}
%     \caption{Arsitektur script fine-tuning dengan komponen utama}
%     \label{fig:fine-tuning-script-architecture}
% \end{figure}

Implementasi script menggunakan framework PyTorch dan Transformers library dari HuggingFace untuk memastikan kompatibilitas dengan model \donut{} dan ekosistem \emph{machine learning} yang luas. Script dirancang dengan beberapa komponen utama: \emph{data loading}, \emph{model initialization}, \emph{training loop}, dan \emph{evaluation pipeline}.

Komponen \emph{data loading} mengimplementasikan \texttt{PaymentProofDataset} class yang secara khusus dirancang untuk menangani format data JSONL dengan struktur yang konsisten. Dataset loader ini memiliki kemampuan untuk memproses gambar dalam berbagai format (JPG, JPEG, PNG) dan melakukan preprocessing yang diperlukan seperti resizing dan normalisasi. Implementasi juga mencakup robust error handling untuk menangani file yang corrupted atau format anotasi yang tidak valid.

Proses inisialisasi model melakukan ekspansi vocabulary untuk mengakomodasi \emph{special tokens} yang spesifik untuk domain pembayaran Indonesia. Script secara otomatis mendeteksi dan menambahkan token baru seperti representasi mata uang rupiah, format tanggal Indonesia, dan terminologi pembayaran lokal. Konfigurasi model disesuaikan dengan kebutuhan pelatihan, termasuk pengaturan \emph{mixed precision} FP16 untuk optimasi memori dan kecepatan pelatihan.

\emph{Training loop} mengimplementasikan strategi yang telah dirancang dengan 30 epoch, \emph{batch size} 1, dan \emph{gradient accumulation} 8 steps. Script mencakup implementasi \emph{early stopping} dengan patience=3 untuk mencegah \emph{overfitting} dan mengoptimalkan penggunaan sumber daya komputasi. Monitoring progres pelatihan dilakukan melalui logging yang comprehensive, mencakup \emph{training loss}, \emph{validation loss}, dan metrik evaluasi khusus.

Script juga mengimplementasikan sistem \emph{checkpoint} otomatis yang menyimpan state model setiap 5 epoch, memungkinkan recovery dari interruption dan analisis evolusi kinerja model. Implementasi mencakup metadata lengkap untuk setiap checkpoint, termasuk konfigurasi pelatihan, metrik performance, dan timestamp untuk audit trail yang lengkap.

Untuk memastikan reproducibility, script mencakup seed management yang konsisten dan logging semua hyperparameter yang digunakan. Dokumentasi inline yang comprehensive memungkinkan maintainability dan adaptasi script untuk eksperimen future. Script dirancang dengan modularitas yang memungkinkan easy integration dengan berbagai \emph{monitoring tools} dan \emph{experiment tracking platforms}.


\subsection{Model yang Telah Di-fine-tune}
\label{subsec:model-fine-tuned}

Hasil dari proses \emph{fine-tuning} menghasilkan dua model yang telah diadaptasi untuk domain pembayaran Indonesia, masing-masing dioptimalkan untuk menangani jenis dokumen yang berbeda. Kedua model ini merupakan hasil adaptasi dari model \donut{} dasar yang telah disesuaikan dengan karakteristik dan kebutuhan spesifik ekstraksi data pembayaran lokal.

% \begin{figure}[htbp]
%     \centering
%     \includegraphics[width=0.9\textwidth]{images/dual-model-architecture.png}
%     \caption{Arsitektur dual-model untuk pemrosesan dokumen pembayaran}
%     \label{fig:dual-model-architecture}
% \end{figure}

Strategi \emph{dual-model} dipilih untuk memaksimalkan akurasi ekstraksi dengan mempertimbangkan karakteristik visual dan struktural yang berbeda antara dokumen pembayaran digital dan struk berbasis kertas. Setiap model dikembangkan dengan \emph{task-specific optimization} yang memungkinkan performa optimal pada domain masing-masing.

\subsubsection{Model QRIS-TF (Ekstraksi dan Klasifikasi)}
\label{subsubsec:model-qris-tf}

Model QRIS-TF merupakan hasil \emph{fine-tuning} dari model \donut{} dasar menggunakan dataset bukti pembayaran digital yang mencakup transaksi QRIS dan transfer. Model ini memiliki kemampuan dual-function untuk melakukan ekstraksi informasi sekaligus klasifikasi jenis transaksi dalam satu proses inference.

% \begin{figure}[htbp]
%     \centering
%     \includegraphics[width=0.7\textwidth]{images/qris-tf-model-output.png}
%     \caption{Contoh output struktural model QRIS-TF}
%     \label{fig:qris-tf-model-output}
% \end{figure}

Karakteristik utama model QRIS-TF mencakup vocabulary yang telah diperluas dengan 14 \emph{special tokens} spesifik untuk domain pembayaran digital Indonesia. Token-token ini mencakup representasi untuk nominal pembayaran (\texttt{<total\_amount>}), waktu transaksi (\texttt{<transaction\_time>}), identifikator transaksi (\texttt{<transaction\_identifier>}), jenis transaksi (\texttt{<type>}), nama target/penerima (\texttt{<target\_name>}), dan aplikasi pembayaran (\texttt{<application>}).

Model menghasilkan output dalam format JSON terstruktur yang mencakup semua field yang diekstraksi beserta confidence score untuk setiap field. Kemampuan klasifikasi terintegrasi memungkinkan model untuk membedakan antara transaksi QRIS dan transfer berdasarkan pola visual dan tekstual yang dipelajari selama proses pelatihan. Akurasi klasifikasi yang tinggi tercapai melalui pembelajaran pola layout yang berbeda antara kedua jenis transaksi.

Optimasi model QRIS-TF mencakup fine-tuning pada 30 epoch dengan early stopping yang menghasilkan konvergensi optimal pada epoch ke-27. Model menunjukkan performa yang konsisten pada berbagai variasi aplikasi pembayaran, termasuk format yang berbeda dari SeaBank, Neobank, BCA, dan Gopay. Robustness model terhadap variasi kualitas gambar dan orientasi dokumen menjadi salah satu keunggulan utama implementasi ini.

\subsubsection{Model CORD-v2 (Ekstraksi)}
\label{subsubsec:model-cord-v2}

Model CORD-v2 merupakan adaptasi dari model \donut{} yang telah di-\emph{fine-tune} menggunakan dataset CORD untuk pemrosesan struk pembayaran berbasis kertas. Model ini difokuskan pada ekstraksi informasi tanpa komponen klasifikasi, disesuaikan dengan karakteristik visual struk yang lebih terstruktur dan konsisten.

% \begin{figure}[htbp]
%     \centering
%     \includegraphics[width=0.7\textwidth]{images/cord-v2-model-output.png}
%     \caption{Contoh output ekstraksi model CORD-v2}
%     \label{fig:cord-v2-model-output}
% \end{figure}

Karakteristik model CORD-v2 mencakup kemampuan untuk mengenali dan mengekstraksi berbagai elemen dari struk pembayaran, termasuk informasi merchant, daftar item, harga individual, subtotal, pajak, dan total pembayaran. Model ini telah dioptimalkan untuk menangani variasi format struk dari berbagai jenis merchant, mulai dari retail modern hingga warung tradisional.

Vocabulary model CORD-v2 mencakup representasi khusus untuk elemen-elemen struk yang umum di Indonesia, termasuk format mata uang rupiah, struktur alamat lokal, dan terminologi retail yang spesifik. Model menghasilkan output yang terstruktur dalam format yang konsisten, memudahkan parsing dan integrasi dengan sistem pencatatan keuangan.

Performa model CORD-v2 menunjukkan akurasi tinggi dalam ekstraksi field-field kritis seperti total pembayaran dan informasi merchant. Model mampu menangani variasi kualitas struk, mulai dari hasil scan yang bersih hingga foto struk dengan pencahayaan yang kurang optimal. Implementasi preprocessing yang robust membantu model dalam menormalkan input sebelum proses ekstraksi.

Kedua model menggunakan format output yang konsisten untuk memudahkan integrasi dengan komponen sistem lainnya. Confidence scoring yang disediakan oleh setiap model memungkinkan implementasi quality control dan user feedback mechanism untuk meningkatkan user experience secara keseluruhan.


\subsection{Framework Evaluasi Terpadu}
\label{subsec:framework-evaluasi-terpadu}

Framework evaluasi terpadu merupakan implementasi konkret dari metodologi evaluasi yang dirancang untuk memberikan penilaian komprehensif terhadap kedua model yang telah dikembangkan. Framework ini mengintegrasikan berbagai metrik evaluasi dalam satu sistem yang kohesif dan dapat dieksekusi secara otomatis.

% \begin{figure}[htbp]
%     \centering
%     \includegraphics[width=0.8\textwidth]{images/evaluation-framework-implementation.png}
%     \caption{Implementasi framework evaluasi terpadu}
%     \label{fig:evaluation-framework-implementation}
% \end{figure}

Implementasi framework menggunakan pendekatan modular yang memungkinkan evaluasi terpisah untuk model QRIS-TF dan CORD-v2, namun tetap memberikan hasil yang dapat dibandingkan. Framework dirancang dengan fleksibilitas untuk menangani berbagai format output dan dapat dengan mudah diperluas untuk metrik evaluasi tambahan di masa depan.

Komponen utama framework mencakup \emph{data loader} yang dapat menangani format dataset yang berbeda, \emph{model inference engine} yang support multiple model types, \emph{metrics calculation module} yang mengimplementasikan semua metrik yang diperlukan, dan \emph{reporting system} yang menghasilkan output evaluasi dalam format yang mudah dipahami.

\emph{Metrics calculation module} mengimplementasikan lima metrik utama: Accuracy, Precision, Recall, F1-Score, dan mCER (Mean Character Error Rate). Untuk model QRIS-TF, framework menghitung metrik klasifikasi untuk mengevaluasi kemampuan membedakan antara transaksi QRIS dan transfer, serta metrik ekstraksi untuk mengevaluasi akurasi field extraction. Model CORD-v2 dievaluasi fokus pada metrik ekstraksi dengan emphasis pada field-field kritis seperti total pembayaran dan informasi merchant.

Framework mengimplementasikan sistem weighting yang memberikan prioritas berbeda untuk field-field yang berbeda. Field \texttt{total\_amount} diberikan bobot 3.0x karena kritikalitasnya dalam pencatatan keuangan, field \texttt{type} dan \texttt{target\_name} diberikan bobot 2.0x, sedangkan field lainnya menggunakan bobot standard 1.0x. Sistem weighting ini memastikan bahwa evaluasi mencerminkan importance relatif dari setiap field dalam konteks aplikasi praktis.

% \begin{figure}[htbp]
%     \centering
%     \includegraphics[width=0.9\textwidth]{images/evaluation-metrics-dashboard.png}
%     \caption{Dashboard hasil evaluasi dengan visualisasi metrik}
%     \label{fig:evaluation-metrics-dashboard}
% \end{figure}

Output framework mencakup detailed performance report yang menampilkan metrik aggregate dan per-field analysis. Report juga mencakup confusion matrix untuk komponen klasifikasi, character error analysis untuk detailed understanding dari kesalahan ekstraksi, dan confidence score distribution yang membantu dalam understanding model uncertainty.

Framework juga mengimplementasikan debugging capabilities yang memungkinkan analysis mendalam terhadap sample-sample yang mengalami kesalahan. Feature ini mencakup side-by-side comparison antara ground truth dan prediction, highlighting field-field yang bermasalah, dan analysis error patterns yang dapat membantu dalam improvement model di masa depan.

Untuk memastikan reproducibility, framework mencakup comprehensive logging yang mencatat semua parameter evaluasi, dataset yang digunakan, dan konfigurasi model. Seed management yang konsisten memastikan bahwa hasil evaluasi dapat direproduksi dengan kondisi yang identik. Framework juga support untuk batch evaluation yang memungkinkan comparison performa across different model checkpoints atau konfigurasi.

Implementasi framework menggunakan efficient memory management untuk menangani dataset yang besar tanpa mengalami memory overflow. Parallel processing capabilities memungkinkan evaluasi yang cepat even pada dataset yang substantial. Error handling yang robust memastikan bahwa evaluation process dapat continue bahkan ketika encounter sample yang bermasalah atau model prediction yang malformed.


\subsection{Backend Service DonutAPI}
\label{subsec:backend-service-donutapi}

DonutAPI merupakan implementasi REST API menggunakan framework FastAPI yang menyediakan layanan inferensi model DONUT untuk ekstraksi data dari dokumen pembayaran. Service ini dirancang dengan arsitektur modular yang memisahkan layer API, business logic, dan model inference untuk memudahkan maintenance dan pengembangan future features.

Arsitektur backend mengadopsi design patterns yang robust termasuk singleton pattern untuk model management, dependency injection untuk service components, dan comprehensive error handling untuk production reliability. Service mengimplementasikan automatic model loading saat startup dengan preloading mechanism yang memastikan kedua model siap digunakan sebelum menerima request dari client.

Application lifecycle management menggunakan FastAPI's lifespan context manager yang mengatur startup dan shutdown processes secara graceful. Startup sequence meliputi model preloading, device detection, dan system health verification. Shutdown sequence memastikan proper cleanup dari model resources dan GPU memory clearing jika menggunakan CUDA acceleration.

**Diagram yang disarankan**: Service architecture diagram menunjukkan layer separation, request flow, dan component interactions dalam DonutAPI.

\subsubsection{Endpoint \texttt{/predict/}}
\label{subsubsec:endpoint-predict-full}

Endpoint \texttt{/predict/} menyediakan full prediction service dengan detailed response yang mencakup metadata lengkap tentang inference process. Endpoint ini dirancang untuk use cases yang memerlukan informasi comprehensive tentang model performance dan processing details.

Request validation mengimplementasikan comprehensive checks termasuk file format validation (JPG, JPEG, PNG, WebP, BMP), file size limits maksimum 10MB, dan basic file integrity verification. Query parameter \texttt{model} bersifat required dan harus berupa \texttt{ModelType.BASE} atau \texttt{ModelType.CUSTOM} untuk menentukan model mana yang akan digunakan untuk inference.

Response structure untuk endpoint ini mencakup:
\begin{itemize}
    \item \texttt{success}: Boolean indicator untuk status inference
    \item \texttt{result}: Object containing extracted data dan metadata
    \item \texttt{model\_info}: Informasi detail tentang model yang digunakan
    \item \texttt{processing\_time\_seconds}: Waktu yang diperlukan untuk inference
    \item \texttt{device}: Hardware device yang digunakan untuk inference (CPU/GPU)
    \item \texttt{raw\_sequence}: Token sequence mentah sebelum postprocessing (khusus custom model)
    \item \texttt{confidence\_score}: Skor confidence untuk kualitas ekstraksi (khusus custom model)
\end{itemize}

Error handling untuk endpoint ini mengimplementasikan structured error responses dengan HTTP status codes yang appropriate dan detailed error messages untuk debugging purposes. Request tracing menggunakan unique request ID yang memudahkan correlation antara logs dan monitoring metrics.

\subsubsection{Endpoint \texttt{/predict/simple}}
\label{subsubsec:endpoint-predict-simple}

Endpoint \texttt{/predict/simple} menyediakan simplified response yang focus pada extracted data untuk mobile application consumption. Response format yang streamlined mengurangi bandwidth usage dan mempercepat parsing di sisi client application.

Response structure untuk simple endpoint hanya mencakup essential information:
\begin{itemize}
    \item \texttt{success}: Boolean indicator untuk status inference
    \item \texttt{data}: Object containing extracted data dalam raw format dan cleaned format
    \item \texttt{error}: Error message jika inference gagal
\end{itemize}

Data cleaning dan normalization dilakukan secara otomatis pada endpoint ini untuk memastikan client application menerima data yang ready-to-use. Proses cleaning mencakup currency formatting untuk amount fields, date parsing untuk timestamp fields, dan text normalization untuk name fields.

Custom model response pada endpoint simple mencakup automatic classification result yang menentukan apakah dokumen adalah QRIS atau transfer berdasarkan extracted type field. Base model response menyediakan structured receipt data dengan itemized information yang sudah dinormalisasi untuk mobile display.

\subsubsection{Model Selection dan Inference Service}
\label{subsubsec:model-selection-inference}

DonutAPI mengimplementasikan model selection logic berdasarkan query parameter 'model' yang menentukan apakah menggunakan Internal Model (QRIS-TF) atau Deployed Model (CORD-v2) berdasarkan document type yang akan diproses. Model selection ditentukan oleh receipt type yang dispecify dalam endpoint query parameters, memberikan control yang explicit kepada client application untuk memilih model yang sesuai dengan jenis dokumen.

Deployment architecture menggunakan three-tier approach dengan Android client, Docker container hosting FastAPI service dengan Internal Model, dan external HuggingFace service untuk Deployed Model. Arsitektur ini memungkinkan scalability dan maintainability yang optimal dengan separation between local inference dan cloud-based model serving.

Inference service menggunakan singleton pattern untuk kedua model instances yang memastikan memory efficiency dan menghindari multiple model loading. Model initialization menggunakan lazy loading approach yang hanya memuat model ketika pertama kali dibutuhkan, optimizing startup time dan memory usage.

Error recovery mechanism mengimplementasikan graceful degradation di mana jika primary model gagal, system secara otomatis fallback ke alternative processing atau memberikan informative error message kepada client. Fallback logic untuk custom model dapat menggunakan base model jika custom weights gagal dimuat.

Service juga mengimplementasikan resource monitoring yang track GPU memory usage, CPU utilization, dan inference latency untuk operational insights. Health check endpoints \texttt{/status} dan \texttt{/} menyediakan comprehensive system status termasuk model readiness, hardware information, dan performance metrics.

**Diagram yang disarankan**: API request/response flow diagram menunjukkan validation, model selection, inference, dan response formatting untuk kedua endpoints.

Deployment configuration menggunakan Docker containerization dengan multi-stage builds yang mengoptimalkan image size dan security. Container includes automatic device detection untuk CPU dan GPU environments dengan graceful fallback jika CUDA tidak tersedia. Production deployment menggunakan Uvicorn ASGI server dengan process management yang robust untuk high availability scenarios.


\subsection{Aplikasi Mobile TrackMyBills}
\label{subsec:aplikasi-mobile-trackmybills}

TrackMyBills merupakan aplikasi \emph{mobile} berbasis Flutter. Aplikasi ini berfungsi sebagai antarmuka pengguna untuk sistem ekstraksi data pembayaran dengan integrasi fungsi kamera, akses galeri, \emph{sharing intent}, dan pemanggilan API. TrackMyBills dibangun dengan menggunakan pola \emph{Model-View-Controller} (MVC) yang memisahkan antarmuka, model yang digunakan, dan logika fitur.

\subsubsection{Metode Masukan Gambar}
\label{subsubsec:metode-masukan-gambar}

TrackMyBills menyediakan tiga metode untuk menerima masukan gambar, yaitu melalui kamera, galeri, dan \emph{sharing intent}. TrackMyBills menggunakan \emph{package} \texttt{camera} untuk mengakses kamera perangkat, memungkinkan pengguna untuk mengambil foto bukti pembayaran secara langsung. Untuk bisa menggunakan fitur ini, pengguna harus memberikan izin akses kamera pada aplikasi. Selain menggunakan kamera, TrackMyBills juga menyediakan opsi untuk memilih gambar dari galeri perangkat. Pengguna dapat memilih gambar yang telah disimpan sebelumnya di galeri mereka. Fitur ini diimplementasikan dengan menggunakan \emph{package} \texttt{image\_picker} untuk memilih gambar dari galeri. Fitur ini memerlukan izin akses penyimpanan. TrackMyBills juga mendukung \emph{sharing intent} yang memungkinkan pengguna untuk langsung membagikan tangkapan layar atau gambar bukti pembayaran dari aplikasi lain ke TrackMyBills. Fitur ini memanfaatkan \emph{intent} Android untuk menerima gambar yang dibagikan oleh aplikasi lain. Fitur ini menggunakan \emph{package} \texttt{receive\_sharing\_intent} untuk menangani gambar yang dibagikan.

\subsubsection{Pemrosesan Gambar dan API \emph{Communication}}
\label{subsubsec:pemrosesan-gambar}

Pemrosesan gambar dilakukan dengan menggunakan \emph{package} \texttt{crop\_your\_image} yang menyediakan antarmuka pengguna untuk memotong gambar sebelum diunggah. Pengguna dapat memilih area yang relevan dari gambar bukti pembayaran untuk memastikan bahwa hanya informasi penting yang dikirim ke layanan \emph{backend}. 

Setelah gambar dipotong, TrackMyBills mengirimkan gambar tersebut ke layanan \emph{backend} menggunakan \emph{package} \texttt{http}. TrackMyBills melakukan \emph{request} HTTP POST ke \emph{endpoint} URL API layanan \emph{backend}. Gambar yang diunggah akan dikirim dalam format \emph{multipart/form-data} yang memungkinkan pengiriman \emph{file} bersama dengan data lainnya. Pemanggilan API mengimplementasikan mekanisme \emph{retry} untuk menangani kegagalan jaringan dengan \emph{package} \texttt{retry}. Jika terjadi kegagalan, TrackMyBills akan mencoba mengirim ulang permintaan hingga tiga kali sebelum akhirnya memunculkan \emph{dummy data} saat upaya gagal. Respons yang diterima dari layanan \emph{backend} berupa data terstruktur dalam format JSON yang berisi informasi yang diekstrak dari gambar bukti pembayaran.

\subsubsection{Manajemen Data dan Penyimpanan}
\label{subsubsec:manajemen-data-dan-penyimpanan}

TrackMyBills akan memetakan seluruh data yang diterima dari layanan \emph{backend} ke dalam model transaksi tergantung pada jenis bukti pembayaran terkait yang diunggah. Data yang sudah dipetakan ditampilkan kepada pengguna untuk dikonfirmasi. Pengguna memiliki kebebasan untuk mengubah data tersebut jika terdapat kesalahan atau ketidakakuratan. Pengguna dapat menekan tombol "Save Transaction" untuk mengonfirmasi dan menyimpan data transaksi yang telah dipetakan. Data transaksi yang telah tersimpan akan ditampilkan pada halaman pengeluaran dan diklasifikasikan berdasarkan kategori yang telah dipilih oleh pengguna. TrackMyBills menyediakan fitur untuk menampilkan total pengeluaran pengguna dan total pengeluaran per kategori, sehingga pengguna dapat dengan mudah melacak dan mengelola pengeluaran mereka.

TrackMyBills menyimpan data pengeluaran pengguna secara lokal menggunakan \emph{package} \texttt{shared\_preferences}. Data yang disimpan mencakup seluruh informasi terkait dalam bentuk JSON. Penyimpanan ini memungkinkan pengguna untuk mengakses data mereka secara luring tanpa perlu terhubung ke internet dan mengimplementasikan \emph{data persistence} secara lokal tanpa perlu khawatir kehilangan data. SharedPreference menyimpan data secara lokal dalam gawai pengguna dan pada lokasi yang hanya dapat diakses oleh aplikasi terkait. Hal ini memastikan bahwa data pengguna tetap aman dan tidak dapat diakses oleh aplikasi lain.



\subsection{Integrasi Sistem Keseluruhan}
\label{subsec:integrasi-sistem-keseluruhan}

Integrasi sistem keseluruhan merepresentasikan implementasi end-to-end dari semua komponen yang telah dikembangkan, menciptakan ecosystem yang cohesive untuk ekstraksi data pembayaran. Integrasi ini menggabungkan aplikasi mobile, backend service, dan model inference dalam satu workflow yang seamless.

% \begin{figure}[htbp]
%     \centering
%     \includegraphics[width=\textwidth]{images/complete-system-integration.png}
%     \caption{Integrasi sistem keseluruhan dengan alur data end-to-end}
%     \label{fig:complete-system-integration}
% \end{figure}

Alur integrasi dimulai dari user interaction pada aplikasi mobile dan berakhir dengan structured data yang siap untuk analysis atau storage. Setiap step dalam alur ini telah dioptimalkan untuk memberikan user experience yang optimal sambil maintaining data accuracy dan system reliability.

\emph{Data flow} mengikuti pattern yang consistent: image capture/sharing → preprocessing → API transmission → model selection → inference → response formatting → result display → local storage. Setiap transition antar komponen mengimplementasikan proper error handling dan fallback mechanisms yang memastikan system robustness.

\emph{Automatic model selection} merupakan salah satu achievement utama dari integrasi ini. System dapat secara intelligent menentukan model yang appropriate berdasarkan image characteristics tanpa requiring user input. Logic selection menganalisis visual patterns, image dimensions, dan content characteristics untuk routing ke model yang optimal.

% \begin{figure}[htbp]
%     \centering
%     \includegraphics[width=0.9\textwidth]{images/system-data-flow.png}
%     \caption{Alur data dan komunikasi antar komponen sistem}
%     \label{fig:system-data-flow}
% \end{figure}

\emph{Performance optimization} pada level system mencakup connection pooling untuk API communications, efficient memory management yang prevent memory leaks, dan caching strategies yang reduce redundant processing. System dirancang untuk handle concurrent requests dengan graceful degradation ketika resource constraints encountered.

Quality assurance mechanisms terintegrasi pada multiple levels. Image validation pada mobile app level memastikan quality minimal sebelum transmission. API validation memverifikasi data integrity dan format compliance. Model confidence scoring memberikan quality indicators yang dapat digunakan untuk automated quality control atau user feedback.

\emph{Error recovery strategies} mengimplementasikan multi-tier fallback approach. Jika primary model gagal, system automatically attempts dengan alternative model. Jika API communication fails, aplikasi mobile menyimpan request untuk retry ketika connection restored. Jika model inference menghasilkan low confidence results, system dapat prompt user untuk image retaking atau manual verification.

\emph{Monitoring dan logging} terintegrasi across all components memberikan comprehensive visibility into system performance. Request tracing menggunakan unique identifiers yang dapat ditrack dari mobile app hingga model inference dan back. Performance metrics collection memungkinkan identification dari bottlenecks dan optimization opportunities.

% \begin{figure}[htbp]
%     \centering
%     \includegraphics[width=0.8\textwidth]{images/system-monitoring-dashboard.png}
%     \caption{Dashboard monitoring sistem terintegrasi}
%     \label{fig:system-monitoring-dashboard}
% \end{figure}

\emph{Scalability considerations} dalam integrasi mencakup stateless design yang memungkinkan horizontal scaling, efficient resource utilization yang minimize computational overhead, dan modular architecture yang facilitate addition dari new models atau features tanpa major system changes.

Security implementation mencakup end-to-end encryption untuk data transmission, secure API authentication, dan proper data sanitization pada all entry points. Privacy protection ensures bahwa user data tidak retained longer than necessary dan proper anonymization untuk analytics purposes.

\emph{Deployment strategy} menggunakan containerized approach dengan Docker yang memungkinkan consistent deployment across different environments. Configuration management memungkinkan easy adjustment dari system parameters tanpa code changes. Health checking dan automatic restart capabilities ensure high availability dalam production deployment.

User feedback integration memungkinkan continuous improvement dari system accuracy. Feedback mechanism memungkinkan users untuk report incorrect extractions atau suggest improvements, creating feedback loop yang dapat digunakan untuk future model improvements atau system enhancements.

\emph{Backward compatibility} considerations memastikan bahwa future updates tidak break existing functionality. API versioning strategy memungkinkan smooth migration ketika new features added atau existing features enhanced. Database migration scripts memastikan data integrity ketika schema changes required.

System testing strategy mencakup unit testing untuk individual components, integration testing untuk component interactions, end-to-end testing untuk complete user workflows, dan performance testing untuk load handling capabilities. Automated testing pipeline memastikan bahwa system changes tidak introduce regressions atau performance degradations.

Hasil integrasi adalah system yang not only technically sound tetapi juga delivers real value kepada target users. Gen Z users dapat dengan mudah digitize payment information dengan minimal friction, enabling better financial tracking dan budgeting habits yang crucial untuk financial literacy development.
