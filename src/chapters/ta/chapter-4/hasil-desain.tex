\section{Hasil Desain}
\label{sec:hasil-desain}

Hasil desain adalah subbab yang menjelaskan mengenai artefak yang dihasilkan oleh sistem. Subbab ini sejalan dengan tahap \emph{Design and Development} pada metodologi \dsrm{} yang digunakan sebagai metodologi dalam tugas akhir ini. Artefak sistem yang dijelaskan pada subbab ini merupakan implementasi dari tahapan desain yang telah dirancang pada \autoref{sec:tahapan-desain}. 

Sistem ini memiliki tiga artefak utama yang saling terintegrasi untuk memberikan solusi \emph{end-to-end} dalam pengembangan sistem pencatatan pengeluaran berbasis \emph{mobile}. Artefak-artefak tersebut meliputi model \donut{} yang telah disesuaikan untuk domain pembayaran Indonesia, layanan \emph{backend} (DonutAPI) yang menyediakan inferensi model melalui REST API, dan aplikasi \emph{mobile} TrackMyBills yang menjadi antarmuka pengguna.

\subsection{Model yang Digunakan}
\label{subsec:model-yang-digunakan}

Sistem TrackMyBills menggunakan pendekatan dual-model yang menggunakan arsitektur model \donut{} yang telah disesuaikan dengan karakteristik dokumen pembayaran Indonesia. Strategi dual-model dipilih untuk mengoptimalkan akurasi ekstraksi data dengan mempertimbangkan perbedaan signifikan antara dokumen pembayaran digital dan struk berbasis kertas dalam hal layout, struktur informasi, dan kualitas visual.

Kedua model yang diimplementasikan menggunakan arsitektur transformer yang menggabungkan vision encoder dan text decoder untuk memahami dokumen secara holistik. Vision encoder menggunakan Swin Transformer untuk memproses informasi visual dokumen, sementara text decoder menggunakan BART untuk menghasilkan output terstruktur dalam format JSON. Kombinasi ini memungkinkan model untuk memahami tidak hanya teks yang terbaca, tetapi juga layout dan konteks visual dokumen secara keseluruhan.

**Diagram yang disarankan**: Arsitektur dual-model dengan visualisasi jalur inference untuk setiap jenis dokumen, menunjukkan preprocessing, vision encoding, text decoding, dan postprocessing untuk masing-masing model.

\subsubsection{Model DONUT Custom untuk Pembayaran Digital}
\label{subsubsec:model-custom}

Model custom merupakan hasil fine-tuning dari model DONUT base yang telah dioptimalkan khusus untuk dokumen pembayaran digital Indonesia, termasuk QRIS dan transfer bank. Model ini dilatih menggunakan dataset yang dikumpulkan dari berbagai platform pembayaran populer di Indonesia, mencakup variasi layout dan format yang beragam.

Konfigurasi model custom menggunakan task prompt khusus \texttt{<s\_payment\_proof>} yang memberikan konteks spesifik untuk jenis dokumen yang akan diproses. Model ini memiliki vocabulary yang diperluas dengan 14 special tokens yang disesuaikan dengan domain pembayaran Indonesia:
\begin{enumerate}
    \item \texttt{<s\_total\_amount>} dan \texttt{</s\_total\_amount>} untuk nominal pembayaran
    \item \texttt{<s\_transaction\_time>} dan \texttt{</s\_transaction\_time>} untuk waktu transaksi
    \item \texttt{<s\_transaction\_identifier>} dan \texttt{</s\_transaction\_identifier>} untuk ID transaksi
    \item \texttt{<s\_type>} dan \texttt{</s\_type>} untuk jenis transaksi (QRIS atau transfer)
    \item \texttt{<s\_target\_name>} dan \texttt{</s\_target\_name>} untuk nama penerima
    \item \texttt{<s\_application>} dan \texttt{</s\_application>} untuk aplikasi pembayaran
    \item \texttt{<s\_payment\_proof>} dan \texttt{</s\_payment\_proof>} untuk markup dokumen
\end{enumerate}

Model dilatih dengan konfigurasi generation yang optimal menggunakan beam search dengan 4 beams, repetition penalty 1.2, dan maximum length 512 tokens. Proses fine-tuning dilakukan selama 30 epoch dengan early stopping yang secara otomatis menghentikan training pada epoch ke-27 ketika validation loss mencapai konvergensi optimal.

Implementasi model custom mencakup robust text cleaning dan error handling yang dapat menangani berbagai masalah umum dalam OCR hasil, termasuk encoding issues, malformed token sequences, dan incomplete extractions. Sistem juga mengimplementasikan multiple generation attempts dengan fallback configurations untuk memastikan reliable inference bahkan pada dokumen dengan kualitas suboptimal.

\subsubsection{Model DONUT Base untuk Struk Kertas}
\label{subsubsec:model-base}

Model base menggunakan DONUT pre-trained model \texttt{naver-clova-ix/donut-base-finetuned-cord-v2} yang telah dioptimalkan untuk pemrosesan receipt dan dokumen semi-terstruktur. Model ini dipilih karena sudah memiliki pemahaman yang baik terhadap struktur receipt umum dan dapat langsung digunakan untuk dokumen struk pembayaran tanpa fine-tuning tambahan.

Konfigurasi model base menggunakan task prompt \texttt{<s\_cord-v2>} yang mengikuti format standar CORD dataset. Model ini mampu mengekstraksi informasi kompleks dari struk pembayaran, termasuk:
\begin{itemize}
    \item Informasi merchant dan alamat
    \item Itemized list dengan nama, quantity, dan harga individual
    \item Subtotal, pajak, dan total pembayaran
    \item Metadata transaksi seperti tanggal dan waktu
\end{itemize}

Model base mengimplementasikan preprocessing yang robust untuk optimasi ukuran gambar dengan maximum dimension 2048 pixels dan automatic resizing menggunakan Lanczos resampling untuk mempertahankan kualitas visual. Generation configuration menggunakan single beam search dengan early stopping untuk optimasi kecepatan inference sambil mempertahankan akurasi yang tinggi.

Output model base berupa struktur JSON yang kompleks dan hierarkis, sesuai dengan format CORD dataset yang mencerminkan structure receipt yang lebih detailed dibandingkan format simplified untuk payment proof. Model ini juga menggunakan token-to-JSON converter yang robust untuk menghandle various format receipt dari different merchants.

% **Diagram yang disarankan**: Comparison matrix antara kedua model menunjukkan input types, special tokens, output structure, dan use cases masing-masing model.

Kedua model menggunakan automatic device detection yang dapat beradaptasi dengan hardware yang tersedia, baik CPU maupun GPU inference. Implementasi singleton pattern memastikan model hanya dimuat sekali dalam memory untuk mengoptimalkan resource usage dan mengurangi latency pada request subsequent.


\subsection{Backend Service (DonutAPI)}
\label{subsec:backend-service-donutapi}

DonutAPI merupakan implementasi \emph{backend service} yang menyediakan layanan ekstraksi data pembayaran melalui RESTful API. Service ini dirancang menggunakan framework FastAPI untuk memberikan performa tinggi, dokumentasi otomatis, dan type safety yang memudahkan integration dan maintenance.

% \begin{figure}[htbp]
%     \centering
%     \includegraphics[width=0.9\textwidth]{images/donutapi-architecture.png}
%     \caption{Arsitektur DonutAPI dengan komponen utama}
%     \label{fig:donutapi-architecture}
% \end{figure}

Arsitektur DonutAPI menggunakan pattern layered architecture yang memisahkan concern antara API layer, business logic layer, dan model inference layer. Pemisahan ini memungkinkan maintainability yang baik dan facilitates future enhancements tanpa mempengaruhi komponen lain.

API layer mengimplementasikan dua endpoint utama: \texttt{/predict/} untuk full prediction dengan detailed metadata, dan \texttt{/predict/simple} untuk simplified response yang focus pada extracted data saja. Kedua endpoint support parameter \texttt{model} yang memungkinkan client untuk specify model mana yang akan digunakan (base atau custom), memberikan flexibility dalam processing different document types.

\emph{Request handling} mengimplementasikan comprehensive validation yang mencakup file format validation (JPG, JPEG, PNG, WebP, BMP), file size limits (maksimum 10MB), dan image dimension constraints (maksimum 2048x2048 pixels). Validation ini memastikan bahwa hanya input yang valid yang diproses oleh model, reducing computational waste dan improving system reliability.

% \begin{figure}[htbp]
%     \centering
%     \includegraphics[width=0.8\textwidth]{images/api-request-flow.png}
%     \caption{Alur pemrosesan request dalam DonutAPI}
%     \label{fig:api-request-flow}
% \end{figure}

Model inference layer mengimplementasikan \emph{singleton pattern} untuk kedua model (QRIS-TF dan CORD-v2) untuk mencegah multiple instances yang dapat menyebabkan memory issues. Model loading menggunakan lazy initialization yang hanya memuat model ketika pertama kali dibutuhkan, optimizing startup time dan memory usage.

Service mengimplementasikan automatic model selection logic yang menganalisis karakteristik input image untuk menentukan model yang paling appropriate. Logic ini menggunakan simple heuristics berdasarkan image properties dan dapat with high accuracy menentukan apakah document adalah digital payment proof atau paper receipt.

Error handling dan response formatting mengikuti standard industri dengan HTTP status codes yang appropriate dan error messages yang informative. Service mengimplementasikan graceful degradation di mana jika primary model gagal, system secara otomatis fallback ke alternative processing atau memberikan informative error message kepada client.

% \begin{figure}[htbp]
%     \centering
%     \includegraphics[width=0.7\textwidth]{images/api-response-format.png}
%     \caption{Format respons API dengan metadata lengkap}
%     \label{fig:api-response-format}
% \end{figure}

Response formatting menggunakan Pydantic schemas untuk memastikan type safety dan consistent data structure. Full response format mencakup extracted data, confidence scores, processing metadata, dan model information. Simple response format hanya mengembalikan cleaned extracted data yang siap untuk consumption oleh mobile application.

Service juga mengimplementasikan health check endpoints (\texttt{/health} dan \texttt{/health/detailed}) yang memungkinkan monitoring system status dan model availability. Endpoints ini crucial untuk deployment dalam production environment dan integration dengan load balancers atau orchestration systems.

Logging system menggunakan structured JSON logging dengan request tracing yang memungkinkan debugging dan monitoring yang effective. Setiap request diberikan unique ID yang dapat ditrack across all log entries, facilitating troubleshooting dan performance analysis.

Security features mencakup CORS configuration untuk cross-origin requests, optional API key authentication, dan input sanitization untuk mencegah malicious uploads. File validation tidak hanya check format dan size, tetapi juga verify content integrity untuk memastikan bahwa uploaded files adalah valid images.

Deployment configuration menggunakan Docker containerization yang memungkinkan consistent deployment across different environments. Container includes all dependencies dan optimized untuk both CPU dan GPU inference, dengan automatic device detection yang menggunakan GPU jika available dan fallback ke CPU processing jika diperlukan.


\subsection{Aplikasi Mobile TrackMyBills}
\label{subsec:aplikasi-mobile-trackmybills}

TrackMyBills merupakan aplikasi \emph{mobile} berbasis Flutter. Aplikasi ini berfungsi sebagai antarmuka pengguna untuk sistem ekstraksi data pembayaran dengan integrasi fungsi kamera, akses galeri, \emph{sharing intent}, dan pemanggilan API. TrackMyBills dibangun dengan menggunakan pola \emph{Model-View-Controller} (MVC) yang memisahkan antarmuka, model yang digunakan, dan logika fitur. Flutter memiliki \emph{built-in state management} melalui \emph{widget}-nya, yaitu StatefulWidget untuk memudahkan pengelolaan \emph{state} dari sistem. 

% Design system aplikasi mengadopsi Material Design 3 principles dengan color scheme yang vibrant dan typography yang readable untuk mencerminkan aesthetic preferences Gen Z. Interface menggunakan card-based layouts dengan rounded corners dan subtle shadows yang memberikan modern appearance sambil maintaining functional clarity untuk financial data display.

% **Diagram yang disarankan**: App architecture diagram menunjukkan MVC pattern implementation, state management flow, dan component interactions dalam aplikasi mobile.

\subsubsection{Metode Masukan Gambar}
\label{subsubsec:metode-masukan-gambar}

TrackMyBills menyediakan tiga metode untuk menerima masukan gambar, yaitu melalui kamera, galeri, dan \emph{sharing intent}. 

Camera capture menggunakan native Android camera API dengan custom UI overlay yang memberikan guidance visual untuk optimal document framing. Feature ini mengimplementasikan automatic focus dan flash control dengan preview functionality yang memungkinkan pengguna untuk review foto sebelum processing. Camera interface juga menyediakan grid lines dan document detection hints untuk membantu pengguna mengambil foto dengan kualitas optimal.

Gallery selection mengintegrasikan dengan Android's native file picker untuk memilih existing images dari device storage. Feature ini support multiple file formats yang compatible dengan backend service dan mengimplementasikan automatic file validation sebelum upload. Image preview dengan zoom functionality memungkinkan pengguna untuk verify quality dan content sebelum submitting untuk processing.

Android sharing integration merupakan unique feature yang specifically designed untuk Gen Z behavior patterns. Aplikasi register sebagai sharing target untuk image MIME types, memungkinkan pengguna untuk directly share screenshots dari payment apps seperti Gopay, BCA Mobile, atau SeaBank ke TrackMyBills. Integration ini menggunakan Intent filters dan custom sharing UI yang seamlessly handle incoming shared content.

\subsubsection{Image Processing dan API Communication}
\label{subsubsec:image-processing-api}

Image processing workflow mengimplementasikan automatic optimization untuk efficient transmission ke backend API. Pre-upload processing mencakup image compression menggunakan adaptive quality settings yang balance antara file size dan visual quality. Compression algorithm mempertimbangkan original image dimensions dan content complexity untuk determine optimal compression ratio.

Image cropping functionality menggunakan UCrop library yang menyediakan intuitive cropping interface dengan predefined aspect ratios untuk different document types. Cropping feature particularly useful untuk paper receipts yang mungkin memerlukan framing adjustment untuk optimal OCR results. Real-time preview dengan zoom dan pan capabilities memudahkan precision cropping untuk various document sizes.

API communication menggunakan HTTP client dengan comprehensive retry mechanism dan exponential backoff untuk handling network issues. Request timeout configuration menggunakan adaptive values berdasarkan file size dan connection quality. Upload progress indicators menggunakan stream-based approach yang memberikan real-time feedback kepada pengguna selama transmission process.

Error handling mengimplementasikan fallback mechanism yang menampilkan mock data ketika backend API tidak tersedia. Strategi ini memastikan aplikasi tetap functional bahkan dalam kondisi network connectivity yang tidak optimal, memberikan user experience yang konsisten untuk pengguna Gen Z yang mengharapkan aplikasi selalu responsive.

\subsubsection{Data Management dan Storage}
\label{subsubsec:data-management}

Data persistence menggunakan SharedPreferences untuk lightweight local storage yang appropriate untuk financial transaction data. Storage architecture mengimplementasikan encrypted data containers untuk sensitive information seperti transaction amounts dan merchant names. Encryption menggunakan platform-native security features dengan automatic key management.

% Transaction management mengimplementasikan tiga method utama: addTransaction() untuk menambah transaksi baru dengan automatic state notification, loadTransactions() untuk loading data dari storage dengan comprehensive error handling, dan _saveTransactions() untuk persistence dalam format JSON. Receipt type preference management menggunakan loadReceiptTypePreference() dan saveReceiptTypePreference() untuk user customization.

Transaction history organization menggunakan structured data format yang enable efficient querying dan sorting berdasarkan various criteria seperti date, amount, atau transaction type. Data model support untuk both digital payment extractions dan receipt-based transactions dengan flexible schema yang dapat accommodate different data structures.

Automatic categorization menggunakan rule-based logic yang analyze extracted merchant names dan transaction patterns untuk suggest appropriate expense categories. Category suggestions dapat di-customize oleh pengguna dan learning mechanism dapat improve suggestion accuracy over time berdasarkan user preferences.

Local data backup dan restore functionality memungkinkan pengguna untuk export transaction history dalam standard formats seperti CSV atau JSON. Import capabilities mendukung data migration dari other expense tracking applications atau manual transaction logs.

\subsubsection{User Interface dan Experience Features}
\label{subsubsec:ui-ux-features}

Results display mengimplementasikan structured card layouts yang menampilkan extracted information dalam easily scannable format. Editable fields memungkinkan pengguna untuk correction atau addition information yang mungkin not captured perfectly oleh OCR process. Confidence indicators menggunakan visual cues seperti color coding atau icon badges untuk help users understand extraction reliability.

Expense tracking dashboard menyediakan overview dari spending patterns dengan basic analytics yang relevant untuk Gen Z financial awareness. Charts dan visualizations menggunakan simple bar charts dan pie charts yang easy to understand tanpa overwhelming complexity. Spending summaries available dalam different time periods seperti weekly, monthly, atau custom date ranges.

Search dan filter functionality memungkinkan quick access ke specific transactions berdasarkan various criteria. Search implementation menggunakan fuzzy matching untuk handle typos atau partial merchant names. Filter options mencakup date ranges, amount ranges, transaction types, dan custom categories.

Settings dan preferences screen menyediakan customization options untuk notification preferences, default categories, currency formatting, dan data retention policies. User profile management minimal sesuai dengan local-first approach yang tidak memerlukan user accounts atau cloud synchronization.

**Diagram yang disarankan**: User journey flowchart menunjukkan different paths dari image capture hingga transaction storage, termasuk error handling dan retry mechanisms.

Performance optimizations mencakup efficient image handling dengan automatic memory management, background processing untuk non-critical tasks, dan caching strategies untuk frequently accessed data. Aplikasi dioptimalkan untuk smooth performance pada mid-range Android devices yang commonly used oleh target demographic dengan memory footprint yang minimal dan battery usage yang efficient.


\subsection{Alur Kerja Sistem}
\label{subsec:alur-kerja-sistem}

Alur kerja sistem TrackMyBills dirancang untuk mengakomodasi berbagai skenario penggunaan yang mencerminkan behavior patterns pengguna Gen Z dalam melakukan transaksi digital. Sistem mengimplementasikan lima flow utama yang memberikan flexibility maksimal dalam capture, processing, dan storage data transaksi. Setiap flow dioptimalkan untuk specific use cases sambil mempertahankan consistency dalam user experience dan data quality.

Integration antara mobile application dan backend service menggunakan RESTful API dengan robust error handling dan automatic retry mechanisms. Workflow management memastikan bahwa setiap step dalam processing pipeline dapat handle various edge cases dan provide meaningful feedback kepada pengguna. System juga mengimplementasikan offline capability untuk scenarios di mana network connectivity tidak optimal.

**Diagram yang disarankan**: System workflow diagram menunjukkan kelima flow utama dengan decision points, error handling paths, dan data flow antara mobile app dan backend service.

\subsubsection{Flow 1: Sharing dan Ekstraksi dari Aplikasi Pembayaran}
\label{subsubsec:flow-sharing}

Flow sharing merupakan primary use case yang dirancang khusus untuk mengakomodasi habit Gen Z yang frequently menggunakan multiple payment applications. Ketika pengguna melakukan pembayaran menggunakan aplikasi seperti Gopay, BCA Mobile, atau SeaBank, mereka dapat langsung share screenshot confirmation ke TrackMyBills melalui Android's native sharing mechanism.

Proses dimulai ketika pengguna tap share button pada payment confirmation screen di aplikasi pembayaran dan memilih TrackMyBills dari sharing options. Aplikasi TrackMyBills secara otomatis menerima shared image dan melakukan preprocessing untuk memverifikasi bahwa content merupakan valid payment document. Image preprocessing mencakup automatic rotation correction, quality assessment, dan format normalization sebelum dikirim ke backend service.

Backend service menerima image dan melakukan automatic model selection berdasarkan image characteristics. Untuk digital payment screenshots, system secara otomatis menggunakan custom DONUT model yang telah dioptimalkan untuk payment proof extraction. Model melakukan inference dan mengekstraksi key information seperti transaction amount, timestamp, merchant name, transaction ID, dan payment application. Extracted data kemudian dikirim kembali ke mobile application dalam structured format yang ready untuk storage dan display.

Mobile application menerima response dari backend dan melakukan data validation serta cleaning untuk memastikan consistency dengan local data format. User dapat review extracted information dan melakukan manual corrections jika diperlukan sebelum menyimpan transaction record ke local storage. Automatic categorization suggestions ditampilkan berdasarkan merchant name dan transaction pattern untuk memudahkan expense tracking.

Workflow ini mengimplementasikan comprehensive error handling untuk scenarios seperti invalid image format, network connectivity issues, atau extraction failures. Fallback mechanisms memungkinkan user untuk manual input atau retry dengan different processing options jika automatic extraction tidak successful.

\subsubsection{Flow 2: Camera Capture untuk Struk Kertas}
\label{subsubsec:flow-camera-receipt}

Flow camera capture untuk struk kertas dirancang untuk handling physical receipts yang masih common dalam retail transactions di Indonesia. User menggunakan in-app camera functionality yang dilengkapi dengan document detection guidelines dan automatic focus optimization untuk memastikan capture quality yang optimal.

Camera interface menyediakan visual aids berupa grid lines dan document boundary detection yang membantu user melakukan framing yang proper untuk paper receipts. Real-time preview dengan exposure control memungkinkan user untuk adjust lighting conditions sebelum capture. Automatic flash detection menggunakan ambient light sensors untuk determine optimal flash settings tanpa overexposing text content pada receipt.

Setelah photo capture, aplikasi melakukan immediate quality assessment menggunakan blur detection dan contrast analysis. Jika image quality tidak memenuhi threshold minimum, user akan diberi option untuk retake photo dengan suggestions untuk improvement. Preprocessing pada mobile app mencakup automatic cropping, perspective correction, dan contrast enhancement untuk optimizing OCR accuracy.

Backend service menerima processed image dan menggunakan base DONUT model yang telah pre-trained untuk receipt processing. Model mengekstraksi comprehensive information termasuk merchant details, itemized purchases dengan quantities dan prices, subtotals, tax information, dan total amount. Complex receipt structures dengan multiple items dan promotions dapat di-handle dengan robust parsing logic.

Response processing pada mobile app mengorganisasi extracted data menjadi structured transaction record dengan itemized breakdown yang dapat di-expand untuk detailed view. User dapat edit individual line items, add missing information, atau correct OCR errors sebelum finalizing transaction record. Automatic merchant categorization menggunakan business type detection berdasarkan extracted merchant information dan item categories.

\subsubsection{Flow 3: Upload Screenshot atau Captured Struk}
\label{subsubsec:flow-upload-receipt}

Flow upload untuk existing images memberikan flexibility bagi user yang sudah memiliki saved photos dari receipts atau payment confirmations di device storage. User dapat access gallery picker yang terintegrasi dengan Android's file system untuk memilih images dari various sources termasuk photo gallery, downloads folder, atau cloud storage applications.

File selection interface mengimplementasikan smart filtering yang automatically highlight images yang likely merupakan payment documents berdasarkan metadata analysis dan basic image content recognition. Preview functionality dengan zoom capabilities memungkinkan user untuk verify image content dan quality sebelum submitting untuk processing. Batch selection support memungkinkan processing multiple receipts dalam single session untuk efficiency.

Image validation process pada mobile app mencakup format verification, file size checking, dan basic content analysis untuk ensure suitability untuk OCR processing. Images dengan poor quality atau invalid format akan di-reject dengan specific feedback tentang issues dan suggestions untuk resolution. Pre-upload optimization menggunakan intelligent compression yang preserves text readability sambil minimizing file size.

Similar dengan camera capture flow, backend processing menggunakan base DONUT model untuk receipt analysis dengan comprehensive extraction capabilities. Additional preprocessing steps untuk uploaded images mencakup automatic orientation detection dan correction, noise reduction, dan adaptive thresholding untuk optimize text recognition accuracy pada various image qualities.

Post-processing pada mobile app mencakup duplicate detection logic yang compare newly processed transactions dengan existing records untuk prevent double entries. User dapat choose untuk merge duplicates, keep separate records, atau replace existing entries berdasarkan accuracy comparison. Advanced editing capabilities memungkinkan detailed modification dari extracted data structure.

\subsubsection{Flow 4: Camera Capture untuk QRIS dan Transfer}
\label{subsubsec:flow-camera-digital}

Flow camera capture untuk digital payment confirmations dirancang untuk scenarios di mana user memiliki payment confirmation yang displayed pada another device atau printed payment proof. Camera functionality menggunakan specialized settings yang optimized untuk screen capture dengan reduced glare dan automatic white balance adjustment untuk LCD/OLED displays.

Document detection algorithm dapat recognize payment confirmation layouts dan provide automatic framing suggestions untuk optimal capture area. Screen detection capabilities menggunakan edge detection untuk automatically crop irrelevant portions seperti notification bars atau navigation elements yang tidak relevant untuk payment information. Anti-glare optimization menggunakan polarization techniques dan exposure bracketing untuk minimize reflection issues.

Preprocessing pada mobile app mengimplementasikan screen content enhancement yang specifically tuned untuk digital display characteristics. This includes automatic gamma correction, contrast adjustment, dan text sharpening yang improve OCR accuracy untuk screen-captured content. Color space normalization ensures consistent processing regardless dari source display technology.

Backend processing menggunakan custom DONUT model yang telah fine-tuned untuk Indonesian digital payment formats. Model dapat distinguish antara QRIS dan transfer confirmations dan extract appropriate field sets untuk masing-masing transaction type. Advanced parsing logic handle various layout formats dari different banking applications dengan robust field recognition capabilities.

Response validation pada mobile app mengimplementasikan transaction type verification yang cross-check extracted information dengan expected patterns untuk QRIS versus transfer transactions. Automatic merchant name cleanup menggunakan business directory matching untuk standardize merchant representations. Currency formatting dan amount validation ensure data consistency dengan local financial conventions.

\subsubsection{Flow 5: Upload Screenshot atau Captured QRIS dan Transfer}
\label{subsubsec:flow-upload-digital}

Flow upload untuk digital payment screenshots merupakan alternative untuk flow 1 yang mengakomodasi scenarios di mana sharing functionality tidak available atau user prefer manual selection process. Gallery access menggunakan enhanced filtering yang specifically identify payment-related screenshots berdasarkan filename patterns, metadata analysis, dan thumbnail content recognition.

Smart categorization dalam file picker dapat automatically group images berdasarkan likely content type, memudahkan user untuk quickly locate payment confirmations among large photo collections. Preview interface dengan metadata display menunjukkan image properties, capture timestamp, dan basic content hints untuk membantu selection process. Multi-select capabilities dengan batch processing support memungkinkan efficient handling dari multiple payment confirmations.

Quality assessment untuk uploaded screenshots menggunakan specialized algorithms yang understand digital payment layout characteristics. This includes text clarity verification, layout integrity checking, dan completeness assessment untuk ensure extracted information akan accurate dan complete. Automatic enhancement capabilities dapat improve low-quality screenshots dengan adaptive filtering dan contrast optimization.

Backend inference menggunakan custom model dengan identical processing pipeline seperti flow 1, ensuring consistent extraction quality regardless dari input method. Advanced field validation menggunakan domain-specific rules untuk Indonesian payment systems, including format verification untuk transaction IDs, amount range checking, dan timestamp validation. Cross-platform compatibility handling memastikan accurate processing untuk screenshots dari various banking dan payment applications.

Final data integration pada mobile app mengimplementasikan intelligent deduplication yang dapat recognize same transactions yang mungkin captured atau uploaded multiple times. Smart merging capabilities memungkinkan consolidation dari partial extractions menjadi complete transaction records. Comprehensive audit trail tracking memungkinkan user untuk trace data origin dan modification history untuk transparency dan accuracy verification.
