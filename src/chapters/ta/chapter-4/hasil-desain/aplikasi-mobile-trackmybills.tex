\subsection{Aplikasi Mobile TrackMyBills}
\label{subsec:aplikasi-mobile-trackmybills}

TrackMyBills merupakan aplikasi mobile berbasis Flutter yang dirancang khusus untuk target pengguna Gen Z Indonesia dengan focus pada user experience yang intuitive dan modern. Aplikasi ini berfungsi sebagai frontend interface untuk sistem ekstraksi data pembayaran dengan mengintegrasikan camera functionality, file management, dan API communication dalam satu platform yang cohesive.

Arsitektur aplikasi menggunakan Model-View-Controller (MVC) pattern yang memisahkan business logic dari presentation layer untuk maintainability yang optimal. State management menggunakan provider pattern untuk reactive UI updates dan efficient data flow across different screens. Navigation mengimplementasikan bottom navigation bar dengan consistent design language yang familiar bagi pengguna Android modern.

Design system aplikasi mengadopsi Material Design 3 principles dengan color scheme yang vibrant dan typography yang readable untuk mencerminkan aesthetic preferences Gen Z. Interface menggunakan card-based layouts dengan rounded corners dan subtle shadows yang memberikan modern appearance sambil maintaining functional clarity untuk financial data display.

**Diagram yang disarankan**: App architecture diagram menunjukkan MVC pattern implementation, state management flow, dan component interactions dalam aplikasi mobile.

\subsubsection{Input Methods dan User Interaction}
\label{subsubsec:input-methods}

Aplikasi mengimplementasikan tiga metode input yang memberikan maximum flexibility bagi pengguna dalam melakukan capture dan upload dokumen pembayaran:

Camera capture menggunakan native Android camera API dengan custom UI overlay yang memberikan guidance visual untuk optimal document framing. Feature ini mengimplementasikan automatic focus dan flash control dengan preview functionality yang memungkinkan pengguna untuk review foto sebelum processing. Camera interface juga menyediakan grid lines dan document detection hints untuk membantu pengguna mengambil foto dengan kualitas optimal.

Gallery selection mengintegrasikan dengan Android's native file picker untuk memilih existing images dari device storage. Feature ini support multiple file formats yang compatible dengan backend service dan mengimplementasikan automatic file validation sebelum upload. Image preview dengan zoom functionality memungkinkan pengguna untuk verify quality dan content sebelum submitting untuk processing.

Android sharing integration merupakan unique feature yang specifically designed untuk Gen Z behavior patterns. Aplikasi register sebagai sharing target untuk image MIME types, memungkinkan pengguna untuk directly share screenshots dari payment apps seperti Gopay, BCA Mobile, atau SeaBank ke TrackMyBills. Integration ini menggunakan Intent filters dan custom sharing UI yang seamlessly handle incoming shared content.

\subsubsection{Image Processing dan API Communication}
\label{subsubsec:image-processing-api}

Image processing workflow mengimplementasikan automatic optimization untuk efficient transmission ke backend API. Pre-upload processing mencakup image compression menggunakan adaptive quality settings yang balance antara file size dan visual quality. Compression algorithm mempertimbangkan original image dimensions dan content complexity untuk determine optimal compression ratio.

Image cropping functionality menggunakan UCrop library yang menyediakan intuitive cropping interface dengan predefined aspect ratios untuk different document types. Cropping feature particularly useful untuk paper receipts yang mungkin memerlukan framing adjustment untuk optimal OCR results. Real-time preview dengan zoom dan pan capabilities memudahkan precision cropping untuk various document sizes.

API communication menggunakan HTTP client dengan comprehensive retry mechanism dan exponential backoff untuk handling network issues. Request timeout configuration menggunakan adaptive values berdasarkan file size dan connection quality. Upload progress indicators menggunakan stream-based approach yang memberikan real-time feedback kepada pengguna selama transmission process.

Error handling mengimplementasikan user-friendly error messages dengan actionable suggestions untuk common issues seperti network connectivity, file format, atau server errors. Offline detection memungkinkan aplikasi untuk queue requests ketika network unavailable dan automatic retry ketika connectivity restored.

\subsubsection{Data Management dan Storage}
\label{subsubsec:data-management}

Data persistence menggunakan SharedPreferences untuk lightweight local storage yang appropriate untuk financial transaction data. Storage architecture mengimplementasikan encrypted data containers untuk sensitive information seperti transaction amounts dan merchant names. Encryption menggunakan platform-native security features dengan automatic key management.

Transaction history organization menggunakan structured data format yang enable efficient querying dan sorting berdasarkan various criteria seperti date, amount, atau transaction type. Data model support untuk both digital payment extractions dan receipt-based transactions dengan flexible schema yang dapat accommodate different data structures.

Automatic categorization menggunakan rule-based logic yang analyze extracted merchant names dan transaction patterns untuk suggest appropriate expense categories. Category suggestions dapat di-customize oleh pengguna dan learning mechanism dapat improve suggestion accuracy over time berdasarkan user preferences.

Local data backup dan restore functionality memungkinkan pengguna untuk export transaction history dalam standard formats seperti CSV atau JSON. Import capabilities mendukung data migration dari other expense tracking applications atau manual transaction logs.

\subsubsection{User Interface dan Experience Features}
\label{subsubsec:ui-ux-features}

Results display mengimplementasikan structured card layouts yang menampilkan extracted information dalam easily scannable format. Editable fields memungkinkan pengguna untuk correction atau addition information yang mungkin not captured perfectly oleh OCR process. Confidence indicators menggunakan visual cues seperti color coding atau icon badges untuk help users understand extraction reliability.

Expense tracking dashboard menyediakan overview dari spending patterns dengan basic analytics yang relevant untuk Gen Z financial awareness. Charts dan visualizations menggunakan simple bar charts dan pie charts yang easy to understand tanpa overwhelming complexity. Spending summaries available dalam different time periods seperti weekly, monthly, atau custom date ranges.

Search dan filter functionality memungkinkan quick access ke specific transactions berdasarkan various criteria. Search implementation menggunakan fuzzy matching untuk handle typos atau partial merchant names. Filter options mencakup date ranges, amount ranges, transaction types, dan custom categories.

Settings dan preferences screen menyediakan customization options untuk notification preferences, default categories, currency formatting, dan data retention policies. User profile management minimal sesuai dengan local-first approach yang tidak memerlukan user accounts atau cloud synchronization.

**Diagram yang disarankan**: User journey flowchart menunjukkan different paths dari image capture hingga transaction storage, termasuk error handling dan retry mechanisms.

Performance optimizations mencakup efficient image handling dengan automatic memory management, background processing untuk non-critical tasks, dan caching strategies untuk frequently accessed data. Aplikasi dioptimalkan untuk smooth performance pada mid-range Android devices yang commonly used oleh target demographic dengan memory footprint yang minimal dan battery usage yang efficient.
