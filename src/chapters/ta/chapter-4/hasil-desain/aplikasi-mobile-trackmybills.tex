\subsection{Aplikasi Mobile TrackMyBills}
\label{subsec:aplikasi-mobile-trackmybills}

TrackMyBills merupakan aplikasi \emph{mobile} berbasis Flutter. Aplikasi ini berfungsi sebagai antarmuka pengguna untuk sistem ekstraksi data pembayaran dengan integrasi fungsi kamera, akses galeri, \emph{sharing intent}, dan pemanggilan API. TrackMyBills dibangun dengan menggunakan pola \emph{Model-View-Controller} (MVC) yang memisahkan antarmuka, model yang digunakan, dan logika fitur.

\subsubsection{Metode Masukan Gambar}
\label{subsubsec:metode-masukan-gambar}

TrackMyBills menyediakan tiga metode untuk menerima masukan gambar, yaitu melalui kamera, galeri, dan \emph{sharing intent}. TrackMyBills menggunakan \emph{package} \texttt{camera} untuk mengakses kamera perangkat, memungkinkan pengguna untuk mengambil foto bukti pembayaran secara langsung. Untuk bisa menggunakan fitur ini, pengguna harus memberikan izin akses kamera pada aplikasi. Selain menggunakan kamera, TrackMyBills juga menyediakan opsi untuk memilih gambar dari galeri perangkat. Pengguna dapat memilih gambar yang telah disimpan sebelumnya di galeri mereka. Fitur ini diimplementasikan dengan menggunakan \emph{package} \texttt{image\_picker} untuk memilih gambar dari galeri. Fitur ini memerlukan izin akses penyimpanan. TrackMyBills juga mendukung \emph{sharing intent} yang memungkinkan pengguna untuk langsung membagikan tangkapan layar atau gambar bukti pembayaran dari aplikasi lain ke TrackMyBills. Fitur ini memanfaatkan \emph{intent} Android untuk menerima gambar yang dibagikan oleh aplikasi lain. Fitur ini menggunakan \emph{package} \texttt{receive\_sharing\_intent} untuk menangani gambar yang dibagikan.

\subsubsection{Pemrosesan Gambar dan API \emph{Communication}}
\label{subsubsec:pemrosesan-gambar}

Pemrosesan gambar dilakukan dengan menggunakan \emph{package} \texttt{crop\_your\_image} yang menyediakan antarmuka pengguna untuk memotong gambar sebelum diunggah. Pengguna dapat memilih area yang relevan dari gambar bukti pembayaran untuk memastikan bahwa hanya informasi penting yang dikirim ke layanan \emph{backend}. 

Setelah gambar dipotong, TrackMyBills mengirimkan gambar tersebut ke layanan \emph{backend} menggunakan \emph{package} \texttt{http}. TrackMyBills melakukan \emph{request} HTTP POST ke \emph{endpoint} URL API layanan \emph{backend}. Gambar yang diunggah akan dikirim dalam format \emph{multipart/form-data} yang memungkinkan pengiriman \emph{file} bersama dengan data lainnya. Pemanggilan API mengimplementasikan mekanisme \emph{retry} untuk menangani kegagalan jaringan dengan \emph{package} \texttt{retry}. Jika terjadi kegagalan, TrackMyBills akan mencoba mengirim ulang permintaan hingga tiga kali sebelum akhirnya memunculkan \emph{dummy data} saat upaya gagal. Respons yang diterima dari layanan \emph{backend} berupa data terstruktur dalam format JSON yang berisi informasi yang diekstrak dari gambar bukti pembayaran.

\subsubsection{Manajemen Data dan Penyimpanan}
\label{subsubsec:manajemen-data-dan-penyimpanan}

TrackMyBills akan memetakan seluruh data yang diterima dari layanan \emph{backend} ke dalam model transaksi tergantung pada jenis bukti pembayaran terkait yang diunggah. Data yang sudah dipetakan ditampilkan kepada pengguna untuk dikonfirmasi. Pengguna memiliki kebebasan untuk mengubah data tersebut jika terdapat kesalahan atau ketidakakuratan. Pengguna dapat menekan tombol "Save Transaction" untuk mengonfirmasi dan menyimpan data transaksi yang telah dipetakan. Data transaksi yang telah tersimpan akan ditampilkan pada halaman pengeluaran dan diklasifikasikan berdasarkan kategori yang telah dipilih oleh pengguna. TrackMyBills menyediakan fitur untuk menampilkan total pengeluaran pengguna dan total pengeluaran per kategori sehingga pengguna dapat dengan mudah melacak dan mengelola pengeluaran mereka.

TrackMyBills menyimpan data pengeluaran pengguna secara lokal menggunakan \emph{package} \texttt{shared\_preferences}. Data yang disimpan mencakup seluruh informasi terkait dalam bentuk JSON. Penyimpanan ini memungkinkan pengguna untuk mengakses data mereka secara luring tanpa perlu terhubung ke internet dan mengimplementasikan \emph{data persistence} secara lokal tanpa perlu khawatir kehilangan data. SharedPreference menyimpan data secara lokal dalam gawai pengguna dan pada lokasi yang hanya dapat diakses oleh aplikasi terkait. Hal ini memastikan bahwa data pengguna tetap aman dan tidak dapat diakses oleh aplikasi lain.

