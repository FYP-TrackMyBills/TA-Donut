\subsection{Framework Evaluasi Terpadu}
\label{subsec:framework-evaluasi-terpadu}

Framework evaluasi terpadu merupakan implementasi konkret dari metodologi evaluasi yang dirancang untuk memberikan penilaian komprehensif terhadap kedua model yang telah dikembangkan. Framework ini mengintegrasikan berbagai metrik evaluasi dalam satu sistem yang kohesif dan dapat dieksekusi secara otomatis.

% \begin{figure}[htbp]
%     \centering
%     \includegraphics[width=0.8\textwidth]{images/evaluation-framework-implementation.png}
%     \caption{Implementasi framework evaluasi terpadu}
%     \label{fig:evaluation-framework-implementation}
% \end{figure}

Implementasi framework menggunakan pendekatan modular yang memungkinkan evaluasi terpisah untuk model QRIS-TF dan CORD-v2, namun tetap memberikan hasil yang dapat dibandingkan. Framework dirancang dengan fleksibilitas untuk menangani berbagai format output dan dapat dengan mudah diperluas untuk metrik evaluasi tambahan di masa depan.

Komponen utama framework mencakup \emph{data loader} yang dapat menangani format dataset yang berbeda, \emph{model inference engine} yang support multiple model types, \emph{metrics calculation module} yang mengimplementasikan semua metrik yang diperlukan, dan \emph{reporting system} yang menghasilkan output evaluasi dalam format yang mudah dipahami.

\emph{Metrics calculation module} mengimplementasikan lima metrik utama: Accuracy, Precision, Recall, F1-Score, dan mCER (Mean Character Error Rate). Untuk model QRIS-TF, framework menghitung metrik klasifikasi untuk mengevaluasi kemampuan membedakan antara transaksi QRIS dan transfer, serta metrik ekstraksi untuk mengevaluasi akurasi field extraction. Model CORD-v2 dievaluasi fokus pada metrik ekstraksi dengan emphasis pada field-field kritis seperti total pembayaran dan informasi merchant.

Framework mengimplementasikan sistem weighting yang memberikan prioritas berbeda untuk field-field yang berbeda. Field \texttt{total\_amount} diberikan bobot 3.0x karena kritikalitasnya dalam pencatatan keuangan, field \texttt{type} dan \texttt{target\_name} diberikan bobot 2.0x, sedangkan field lainnya menggunakan bobot standard 1.0x. Sistem weighting ini memastikan bahwa evaluasi mencerminkan importance relatif dari setiap field dalam konteks aplikasi praktis.

% \begin{figure}[htbp]
%     \centering
%     \includegraphics[width=0.9\textwidth]{images/evaluation-metrics-dashboard.png}
%     \caption{Dashboard hasil evaluasi dengan visualisasi metrik}
%     \label{fig:evaluation-metrics-dashboard}
% \end{figure}

Output framework mencakup detailed performance report yang menampilkan metrik aggregate dan per-field analysis. Report juga mencakup confusion matrix untuk komponen klasifikasi, character error analysis untuk detailed understanding dari kesalahan ekstraksi, dan confidence score distribution yang membantu dalam understanding model uncertainty.

Framework juga mengimplementasikan debugging capabilities yang memungkinkan analysis mendalam terhadap sample-sample yang mengalami kesalahan. Feature ini mencakup side-by-side comparison antara ground truth dan prediction, highlighting field-field yang bermasalah, dan analysis error patterns yang dapat membantu dalam improvement model di masa depan.

Untuk memastikan reproducibility, framework mencakup comprehensive logging yang mencatat semua parameter evaluasi, dataset yang digunakan, dan konfigurasi model. Seed management yang konsisten memastikan bahwa hasil evaluasi dapat direproduksi dengan kondisi yang identik. Framework juga support untuk batch evaluation yang memungkinkan comparison performa across different model checkpoints atau konfigurasi.

Implementasi framework menggunakan efficient memory management untuk menangani dataset yang besar tanpa mengalami memory overflow. Parallel processing capabilities memungkinkan evaluasi yang cepat even pada dataset yang substantial. Error handling yang robust memastikan bahwa evaluation process dapat continue bahkan ketika encounter sample yang bermasalah atau model prediction yang malformed.
