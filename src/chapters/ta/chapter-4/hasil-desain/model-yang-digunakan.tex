\subsection{Model yang Digunakan}
\label{subsec:model-yang-digunakan}

Sistem TrackMyBills menggunakan pendekatan dual-model yang menggunakan arsitektur model \donut{} yang telah disesuaikan dengan karakteristik dokumen pembayaran Indonesia. Strategi dual-model dipilih untuk mengoptimalkan akurasi ekstraksi data dengan mempertimbangkan perbedaan signifikan antara dokumen pembayaran digital dan struk berbasis kertas dalam hal layout, struktur informasi, dan kualitas visual.

Kedua model yang diimplementasikan menggunakan arsitektur transformer yang menggabungkan vision encoder dan text decoder untuk memahami dokumen secara holistik. Vision encoder menggunakan Swin Transformer untuk memproses informasi visual dokumen, sementara text decoder menggunakan BART untuk menghasilkan output terstruktur dalam format JSON. Kombinasi ini memungkinkan model untuk memahami tidak hanya teks yang terbaca, tetapi juga layout dan konteks visual dokumen secara keseluruhan.

**Diagram yang disarankan**: Arsitektur dual-model dengan visualisasi jalur inference untuk setiap jenis dokumen, menunjukkan preprocessing, vision encoding, text decoding, dan postprocessing untuk masing-masing model.

\subsubsection{Model DONUT Custom untuk Pembayaran Digital}
\label{subsubsec:model-custom}

Model custom merupakan hasil fine-tuning dari model DONUT base yang telah dioptimalkan khusus untuk dokumen pembayaran digital Indonesia, termasuk QRIS dan transfer bank. Model ini dilatih menggunakan dataset yang dikumpulkan dari berbagai platform pembayaran populer di Indonesia, mencakup variasi layout dan format yang beragam.

Konfigurasi model custom menggunakan task prompt khusus \texttt{<s\_payment\_proof>} yang memberikan konteks spesifik untuk jenis dokumen yang akan diproses. Model ini memiliki vocabulary yang diperluas dengan 14 special tokens yang disesuaikan dengan domain pembayaran Indonesia:
\begin{enumerate}
    \item \texttt{<s\_total\_amount>} dan \texttt{</s\_total\_amount>} untuk nominal pembayaran
    \item \texttt{<s\_transaction\_time>} dan \texttt{</s\_transaction\_time>} untuk waktu transaksi
    \item \texttt{<s\_transaction\_identifier>} dan \texttt{</s\_transaction\_identifier>} untuk ID transaksi
    \item \texttt{<s\_type>} dan \texttt{</s\_type>} untuk jenis transaksi (QRIS atau transfer)
    \item \texttt{<s\_target\_name>} dan \texttt{</s\_target\_name>} untuk nama penerima
    \item \texttt{<s\_application>} dan \texttt{</s\_application>} untuk aplikasi pembayaran
    \item \texttt{<s\_payment\_proof>} dan \texttt{</s\_payment\_proof>} untuk markup dokumen
\end{enumerate}

Model dilatih dengan konfigurasi generation yang optimal menggunakan beam search dengan 4 beams, repetition penalty 1.2, dan maximum length 512 tokens. Proses fine-tuning dilakukan selama 30 epoch dengan early stopping yang secara otomatis menghentikan training pada epoch ke-27 ketika validation loss mencapai konvergensi optimal.

Implementasi model custom mencakup robust text cleaning dan error handling yang dapat menangani berbagai masalah umum dalam OCR hasil, termasuk encoding issues, malformed token sequences, dan incomplete extractions. Sistem juga mengimplementasikan multiple generation attempts dengan fallback configurations untuk memastikan reliable inference bahkan pada dokumen dengan kualitas suboptimal.

\subsubsection{Model DONUT Base untuk Struk Kertas}
\label{subsubsec:model-base}

Model base menggunakan DONUT pre-trained model \texttt{naver-clova-ix/donut-base-finetuned-cord-v2} yang telah dioptimalkan untuk pemrosesan receipt dan dokumen semi-terstruktur. Model ini dipilih karena sudah memiliki pemahaman yang baik terhadap struktur receipt umum dan dapat langsung digunakan untuk dokumen struk pembayaran tanpa fine-tuning tambahan.

Konfigurasi model base menggunakan task prompt \texttt{<s\_cord-v2>} yang mengikuti format standar CORD dataset. Model ini mampu mengekstraksi informasi kompleks dari struk pembayaran, termasuk:
\begin{itemize}
    \item Informasi merchant dan alamat
    \item Itemized list dengan nama, quantity, dan harga individual
    \item Subtotal, pajak, dan total pembayaran
    \item Metadata transaksi seperti tanggal dan waktu
\end{itemize}

Model base mengimplementasikan preprocessing yang robust untuk optimasi ukuran gambar dengan maximum dimension 2048 pixels dan automatic resizing menggunakan Lanczos resampling untuk mempertahankan kualitas visual. Generation configuration menggunakan single beam search dengan early stopping untuk optimasi kecepatan inference sambil mempertahankan akurasi yang tinggi.

Output model base berupa struktur JSON yang kompleks dan hierarkis, sesuai dengan format CORD dataset yang mencerminkan structure receipt yang lebih detailed dibandingkan format simplified untuk payment proof. Model ini juga menggunakan token-to-JSON converter yang robust untuk menghandle various format receipt dari different merchants.

% **Diagram yang disarankan**: Comparison matrix antara kedua model menunjukkan input types, special tokens, output structure, dan use cases masing-masing model.

Kedua model menggunakan automatic device detection yang dapat beradaptasi dengan hardware yang tersedia, baik CPU maupun GPU inference. Implementasi singleton pattern memastikan model hanya dimuat sekali dalam memory untuk mengoptimalkan resource usage dan mengurangi latency pada request subsequent.
