\subsection{Perancangan Evaluasi Model}
\label{subsec:perancangan-evaluasi-model}

Perancangan evaluasi model pada penelitian ini mengadopsi pendekatan evaluasi terpadu (\textit{unified evaluation}) yang dirancang untuk mengukur kinerja model secara komprehensif dan konsisten. Sistem evaluasi ini tidak hanya mengukur akurasi dasar, tetapi juga mempertimbangkan aspek-aspek kritis seperti kualitas karakter yang diekstraksi, cakupan field, dan prioritas bisnis dari setiap field yang diekstraksi. Pendekatan evaluasi terpadu ini memungkinkan perbandingan yang adil antara berbagai model dan memberikan wawasan mendalam tentang kekuatan dan kelemahan masing-masing model dalam konteks aplikasi praktis.

\subsubsection{Arsitektur Evaluasi Terpadu}

Sistem evaluasi terpadu dirancang dengan arsitektur modular yang dapat menangani berbagai jenis tugas pemahaman dokumen secara bersamaan. Arsitektur ini memisahkan logika evaluasi umum dari logika spesifik tugas, sehingga memungkinkan ekstensibilitas untuk tugas-tugas evaluasi baru di masa depan. Komponen utama arsitektur evaluasi meliputi pengatur konfigurasi tugas (\textit{task configuration manager}), pemroses data ground truth (\textit{ground truth processor}), generator prediksi (\textit{prediction generator}), kalkulator metrik (\textit{metrics calculator}), dan agregator hasil (\textit{results aggregator}).

Pengatur konfigurasi tugas bertanggung jawab untuk mengelola parameter evaluasi yang spesifik untuk setiap jenis dokumen. Untuk tugas QRIS-TF, konfigurasi ini mencakup daftar field target seperti \texttt{total\_amount}, \texttt{transaction\_time}, \texttt{transaction\_identifier}, \texttt{type}, \texttt{target\_name}, dan \texttt{application}, serta bobot prioritas untuk setiap field. Pemroses data ground truth menangani normalisasi dan standardisasi format data referensi, memastikan konsistensi dalam perbandingan prediksi dengan ground truth. Generator prediksi mengimplementasikan logika inferensi model dengan parameter yang dioptimalkan untuk setiap tugas, termasuk pengaturan beam search, panjang maksimal sequence, dan strategi decoding.

\subsubsection{Metrik Evaluasi Komprehensif}

Sistem evaluasi mengimplementasikan enam metrik utama yang memberikan gambaran menyeluruh tentang kinerja model. \textit{Accuracy} mengukur persentase field yang diekstraksi dengan benar secara eksak, memberikan indikasi langsung tentang tingkat keberhasilan ekstraksi. \textit{Precision} dan \textit{Recall} dihitung berdasarkan klasifikasi true positive, false positive, dan false negative untuk setiap field, memberikan wawasan tentang kualitas prediksi dan cakupan deteksi. \textit{F1-score} sebagai harmonic mean dari precision dan recall memberikan metrik balanced yang mempertimbangkan kedua aspek tersebut.

\textit{Mean Character Error Rate} (mCER) dihitung sebagai rata-rata edit distance karakter antara prediksi dan ground truth untuk semua field, dinormalisasi terhadap panjang ground truth. Metrik ini sangat penting untuk aplikasi ekstraksi data finansial karena kesalahan satu karakter dalam jumlah uang atau ID transaksi dapat mengakibatkan konsekuensi serius. \textit{Coverage} mengukur persentase field ground truth yang memiliki prediksi yang tidak kosong, memberikan indikasi tentang kemampuan model untuk mengidentifikasi keberadaan informasi dalam dokumen.

\subsubsection{Sistem Pembobotan Field Berdasarkan Prioritas Bisnis}

Implementasi sistem pembobotan field mencerminkan realitas bahwa tidak semua field memiliki tingkat kepentingan yang sama dalam aplikasi praktis. Untuk konteks pembayaran QRIS, field \texttt{total\_amount} diberi bobot tertinggi (3.0) karena kesalahan dalam ekstraksi jumlah pembayaran memiliki dampak langsung terhadap akurasi transaksi keuangan. Field \texttt{type}, \texttt{application}, dan \texttt{target\_name} diberi bobot menengah (2.0) karena penting untuk kategorisasi dan verifikasi transaksi. Field \texttt{transaction\_time} dan \texttt{transaction\_identifier} diberi bobot standar (1.0) karena meskipun berguna untuk audit dan pelacakan, kesalahan pada field ini tidak secara langsung mempengaruhi validitas transaksi.

Sistem pembobotan ini diimplementasikan dalam perhitungan metrik agregat dengan mengalikan kontribusi setiap field terhadap metrik keseluruhan dengan bobot yang sesuai. Pendekatan ini menghasilkan skor evaluasi yang lebih representatif terhadap performa model dalam skenario penggunaan nyata, di mana akurasi field kritis lebih diutamakan daripada field yang kurang kritis.
