\subsubsection{\emph{Scenarios View}}
\label{subsubsec:use-case-view}

\emph{Scenarios View} menunjukkan perilaku sistem dari sudut pandang pengguna melalui pemodelan \emph{use case} dan pemodelan BPMN. \autoref{fig:bpmn-to-be} menunjukkan \emph{Business Process Model and Notation} (BPMN) untuk \emph{use case} utama sistem, yaitu \emph{share-extract-save}. BPMN ini menunjukkan perbedaan alur kerja dari BPMN sistem saat ini (\emph{As-Is}) yang telah ditunjukkan pada \autoref{sec:analisis-kondisi-saat-ini}. Proses manual yang perlu dilakukan pengguna mulai dari mengekstrak data hingga menyimpan hasil pencatatan pengeluaran ke dalam aplikasi telah diotomatisasi pada sistem yang diusulkan.

\begin{figure}[htbp]
    \centering
    \includegraphics[width=1\textwidth]{images/To-be.png}
    \caption{BPMN untuk \emph{use case} utama sistem (\emph{share-extract-save})}
    \label{fig:bpmn-to-be}
\end{figure}

\autoref{fig:use-case-diagram} menunjukkan \emph{use case diagram} sistem pencatatan pengeluaran berbasis \emph{mobile}. Diagram ini menggambarkan interaksi antara pengguna dan sistem pada \emph{use case} yang telah diidentifikasikan dari kebutuhan fungsional. \autoref{tab:use-case-list} menyajikan daftar \emph{use case} yang diidentifikasi dalam sistem beserta kode kebutuhan fungsional yang terkait. 

\emph{Use case} menunjukkan seluruh kasus penggunaan yang dapat dilakukan pengguna terhadap sistem. Kasus-kasus yang didefinisikan tersebut menjadi dasar untuk mengembangkan fitur-fitur dalam sistem untuk memenuhi kebutuhan fungsional dan kebutuhan non-fungsional pengguna. \emph{Scenarios View} ini bertujuan untuk memberikan gambaran yang jelas tentang bagaimana pengguna akan berinteraksi dengan sistem dan bagaimana sistem akan merespons interaksi tersebut.

\begin{figure}[htbp]
    \centering
    \includegraphics[width=.6\textwidth]{images/use-case-diagram.png}
    \caption{\emph{Use case diagram} sistem pencatatan pengeluaran berbasis \emph{mobile}}
    \label{fig:use-case-diagram}
\end{figure}

\begin{table}[h!]
\centering
\caption{Daftar \emph{use case} sistem pencatatan pengeluaran berbasis \emph{mobile}}
\label{tab:use-case-list}
\begin{tabularx}{\textwidth}{|p{1.6cm}|p{1.5cm}|p{2.8cm}|X|}
\hline
\textbf{Kode \emph{Use Case}} & \textbf{Kode FR} & \textbf{\emph{Use Case}} & \textbf{Deskripsi} \\ \hline
UC-01 & FR-01 & Bagikan gambar & Membagi (\emph{share}) gambar ke sistem \\ \hline
UC-02 & FR-01 & Mengambil foto & Mengakses kamera dari aplikasi dan mengambil foto \\ \hline
UC-03 & FR-01 & Unggah gambar & Memberikan izin mengakses foto dan mengunggah gambar dari galeri \\ \hline
UC-04 & FR-02 & Memotong bagian gambar & Memotong bagian gambar yang diperlukan \\ \hline
UC-05 & FR-03 & Ekstrak data & Mengekstrak data dari gambar yang telah diterima aplikasi \\ \hline
UC-06 & FR-04 & Memilih kategori pengeluaran & Memilih kategori dari data pengeluaran yang diekstrak \\ \hline
UC-07 & FR-04 & Mengubah hasil ekstraksi & Mengubah hasil ekstraksi data sebelum data disimpan \\ \hline
UC-08 & FR-05 & Menyimpan hasil ekstraksi & Menyimpan hasil ekstraksi data dari gambar yang diterima \\ \hline
UC-09 & FR-05 & Lihat total pengeluaran & Melihat total pengeluaran dan pengeluaran per kategori \\ \hline
\end{tabularx}
\end{table}