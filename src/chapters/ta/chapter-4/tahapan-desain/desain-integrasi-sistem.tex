\subsection{Desain Integrasi Sistem}
\label{subsec:desain-integrasi-sistem}

Desain integrasi sistem bertujuan untuk menciptakan komunikasi yang seamless antara komponen-komponen sistem, mulai dari aplikasi \textit{mobile} hingga model pemrosesan di \textit{backend}. Integrasi dirancang dengan mempertimbangkan efisiensi, reliabilitas, dan kemudahan penggunaan bagi pengguna Gen Z.

% \begin{figure}[htbp]
%     \centering
%     \includegraphics[width=0.9\textwidth]{images/system-integration-flow.png}
%     \caption{Alur integrasi sistem end-to-end}
%     \label{fig:system-integration-flow}
% \end{figure}

\autoref{fig:system-integration-flow} menunjukkan alur integrasi yang menghubungkan seluruh komponen sistem. Komunikasi antara aplikasi \textit{mobile} dan \textit{backend service} menggunakan protokol HTTP/HTTPS dengan format data \json{} untuk memastikan kompatibilitas dan kemudahan parsing. Desain API mengikuti prinsip \textit{RESTful} dengan \textit{endpoint} yang jelas dan konsisten untuk setiap operasi.

Aplikasi \textit{mobile} dirancang dengan dua mekanisme utama untuk input gambar: pengambilan foto langsung menggunakan kamera dan integrasi dengan sistem \textit{sharing} Android. Sistem \textit{sharing} memungkinkan pengguna untuk membagikan \emph{screenshot} bukti pembayaran dari aplikasi lain secara langsung ke aplikasi TrackMyBills, menciptakan pengalaman pengguna yang natural dan efisien. Mekanisme ini particularly important untuk target pengguna Gen Z yang terbiasa dengan interaksi mobile yang intuitif.

Proses pemilihan model dilakukan secara otomatis di sisi \textit{backend} berdasarkan analisis karakteristik gambar input. Sistem menggunakan heuristik sederhana yang mengidentifikasi pola visual untuk menentukan apakah dokumen merupakan bukti pembayaran digital atau struk berbasis kertas. Pendekatan ini menghilangkan kebutuhan pengguna untuk secara manual memilih jenis dokumen, menyederhanakan \textit{user flow} dan mengurangi potensi kesalahan.

Desain \textit{error handling} dan \textit{fallback mechanism} memastikan sistem tetap responsif bahkan ketika terjadi kegagalan pada salah satu komponen. Jika model utama gagal memproses dokumen, sistem secara otomatis mencoba menggunakan model alternatif atau memberikan respon yang informatif kepada pengguna. Implementasi \textit{timeout} yang sesuai mencegah aplikasi \textit{mobile} menunggu terlalu lama untuk respons dari \textit{backend}.

Keamanan komunikasi dijamin melalui validasi input yang ketat, sanitasi data, dan implementasi \textit{rate limiting} untuk mencegah penyalahgunaan sistem. Format respons dirancang konsisten dengan menyertakan metadata yang membantu aplikasi \textit{mobile} dalam menampilkan hasil ekstraksi dengan cara yang user-friendly. Desain ini memastikan bahwa integrasi sistem tidak hanya berfungsi secara teknis, tetapi juga memberikan pengalaman pengguna yang optimal.
