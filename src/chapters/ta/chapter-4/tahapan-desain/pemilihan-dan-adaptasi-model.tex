\subsection{Pemilihan dan Adaptasi Model}
\label{subsec:pemilihan-adaptasi-model}

Pemilihan model merupakan keputusan strategis yang menentukan keberhasilan sistem ekstraksi data pembayaran. Berdasarkan analisis pemilihan solusi pada \autoref{sec:analisis-pemilihan-solusi}, \donut{} dipilih sebagai model utama karena pendekatan \emph{end-to-end} yang tidak memerlukan \ocr{} dan kemampuannya dalam memahami dokumen semi-terstruktur.

% \begin{figure}[htbp]
%     \centering
%     \includegraphics[width=0.7\textwidth]{images/model-selection-tree.png}
%     \caption{Pohon keputusan pemilihan model berdasarkan jenis dokumen}
%     \label{fig:model-selection-tree}
% \end{figure}

Strategi adaptasi model menggunakan pendekatan \emph{dual-model} yang disesuaikan dengan karakteristik dokumen yang berbeda. \autoref{fig:model-selection-tree} menunjukkan logika pemilihan model berdasarkan jenis dokumen input. Untuk dokumen pembayaran digital (QRIS dan transfer), sistem menggunakan model yang di-\emph{fine-tune} khusus pada \dataset{} QRIS-TF yang mampu melakukan ekstraksi sekaligus klasifikasi jenis pembayaran. Sedangkan untuk struk pembayaran berbasis kertas, sistem menggunakan model yang di-\emph{fine-tune} pada \dataset{} CORD-v2 yang fokus pada ekstraksi informasi tanpa klasifikasi.

Model \textit{QRIS-TF} dirancang untuk menangani kompleksitas dokumen pembayaran digital yang memiliki variasi layout dan format dari berbagai aplikasi pembayaran. Model ini tidak hanya mengekstraksi informasi seperti nominal pembayaran, waktu transaksi, dan identifikator transaksi, tetapi juga mengklasifikasikan jenis transaksi (QRIS atau transfer). Kemampuan klasifikasi terintegrasi ini menghilangkan kebutuhan untuk langkah pemrosesan tambahan dan meningkatkan efisiensi sistem secara keseluruhan.

Model \textit{CORD-v2} dioptimalkan untuk menangani struk pembayaran berbasis kertas yang memiliki karakteristik visual yang berbeda dari dokumen digital. Fokus pada ekstraksi informasi memungkinkan model ini mencapai akurasi tinggi dalam mengenali elemen-elemen penting seperti nama toko, total pembayaran, dan detail item pembelian.

Adaptasi kedua model dilakukan melalui ekspansi kosakata dengan menambahkan \emph{special tokens} yang spesifik untuk domain pembayaran Indonesia. \emph{Special tokens} ini mencakup representasi untuk mata uang rupiah, format tanggal Indonesia, dan terminologi pembayaran lokal. Proses adaptasi juga melibatkan penyesuaian \emph{task prompt} yang mengarahkan model untuk menghasilkan output dalam format yang konsisten dengan kebutuhan sistem.
