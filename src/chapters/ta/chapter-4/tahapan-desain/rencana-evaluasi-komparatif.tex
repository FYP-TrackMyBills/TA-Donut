\subsubsection{Rencana Evaluasi Komparatif}
\label{subsubsec:rencana-evaluasi-komparatif}

Evaluasi komparatif dirancang untuk memberikan insight mendalam mengenai trade-offs antara model size, inference speed, implementation complexity, dan accuracy retention. Evaluasi akan dilakukan menggunakan CORD v2 dataset sebagai benchmark standard untuk memastikan comparability dengan penelitian existing.

\paragraph{Experimental Setup}
\begin{enumerate}
    \item \textbf{Baseline Model}: PyTorch Donut fine-tuned model sebagai reference point
    \item \textbf{Conversion Targets}: ONNX standard, INT8 quantized, UINT8 quantized, FP16 precision
    \item \textbf{Evaluation Dataset}: 100 samples dari CORD v2 validation set untuk quick assessment, dengan opsi full evaluation pada seluruh validation set
    \item \textbf{Hardware Target}: Representative Android devices dengan various computational capabilities
\end{enumerate}

Hasil evaluasi ini akan menjadi dasar untuk pengambilan keputusan deployment strategy yang optimal, dengan mempertimbangkan keseimbangan antara performance, efficiency, dan implementation feasibility untuk context aplikasi mobile receipt scanning.