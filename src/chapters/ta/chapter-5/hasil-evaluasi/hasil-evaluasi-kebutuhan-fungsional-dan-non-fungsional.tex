\subsection{Hasil Evaluasi Kebutuhan Fungsional dan Non-Fungsional}
\label{subsec:hasil-evaluasi-kebutuhan-fungsional-dan-non-fungsional}

Seluruh kebutuhan fungsional telah diuji dengan menggunakan skenario uji yang telah dipersiapkan pada \autoref{subsec:evaluasi-kebutuhan-fungsional-dan-non-fungsional}. \autoref{tab:hasil-uji-fungsionalitas} menunjukkan hasil evaluasi uji fungsionalitas berdasarkan skenario uji yang telah dilakukan. Status pengujian setiap skenario ditunjukkan pada kolom Status Keberhasilan. Hasil evaluasi menunjukkan bahwa seluruh skenario uji fungsionalitas berhasil dijalankan sesuai dengan yang diharapkan.

\begin{table}[h!]
\caption{Hasil uji fungsionalitas}
\label{tab:hasil-uji-fungsionalitas}
\begin{tabularx}{\linewidth}{|p{2cm}|X|p{2.5cm}|}
\hline
\textbf{Kode Skenario} & \textbf{Skenario Pengujian} & \textbf{Status Keberhasilan} \\
\hline
S-01 & Penguji membagikan gambar struk format PNG dari aplikasi galeri. & Berhasil \\
\hline
S-02 & Penguji mencoba membagikan \emph{file} PDF ke aplikasi. & Berhasil \\
\hline
S-03 & Penguji mengambil foto struk dalam kondisi pencahayaan yang baik dan fokus yang jelas. & Berhasil \\
\hline
S-04 & Penguji membuka kamera dari dalam aplikasi dan membatalkannya sebelum mengambil foto. & Berhasil \\
\hline
S-05 & Penguji membuka kamera dari dalam aplikasi dan mencoba menggunakan \emph{flash} untuk mengambil foto. & Berhasil \\
\hline
S-06 & Penguji mengunggah \emph{screenshot} bukti transfer dari galeri. & Berhasil \\
\hline
\end{tabularx}
\end{table}

\begin{table}[h!]
\ContinuedFloat
\caption{Hasil uji fungsionalitas (lanjutan)}
\begin{tabularx}{\linewidth}{|p{2cm}|X|p{2.5cm}|}
\hline
\textbf{Kode Skenario} & \textbf{Skenario Pengujian} & \textbf{Status Keberhasilan} \\
\hline
S-07 & Penguji mengunggah gambar dengan resolusi sangat rendah atau buram. & Berhasil \\
\hline
S-08 & Penguji mengunggah gambar \emph{screenshot} bukti pembayaran QRIS. & Berhasil \\
\hline
S-09 & Penguji memotong gambar untuk menyertakan seluruh area informasi yang relevan. & Berhasil \\
\hline
S-10 & Penguji melanjutkan proses tanpa memotong gambar sama sekali. & Berhasil \\
\hline
S-11 & Penguji menjalankan ekstraksi pada gambar QRIS yang jelas dan standar. & Berhasil \\
\hline
S-12 & Penguji memilih kategori "Others" dari daftar pilihan. & Berhasil \\
\hline
S-13 & Penguji mengubah pilihan kategori dari "Food" menjadi "Bills" sebelum menyimpan. & Berhasil \\
\hline
S-14 & Penguji mengubah data "Total Amount" hasil ekstraksi karena tidak akurat. & Berhasil \\
\hline
S-15 & Penguji mencoba memasukkan teks (bukan angka) ke dalam kolom "Total Amount". & Berhasil \\
\hline
S-16 & Penguji menekan tombol "Save Transaction" setelah semua data terisi dengan benar. & Berhasil \\
\hline
S-17 & Penguji mencoba menyimpan transaksi dengan kolom "Total Amount" yang masih kosong. & Berhasil \\
\hline
S-18 & Penguji membuka halaman total pengeluaran setelah menyimpan transaksi baru. & Berhasil \\
\hline
S-19 & Penguji mencoba untuk mengakses halaman total pengeluaran dari halaman utama. & Berhasil \\
\hline
\end{tabularx}
\end{table}