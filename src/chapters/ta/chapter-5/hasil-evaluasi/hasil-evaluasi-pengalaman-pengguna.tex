\subsection{Hasil Evaluasi Pengalaman Pengguna}
\label{subsec:hasil-evaluasi-pengalaman-pengguna}

\autoref{tab:hasil-evaluasi-pengalaman-pengguna} menunjukkan hasil evaluasi pengalaman pengguna berdasarkan kuesioner yang telah disebarkan kepada 15 responden. Kuesioner ini bertujuan untuk mengukur pengalaman pengguna dengan metrik SUS dalam menggunakan aplikasi TrackMyBills dan mengumpulkan umpan balik yang diberikan oleh pengguna. 

\begin{table}[h!]
\centering
\caption{Hasil evaluasi pengalaman pengguna}
\label{tab:hasil-evaluasi-pengalaman-pengguna}
\begin{tabularx}{\linewidth}{|l|*{11}{c|}}
\hline
\textbf{ID} & 1 & 2 & 3 & 4 & 5 & 6 & 7 & 8 & 9 & 10 & 11 \\ \hline
\textbf{SUS} & 70,0 & 60,0 & 85,0 & 75,0 & 65,0 & 77,5 & 85,0 & 67,5 & 80,0 & 67,5 & 67,5 \\ \hline
\end{tabularx}
\end{table}

\begin{table}[h!]
    \ContinuedFloat
\centering
\caption{Hasil evaluasi pengalaman pengguna (lanjutan)}
\begin{tabularx}{\textwidth}{|l|*{4}{c|}X|}
\hline
\textbf{ID} & 12 & 13 & 14 & 15 & Keseluruhan \\ \hline
\textbf{SUS} & 65,0 & 85,0 & 50,0 & 77,5 & 71,83 \\ \hline
\end{tabularx}
\end{table}

Hasil evaluasi menunjukkan bahwa rata-rata nilai SUS yang diperoleh adalah 71,83. Nilai ini berada di atas ambang batas atas 68. Dalam literatur SUS, skor 71,83 menunjukkan bahwa sistem memiliki usabilitas yang baik. Hasil ini menunjukkan bahwa aplikasi TrackMyBills memiliki tingkat kepuasan pengguna yang baik dan dinilai \emph{usable} dan tidak membingungkan oleh mayoritas responden.

Responden juga diberikan pertanyaan untuk melakukan evaluasi secara kualitatif. 80\% responden menyatakan bahwa mereka akan menggunakan aplikasi ini untuk mempermudah pencatatan pengeluaran mereka dan 60\% responden menyatakan bahwa mereka merasa akan menjadi lebih disiplin dalam mencatat pengeluaran dengan adanya aplikasi TrackMyBills
