\subsection{Metode Evaluasi Kebutuhan Fungsional dan Non-Fungsional}
\label{subsec:evaluasi-kebutuhan-fungsional-dan-non-fungsional}
Evaluasi kebutuhan fungsional dan non-fungsional dilakukan untuk memastikan bahwa sistem memenuhi kebutuhan fungsional dan kebutuhan non-fungsional yang telah ditetapkan pada \autoref{sec:analisis-kebutuhan}. Evaluasi dilakukan dengan menetapkan keberhasilan pengguna menyelesaikan skenario yang telah dipersiapkan. Skenario dirancang dengan disesuaikan pada kasus uji yang telah didefinisikan pada \autoref{subsubsec:use-case-view}. Berdasarkan skenario-skenario yang telah dirancang tersebut, hasil yang diharapkan dari skenario tersebut akan ditentukan. \autoref{tab:skenario-uji-fungsional} menunjukkan skenario uji fungsionalitas yang akan digunakan untuk melakukan evaluasi sistem.
\begin{table}[h!]
\caption{Skenario uji fungsionalitas}
\label{tab:skenario-uji-fungsional}
\begin{tabularx}{\linewidth}{|p{1.5cm}|p{1.5cm}|p{4.5cm}|X|}
\hline
\textbf{Kode Skenario} & \textbf{Kode \emph{Use Case}} & \textbf{Skenario Pengujian} & \textbf{Hasil yang Diharapkan} \\
\hline
S-01 & UC-01 & Penguji membagikan gambar struk format PNG dari aplikasi galeri. & Aplikasi berhasil menerima dan menampilkan pratinjau gambar PNG tersebut. \\
\hline
S-02 & UC-01 & Penguji mencoba membagikan \emph{file} PDF ke aplikasi. & Aplikasi tidak akan muncul sebagai opsi aplikasi yang dapat dibagikan. \\
\hline
S-03 & UC-02 & Penguji mengambil foto struk dalam kondisi pencahayaan yang baik dan fokus yang jelas. & Aplikasi berhasil menangkap gambar dengan jelas dan menampilkan pratinjau. \\
\hline
S-04 & UC-02 & Penguji membuka kamera dari dalam aplikasi dan membatalkannya sebelum mengambil foto. & Aplikasi kembali ke layar sebelumnya. \\
\hline
S-05 & UC-02 & Penguji membuka kamera dari dalam aplikasi dan mencoba menggunakan \emph{flash} untuk mengambil foto. & Aplikasi menyalakan \emph{flash} dan berhasil mengambil foto. \\
\hline
S-06 & UC-03 & Penguji mengunggah \emph{screenshot} bukti transfer dari galeri. & Aplikasi berhasil memuat dan menampilkan tangkapan layar tersebut. \\
\hline
S-07 & UC-03 & Penguji mengunggah gambar dengan resolusi sangat rendah atau buram. & Aplikasi tetap berhasil menampilkan gambar, meskipun kualitasnya rendah. \\
\hline
S-08 & UC-03 & Penguji mengunggah gambar \emph{screenshot} bukti pembayaran QRIS. & Aplikasi berhasil memuat dan menampilkan tangkapan layar tersebut.\\
\hline
S-09 & UC-04 & Penguji memotong gambar untuk menyertakan seluruh area informasi yang relevan. & Sistem berhasil menyimpan area gambar yang telah dipotong dan menampilkannya untuk langkah selanjutnya. \\
\hline

\end{tabularx}
\end{table}

\begin{table}[h!]
\ContinuedFloat
\caption{Skenario uji fungsionalitas (lanjutan)}
\begin{tabularx}{\linewidth}{|p{1.5cm}|p{1.5cm}|p{4.5cm}|X|}
\hline
\textbf{Kode Skenario} & \textbf{Kode \emph{Use Case}} & \textbf{Skenario Pengujian} & \textbf{Hasil yang Diharapkan} \\
\hline
S-10 & UC-04 & Penguji melanjutkan proses tanpa memotong gambar sama sekali. & Sistem menggunakan keseluruhan gambar asli untuk proses ekstraksi data. \\
\hline
S-11 & UC-05 & Penguji menjalankan ekstraksi pada gambar QRIS yang jelas dan standar. & Sistem berhasil mengekstrak informasi total pembayaran dan tanggal dengan akurat. \\
\hline
S-12 & UC-06 & Penguji memilih kategori "Others" dari daftar pilihan. & Sistem berhasil menetapkan kategori "Others" untuk transaksi tersebut. \\
\hline
S-13 & UC-06 & Penguji mengubah pilihan kategori dari "Food" menjadi "Bills" sebelum menyimpan. & Sistem berhasil memperbarui kategori menjadi "Bills". \\
\hline
S-14 & UC-07 & Penguji mengubah data "Total Amount" hasil ekstraksi karena tidak akurat. & Sistem berhasil memperbarui kolom "Total Amount" sesuai input manual pengguna. \\
\hline
S-15 & UC-07 & Penguji mencoba memasukkan teks (bukan angka) ke dalam kolom "Total Amount". & Sistem tidak memberikan opsi untuk memasukkan teks. \\
\hline
S-16 & UC-08 & Penguji menekan tombol "Save Transaction" setelah semua data terisi dengan benar. & Sistem berhasil menyimpan data, menampilkan pesan sukses, dan mengarahkan pengguna ke halaman total pengeluaran. \\
\hline
S-17 & UC-08 & Penguji mencoba menyimpan transaksi dengan kolom "Total Amount" yang masih kosong. & Sistem menampilkan pesan kesalahan yang meminta pengguna melengkapi data dan transaksi tidak disimpan. \\
\hline
S-18 & UC-09 & Penguji membuka halaman total pengeluaran setelah menyimpan transaksi baru. & Total pengeluaran dan total pada kategori yang bersangkutan berhasil diperbarui secara akurat. \\
\hline
S-19 & UC-09 & Penguji mencoba untuk mengakses halaman total pengeluaran dari halaman utama. & Sistem berhasil menampilkan halaman total pengeluaran. \\
\hline
\end{tabularx}
\end{table}

% \begin{table}[h!]
% \caption{Skenario uji kebutuhan non-fungsional}
% \begin{tabularx}{\linewidth}{|p{1.5cm}|p{1.5cm}|p{4cm}|X|}
% \hline
% \textbf{Kode Kebutuhan} & \textbf{Kode Skenario} & \textbf{Skenario Pengujian} & \textbf{Hasil yang Diharapkan} \\
% \hline
% NF-01 & S-06 & Penguji menguji aplikasi pada perangkat Android dengan spesifikasi yang umum dimiliki Gen Z. & Aplikasi dapat berjalan dengan baik pada perangkat Android dengan spesifikasi tersebut. \\
% \hline
% NF-02 & S-07 & Penguji melakukan evaluasi \emph{System Usability Scale} (SUS) untuk menilai \emph{usability} aplikasi. & Hasil evaluasi SUS Score lebih besar dari 68. \\
% \hline
% NF-03 & S-08 & Penguji memasukkan gambar berkualitas jelas dan gambar yang \emph{blur} untuk melihat respons sistem. & Sistem tetap dapat memproses masukan gambar dengan kualitas yang jelas dan tidak blur dengan kekurangan kinerja. \\
% \hline
% NF-04 & S-09 & Penguji mencoba menjalankan aplikasi pada sistem operasi Android. & Sistem dapat beroperasi pada sistem operasi Android. \\
% \hline
% NF-05 & S-10 & Penguji menjalankan set data evaluasi untuk mengukur metrik kinerja sistem. & Hasil evaluasi \accuracy{}, \precision, \recall, dan \fscore{} lebih besar daripada 70\%, dan \mcer{} lebih kecil dari 20\%. \\
% \hline
% \end{tabularx}
% \end{table}