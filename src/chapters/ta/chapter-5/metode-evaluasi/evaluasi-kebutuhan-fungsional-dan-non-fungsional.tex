\subsection{Metode Evaluasi Kebutuhan Fungsional dan Non-Fungsional}
\label{subsec:evaluasi-kebutuhan-fungsional-dan-non-fungsional}
Evaluasi kebutuhan fungsional dan non-fungsional dilakukan untuk memastikan bahwa sistem memenuhi kebutuhan fungsional dan kebutuhan non-fungsional yang telah ditetapkan pada \autoref{sec:analisis-kebutuhan}. Evaluasi kebutuhan fungsional dan non-fungsional dilakukan dengan menetapkan pemenuhan kebutuhan dengan skenario yang telah ditetapkan dan hasil yang diharapkan dari skenario tersebut. Kebutuhan fungsional mencakup fitur-fitur yang harus ada dalam aplikasi, sedangkan kebutuhan non-fungsional mencakup aspek-aspek seperti \emph{usability}, \emph{compatibility}, \emph{supportability}, dan \emph{performance} yang mendukung pengalaman pengguna secara keseluruhan. \autoref{tab:skenario-uji-fungsional} menunjukkan skenario uji fungsionalitas dan hasil yang diharapkan dari sistem.

\begin{table}[h!]
\caption{Skenario uji fungsionalitas (1 Skenario per \emph{Use Case})}
\label{tab:skenario-uji-fungsional}
\begin{tabularx}{\linewidth}{|p{1.5cm}|p{1.5cm}|p{4.5cm}|X|}
\hline
\textbf{Kode Skenario} & \textbf{Kode \emph{Use Case}} & \textbf{Skenario Pengujian} & \textbf{Hasil yang Diharapkan} \\
\hline
SF-01 & UC-01 & Penguji membagikan gambar struk dari galeri foto ke aplikasi. & Aplikasi berhasil menerima dan menunjukkan gambar yang dibagikan. \\
\hline
SF-02 & UC-02 & Penguji menekan tombol untuk mengakses kamera di dalam aplikasi, lalu mengambil foto struk. & Aplikasi berhasil mengakses kamera, mengambil gambar, dan menampilkan hasil foto. \\
\hline
SF-03 & UC-03 & Penguji menekan tombol untuk mengunggah di dalam aplikasi, lalu memilih gambar dari galeri. & Aplikasi berhasil mengakses dan menampilkan gambar yang dipilih dari galeri. \\
\hline
SF-04 & UC-04 & Setelah gambar tampil, penguji menyesuaikan area potong untuk fokus pada data transaksi. & Sistem berhasil memotong gambar sesuai area yang dipilih pengguna. \\
\hline
SF-05 & UC-05 & Penguji menekan tombol untuk mengonfirmasi gambar yang ingin diekstrak. & Sistem berhasil mengekstrak teks dari gambar dan mengisinya ke kolom-kolom data yang tersedia. \\
\hline
SF-06 & UC-06 & Penguji memilih kategori "Bills" dari daftar pilihan kategori yang tersedia. & Sistem berhasil menetapkan kategori "Bills" pada data transaksi tersebut. \\
\hline
SF-07 & UC-07 & Penguji mengubah angka pada kolom "Total Amount" yang diekstrak secara otomatis. & Sistem berhasil memperbarui nilai pada kolom "Total Amount" sesuai input manual dari pengguna. \\
\hline
SF-08 & UC-08 & Setelah semua data benar, penguji menekan tombol "Save Transaction". & Sistem menyimpan data transaksi dan menunjukkan halaman total pengeluaran. \\
\hline
SF-09 & UC-09 & Penguji membuka halaman pengeluaran. & Sistem menampilkan total pengeluaran dan rincian pengeluaran per kategori yang telah diperbarui dengan data yang baru saja disimpan. \\
\hline
\end{tabularx}
\end{table}

% \begin{table}[h!]
% \caption{Skenario uji kebutuhan non-fungsional}
% \begin{tabularx}{\linewidth}{|p{1.5cm}|p{1.5cm}|p{4cm}|X|}
% \hline
% \textbf{Kode Kebutuhan} & \textbf{Kode Skenario} & \textbf{Skenario Pengujian} & \textbf{Hasil yang Diharapkan} \\
% \hline
% NF-01 & S-06 & Penguji menguji aplikasi pada perangkat Android dengan spesifikasi yang umum dimiliki Gen Z. & Aplikasi dapat berjalan dengan baik pada perangkat Android dengan spesifikasi tersebut. \\
% \hline
% NF-02 & S-07 & Penguji melakukan evaluasi \emph{System Usability Scale} (SUS) untuk menilai \emph{usability} aplikasi. & Hasil evaluasi SUS Score lebih besar dari 68. \\
% \hline
% NF-03 & S-08 & Penguji memasukkan gambar berkualitas jelas dan gambar yang \emph{blur} untuk melihat respons sistem. & Sistem tetap dapat memproses masukan gambar dengan kualitas yang jelas dan tidak blur dengan kekurangan kinerja. \\
% \hline
% NF-04 & S-09 & Penguji mencoba menjalankan aplikasi pada sistem operasi Android. & Sistem dapat beroperasi pada sistem operasi Android. \\
% \hline
% NF-05 & S-10 & Penguji menjalankan set data evaluasi untuk mengukur metrik kinerja sistem. & Hasil evaluasi \accuracy{}, \precision, \recall, dan \fscore{} lebih besar daripada 70\%, dan \mcer{} lebih kecil dari 20\%. \\
% \hline
% \end{tabularx}
% \end{table}