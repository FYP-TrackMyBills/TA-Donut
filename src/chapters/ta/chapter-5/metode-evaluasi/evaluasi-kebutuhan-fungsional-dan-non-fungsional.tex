\subsection{Evaluasi Kebutuhan Fungsional dan Non-Fungsional}
\label{subsec:evaluasi-kebutuhan-fungsional-dan-non-fungsional}
Evaluasi kebutuhan fungsional dan non-fungsional dilakukan untuk memastikan bahwa sistem memenuhi kebutuhan fungsional dan kebutuhan non-fungsional yang telah ditetapkan pada \autoref{sec:analisis-kebutuhan}. Kebutuhan fungsional mencakup fitur-fitur yang harus ada dalam aplikasi, sedangkan kebutuhan non-fungsional mencakup aspek-aspek seperti \emph{usability}, \emph{compatibility}, \emph{supportability}, dan \emph{performance} yang mendukung pengalaman pengguna secara keseluruhan. \autoref{tab:skenario-uji-fungsional} menunjukkan skenario uji fungsionalitas yang digunakan untuk mengevaluasi kebutuhan fungsional dan non-fungsional sistem.

\begin{table}[h!]
\caption{Skenario uji fungsionalitas}
\label{tab:skenario-uji-fungsional}
\begin{tabularx}{\linewidth}{|p{1.5cm}|p{1.5cm}|p{4cm}|X|}
\hline
\textbf{Kode Kebutuhan} & \textbf{Kode Skenario} & \textbf{Skenario Pengujian} & \textbf{Hasil yang Diharapkan} \\
\hline
FR-01 & SF-01 & Penguji memasukkan gambar bukti transaksi (struk fisik atau tangkapan layar digital). & Sistem harus dapat menerima masukan berupa gambar bukti transaksi, yaitu struk fisik cetakan dan tangkapan layar bukti pembayaran digital, yaitu QRIS dan transfer. \\
\hline
FR-02 & SF-02 & Penguji memilih dan memotong area relevan pada gambar yang diunggah. & Sistem harus dapat memotong area relevan transaksi berdasarkan keinginan pengguna dari gambar yang diunggah. \\
\hline
FR-03 & SF-03 & Penguji menjalankan fungsi ekstraksi pada gambar yang sudah dipotong. & Sistem harus dapat mengekstrak dan mengklasifikasikan informasi kunci, yaitu total pembayaran dan tipe transaksi dan target dari QRIS dan transfer dan total pembayaran dari struk pembayaran. \\
\hline
FR-04 & SF-04 & Penguji memeriksa hasil ekstraksi dan mengoreksi data yang tidak akurat. & Sistem harus menampilkan hasil ekstraksi kepada pengguna dan menyediakan fungsi mengoreksi data jika terdapat ketidakakuratan. \\
\hline
FR-05 & SF-05 & Penguji menyimpan data transaksi yang telah diverifikasi. & Sistem harus dapat menyimpan data transaksi yang telah diverifikasi dan menampilkannya dalam riwayat pengeluaran yang terstruktur. \\
\hline
NF-01 & S-06 & Penguji mengevaluasi kemudahan penggunaan antarmuka dengan melakukan proses dari unggah, tinjau, hingga simpan. & Antarmuka pengguna (UI) harus minimalis dan intuitif, memungkinkan pengguna mengubah gambar bukti bayar menjadi data transaksi dalam beberapa langkah sederhana (unggah, tinjau, simpan). \\
\hline
\end{tabularx}
\end{table}

\begin{table}[h!]
\ContinuedFloat
\caption{Skenario uji fungsionalitas (lanjutan)}
\begin{tabularx}{\linewidth}{|p{1.5cm}|p{1.5cm}|p{4cm}|X|}
\hline
\textbf{Kode Kebutuhan} & \textbf{Kode Skenario} & \textbf{Skenario Pengujian} & \textbf{Hasil yang Diharapkan} \\
\hline
NF-02 & S-07 & Penguji melakukan evaluasi \emph{System Usability Scale} (SUS) untuk menilai \emph{usability} aplikasi. & Hasil evaluasi SUS Score lebih besar dari 68. \\
\hline
NF-03 & S-08 & Penguji memasukkan gambar berkualitas jelas dan gambar yang \emph{blur} untuk melihat respons sistem. & Sistem tetap dapat memproses masukan gambar dengan kualitas yang jelas dan tidak blur dengan kekurangan kinerja. \\
\hline
NF-04 & S-09 & Penguji mencoba menjalankan aplikasi pada sistem operasi Android. & Sistem dapat beroperasi pada sistem operasi Android. \\
\hline
NF-05 & S-10 & Penguji menjalankan set data evaluasi untuk mengukur metrik kinerja sistem. & Hasil evaluasi \accuracy{}, \precision, \recall, dan \fscore{} lebih besar daripada 70\%, dan \mcer{} lebih kecil dari 20\%. \\
\hline
\end{tabularx}
\end{table}