\subsection{Evaluasi Kinerja Model}
\label{subsec:evaluasi-kinerja-model}

Evaluasi kinerja model dilakukan dengan membandingkan hasil prediksi model dengan \emph{ground truth} (GT) yang telah disiapkan. Proses evaluasi ini bertujuan untuk mengukur seberapa baik model dalam mengenali dan mengekstrak informasi dari dokumen pembayaran yang telah dilatih sebelumnya. \autoref{tab:definisi-kategori-prediksi} menunjukkan definisi dari setiap kategori dari prediksi yang dihasilkan.

\begin{table}[h!]
    \centering
    \caption{Definisi kategori prediksi model}
    \label{tab:definisi-kategori-prediksi}
    \begin{tabularx}{\textwidth}{|p{2.5cm}|X|}
        \hline
        \textbf{Kategori} & \textbf{Definisi} \\ \hline
        \emph{True Positive} (TP) & Seluruh atribut yang ada di \emph{Ground Truth} (GT) terprediksi dengan nilai yang benar. \\ \hline
        \emph{False Positive} (FP) & Terdapat atribut yang ada di GT dengan nilai yang tidak kosong, tetapi tidak muncul di hasil prediksi atau muncul dengan nilai kosong. \\ \hline
        \emph{False Negative} (FN) & Terdapat atribut yang ada di GT dengan nilai yang tidak kosong, tetapi tidak muncul di hasil prediksi atau muncul dengan nilai kosong. \\ \hline
        \emph{True Negative} (TN) & Terdapat atribut yang tidak ada di GT dan tidak muncul di hasil prediksi. \\ \hline
    \end{tabularx}
\end{table}

Evaluasi kinerja model akan mengabaikan keberadaan \emph{True Negative} (TN). Jumlah informasi potensial yang dapat diekstrak dari dokumen sangat banyak dalam ekstraksi informasi dokumen sehingga nilai dari TN akan menjadi sangat besar. Oleh karena itu, evaluasi kinerja model akan berfokus pada informasi yang tertera pada \emph{Ground Truth} (GT). Metrik yang bergantung pada TN, seperti \accuracy, tidak menunjukkan kinerja model yang berarti dalam konteks ini dibandingkan dengan metrik lainnya. 

Evaluasi akan dilakukan terhadap dua jenis model yang digunakan, yaitu model \donut{} yang telah dilatih dengan \dataset{} QRIS dan transfer bank, serta model \donut{} yang telah dilatih dengan \dataset{} CORD-v2. Hasil evaluasi akan diukur dengan menggunakan metrik-metrik standar dalam evaluasi model klasifikasi, yaitu \accuracy, \precision, \recall, dan \fscore. 

Selain metrik-metrik standar tersebut, terdapat juga metrik \mcer{} yang digunakan untuk mengukur berapa banyak perubahan karakter yang diperlukan dari teks prediksi menjadi teks sebenarnya dari \emph{Ground Truth}. Metrik-metrik ini akan memberikan gambaran yang jelas mengenai kinerja model dalam mengenali dan mengekstrak informasi dari dokumen pembayaran yang telah dilatih sebelumnya.

Setiap model yang dievaluasi memiliki ketentuan bobot untuk setiap atribut yang diekstrak. Hal ini dilakukan untuk memberikan bobot yang sesuai pada setiap atribut berdasarkan pentingnya informasi tersebut dalam konteks dokumen pembayaran. \autoref{tab:base-model-field-weights} menunjukkan ketentuan bobot untuk setiap atribut yang diekstrak pada model \emph{base model} dan \autoref{tab:custom-model-field-weights} menunjukkan ketentuan bobot untuk setiap atribut yang diekstrak pada model \emph{custom model}.

\begin{table}[h!]
    \centering
    \caption{Ketentuan bobot atribut yang diekstrak \emph{base model}}
    \label{tab:base-model-field-weights}
    \begin{tabularx}{\textwidth}{|X|X|}
        \hline
        \textbf{Atribut} & \textbf{Bobot} \\ \hline
        Total transaksi & 2.0 \\ \hline
        Nama menu & 1.0 \\ \hline
        Jumlah menu & 1.0 \\ \hline
        Harga menu & 1.0 \\ \hline 
    \end{tabularx}
\end{table}

\begin{table}[h!]
    \centering
    \caption{Ketentuan bobot atribut yang diekstrak \emph{custom model}}
    \label{tab:custom-model-field-weights}
    \begin{tabularx}{\textwidth}{|X|X|}
        \hline
        \textbf{Atribut} & \textbf{Bobot} \\ \hline
        Total transaksi & 3.0 \\ \hline
        Target transaksi & 2.0 \\ \hline
        Tipe transaksi & 2.0 \\ \hline
        Aplikasi & 1.0 \\ \hline
        Waktu transaksi & 1.0 \\ \hline
        ID transaksi & 1.0 \\ \hline
    \end{tabularx}
\end{table}
