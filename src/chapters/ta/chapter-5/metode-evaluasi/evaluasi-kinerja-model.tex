\subsection{Evaluasi Kinerja Model}
\label{subsec:evaluasi-kinerja-model}

Evaluasi kinerja model dilakukan dengan membandingkan hasil prediksi model dengan \emph{ground truth} (GT) yang telah disiapkan. Proses evaluasi ini bertujuan untuk mengukur seberapa baik model dalam mengenali dan mengekstrak informasi dari dokumen pembayaran yang telah dilatih sebelumnya. \autoref{tab:definisi-kategori-prediksi} menunjukkan definisi dari setiap kategori dari prediksi yang dihasilkan.

\begin{table}[h!]
    \centering
    \caption{Definisi kategori prediksi model}
    \label{tab:definisi-kategori-prediksi}
    \begin{tabularx}{\textwidth}{|p{2.5cm}|X|}
        \hline
        \textbf{Kategori} & \textbf{Definisi} \\ \hline
        \emph{True Positive} (TP) & Seluruh atribut yang ada di \emph{Ground Truth} (GT) terprediksi dengan nilai yang benar. \\ \hline
        \emph{False Positive} (FP) & Terdapat atribut yang \textbf{tidak} ada di GT, tetapi \textbf{muncul} di hasil prediksi. \\ \hline
        \emph{False Negative} (FN) & Terdapat atribut yang \emph{ada} di GT dengan nilai yang tidak kosong, tetapi diprediksi dengan salah (kosong, tidak terprediksi, atau tidak sesuai dengan GT). \\ \hline
        \emph{True Negative} (TN) & Terdapat atribut yang tidak ada di GT dan tidak muncul di hasil prediksi. Metrik ini jarang digunakan untuk evaluasi pada ekstraksi informasi dokumen. \\ \hline
    \end{tabularx}
\end{table}

Evaluasi kinerja model akan mengabaikan keberadaan \emph{True Negative} (TN). Jumlah informasi potensial yang dapat diekstrak dari dokumen sangat banyak dalam ekstraksi informasi dokumen sehingga nilai dari TN akan menjadi sangat besar. Oleh karena itu, evaluasi kinerja model akan berfokus pada informasi yang tertera pada \emph{Ground Truth} (GT). Metrik yang bergantung pada TN, seperti \accuracy, tidak menunjukkan kinerja model yang berarti dalam konteks ini dibandingkan dengan metrik lainnya. 

Evaluasi akan dilakukan terhadap dua jenis model yang digunakan, yaitu model \donut{} yang telah dilatih dengan \dataset{} QRIS dan transfer bank, serta model \donut{} yang telah dilatih dengan \dataset{} CORD-v2. Hasil evaluasi akan diukur dengan menggunakan metrik-metrik standar dalam evaluasi model klasifikasi, yaitu \accuracy, \precision, \recall, dan \fscore. 

Selain metrik-metrik standar tersebut, terdapat juga metrik \mcer{} yang digunakan untuk mengukur berapa banyak perubahan karakter yang diperlukan dari teks prediksi menjadi teks sebenarnya dari \emph{Ground Truth}. Metrik-metrik ini akan memberikan gambaran yang jelas mengenai kinerja model dalam mengenali dan mengekstrak informasi dari dokumen pembayaran yang telah dilatih sebelumnya.

Setiap model yang dievaluasi memiliki ketentuan bobot untuk setiap atribut yang diekstrak. Hal ini dilakukan untuk memberikan bobot yang sesuai pada setiap atribut berdasarkan pentingnya informasi tersebut dalam konteks dokumen pembayaran. Penetapan bobot didasarkan pada analisis dampak kesalahan dan kebutuhan pengguna dalam sistem pencatatan pengeluaran.

Penentuan bobot untuk setiap atribut yang diekstrak menggunakan pendekatan \emph{Risk-Impact Assessment Framework}. Pendekatan ini menganalisis dampak kesalahan klasifikasi terhadap fungsionalitas sistem pencatatan pengeluaran secara sistematis. Kerangka kerja ini mengevaluasi setiap atribut berdasarkan dampak bisnis, dampak pengalaman pengguna, dan biaya pemulihan kesalahan. \autoref{tab:risk-classification} menunjukkan definisi klasifikasi yang digunakan. \autoref{tab:risk-impact-base-model} menunjukkan analisis terhadap atribut yang diekstrak pada model \emph{base model} dan \autoref{tab:risk-impact-custom-model} menunjukkan analisis terhadap atribut yang diekstrak pada model \emph{custom model}.

\begin{table}[h!]
    \centering
    \caption{Klasifikasi Tingkat Risiko untuk Setiap Kriteria Evaluasi}
    \label{tab:risk-classification}
    % --- THE FIX IS HERE ---
    % We use the standard `tabular` environment
    % and the `m{width}` column type for vertical centering.
    \begin{tabular}{|m{3cm}|l|m{7.5cm}|}
        \hline
        \textbf{Kriteria} & \textbf{Tingkat Risiko} & \textbf{Definisi} \\ \hline
        
        \multirow{2}{=}{Dampak Bisnis} & TINGGI & Kesalahan langsung mempengaruhi keputusan finansial atau akurasi pencatatan keuangan \\ \cline{2-3}
        & RENDAH & Tidak mempengaruhi keputusan finansial atau analisis pengeluaran utama \\ \hline
        
        % Note: I removed the manual \newline. The m{} column handles wrapping.
        \multirow{2}{=}{Dampak Pengalaman Pengguna} & TINGGI & Tidak ada mekanisme verifikasi alternatif atau kesalahan sulit dideteksi melalui konteks \\ \cline{2-3}
        & RENDAH & Kesalahan mudah dideteksi atau tidak mengganggu fungsionalitas sistem \\ \hline

        \multirow{2}{=}{Biaya Pemulihan Kesalahan} & TINGGI & Memerlukan upaya signifikan untuk memverifikasi dan memperbaiki kesalahan \\ \cline{2-3}
        & RENDAH & Mudah diverifikasi, dikoreksi, atau diabaikan tanpa dampak signifikan \\ \hline
    \end{tabular}
\end{table}

\begin{table}[h!]
    \centering
    \caption{Analisis \emph{Risk-Impact} untuk \emph{Base Model}}
    \label{tab:risk-impact-base-model}
    \begin{tabularx}{\textwidth}{|p{3cm}|p{1cm}|p{2cm}|p{2.5cm}|X|}
        \hline
        \textbf{Atribut} & \textbf{Bobot} & \textbf{Dampak Bisnis} & \textbf{Dampak Pengalaman Pengguna} & \textbf{Biaya Pemulihan Kesalahan} \\ \hline
        Total transaksi & 2.0 & TINGGI & TINGGI & RENDAH \\ \hline
        Nama menu & 1.0 & RENDAH & RENDAH & RENDAH \\ \hline
        Jumlah menu & 1.0 & RENDAH & RENDAH & RENDAH \\ \hline
        Harga menu & 1.0 & RENDAH & RENDAH & RENDAH \\ \hline
    \end{tabularx}
\end{table}

\begin{table}[h!]
    \centering
    \caption{Analisis \emph{Risk-Impact} untuk \emph{Custom Model}}
    \label{tab:risk-impact-custom-model}
    \begin{tabularx}{\textwidth}{|p{3cm}|p{1cm}|p{2cm}|p{2.5cm}|X|}
        \hline
        \textbf{Atribut} & \textbf{Bobot} & \textbf{Dampak Bisnis} & \textbf{Dampak Pengalaman Pengguna} & \textbf{Biaya Pemulihan Kesalahan} \\ \hline
        Total transaksi & 3.0 & TINGGI & TINGGI & TINGGI \\ \hline
        Target transaksi & 2.0 & TINGGI & TINGGI & RENDAH \\ \hline
        Tipe transaksi & 2.0 & TINGGI & TINGGI & RENDAH \\ \hline
        Aplikasi & 1.0 & RENDAH & RENDAH & RENDAH \\ \hline
        Waktu transaksi & 1.0 & RENDAH & RENDAH & RENDAH \\ \hline
        ID transaksi & 1.0 & RENDAH & RENDAH & RENDAH \\ \hline
    \end{tabularx}
\end{table}

% Perbedaan bobot total transaksi antara kedua model (3.0 vs 2.0) mencerminkan perbedaan signifikan dalam \emph{User Experience Impact} dan \emph{Error Recovery Cost}. Model QRIS-TF memiliki risiko lebih tinggi karena tidak tersedianya mekanisme verifikasi redundan yang dimiliki oleh model CORD-v2. Penentuan bobot ini sejalan dengan prinsip user-centered design dimana atribut yang paling mempengaruhi keputusan keuangan pengguna diberikan prioritas evaluasi yang lebih tinggi \parencite{norman2013design}.

% \autoref{tab:base-model-field-weights} menunjukkan rangkuman ketentuan bobot untuk setiap atribut yang diekstrak pada model \emph{base model} dan \autoref{tab:custom-model-field-weights} menunjukkan rangkuman ketentuan bobot untuk setiap atribut yang diekstrak pada model \emph{custom model}.

% \begin{table}[h!]
%     \centering
%     \caption{Ketentuan bobot atribut yang diekstrak \emph{base model}}
%     \label{tab:base-model-field-weights}
%     \begin{tabularx}{\textwidth}{|X|X|}
%         \hline
%         \textbf{Atribut} & \textbf{Bobot} \\ \hline
%         Total transaksi & 2.0 \\ \hline
%         Nama menu & 1.0 \\ \hline
%         Jumlah menu & 1.0 \\ \hline
%         Harga menu & 1.0 \\ \hline 
%     \end{tabularx}
% \end{table}

% \begin{table}[h!]
%     \centering
%     \caption{Ketentuan bobot atribut yang diekstrak \emph{custom model}}
%     \label{tab:custom-model-field-weights}
%     \begin{tabularx}{\textwidth}{|X|X|}
%         \hline
%         \textbf{Atribut} & \textbf{Bobot} \\ \hline
%         Total transaksi & 3.0 \\ \hline
%         Target transaksi & 2.0 \\ \hline
%         Tipe transaksi & 2.0 \\ \hline
%         Aplikasi & 1.0 \\ \hline
%         Waktu transaksi & 1.0 \\ \hline
%         ID transaksi & 1.0 \\ \hline
%     \end{tabularx}
% \end{table}