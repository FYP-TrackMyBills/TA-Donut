\subsection{Evaluasi Pengalaman Pengguna}
\label{subsec:evaluasi-pengalaman-pengguna}
Evaluasi pengalaman pengguna dilakukan untuk memastikan bahwa sistem yang dibangun memenuhi kebutuhan dan memberikan pengalaman yang baik. Evaluasi ini dilakukan dengan mengacu pada kebutuhan fungsional dan non-fungsional yang telah diidentifikasi sebelumnya. Evaluasi pengalaman pengguna dilakukan dengan cara menguji sistem secara langsung kepada pengguna dan mengisi kuesioner. Pengujian dilakukan pada 15 pengguna yang memenuhi profil yang diperlukan, yaitu pengguna Gen Z. Pengujian dilakukan dengan memberikan skenario pengujian yang telah ditetapkan sebelumnya. 

\autoref{tab:skenario-uji-fungsional} menunjukkan skenario pengujian yang dilakukan untuk menguji pengalaman pengguna. Pengguna kemudian akan diminta untuk mengisi formulir yang telah dipersiapkan. Formulir ini berisi profil pengguna, pakta integritas, sepuluh pernyataan SUS, dan umpan balik mengenai pengalaman pengguna saat menggunakan sistem. Respons dari pengguna kemudian akan digunakan sebagai dasar untuk mengevaluasi pengalaman pengguna terhadap sistem yang telah dibangun dengan menggunakan SUS dan pernyataan kualitatif yang telah diberikan pengguna.