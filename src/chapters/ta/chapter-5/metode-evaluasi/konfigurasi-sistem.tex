\subsection{Konfigurasi Sistem}
\label{subsec:konfigurasi-sistem}

Evaluasi sistem dilakukan menggunakan konfigurasi yang disesuaikan dengan jenis evaluasi yang dilakukan. Evaluasi sistem dilakukan pada komponen model \ml{} dan aplikasi \emph{mobile}. Konfigurasi sistem dibagi menjadi dua lingkungan evaluasi yang berbeda sesuai dengan karakteristik masing-masing komponen.

\subsubsection{Konfigurasi Sistem untuk Evaluasi Kinerja Model}
\label{subsubsec:konfigurasi-sistem-evaluasi-kinerja-model}

Evaluasi kinerja model dilakukan menggunakan \emph{platform} Kaggle Notebook yang menyediakan akses ke GPU untuk tugas \ml. \autoref{tab:ml-config} menunjukkan konfigurasi Kaggle Notebook yang digunakan untuk evaluasi model \donut.

Lingkungan Kaggle dipilih karena menyediakan GPU Tesla T4 yang cukup mumpuni untuk melakukan \emph{fine-tuning} dan evaluasi model \donut{} serta memiliki semua \emph{dependency} yang diperlukan untuk \dl.

\begin{table}[h!]
    \centering
    \caption{Konfigurasi perangkat keras Kaggle Notebook untuk evaluasi model}
    \label{tab:ml-config}
    \begin{tabularx}{\textwidth}{|p{4cm}|X|}
        \hline
        \textbf{Komponen} & \textbf{Spesifikasi} \\
        \hline
        \textit{Platform} & Kaggle Notebook Environment \\
        \hline
        GPU      & 2x NVIDIA Tesla T4 (16GB VRAM total) \\
        \hline
        CPU      & Intel Xeon \\
        \hline
        RAM      & 30GB \\
        \hline
        \textit{Storage}  & 20GB \\
        \hline
    \end{tabularx}
\end{table}

\subsubsection{Konfigurasi Sistem untuk Evaluasi Aplikasi Mobile}
\label{subsubsec:konfigurasi-sistem-evaluasi-aplikasi-mobile}
Evaluasi aplikasi \emph{mobile} dilakukan menggunakan perangkat fisik untuk memastikan pengalaman pengguna yang realistis. \autoref{tab:mobile-config} menunjukkan spesifikasi perangkat yang digunakan untuk evaluasi aplikasi \emph{mobile}. Perangkat yang digunakan untuk melakukan evaluasi pengalaman pengguna bervariasi tergantung pada ketersediaan perangkat. Namun, umumnya perangkat yang digunakan adalah perangkat Android yang umumnya dimiliki Gen Z.

\begin{table}[h!]
    \centering
    \caption{Konfigurasi Perangkat Mobile untuk Evaluasi}
    \label{tab:mobile-config}
    \begin{tabularx}{\linewidth}{|p{4cm}|X|}
        \hline
        \textbf{Spesifikasi} & \textbf{Detail} \\
        \hline
        \textit{Model} & Samsung Galaxy A73 5G \\
        \hline
        \textit{Processor} & Qualcomm Snapdragon 778G \\
        \hline
        RAM & 8GB \\
        \hline
        \textit{Storage} & 256GB \\
        \hline
        \textit{Display} & 6.7" FHD+ AMOLED (1080 x 2400) \\
        \hline
        \textit{Camera} & 108MP \textit{main}, 12MP \textit{ultrawide} \\
        \hline
        \textit{Operating System} & Android 15 \\
        \hline
    \end{tabularx}
\end{table}

Pemilihan Samsung Galaxy A73 sebagai perangkat evaluasi didasarkan pada kapabilitas perwakilan terhadap mid-to-high-range perangkat Android yang umum digunakan Gen Z. Spesifikasi yang cukup mumpuni memastikan bahwa evaluasi tidak terbias oleh keterbatasan perangkat, namun tetap realistis untuk penggunaan sehari-hari.