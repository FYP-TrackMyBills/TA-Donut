\chapter{Kesimpulan dan Saran}

\section{Kesimpulan}
\label{sec:kesimpulan}
Tugas akhir ini bertujuan untuk mengembangkan sistem pencatatan pengeluaran berbasis \emph{mobile} yang dapat membantu pengguna, yaitu Gen Z, dalam mencatat pengeluaran mereka dengan lebih mudah. Sistem ini menggunakan model \donut{} untuk mengekstrak informasi penting dari gambar bukti pembayaran dan menyimpannya dalam format yang terstruktur untuk kemudian dapat ditampilkan kepada pengguna. Hasil evaluasi sistem menunjukkan bahwa aplikasi TrackMyBills \textbf{berhasil} memenuhi seluruh evaluasi yang dilakukan, yaitu dari pengujian fungsionalitas, pengujian kinerja model, dan pengujian pengalaman pengguna. 

Pengujian fungsionalitas \emph{berhasil} menunjukkan bahwa seluruh skenario pengujian dapat dijalankan dengan baik dan sesuai dengan yang diharapkan. Pengujian pengalaman pengguna menunjukkan bahwa aplikasi memiliki tingkat kepuasan pengguna yang baik, dengan nilai SUS di angka \textbf{71,83} rata-rata di atas ambang batas nilai SUS, yaitu pada \textbf{68}. Hal ini menunjukkan bahwa aplikasi dapat memberikan pengalaman pengguna yang baik dan memenuhi kebutuhan pengguna dalam mencatat pengeluaran mereka.

Pengujian kinerja model menunjukkan bahwa kedua model yang digunakan, yaitu \donut{} yang di \emph{fine-tune} pada \dataset{} CORD-v2 (\emph{base model}) dan \donut{} yang di \emph{fine-tune} pada \dataset{} QRIS-TF (\emph{custom model}), menunjukkan hasil yang memuaskan dengan nilai \accuracy, \precision, \recall, \fscore, dan \mcer{} yang melewati standar yang telah ditetapkan. \emph{Base model} menunjukkan angka \accuracy{} \textbf{73,43\%}, \precision{} \textbf{90,53\%}, \recall{} \textbf{79,53\%}, \fscore{} \textbf{84,68\%}, dan \mcer{} \textbf{18,85\%}. \emph{Custom model} menunjukkan angka \accuracy{} \textbf{68,78\%}, \precision{} \textbf{100\%}, \recall{} \textbf{68,78\%}, \fscore{} \textbf{81,50\%}, dan \mcer{} \textbf{17,20\%}. Hal ini menunjukkan bahwa model dapat mengenali dan mengekstrak informasi penting dari gambar struk dengan hasil yang baik. 

\section{Saran}
\label{sec:saran}
Meskipun sistem TrackMyBills telah menunjukkan kinerja yang baik dan mendapatkan umpan balik positif dari pengguna, masih ada beberapa area yang dapat ditingkatkan. Kinerja model \donut{} masih dapat ditingkatkan dengan melatih model pada data yang lebih beragam dan representatif, terutama untuk dokumen pembayaran QRIS dan transfer. 

\datasetfl{} CORD-v2 yang digunakan untuk \emph{fine-tuning} model \donut{} belum dapat memberikan kebebasan kepada model untuk memahami dokumen secara holistik, sehingga model masih memerlukan \emph{cropping} pada gambar bukti pembayaran sebelum diekstrak. Penelitian selanjutnya dapat membuat \dataset{} seperti CORD-v2 tanpa menghilangkan sebagian informasi yang tidak relevan dan melatihnya pada model \donut{} untuk menghilangkan kebutuhan pemotongan gambar.

TrackMyBills masih menggunakan dua model yang berbeda untuk menangani dua kasus yang berbeda, yaitu dokumen pembayaran QRIS dan transfer serta struk pembayaran. Penelitian selanjutnya dapat mencoba untuk menggunakan satu model yang dapat menangani kedua jenis dokumen tersebut dengan \emph{task prompt} yang sesuai untuk masing-masing jenis dokumen untuk mengurangi kompleksitas sistem yang dibangun.

Sistem TrackMyBills masih dapat dikembangkan lebih lanjut dengan menambahkan fitur-fitur penting, sebagai berikut.
\begin{enumerate}
    \item Integrasi dengan layanan perbankan\\~ Fitur ini akan memudahkan pengguna dalam mencatat pengeluaran mereka tanpa perlu mengunggah bukti pembayaran secara manual.
    \item Melihat detail transaksi lampau dan membuat perubahan\\~ Fitur ini akan memungkinkan pengguna untuk mengelola dan memperbaiki data transaksi yang telah dicatat sebelumnya.
    \item Integrasi dengan sistem basis data \emph{cloud}\\~
    Fitur ini akan memungkinkan pengguna untuk mengakses data mereka dari berbagai perangkat dan memastikan data tetap aman meskipun gawai pengguna hilang atau rusak.
\end{enumerate}