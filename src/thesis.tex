%--------------------------------------------------------------------%
%
% Berkas utama templat LaTeX.
%
% author Petra Barus, Peb Ruswono Aryan, Faris Rizki Ekananda
%
%--------------------------------------------------------------------%
%
% Berkas ini berisi struktur utama dokumen LaTeX yang akan dibuat.
%
%--------------------------------------------------------------------%

\documentclass[bahasa, 12pt, a4paper, onecolumn, oneside, final]{report}

\input{config/ta/if-itb-thesis.sty}
%--------------------------------------------------------------------%
%
% Hypenation untuk Bahasa Indonesia
%
% @author Petra Barus
%
%--------------------------------------------------------------------%
%
% Secara otomatis LaTeX dapat langsung memenggal kata dalam dokumen,
% tapi sering kali terdapat kesalahan dalam pemenggalan kata. Untuk
% memperbaiki kesalahan pemenggalan kata tertentu, cara pemenggalan
% kata tersebut dapat ditambahkan pada dokumen ini. Pemenggalan
% dilakukan dengan menambahkan karakter '-' pada suku kata yang
% perlu dipisahkan.
%
% Contoh pemenggalan kata 'analisa' dilakukan dengan 'a-na-li-sa'
%
%--------------------------------------------------------------------%

\hyphenation {
	% A
	%
	a-na-li-sis
	a-pli-ka-si
	a-lo-ka-si
	an-ta-ra
	ada-nya
	a-kan
	a-da-lah
	% B
	%
	be-be-ra-pa
	ber-ge-rak
	be-ri-kut
	ber-ko-mu-ni-ka-si
	bu-ah
	Bouvet
	ber-fo-kus
	ber-fung-si
	ber-ja-lan
	bound-ing
	bi-di-rec-tio-nal
	% C
	%
	ca-ri
	Carzaniga
	cloud
	CloudFormation
	con-tain-er
	ClusterIP
	% D
	%
	da-e-rah
	di-nya-ta-kan
	de-fi-ni-si
	di-bu-tuh-kan
	di-gu-na-kan
	di-tam-bah-kan-nya
	di-tem-pat-kan
	di-la-ku-kan
	di-kem-bang-kan
	di-im-ple-men-ta-si-kan
	da-pat
	di-gi-tal
	di-ka-te-go-ri-kan
	de-ngan
	di-kem-bang-kan
	da-ta
	di-lun-cur-kan
	di-a-pli-ka-si-kan
	di-wan
	di-pro-yek-si-kan
	di-ban-ding-kan
	% E
	%
	e-ner-gi
	eks-klu-sif
	eks-ter-nal
	ek-strak-si
	% F
	FastAPI
	%
	fa-si-li-tas
	% G
	%
	ga-bung-an
	% H
	%
	ha-lang-an
	ha-sil
	hell
	hyperspectral
	% I
	% 
	i-nduk
	in-for-ma-si
	im-ple-men-ta-si
	% J
	%
	jenis
	% K
	%
	kom-po-si-si
	kom-pre-si
	ka-lang-an
	ka-re-na
	ke-sa-ba-ran-nya
	ka-me-ra
	ku-a-li-tas
	ke-nang-an
	kom-plek-si-tas
	ke-ti-ka
	ke-le-bi-han-nya
	ke-gi-a-tan
	ko-mu-ni-ka-si
	ke-cil
	ke-du-a
	ke-ce-pa-tan
	% L
	%
	la-ya-nan
	language
	% M
	%
	me-mer-lu-kan
	me-nge-mu-di
	mem-be-ri-kan
	me-ngu-rang-i
	meng-eva-lu-a-si
	me-nge-na-i
	me-nge-lo-la
	men-da-lam
	men-ja-lan-kan
	mak-si-mal
	me-nye-le-sai-kan
	me-ngun-jung-i
	men-du-kung
	me-nu-rut
	me-la-ku-kan
	mem-bu-at
	men-daf-tar-kan
	meng-u-sul-kan
	me-mi-li-ki
	meng-gu-na-kan
	men-ja-di
	me-ru-pa-kan
	men-ja-ga
	me-mu-dah-kan
	me-ne-rus-kan
	mem-pro-ses
	meng-ha-sil-kan
	meng-e-va-lu-a-si
	me-ngem-bang-kan
	ma-chine
	me-nun-juk-kan
	me-re-pre-sen-ta-si-kan
	mem-fo-kus-kan
	me-nam-bah-kan
	% N
	%
	Na-mun
	% O
	%
	ob-so-lete
	or-kes-tra-si
	o-to-ma-ti-sa-si
	out-put
	% P
	%
	peng-a-lam-an
	pe-mro-ses-an
	pro-vi-der
	pe-ru-sa-ha-an
	pe-rang-kat
	pro-ses
	plat-form
	pro-duk-si
	pe-ne-li-tian
	pe-ru-ba-han
	pa-ra-dig-ma
	pe-man-tau-an
	pe-ngum-pu-lan
	pack-age
	per-kem-bang-an-nya
	pe-sat-nya
	peng-u-ji-an
	% Q
	%
	quality
	% R
	r-cnn
	re-le-van
	%
	% S
	Sub-bab
	state
	se-la-in
	stan-dar-di-sasi
	se-cu-ri-ty
	so-lu-si
	se-lu-ruh
	Soft-ware
	soft-ware
	se-buah
	se-ca-ra
	ShopeePay
	re-search
	au-to-re-gress-ive
	seg-men
	%
	% T
	% 
	ter-li-bat
	trans-for-mer
	ter-pi-sah-kan
	ter-mi-nal
	ter-ba-tas
	ter-na-ma
	trans-fer
	% U
	%
	un-tuk
	% V
	%
	% W
	%
	% X
	%
	% Y
	% 
	% Z
	%
}

\input{config/hypenation-en.tex}
%--------------------------------------------------------------------%
%
% Custom Commands and Definitions for Vincent Franstyo's Thesis
% OCR-free Payment Receipt Data Extraction System
%
%--------------------------------------------------------------------%

\newcommand{\namapenulis}{Vincent Franstyo}
\newcommand{\nimpenulis}{18221100}

\newcommand{\namadosbing}{Riza Satria Perdana}
\newcommand{\namadosbingfull}{Riza Satria Perdana, S.T., M.T.}
\newcommand{\nipdosbing}{19700609 199512 1 002}

\newcommand{\namadosenpengsatufull}{Monterico Adrian, S.T., M.T.}
\newcommand{\namadospengduafull}{Atina Putri, S.Kom., M.T.}

\newcommand{\tanggalpengesahan}{18 Juli 2025}

%--------------------------------------------------------------------%
% Technical Terms - Consistent Formatting
%--------------------------------------------------------------------%

% Core Technologies
\newcommand{\cv}{\emph{Computer Vision} (CV)}
\newcommand{\dl}{\emph{deep learning}}
\newcommand{\ml}{\emph{Machine Learning}}
\newcommand{\aifull}{\emph{Artificial Intelligence} (AI)}
\newcommand{\nlp}{\emph{Natural Language Processing} (NLP)}

% Model Architectures
\newcommand{\transformer}{Transformer}
\newcommand{\donut}{Donut}
\newcommand{\donutfull}{\emph{Document Understanding Transformer} (Donut)}
\newcommand{\swin}{Swin Transformer}
\newcommand{\bart}{BART}
\newcommand{\bartfull}{\emph{Bidirectional and Auto-Regressive Transformer} (BART)}
\newcommand{\layoutlm}{LayoutLM}
\newcommand{\bert}{BERT}

% Technical Components
\newcommand{\nn}{Neural Network}
\newcommand{\cnnfull}{\emph{Convolutional Neural Network} (CNN)}
\newcommand{\cnn}{CNN}
\newcommand{\rnn}{\emph{Recurrent Neural Network} (RNN)}
\newcommand{\vit}{\emph{Vision Transformer} (ViT)}

% Development Technologies
\newcommand{\flutter}{Flutter}
\newcommand{\onnx}{ONNX}
\newcommand{\pytorch}{PyTorch}
\newcommand{\tensorflow}{TensorFlow}

%--------------------------------------------------------------------%
% Acronyms - Small Caps Formatting
%--------------------------------------------------------------------%

% Financial/Payment Systems
\newcommand{\qris}{\MakeUppercase{qris}}
\newcommand{\qrisfull}{\MakeUppercase{qris} (\emph{Quick Response Code Indonesian Standard})}

% Technical Acronyms
\newcommand{\ocr}{\MakeUppercase{ocr}}
\newcommand{\ocrfull}{\MakeUppercase{ocr} (\emph{Optical Character Recognition})}
\newcommand{\api}{\MakeUppercase{api}}
\newcommand{\apifull}{\MakeUppercase{api} (\emph{Application Programming Interface})}
\newcommand{\gpu}{\MakeUppercase{gpu}}
\newcommand{\cpu}{\MakeUppercase{cpu}}
\newcommand{\ram}{\MakeUppercase{ram}}
\newcommand{\pdf}{\MakeUppercase{pdf}}
\newcommand{\json}{\MakeUppercase{json}}

% Methodology
\newcommand{\crisp}{\MakeUppercase{crisp-dm}}
\newcommand{\crispfull}{\MakeUppercase{crisp-dm} (\emph{Cross-Industry Standard Process for Data Mining})}

% Evaluation Metrics
\newcommand{\mcer}{\MakeUppercase{mcer}}
\newcommand{\mcerfull}{\MakeUppercase{mcer} (\emph{mean Character Error Rate})}

%--------------------------------------------------------------------%
% Indonesian Banks and Financial Institutions
%--------------------------------------------------------------------%

\newcommand{\bca}{\MakeUppercase{bca}}
\newcommand{\bni}{\MakeUppercase{bni}}
\newcommand{\bri}{\MakeUppercase{bri}}
\newcommand{\seabank}{SeaBank}
\newcommand{\neobank}{Neobank}

% E-wallets
\newcommand{\gopay}{Gopay}
\newcommand{\ovo}{\MakeUppercase{ovo}}
\newcommand{\shopeepay}{ShopeePay}

%--------------------------------------------------------------------%
% University and Academic Terms
%--------------------------------------------------------------------%

\newcommand{\itb}{\MakeUppercase{itb}}
\newcommand{\itbfull}{Institut Teknologi Bandung}
\newcommand{\stei}{\MakeUppercase{stei}}
\newcommand{\steifull}{Sekolah Teknik Elektro dan Informatika}
\newcommand{\stifull}{Sistem dan Teknologi Informasi}

%--------------------------------------------------------------------%
% Document and File Types
%--------------------------------------------------------------------%

\newcommand{\jpg}{\MakeUppercase{jpg}}
\newcommand{\jpeg}{\MakeUppercase{jpeg}}
\newcommand{\png}{\MakeUppercase{png}}
\newcommand{\latex}{\LaTeX}

%--------------------------------------------------------------------%
% Utility Commands
%--------------------------------------------------------------------%

% For Indonesian mixed with English terms
\newcommand{\sistem}{sistem}
\newcommand{\model}{model}
\newcommand{\arsitektur}{arsitektur}
\newcommand{\aplikasi}{aplikasi}

% For emphasis on first mention
\newcommand{\firstmention}[1]{\emph{#1}}

% For file names and code
\newcommand{\filename}[1]{\texttt{#1}}
\newcommand{\code}[1]{\texttt{#1}}

%--------------------------------------------------------------------%
% Thesis-Specific Commands
%--------------------------------------------------------------------%

% Your specific model and dataset names
\newcommand{\donutbase}{\texttt{naver-clova-ix/donut-base-finetuned-cord-v2}}
\newcommand{\mydataset}{Payment and Transfer Proof Dataset}

% Common Indonesian tech phrases
\newcommand{\berbasis}{berbasis}
\newcommand{\menggunakan}{menggunakan}
\newcommand{\implementasi}{implementasi}


\makeatletter

\makeatother

\addbibresource{references.thesis.bib}

\begin{document}

\title{Pengembangan Sistem Pemindaian dan Ekstraksi Data Otomatis dari Struk dan Bukti Pembayaran Berbasis \emph{Computer Vision} dan \emph{Deep Learning} Tanpa Penggunaan OCR}
\date{}
\author{
	\MakeUppercase{\namapenulis \\
		NIM: \nimpenulis}
}

\pagenumbering{roman}
\setcounter{page}{1}

\clearpage
\pagestyle{empty}

\begin{center}
	\smallskip

	\Large \bfseries {\thetitle}
	\vfill

	\Large Laporan Tugas Akhir
	\vfill

	\large Disusun sebagai syarat kelulusan tingkat sarjana
	\vfill

	\large Oleh

	\Large \theauthor

	\vfill
	\begin{figure}[ht]
		\centering
		\includegraphics[width=0.15\textwidth]{cover-ganesha.jpg}
	\end{figure}
	\vfill

	\large
	\uppercase{
		Program Studi Sistem dan Teknologi Informasi \\
		\MakeUppercase{\steifull \\ \itbfull}
	}

	\tanggalpengesahan

\end{center}

\clearpage

\input{chapters/ta/approval-1}
\chapter*{Lembar Pernyataan}

Dengan ini saya menyatakan bahwa:

\begin{enumerate}

	\item Pengerjaan dan penulisan Laporan Tugas Akhir ini dilakukan tanpa menggunakan bantuan yang tidak dibenarkan.
	\item Segala bentuk kutipan dan acuan terhadap tulisan orang lain yang digunakan di dalam penyusunan laporan tugas akhir ini telah dituliskan dengan baik dan benar.
	\item Laporan Tugas Akhir ini belum pernah diajukan pada program pendidikan di perguruan tinggi mana pun.

\end{enumerate}

Jika terbukti melanggar hal-hal di atas, saya bersedia dikenakan sanksi sesuai dengan Peraturan Rektor ITB No. 257 tahun 2019 tentang Penegakan Norma Akademik dan Kemahasiswaan Institut Teknologi Bandung.
\vspace{15mm}

Bandung, \tanggalpengesahan

\includegraphics[width=0.2\textwidth]{images/sign.png} \\
\namapenulis \\
NIM \nimpenulis


\pagestyle{plain}

\clearpage
\chapter*{ABSTRAK}
\addcontentsline{toc}{chapter}{ABSTRAK}
\begin{center}
	\center
	\begin{singlespace}
		\large\bfseries{\thetitle}

		\normalfont\normalsize
		Oleh:

		\bfseries \theauthor
	\end{singlespace}
\end{center}

\begin{singlespace}
	% \small
	QRIS menjadi metode pembayaran yang semakin populer di Indonesia, terutama di kalangan Gen Z. Namun, banyak Gen Z yang masih kesulitan dalam mencatat pengeluaran mereka secara manual. Tugas akhir ini bertujuan untuk mengembangkan sistem pencatatan pengeluaran berbasis \emph{mobile} yang dapat membantu pengguna, yaitu Gen Z, dalam mencatat pengeluaran mereka dengan lebih mudah. Sistem ini menggunakan model \donut{} untuk mengekstrak informasi penting dari gambar bukti pembayaran dan menyimpannya dalam format yang terstruktur untuk kemudian dapat ditampilkan kepada pengguna. Model \donut{} adalah model SOTA \emph{end-to-end} yang dapat digunakan untuk mengekstrak informasi dari dokumen tanpa memerlukan OCR. Metodologi penelitian menggunakan metodologi \dsrm. Pengembangan dilakukan secara bertahap, dimulai dari mengumpulkan data, eksplorasi data, \emph{modelling}, dan evaluasi terhadap model yang dihasilkan. Model \donut{} di \emph{fine-tune} pada \dataset{} CORD-v2 untuk dokumen struk pembayaran kertas dan \dataset{} QRIS-TF untuk dokumen pembayaran QRIS dan transfer. Dengan dua jenis model tersebut, layanan \emph{backend} DonutAPI dikembangkan dengan FastAPI sebagai antarmuka REST API untuk inferensi model. Aplikasi \emph{mobile} TrackMyBills dikembangkan dengan Flutter sebagai antarmuka pengguna. Pengujian pengalaman pengguna menunjukkan bahwa aplikasi memiliki tingkat kepuasan pengguna yang baik, dengan nilai SUS di angka \textbf{71,83} rata-rata di atas ambang batas nilai SUS, yaitu pada \textbf{68}.  \emph{Base model} menunjuk \fscore{} 84,68\%, dan \mcer{} 18,85\%. \emph{Custom model} menunjukkan  \fscore{} 81,50\%, dan \mcer{} 17,20\%. Hasil evaluasi menunjukkan bahwa aplikasi TrackMyBills memiliki tingkat kepuasan pengguna yang baik dan dinilai \emph{usable} dan tidak membingungkan oleh mayoritas responden. Aplikasi ini dapat membantu Gen Z dalam mencatat pengeluaran mereka dengan lebih mudah dan efisien.

	\textbf{Kata kunci: QRIS, Donut, Sistem Pencatatan Pengeluaran, Gen Z, Tanpa OCR}

\end{singlespace}
\clearpage
% \input{chapters/abstract-en}

\chapter*{Kata Pengantar}
\addcontentsline{toc}{chapter}{KATA PENGANTAR}

Puji dan syukur penulis panjatkan kepada Tuhan Yang Maha Esa atas berkat dan rahmatnya, laporan tugas akhir yang berjudul "\thetitle" dapat diselesaikan dalam rangka memenuhi syarat kelulusan tingkat sarjana. Perlu diakui pengerjaan tugas akhir ini didukung oleh banyak pihak. Khususnya, penulis ingin mengucapkan terima kasih kepada:

\begin{enumerate}
	\item Bapak \namadosbingfull, selaku dosen pembimbing atas segala bentuk dukungan yang telah diberikan dan kesabarannya dalam membimbing penulis serta memberikan saran dalam pengerjaan tugas akhir.
	\item Bapak \namadosenpengsatufull{} dan Bapak \namadospengduafull, selaku dosen penguji atas segala masukan dan kritik yang telah diberikan terhadap tugas akhir penulis.
	\item Bapak Ir. I Gusti Bagus Baskara Nugraha, S.T., M.T., Ph.D. dan Ibu Dr. Fetty Fitriyanti Lubis, S.T., M.T.  selaku dosen koordinator tim tugas akhir atas usahanya mengingatkan mahasiswa program studi \stifull{} untuk mengerjakan tugas akhirnya.
	\item Seluruh dosen program studi \stifull{} \itb{} yang telah memberikan ilmu pengetahuan yang sangat berharga bagi penulis.
	\item Ibu Lenny Wijaya dan Alm. Bapak Teddy selaku kedua orangtua penulis atas dukungan yang telah diberikan baik sebelum, selama, dan sesudah penulisan tugas akhir.
	\item Teman-teman SUDO 2021 yang telah menemani, memberikan inspirasi, serta dukungan moral kepada penulis dalam menempuh kuliah pada program studi \stifull.
	\item Kos Padma Homestay yang telah menjadi tempat menulis tugas akhir penulis, berkumpul dan menjalin silaturahmi dengan teman-teman baru, serta menjadi tempat yang nyaman untuk belajar.
	\item Teman-teman terdekat penulis yang tetap konsisten mengajak penulis bermain \textit{game}, berolahraga, dan melakukan kegiatan lain yang menyenangkan sehingga penulis tetap bisa menjaga kesehatan selama mengerjakan tugas akhir dan membuat kenangan-kenangan baru.
	\item Alysia yang telah menemani perjuangan dari TPB hingga saat ini, menjadi \textit{emotional support} di segala situasi, membantu penulis dalam proses menyelesaikan tugas akhir, serta membuat hari-hari menjadi lebih berwarna.
	\item Video Youtube \textit{Study With Me} yang telah menemani penulis dalam mengerjakan tugas akhir, memberikan motivasi, dan membuat penulis tetap fokus dalam menyelesaikan tugas akhir.
	\item Seluruh pihak lain yang tidak bisa disebutkan disini yang telah membantu dalam proses pengerjaan tugas akhir.
\end{enumerate}

Akhir kata, penulis mengucapkan terima kasih kepada semua pihak yang telah terlibat dalam pengerjaan tugas akhir ini. Penulis juga ingin menyampaikan mohon maaf apabila terdapat kesalahan maupun kekurangan dalam laporan tugas akhir ini. Penulis berharap semoga tugas akhir ini dapat bermanfaat bagi pembaca dan riset-riset kedepannya.

\begin{flushright}
	\vspace{0.5cm}
	Bandung, \tanggalpengesahan
	\vspace{1.5cm}

	% \includegraphics[width=0.15\textwidth]{images/sign.png}

	\namapenulis
\end{flushright}

% Temporarily override section formatting for TOC-related sections
% \titleformat*{\section}{\centering\bfseries\Large\MakeUpperCase}
\titlespacing*{\chapter}{0pt}{0pt}{4pt}

% Setting judul toc, lot, lof, bib
\renewcommand{\contentsname}{DAFTAR ISI}
\renewcommand{\listfigurename}{DAFTAR GAMBAR}
\renewcommand{\listtablename}{DAFTAR TABEL}
\renewcommand{\bibname}{DAFTAR PUSTAKA}

% daftar isi, lampiran, gambar, table
\tableofcontents
\listofappendices
\newpage

\listoffigures
\listoftables
\newpage

\listofequations

\newpage

% Uniform section formatting - all levels at 12pt (normalsize)
% MOVED: Section formatting moved after chapter config for proper application
% \titleformat*{\section}{\normalfont\normalsize\bfseries}
% \titleformat*{\subsection}{\normalfont\normalsize\bfseries}
% \titleformat*{\subsubsection}{\normalfont\normalsize\bfseries}

% Custom 4th level formatting (subsubsubsection)
% \titleformat{\subsubsubsection}
% {\normalfont\normalsize\bfseries}{\theparagraph}{1em}{}
% \titlespacing*{\subsubsubsection}
% {0pt}{3.25ex plus 1ex minus .2ex}{1.5ex plus .2ex}
\pagenumbering{arabic}

%----------------------------------------------------------------%
% Konfigurasi Bab
%----------------------------------------------------------------%
\setcounter{page}{1}
\renewcommand{\chaptername}{BAB}
\renewcommand{\thechapter}{\Roman{chapter}}
%----------------------------------------------------------------%

% CRITICAL: Apply uniform section formatting - all levels at 12pt (normalsize)
% This placement after chapter config ensures it overrides document class defaults
\makeatletter
\titleformat*{\section}{\normalfont\normalsize\bfseries}
\titleformat*{\subsection}{\normalfont\normalsize\bfseries}
\titleformat*{\subsubsection}{\normalfont\normalsize\bfseries}

% Custom 4th level formatting (subsubsubsection)
\titleformat{\subsubsubsection}
{\normalfont\normalsize\bfseries}{\theparagraph}{1em}{}
\titlespacing*{\subsubsubsection}
{0pt}{3.25ex plus 1ex minus .2ex}{1.5ex plus .2ex}

% Force font size consistency - override any document class variations
\def\section{\@startsection{section}{1}{\z@}{-3.5ex plus -1ex minus -.2ex}{2.3ex plus .2ex}{\normalfont\normalsize\bfseries}}
\def\subsection{\@startsection{subsection}{2}{\z@}{-3.25ex plus -1ex minus -.2ex}{1.5ex plus .2ex}{\normalfont\normalsize\bfseries}}
\def\subsubsection{\@startsection{subsubsection}{3}{\z@}{-3.25ex plus -1ex minus -.2ex}{1.5ex plus .2ex}{\normalfont\normalsize\bfseries}}
\makeatother

%----------------------------------------------------------------%
% Dafter Bab
% Untuk menambahkan daftar bab, buat berkas bab misalnya `chapter-6` di direktori `chapters`, dan masukkan ke sini.
%----------------------------------------------------------------%
\chapter{Pendahuluan}
\label{chapter:pendahuluan}

\section{Latar Belakang}
\label{sec:latarbelakang}

Transaksi pembayaran merupakan bagian penting dari aktivitas ekonomi, baik dalam lingkup individu maupun organisasi. Proses ini mencakup pengalihan dana antara pihak yang terlibat untuk memenuhi kewajiban finansial. Dalam perkembangannya, metode pembayaran telah bertransformasi dari yang awalnya menggunakan uang tunai dan cek menjadi transfer bank dan sistem yang lebih modern berbasis digital. Transformasi ini bertujuan untuk meningkatkan efisiensi, keamanan, dan kenyamanan dalam melakukan transaksi.

\qrisfull{} menjadi salah satu faktor pesatnya perkembangan teknologi pembayaran digital di Indonesia. \qris{} diluncurkan oleh Bank Indonesia pada tahun 2019 dan menjadi metode pembayaran yang populer dalam waktu kurang dari lima tahun karena kemudahan dan efisiensinya. Generasi Z dan millenial, sebagai pengguna utama metode pembayaran ini, merasakan kemudahan dalam melakukan pembayaran atau pengeluaran uang. Goodstats menunjukkan bahwa 38\% Gen Z menggunakan \qris{} dalam kehidupan sehari-hari, sementara di kalangan millenial angkanya mencapai
25\%.

Pada April 2024, Bank Indonesia melaporkan jumlah pengguna \qris{} mencapai angka 48,12 juta dan jumlah merchant yang menggunakan mencapai 31,61 juta. Total nilai transaksi dengan menggunakan metode pembayaran \qris{} telah mencapai Rp 31,65 triliun, yaitu meningkat sebesar 149,46\% secara tahunan
pada bulan Februari 2024. Peningkatan ini mencerminkan adopsi yang signifikan di kalangan masyarakat, khususnya Gen Z dan millenial, yang cenderung memilih
transaksi non-tunai. Survei dari Jawa Pos Radar Lawu mengungkapkan bahwa 57\% dari kedua generasi ini lebih memilih transaksi non-tunai, dengan \qris{} menjadi salah satu metode yang paling efektif.

\newpage

Kemudahan ini mendorong peningkatan frekuensi transaksi digital di kalangan pengguna yang mengakibatkan kenaikan pada angka volume transaksi. Namun, seiring dengan meningkatnya volume transaksi, belum terdapat sistem dengan mekanisme pencatatan transaksi. Banyak pengguna masih harus melakukan pencatatan manual untuk memantau pengeluaran mereka yang rentan terhadap kesalahan dan memakan waktu. Salah satu solusi potensial untuk masalah ini adalah penerapan teknologi berbasis \cvfull{} dan \dl{} yang dapat melakukan pemindaian dan ekstraksi data dari dokumen gambar secara akurat, seperti \ocrfull, \cnnfull, \transformer, dan lainnya.

\layoutlm{} dan \bert{} merupakan model transformer yang menggabungkan informasi visual dan tekstual untuk memahami dokumen. Namun, kedua model ini masih bergantung pada \ocr. Ketergantungan ini membuat kedua model tersebut kurang efisien. Oleh karena itu, sistem yang dapat mengonversi dan mengekstrak data dari bukti pembayaran berbasis kertas menjadi dokumen digital yang tidak bergantung pada OCR menjadi sistem yang diperlukan.

\section{Rumusan Masalah}
\label{sec:rumusanmasalah}

Berdasarkan latar belakang dan masalah yang dihadapi oleh individu untuk melakukan pengelolaan bukti dan struk pembayaran, dapat dirumuskan masalah yang akan diselesaikan adalah sebagai berikut.

\begin{center}
	"Bagaimana cara mengembangkan sistem yang dapat mengonversi dan mengektraksi data dari bukti dan struk pembayaran yang berbasis kertas dan bervariasi menjadi dokumen digital?"
\end{center}

\section{Tujuan}
\label{sec:tujuan}

Berdasarkan rumusan masalah yang telah didefinisikan pada \autoref{sec:rumusanmasalah}, didapatkan jawaban dari rumusan masalah dan hasil akhir yang ingin diperoleh, yaitu mengembangkan sistem yang dapat digunakan untuk memindai, mengonversi, dan mengekstrak data dari struk dan bukti pembayaran dengan format yang berbeda-beda menjadi dokumen digital.

\section{Batasan Masalah}
\label{sec:batasanmasalah}

Dalam pemenuhan solusi untuk menjawab masalah yang ada, terdapat berbagai jenis batasan yang perlu dipertimbangkan. Batasan-batasan tersebut ditentukan berdasarkan jangka waktu yang tersedia dan lingkup masalah yang akan diselesaikan. Berikut adalah batasan selama pengerjaan tugas akhir ini:
\begin{enumerate}
	\item Sistem hanya dapat memproses dokumen keuangan, seperti bukti pembayaran QRIS, transfer, dan struk yang dicetak dan tidak ditulis
	\item Dokumen keuangan yang diterima hanya dalam format gambar digital dengan format JPG, JPEG, atau PNG dengan kualitas gambar yang jelas dan tidak blur.
	\item  Pengembangan sistem dibatasi untuk memproses dokumen dari bank-bank digital ternama, yaitu SeaBank dan Neobank, bank konvensional ternama, yaitu BCA, aplikasi pembayaran digital populer, yaitu dan \gopay{}.
	\item  Proses ekstraksi data dari struk pembayaran dibatasi pada informasi esensial transaksi yaitu:
	      \begin{enumerate}
		      \item Nominal pembayaran
		      \item ID transaksi
		      \item Aplikasi pembayaran
		      \item Penerima pembayaran
		      \item Tanggal dan waktu transaksi
		      \item Tipe transaksi
	      \end{enumerate}
	\item Sistem dikembangkan untuk memproses satu dokumen bukti pembayaran dalam satu waktu.
	\item Pengujian sistem akan dilakukan dengan dataset minimal 100 sampel bukti pembayaran yang dikumpulkan dengan memperhatikan variasi sampel bukti pembayaran.
	\item Sistem dirancang untuk beroperasi dalam \emph{platform mobile} Android dan tidak mencakup pengembangan untuk \emph{platform desktop} atau iOS.
	\item Implementasi tidak mencakup integrasi dengan sistem akuntansi atau sistem pencatatan keuangan pihak ketiga.
	\item Pengembangan antarmuka pengguna dibatasi pada fitur-fitur dasar yang diperlukan untuk mengunggah gambar, memroses, dan menampilkan hasil ekstraksi data.
\end{enumerate}

\section{Metodologi}
\label{sec:metodologi}

Metodologi yang akan digunakan pada tugas akhir ini adalah \dsrmfull. Metodologi \dsrm{} menjadi metodologi yang cocok untuk digunakan karena menggabungkan prinsip, praktik, dan prosedur yang diperlukan, serta memenuhi tujuan menghasilkan artefak (sistem) yang dapat digunakan untuk mempresentasikan dan mengevaluasi penelitian dalam bidang Sistem Informasi \parencite{peffers2007dsrm}. Tahapan \dsrm{} tergambar pada \autoref{fig:dsrm}.

\begin{figure}[htbp]
	\centering
	\includegraphics[width=.8\textwidth]{images/dsrm.png}
	\caption{Proses dalam Metodologi \dsrm{} \parencite{peffers2007dsrm}.}
	\label{fig:dsrm}
\end{figure}

\dsrm{} mencakup beberapa tahapan, yakni sebagai berikut:
\begin{enumerate}
	\item \emph{Problem Identification and Motivation}~\\
	      Tahapan \emph{problem identification and motivation} bertujuan untuk mengidentifikasi masalah yang ada dan ingin diselesaikan. Pada tahap ini, peneliti melakukan analisis terhadap permasalahan yang ada, yaitu mencari cara untuk mengembangkan sistem yang dapat mengonversi dan mengekstrak data dari bukti-bukti pembayaran digital dan struk pembayaran kertas yang dicetak digital dengan format yang bervariatif menjadi dokumen digital seperti yang telah dijelaskan pada \autoref{sec:latarbelakang} dan \autoref{sec:rumusanmasalah}.
	\item \emph{Define the Objectives for a Solution}~\\
	      Pada tahapan ini, peneliti akan mendefinisikan tujuan dari solusi yang akan dikembangkan. Tujuan dari penelitian ini adalah mengembangkan sistem berbasis \emph{computer vision} dan \dl{} untuk memindai, mengonversi, dan mengekstrak data dari bukti pembayaran dengan format yang berbeda-beda secara akurat dan cepat untuk mendukung pencatatan dan pengelolaan transaksi keuangan secara digital seperti yang telah dijelaskan pada \autoref{sec:tujuan}.
	\item \emph{Design and Development}~\\
	      Tahapan \emph{design and development} merupakan tahapan untuk merancang dan mengembangkan sistem yang akan digunakan untuk menyelesaikan masalah yang telah diidentifikasi. Pada tahap ini, peneliti akan merancang arsitektur sistem, membuat rancangan tahapan desain dari arsitektur sistem, dan mengembangkan sistem sesuai dengan spesifikasi yang telah ditentukan.	Peneliti juga akan mengumpulkan dan melakukan eksplorasi pada data yang akan digunakan untuk melatih model \dl{} yang akan digunakan dalam sistem. Data yang digunakan adalah data bukti pembayaran digital dan struk pembayaran kertas yang dicetak digital dengan format yang bervariatif. Eksplorasi data dilakukan untuk mengidentifikasi variasi data yang diperlukan untuk mengembangkan sistem sesuai dengan spesifikasi yang diinginkan.
	\item \emph{Demonstration}~\\
	      Tahapan \emph{demonstration} merupakan tahapan untuk mendemonstrasikan sistem yang telah dikembangkan. Pada tahap ini, peneliti akan melakukan pengujian terhadap sistem yang telah dikembangkan untuk memastikan bahwa sistem dapat berfungsi sesuai dengan spesifikasi yang telah ditentukan. Pengujian dilakukan dengan menggunakan \dataset{} asli yang telah dikumpulkan.
	\item \emph{Evaluation}~\\
	      Tahapan \emph{evaluation} merupakan tahapan untuk mengevaluasi sistem yang telah dikembangkan. Pada tahap ini, peneliti akan melakukan evaluasi terhadap sistem yang telah dikembangkan dengan metrik yang telah didefinisikan untuk setiap fitur yang ada pada sistem. Evaluasi akan dibagi menjadi dua jenis, yaitu evaluasi terhadap fungsionalitas sistem dan evaluasi terhadap performa model.
	\item \emph{Communication}~\\
	      Tahapan \emph{communication} merupakan tahapan untuk mengkomunikasikan hasil penelitian yang telah dilakukan. Pada tahap ini, peneliti akan menyusun laporan penelitian yang akan digunakan sebagai media untuk mengkomunikasikan hasil penelitian yang telah dilakukan. Laporan penelitian akan mencakup pendahuluan, studi literatur, analisis masalah, desain solusi, evaluasi, dan kesimpulan dan saran yang perlu disampaikan dan diketahui oleh pembaca.
\end{enumerate}

\chapter{Studi Literatur}
\label{chapter:studi-literatur}

\section{Dasar Teori}
\label{sec:dasar-teori}
Subbab dasar teori merupakan subbab yang akan memberikan pengenalan
terhadap permasalahan yang akan dihadapi. Subbab ini juga akan memberikan konteks dan keterangan lebih lanjut mengenai teknologi-teknologi yang akan digunakan atau dipertimbangkan sebagai alternatif solusi dari permasalahan yang diangkat. Berikut merupakan landasan-landasan teori yang akan digunakan dalam penelitian ini.

\subsection{Pemrosesan Dokumen Finansial}
\label{subsec:pemrosesan-dokumen-finansial}

Dokumen struk pembayaran memegang peranan penting dalam pengelolaan
keuangan baik individu maupun institusi. Proses ekstraksi data dari dokumen-dokumen ini sering kali menghadapi tantangan yang signifikan, termasuk keanekaragaman tata letak, kualitas gambar yang bervariasi, dan kebutuhan untuk
mengekstrak informasi dengan akurasi tinggi. Sebagai contoh, elemen-elemen penting seperti total harga, tanggal transaksi, dan nomor rekening sering kali tersebar di berbagai posisi dalam dokumen yang berbeda, sehingga membuat ekstraksi data secara manual menjadi lambat dan rentan terhadap kesalahan.

Teknologi yang umumnya dapat digunakan untuk menanggulangi masalah ini adalah \cv{}, salah satunya \ocr. Secara tradisional, \ocr{} digunakan untuk mengonversi teks dalam gambar menjadi format digital yang dapat diproses lebih lanjut. Namun, OCR memiliki
keterbatasan, terutama dalam menangani dokumen dengan kualitas gambar rendah atau format teks yang tidak standar. Selain itu, ketergantungan pada OCR dapat meningkatkan kompleksitas dan biaya pemrosesan \parencite{kim2021donut}.

\dlfl{} memiliki berbagai jenis algoritma dan model yang mendukung pengembangan sistem tersebut. \dl{} merupakan cabang \ml{} yang memanfaatkan jaringan saraf tiruan berlapis untuk mengenali pola kompleks dalam data, termasuk dokumen finansial. Salah satu model yang umum digunakan adalah \cnn. \cnn{} dirancang untuk memproses data berbentuk gambar dengan mendeteksi fitur lokal seperti teks, garis, atau elemen visual lainnya sehingga cocok untuk ekstraksi data dari dokumen. Namun, \cnn{} terbatas dalam memahami hubungan global antar
elemen dalam dokumen sehingga \transformer{} hadir sebagai solusi yang lebih canggih \parencite{alzubaidi2021review}.

Dengan mekanisme \attention-nya, \transformer{}, seperti LayoutLM dan
Donut, mampu menangkap hubungan kontekstual antar elemen secara global. Hal ini membuat penggunaan \transformer{} dengan struktur kompleks dan informasi
tersebar. \transformer{} memberikan pendekatan yang kuat untuk memastikan akurasi dan efisiensi dalam pemrosesan dokumen finansial.

\subsection{\emph{Deep Learning}}
\label{subsec:dl}

\dlfl{} adalah cabang dari \ml{} yang menggunakan \annfull{} dengan banyak lapisan (\emph{deep}) untuk mempelajari representasi data yang kompleks. Deep Learning memungkinkan komputer untuk menganalisis data tidak terstruktur, seperti teks, gambar, audio, dan video dengan tingkat akurasi yang sangat tinggi. Model-model \dl{} belajar dengan cara memproses data melalui lapisan-lapisan neuron atau saraf yang dirancang untuk mengekstrak fitur-fitur penting, baik yang eksplisit maupun tersembunyi \parencite{Goodfellow-et-al-2016}. 

Terdapat beberapa model dan arsitektur yang dapat digunakan dalam \dl. Setiap model tersebut ditujukan untuk memenuhi kasus yang berbeda-beda. Berikut adalah arsitektur dan model yang relevan dengan kasus ekstraksi data dari dokumen finansial:

\begin{enumerate}
	\item \cnn\--\transformer{} (Kombinasi \cnn{} dan \transformer)~\\
	      \cnn{} dan \transformer{} adalah dua metode yang dapat digunakan secara terpisah atau bersama untuk menangani data visual. \cnn{} adalah salah satu jenis \MakeLowercase{{\nn{}}} yang paling sering digunakan pada data gambar. CNN bisa digunakan untuk mendeteksi dan mengenali fitur signifikan tanpa supervisi manusia pada sebuah gambar \parencite{alzubaidi2021review}. \transformer{} adalah sebuah arsitektur \MakeLowercase{\nn{}} yang meng-\emph{encode} data menjadi fitur-fitur melalui mekanisme \attention. Secara garis besar, \transformer{} akan membagi gambar menjadi beberapa \patch{} dan mengkalkulasi representasinya dan hubungan antar \patch{} tersebut \parencite{han2021transformer}. 
	\item \crnnfull~\\
	      \crnn{} adalah kombinasi dari \cnn{} dan \rnn{} yang merupakan dua model yang paling sering digunakan. \cnn{} digunakan pada tahap awal untuk mengekstrak fitur visual dan gambar. Hasil dari implementasi \cnn{} akan digunakan oleh model \rnn{} sebagai data input untuk untuk menangkap hubungan antar elemen teks, seperti urutan huruf dalam kata atau angka dalam bilangan \parencite{wang2019convolutional}. \crnn{} telah diaplikasikan pada klasifikasi musik, audio, dan klasifikasi data dan teks yang bersifat \emph{hyperspectral}.
	\item Model \objectdetection~\\
	      Model \objectdetection{} adalah salah satu model dalam \dl{} yang digunakan untuk mendeteksi dan mengenali objek-objek tertentu dalam sebuah gambar atau dokumen, termasuk elemen-elemen seperti teks, tabel, atau simbol pada dokumen finansial. Dua jenis utama model \objectdetection{} yang sering digunakan adalah \yolofull{} dan \rcnnfull{}.
	      \begin{enumerate}
		      \item \yolo~\\
		            \yolo{} adalah model \objectdetection{} yang dirancang untuk kecepatan tinggi dan efisiensi. \yolo{} memproses seluruh gambar dalam satu tahap dengan membagi gambar menjadi \grid{}, kemudian memprediksi bounding box dan kelas objek secara bersamaan di setiap \grid{} \parencite{diwan2023object}. Model ini terkenal karena kecepatannya, membuatnya cocok untuk aplikasi \emph{real-time}, meskipun memiliki  kompromi dalam akurasi untuk objek kecil atau yang saling berdekatan.
		      \item \rcnn~\\
		            \rcnn{} menggunakan pendekatan dua tahap. \rcnn{} menghasilkan proposal wilayah yang berpotensi mengandung objek pada tahap pertama. Pada tahap kedua, setiap proposal dianalisis lebih mendalam menggunakan \cnn{} untuk mengklasifikasikan objek dan memperbaiki bounding box. \rcnn{} dan variannya (Fast \rcnn{} dan Faster \rcnn{}) menawarkan akurasi tinggi, tetapi memerlukan waktu komputasi lebih lama sehingga lebih cocok untuk aplikasi yang tidak membutuhkan deteksi \emph{real-time} \parencite{xie2021oriented}. 
	      \end{enumerate}
\end{enumerate}

\subsection{Transformer}
\label{subsec:transformer}

\transformer{} adalah suatu jenis arsitektur jarigan yang baru dalam \dl{} yang digunakan untuk mentransformasi sebuah deretan data menjadi sesuatu dengan karakteristik, seperti panjang atau format, yang berbeda. Transformer tidak menggunakan lapisan rekursif (\rnn) atau konvolusi (\cnn).  Transformer menggunakan mekanisme \selfattention{} yang memungkinkannya untuk memahami hubungan antar elemen dalam sebuah deretan atau urutan
(\sequence). Tidak seperti pada \rnn{}, mekanisme \selfattention tidak perlu memperhatikan jarak antar elemen. Hal ini membuat \transformer{} menjadi sangat efisien untuk memproses data sekuensial. \transformer{} banyak digunakan pada proses penerjemahan bahasa, rangkuman teks, dan pengenalan suara.

\transformer{} terdiri atas dua komponen utama, yaitu \encoder{} dan \decoder. \encoderfl{} bertugas untuk membaca dan memproses sequence masukan, sementara \decoder{} menghasilkan \sequence{} keluaran berdasarkan informasi dari \encoder.

\encoderfl{} dan \decoder{} terdiri dari beberapa lapisan yang identik, dengan masing-masing memiliki dua sub-lapisan utama, yaitu:

\begin{enumerate}
	\item \mha~\\
	      Sublapisan ini menggunakan mekanisme \selfattention{} untuk memahami hubungan antar elemen dalam sebuah \sequence{}. Misalnya, dalam sebuah kalimat, mekanisme ini dapat mengenali bahwa kata "dia"
	      merujuk pada "ibu" meskipun terdapat kata-kata lain di antaranya.
	      \mha{} merupakan gabungan beberapa lapisan (\layer)
	      \emph{Scaled Dot-Product Attention} atau akan disingkat menjadi \attention. Perbandingan proses pada \mha{} dan \emph{Scaled Dot-Product Attention} dapat dilihat pada \autoref{fig:attention}

	      \begin{figure}[htbp]
		      \centering
		      \includegraphics[width=.8\textwidth]{images/attentionmha.png}
		      \caption{\emph{Scaled Dot-Product Attention} (kiri) dan \mha{} (kanan) yang merupakan beberapa \layer{} \attention{} berjalan paralel \parencite{vaswani2017attention}}
		      \label{fig:attention}
	      \end{figure}

	      $Q$ (\emph{Query}), $K$ (\emph{Key}), dan $V$ (\emph{Value}) adalah representasi vektor dari setiap elemen dalam sebuah \sequence. Simbol $d_k$ adalah dimensi dari vektor $K$, yang digunakan untuk menskalakan hasil perkalian \emph{dot-product} $QK^\mathsf{T}$ agar proses pelatihan tetap stabil. Fungsi Softmax kemudian mengubah skor tersebut menjadi bobot probabilistik untuk menentukan elemen mana yang paling relevan untuk diperhatikan.

	      \pagebreak

	      Perhitungan \attention{} dilakukan dengan formula seperti yang ditunjukkan pada persamaan \eqref{eq:attention-softmax} \parencite{vaswani2017attention} yang didefinisikan sebagai berikut:

	      \begin{equation}
		      \label{eq:attention-softmax}
		      \operatorname{Attention}(Q, K, V) = \operatorname{SoftMax}\left(\frac{QK^\mathsf{T}}{\sqrt{d_k}}\right)V
	      \end{equation}
		  \addcontentsline{loe}{myequations}{\protect\numberline{\theequation}Persamaan Softmax}

	\item \ffnfull~\\
	      Setelah \attention{} dihitung, informasi dari setiap elemen diproses melalui jaringan \emph{feed-forward} yang sama untuk setiap posisi dalam urutan. Jaringan ini terdiri dari dua lapisan linear dengan fungsi aktivasi ReLU pada persamaan \eqref{eq:ffn} \parencite{vaswani2017attention}:

	      \begin{equation}
		      \label{eq:ffn}
		      \operatorname{FFN}(x) = \max(0, xW_1 + b_1)W_2 + b_2
	      \end{equation}
		  \addcontentsline{loe}{myequations}{\protect\numberline{\theequation}Fungsi Aktivasi ReLU pada \ffn}

	      \decoderfl{} memiliki perbedaan signifikan dibandingkan dengan \encoder.
	      \encoderfl{} terdiri dari beberapa lapisan yang masing-masing memiliki mekanisme \selfattention{} dan FFN. Setiap elemen dalam \sequence{} dapat saling “memperhatikan”. \decoderfl{} memiliki lapisan tambahan yang \encoder{}\--\decoder{} \attention. Lapisan ini memungkinkan \decoder{} untuk "memperhatikan" keluaran dari \encoder. \decoderfl{} memiliki mekanisme \emph{masking} untuk memastikan bahwa posisi saat ini hanya bergantung pada posisi sebelumnya.

	      \transformer{} memiliki keunggulan dibandingkan dengan metode konvensional pemrosesan sekuensial, yaitu \rnn. \transformer{} tidak bergantung
	      pada pemrosesan berurutan. \transformer{} dapat memproses seluruh urutan secara bersamaan, membuat pelatihan jauh lebih cepat. Implementasi \transformer{} menunjukkan hasil terbaik di berbagai kasus, seperti penerjemahan bahasa,
	      dibandingkan dengan arsitektur lainnya.

\end{enumerate}


\subsection{Transformer}
\label{subsec:transformer}

Transformer adalah suatu jenis arsitektur jarigan yang baru dalam deep learning yang digunakan untuk mentransformasi sebuah deretan data menjadi sesuatu dengan karakteristik, seperti panjang atau format, yang berbeda. Transformer tidak menggunakan lapisan rekursif (RNN) atau konvolusi (CNN).  Transformer menggunakan mekanisme self-attention yang memungkinkannya 
untuk memahami hubungan antar elemen dalam sebuah deretan atau urutan 
(sequence). Tidak seperti pada RNN, mekanisme self-attention tidak perlu 
memperhatikan jarak antar elemen. Hal ini membuat transformer menjadi sangat efisien untuk memproses data sekuensial. Transformer banyak digunakan pada proses penerjemahan bahasa, rangkuman teks, dan pengenalan suara.  

Transformer terdiri atas dua komponen utama, yaitu encoder dan decoder. Encoder bertugas untuk membaca dan memproses sequence masukan, sementara decoder menghasilkan sequence keluaran berdasarkan informasi dari encoder.  

Encoder dan decoder terdiri dari beberapa lapisan yang identik, dengan 
masing-masing memiliki dua sublapisan utama, yaitu:


\subsection{\swin}
\label{subsec:swin}

Swin Transformer adalah sebuah arsitektur Vision Transformer yang 
dirancang untuk backbone dalam berbagai tugas computer vision seperti klasifikasi 
gambar, deteksi objek, dan segmentasi semantik. Nama "Swin" berasal dari konsep 
"Shifted Window" yang menjadi elemen utama dalam desainnya. Tidak seperti 
Vision Transformer (ViT) yang menggunakan metode self-attention secara global 
pada seluruh gambar, Swin Transformer memperkenalkan self-attention berbasis 
local window yang secara signifikan mengurangi kompleksitas komputasi (Liu dkk., 2021).

\subsection{\bartfull}
\label{subsec:bart}

\bartfull adalah adalah model \ml{} yang dirancang untuk memahami, menghasilkan, dan merekonstruksi teks dalam bahasa alami (\emph{natural language}). \bart{} merupakan \emph{denoising autoencoder} yang dilatih untuk memperbaiki teks yang telah dirusak oleh berbagai transformasi sehingga mampu mengembalikan teks ke bentuk aslinya \parencite{lewis2019bart}. Arsitektur \bart{} menggabungkan kelebihan model \bert{} yang memiliki \emph{encoder bidirectional} dan GPT yang menggunakan \emph{decoder autoregressive}. Hal ini menjadikannya sangat fleksibel untuk berbagai tugas \nlpfull.

\bart{} dibangun di atas arsitektur transformer yang terdiri atas dua 
komponen utama, yaitu \emph{bidirectional encoder} dan \textit{autoregressive decoder}. 
\emph{Bidirectional encoder} akan mengolah teks input dengan cara memahami hubungan antar token secara dua arah. \emph{Autoregressive decoder} akan menghasilkan teks secara berurutan, token demi token, dengan mempertimbangkan \emph{sequence} yang sudah dihasilkan.

\autoref{fig:bart} menunjukkan cara \bart{} bekerja. Teks input akan dirusak terlebih dahulu, kemudian dibaca oleh \encoder. Hasil bacaan \encoder{} akan diteruskan ke \decoder{} untuk mengembalikan bagian teks yang “dirusak”. Dengan kemampuannya untuk merekonstruksi teks dari masukan yang rusak, \bart{} menjadi suatu \transformer{} yang sangat baik untuk memahami struktur bahasa dan menghasilkan teks yang koheren walaupun masukan dinilai 
rusak. Penggunaan \bart{} mirip dengan \bert{} dan GPT, seperti klasifikasi teks, generasi teks, dan penejermahan teks. 

\begin{figure}
\centering
\includegraphics[width=0.8\textwidth]{images/bart.png}
\caption{Cara kerja \bart{} \parencite{lewis2019bart}.}
\label{fig:bart}
\end{figure}




% \subsection{\onnx}
\label{subsec:onnx}

\emph{Open Neural Network Exchange} \onnx{} adalah format standar \emph{open-source} untuk merepresentasikan model \ml{} yang memungkinkan interoperabilitas antara berbagai \emph{framework} \dl. Dikembangkan oleh Facebook dan Microsoft pada tahun 2017, \onnx{} telah berkembang menjadi proyek yang lulus dari Linux Foundation AI dengan dukungan dari perusahaan teknologi besar termasuk IBM, Intel, AMD, ARM, Qualcomm, dan NVIDIA \parencite{onnxgithub2019}.

\onnx{} berfungsi sebagai representasi universal struktur komputasi dari \nn. Komponen arsitektur inti \onnx{} terdiri dari beberapa elemen penting. \emph{Node} merepresentasikan operasi matematika. \emph{Edge} merepresentasikan tensor yang mengalir antar operasi. \emph{Initializer} berfungsi untuk menyimpan bobot model dan konstanta yang diperlukan.

% Konsep teknis utama \onnx{} meliputi representasi menengah (\emph{Intermediate Representation}) yang berfungsi sebagai representasi universal untuk menangkap struktur graf komputasi dari jaringan neural. \onnx{} dibangun menggunakan format serialisasi \emph{Protocol Buffers} (protobuf) Google untuk penyimpanan dan transmisi model yang efisien. Format ini juga mendefinisikan \emph{Operator Set} (Opset) yang merupakan koleksi operator berversi untuk mempertahankan kompatibilitas mundur. Arsitektur \onnx{} merepresentasikan jaringan neural sebagai graf terarah asiklik (DAG) dimana \emph{node} merepresentasikan operasi dan \emph{edge} merepresentasikan aliran data.

% \subsubsection{Interoperabilitas dan Optimasi}

\onnx{} menyediakan interoperabilitas yang memungkinkan konversi model ke berbagai format, seperti \pytorch, \tensorflow, Keras, Scikit-learn, dan \emph{framework} lainnya. Hal ini memungkinkan \emph{deployment} tunggal yang memungkinkan pengembang untuk melakukan pelatihan model pada \emph{framework} apapun dan di-\emph{deploy} menggunakan \onnx Runtime. Format ini bersifat agnostik terhadap \emph{hardware} dan memungkinkan model berjalan di berbagai jenis \emph{hardware} seperti CPU, GPU, dan akselerator khusus seperti TPU.

% Dari aspek optimasi, \onnx{} menyediakan optimasi graf otomatis yang mencakup \emph{constant folding}, \emph{operator fusion}, dan eliminasi \emph{node} redundan. Format ini juga menyediakan optimasi spesifik \emph{hardware} yang memanfaatkan \emph{kernel} khusus untuk platform \emph{hardware} berbeda. Efisiensi memori juga menjadi keunggulan dengan alokasi memori yang dioptimalkan dan manajemen \emph{lifecycle} tensor yang efisien.

% \subsubsection{Aplikasi dalam \ml dan \cv}

% \onnx memiliki aplikasi luas dalam berbagai tugas \ml dan \cv. Dalam klasifikasi gambar, \onnx mendukung model seperti ResNet, EfficientNet, dan MobileNet untuk klasifikasi \emph{real-time}. Untuk deteksi objek, format ini kompatibel dengan model \yolo, SSD, dan \rcnn untuk berbagai aplikasi. Aplikasi lainnya mencakup segmentasi semantik dengan model U-Net dan DeepLab, sistem \ocr dan pemahaman dokumen untuk deteksi dan pengenalan teks, serta pengenalan wajah untuk sistem \emph{real-time} dan biometrik.

% Implementasi industri \onnx mencakup berbagai layanan besar seperti layanan Microsoft yang meliputi pencarian Bing, aplikasi Office, dan Azure Cognitive Services. Dalam industri otomotif, \onnx digunakan dalam sistem mengemudi otonom. Sektor kesehatan memanfaatkan \onnx untuk analisis pencitraan medis dan sistem diagnostik, sementara manufaktur menggunakannya untuk sistem kontrol kualitas dan deteksi cacat \parencite{onnxruntime2020}.


\subsection{\flutter}
\label{subsec:flutter}

\flutter{} adalah SDK (\emph{Software Development Kit}) UI \emph{open-source} dari Google yang diluncurkan pada tahun 2017, dirancang sebagai \emph{framework cross-platform} yang memungkinkan pengembang membuat aplikasi yang dikompilasi secara \emph{native} untuk mobile, web, desktop, dan perangkat \emph{embedded} dari satu basis kode. \emph{Framework} ini menggunakan bahasa pemrograman Dart dan mengimplementasikan paradigma UI reaktif dan deklaratif \parencite{flutter2021}.

\flutter memiliki beberapa karakteristik teknis utama yang membedakannya dari \emph{framework} pengembangan aplikasi lainnya, yaitu arsitektur \emph{single codebase} yang memungkinkan . Arsitektur basis kode tunggal memungkinkan pendekatan tulis sekali dan \emph{deploy} di mana saja (\emph{write once, deploy anywhere}), memberikan efisiensi tinggi dalam pengembangan. Model pemrograman reaktif mengimplementasikan paradigma UI = f(state) dimana antarmuka pengguna bereaksi terhadap perubahan \emph{state} secara otomatis. Kompilasi \emph{native} memungkinkan \flutter mengkompilasi ke kode mesin ARM/Intel untuk kinerja optimal yang mendekati aplikasi \emph{native} asli.

Arsitektur berbasis \emph{widget} menjadi ciri khas \flutter dimana segala sesuatu adalah \emph{widget} yang merupakan deskripsi \emph{immutable} dari komponen UI. Fitur \emph{hot reload} memungkinkan \emph{stateful hot reload} selama development untuk iterasi cepat tanpa kehilangan \emph{state} aplikasi, meningkatkan produktivitas pengembang secara signifikan.

\subsubsection{Arsitektur Berlapis}

\flutter terdiri dari empat \layer utama yang saling terintegrasi. \emph{Layer Framework} yang ditulis dalam Dart mencakup library \emph{widget} Material Design dan Cupertino untuk implementasi desain platform-specific. \emph{Layer widget} menyediakan abstraksi komposisi untuk membangun UI yang kompleks, sementara \emph{layer rendering} menangani manajemen \emph{layout} dan perhitungan posisi elemen. \emph{Layer foundation} menyediakan layanan inti seperti animasi, \emph{painting}, dan \emph{gestures} yang fundamental untuk interaksi pengguna.

\emph{Layer Engine} yang ditulis dalam C++ berisi mesin grafis Skia untuk \emph{rendering} dengan transisi menuju Impeller untuk performa yang lebih baik. Runtime Dart dan \emph{virtual machine} berjalan pada layer ini untuk eksekusi kode aplikasi. \emph{Layout} teks dan operasi \emph{file} I/O juga dikelola pada layer ini, bersama dengan implementasi tingkat rendah dari \api inti \flutter.

\emph{Layer Embedder} bersifat spesifik platform dan menangani integrasi dengan sistem operasi yang mendasarinya. Untuk Android menggunakan Java/C++, sedangkan iOS menggunakan Swift/Objective-C. Layer ini bertanggung jawab untuk koordinasi layanan sistem operasi, manajemen \emph{event loop}, dan eksposur \api spesifik platform untuk fungsionalitas yang tidak tersedia secara cross-platform.

\subsubsection{\emph{Cross-Platform Development}}

\flutter menyediakan efisiensi pengembangan yang signifikan dengan pengurangan waktu pengembangan hingga 50\% dibandingkan pengembangan \emph{native} tradisional. Satu tim pengembang dapat menargetkan \emph{multiple platform} secara bersamaan, mengurangi kebutuhan tenaga kerja dan kompleksitas manajemen proyek. Lingkungan pengembangan dan \emph{tooling} yang terpadu memungkinkan konsistensi dalam proses development, sementara logika bisnis dan komponen UI dapat dibagi antar platform.

Konsistensi menjadi aspek penting dengan UI yang identik di semua platform, memberikan pengalaman pengguna yang seragam. Perilaku dan kinerja yang konsisten mengurangi bug spesifik platform dan memberikan pengalaman \emph{brand} yang terpadu. Hal ini sangat penting untuk aplikasi komersial yang memerlukan identitas visual yang kuat di berbagai platform.

\subsubsection{Integrasi dengan \ml dan \cv}

\flutter menyediakan beberapa jalur untuk integrasi \ml yang relevan untuk aplikasi \cv. Integrasi \tensorflow Lite dimungkinkan melalui paket \texttt{tflite\_flutter} yang memungkinkan inferensi model langsung pada perangkat. Dukungan tersedia untuk model kustom dan \emph{pre-trained}, dengan kompatibilitas \emph{multi-platform} untuk Android, iOS, dan Desktop. Akselerasi \emph{hardware} juga didukung melalui \gpu dan \emph{Neural Processing Units} untuk inferensi yang lebih cepat.

Firebase ML Kit menyediakan \api ML \emph{on-device} untuk tugas umum seperti pengenalan teks, deteksi wajah, dan \emph{scanning barcode}. \emph{Labeling} gambar dan deteksi \emph{landmark} juga tersedia dengan kemampuan pemrosesan \emph{real-time} yang memadai untuk aplikasi produksi. Kemampuan \cv dalam \flutter mencakup pemrosesan \emph{stream} kamera \emph{real-time} untuk aplikasi yang memerlukan analisis video langsung. Klasifikasi gambar dan deteksi objek dapat diimplementasikan dengan mudah, begitu juga dengan implementasi \ocr untuk pengenalan teks. Pengenalan wajah dan autentikasi biometrik juga didukung untuk aplikasi keamanan \parencite{flutteronnx2022}.


\subsection{FastAPI}
\label{subsec:fastapi}

FastAPI adalah \emph{framework} \emph{web} Python modern dan berperforma tinggi untuk membangun \api{} dengan Python 3.7+ berdasarkan \emph{type hints} standar Python. Dibuat oleh Sebastián Ramírez, \emph{framework} ini merepresentasikan evolusi signifikan dalam pengembangan \emph{web} Python. FastAPI menggabungkan kesederhanaan Flask dengan kekuatan pemrograman \emph{asynchronous} modern \parencite{ramirez2020fastapi}.

FastAPI memiliki integrasi sistem \emph{type} sebagai fitur unggulan yang memanfaatkan sistem \emph{type hint} Python untuk validasi data otomatis, serialisasi, dan pembuatan dokumentasi secara otomatis. FastAPI memiliki \emph{tools} dokumentasi yang sudah terintegrasi, yaitu OpenAPI (sebelumnya Swagger). \emph{Framework} ini juga menggunakan skema \json yang memberikan kompatibilitas yang luas dengan ekosistem pengembangan modern. FastAPI dibangun di atas \emph{Asynchronous Server Gateway Interface} (ASGI) yang mewakili kemajuan arsitektur yang signifikan dibandingkan \emph{framework} berbasis \emph{Web Server Gateway Interface} (WSGI) tradisional \parencite{ramirez2020fastapi}. 

% Komponen arsitektur ASGI meliputi \emph{scope} yang berisi informasi \emph{scope} koneksi dengan tipe protokol dan metadata yang diperlukan. \emph{Receive} berfungsi sebagai \emph{callable awaitable} untuk menerima \emph{event} dari \emph{client}, sementara \emph{send} bertindak sebagai \emph{callable awaitable} untuk mengirim respons ke \emph{client}.

% Arsitektur berlapis FastAPI terdiri dari beberapa \layer terintegrasi yang bekerja secara harmonis. \emph{Application Layer} merupakan aplikasi inti FastAPI yang menangani \emph{routing} dan \emph{dependency injection} dengan sistem yang sophisticated. \emph{Framework Layer} menggunakan Starlette yang menyediakan kompatibilitas ASGI dan fungsionalitas \emph{framework} web yang diperlukan. \emph{Validation Layer} memanfaatkan Pydantic untuk validasi dan serialisasi data dengan performa tinggi, sementara \emph{Server Layer} menggunakan \emph{server} ASGI Uvicorn untuk penanganan \emph{request} HTTP yang efisien.

% \subsubsection{\emph{Async Programming} dan Kinerja}

% Model \emph{async processing} membedakan FastAPI dari \emph{framework synchronous} tradisional dengan memproses \emph{request} secara asinkron. \emph{Event loop single-threaded} menangani berbagai koneksi bersamaan tanpa overhead context switching yang tinggi. \emph{Coroutines} memungkinkan fungsi \emph{async} yang dapat dihentikan dan dilanjutkan sesuai kebutuhan, sementara \emph{non-blocking} I/O memberikan penanganan operasi I/O yang efisien tanpa memblokir \emph{thread} utama.

% Analisis kinerja berdasarkan \emph{benchmark} TechEmpower independen menunjukkan kinerja superior FastAPI dalam berbagai aspek. \emph{Throughput} FastAPI mencapai sekitar 3 kali lebih tinggi dari Flask dalam kondisi beban tinggi. P99 \emph{latency} yang lebih rendah dicapai karena pemrosesan \emph{async} yang efisien, sementara efisiensi memori menunjukkan \emph{footprint} memori yang berkurang per \emph{request}. Karakteristik \emph{scaling} linear dengan \emph{concurrent requests} memungkinkan aplikasi menangani beban yang meningkat dengan graceful degradation.

% \subsubsection{Integrasi dengan Sistem \cv dan \ml}

% FastAPI menyediakan arsitektur optimal untuk \emph{deployment} model \ml dalam konteks \cv dengan berbagai kemampuan teknis. Pemrosesan \emph{async} memungkinkan pemrosesan gambar \emph{non-blocking} untuk \emph{throughput} tinggi, sangat penting untuk aplikasi yang menangani volume gambar besar. Manajemen memori yang efisien memungkinkan penanganan data gambar besar tanpa degradasi performa yang signifikan. Dukungan \emph{streaming} memungkinkan kemampuan pemrosesan video \emph{real-time} untuk aplikasi yang memerlukan analisis kontinyu, sementara pemrosesan bersamaan memungkinkan \emph{multiple pipeline} pemrosesan gambar berjalan secara paralel.

% Arsitektur \emph{model serving} yang disediakan FastAPI mencakup tahapan yang komprehensif. \emph{Model loading} terjadi saat \emph{startup} aplikasi untuk memuat model \ml ke memori dengan efisien. \emph{Preprocessing pipeline} menangani tahap preprocessing gambar sebelum inferensi untuk memastikan format data yang sesuai. Inferensi model dilakukan dengan eksekusi model \ml untuk prediksi yang akurat dan cepat. \emph{Postprocessing} memproses hasil prediksi untuk format yang diinginkan sesuai kebutuhan aplikasi, dan \emph{response serialization} melakukan serialisasi hasil ke format \json atau format lainnya sesuai spesifikasi \api.

% FastAPI mendukung integrasi dengan berbagai \emph{library} \ml Python seperti \pytorch, \tensorflow, dan \onnx Runtime, menjadikannya pilihan yang sesuai untuk \emph{deployment} model \cv dalam lingkungan produksi \parencite{techempowerbenchmark2023}.


\subsection{Metrik Evaluasi}
\label{subsec:metrik-evaluasi}

Evaluasi kinerja sistem \ml{} memerlukan metrik yang komprehensif dan sesuai dengan domain aplikasi. Sistem pemahaman dokumen dan pengenalan teks memerlukan kombinasi metrik klasifikasi standar dan metrik khusus untuk evaluasi pemahaman dokumen. Metrik evaluasi yang digunakan dalam penelitian ini mencakup \accuracy, \precision, \recall, \fscore, \coverage, dan \mcer{} (\emph{Mean Character Error Rate}).

\subsubsection{Metrik Klasifikasi Standar}

\accuracyfl{} mengukur proporsi prediksi yang benar baik positif maupun negatif di seluruh \emph{instance} dalam \dataset. Formula \accuracy{} didefinisikan sebagai perbandingan antara jumlah prediksi benar dengan total prediksi yang dibuat. Dalam konteks klasifikasi biner, \accuracy{} dihitung menggunakan matriks yang terdiri dari \emph{True Positive} (TP), \emph{True Negative} (TN), \emph{False Positive} (FP), dan \emph{False Negative} (FN). Persamaan \eqref{eq:accuracy} menunjukkan perhitungan \accuracy{} \parencite{jayaswal2020evalmetrics}. 

\begin{equation}
    \label{eq:accuracy}
\text{Accuracy} = \frac{TP + TN}{TP + TN + FP + FN}
\end{equation}
\addcontentsline{loe}{myequations}{\protect\numberline{\theequation}Persamaan \accuracy}

\precisionfl{} mengukur proporsi prediksi positif yang benar di antara semua prediksi positif yang dibuat model. Metrik ini sangat penting ketika biaya \emph{false positive} tinggi dalam aplikasi tertentu. Persamaan \eqref{eq:precision} menunjukkan perhitungan \precision. Perhitungan \precision{} didefinisikan sebagai rasio antara \emph{True Positive} (TP) dengan jumlah total prediksi positif yang dibuat, yaitu jumlah \emph{True Positive} ditambah \emph{False Positive} \parencite{jayaswal2020evalmetrics}. \precisionfl{} tinggi menunjukkan bahwa model meminimalkan \emph{false positive}, memastikan prediksi positif kemungkinan besar benar dan dapat diandalkan untuk pengambilan keputusan.

\begin{equation}
    \label{eq:precision}
\text{Precision} = \frac{TP}{TP + FP}
\end{equation}
\addcontentsline{loe}{myequations}{\protect\numberline{\theequation}Persamaan \precision}

\recallfl{} mengukur proporsi \emph{instance} positif aktual yang berhasil diidentifikasi dengan benar oleh model. Metrik ini kritikal ketika biaya \emph{false negative} tinggi dalam sistem yang memerlukan deteksi lengkap. Persamaan \eqref{eq:recall} menunjukkan rumus yang dapat digunakan untuk menghitung \recall{} \parencite{jayaswal2020evalmetrics}.\recallfl{} tinggi berarti model menangkap sebagian besar \emph{instance} positif, meminimalkan kemungkinan melewatkan data penting yang harus dideteksi.

\begin{equation}
    \label{eq:recall}
\text{Recall} = \frac{TP}{TP + FN}
\end{equation}
\addcontentsline{loe}{myequations}{\protect\numberline{\theequation}Persamaan \recall}

\fscore{} adalah rata-rata harmonik dari \precision{} dan \recall{}. \fscore{} memberikan ukuran seimbang yang mempertimbangkan kedua metrik secara setara. Persamaan \eqref{eq:fscore} menunjukkan rumus yang dapat digunakan untuk menghitung \fscore{} \parencite{jayaswal2020evalmetrics}. \fscore{} tinggi menunjukkan bahwa model tidak hanya akurat dalam prediksi positif tetapi juga menangkap sebagian besar \emph{instance} positif yang relevan. \fscore{} memerlukan \precision{} dan \recall{} tinggi untuk mencapai skor tinggi. \fscore{} akan menjadi nol jika salah satu dari \precision{} atau \recall{} adalah nol.

\begin{equation}
    \label{eq:fscore}
\text{F1-Score} = 2 \times \frac{\text{Precision} \times \text{Recall}}{\text{Precision} + \text{Recall}}
\end{equation}
\addcontentsline{loe}{myequations}{\protect\numberline{\theequation}Persamaan \fscore}

% \input{src/chapters/ta/chapter-2/dasar-teori/ted.tex}

\subsubsection{\mcer{} (\emph{Mean Character Error Rate})}

\mcer{} adalah rata-rata \emph{Character Error Rate} (CER) di beberapa dokumen atau segmen teks yang memberikan ukuran komprehensif akurasi deteksi teks \parencite{neudecker2021survey}. Metrik ini sangat relevan untuk sistem pengenalan teks yang bergantung pada akurasi pengenalan karakter. Persamaan \eqref{eq:mcer} menunjukkan rumus yang digunakan untuk menghitung \mcer{} dengan $\text{CER}_i$ adalah \emph{Character Error Rate} untuk dokumen $i$. Metrik ini mengukur kesalahan pengenalan karakter pada tingkat granular dan memberikan informasi tentang akurasi pengenalan teks pada level karakter.

\begin{equation}
    \label{eq:mcer}
\text{mCER} = \frac{\sum{S_i + D_i + I_i}}{\sum{N_i}}
\end{equation}
\addcontentsline{loe}{myequations}{\protect\numberline{\theequation}Persamaan \emph{Mean Character Error Rate} (mCER)}

Persamaan \eqref{eq:cer} menunjukkan cara perhitungan CER dengan parameter yang meliputi $S$ yang merepresentasikan jumlah substitusi karakter, $D$ untuk jumlah \emph{deletion} atau penghapusan karakter, $I$ untuk jumlah \emph{insertion} atau penambahan karakter, dan $N$ sebagai total jumlah karakter dalam \emph{Ground Truth}.

\begin{equation}
    \label{eq:cer}
\text{CER}_i = \frac{S_i + D_i + I_i}{N_i}
\end{equation}
\addcontentsline{loe}{myequations}{\protect\numberline{\theequation}Persamaan \emph{Character Error Rate} (CER)}

% \mcer{} digunakan secara luas dalam berbagai aplikasi evaluasi \ocr. \emph{Benchmark} mesin \ocr{} menggunakan \mcer{} untuk membandingkan kinerja algoritma yang berbeda dan memantau perkembangan dari waktu ke waktu. Proyek digitalisasi menggunakan metrik ini untuk menilai kualitas hasil konversi dokumen fisik ke digital.

% \subsubsection{Integrasi Metrik untuk Evaluasi Sistem}

% Evaluasi sistem pemahaman dokumen yang komprehensif memerlukan pendekatan \emph{multi-level} yang mengintegrasikan berbagai metrik. Tingkat karakter menggunakan \mcer untuk mengukur akurasi pengenalan teks individual dan memastikan setiap karakter dapat dibaca dengan benar. Tingkat \emph{field} memanfaatkan \accuracy, \precision, \recall, dan F1-\emph{score} untuk evaluasi ekstraksi \emph{field} spesifik dalam dokumen terstruktur. Tingkat struktur menggunakan \ted untuk mengevaluasi pemahaman struktur dokumen keseluruhan dan hubungan antar elemen. Tingkat sistem mengkombinasikan semua metrik untuk evaluasi kinerja sistem secara holistik.

% Interpretasi dan \emph{trade-off} antar metrik memberikan perspektif berbeda tentang kinerja sistem. \accuracy memberikan gambaran umum kinerja tetapi dapat menyesatkan pada data tidak seimbang sehingga perlu dikombinasikan dengan metrik lain. \precision menjadi penting ketika biaya \emph{false positive} tinggi dalam aplikasi yang memerlukan presisi tinggi. \recall kritikal ketika biaya \emph{false negative} tinggi dalam sistem yang memerlukan deteksi lengkap. F1-\emph{score} memberikan keseimbangan antara \precision dan \recall untuk evaluasi menyeluruh. \ted mengukur pemahaman struktur dokumen yang kompleks dan hubungan hierarkis antar elemen, sementara \mcer memberikan ukuran granular akurasi pengenalan teks pada level karakter.

% Penggunaan metrik evaluasi yang komprehensif memastikan bahwa sistem pemahaman dokumen tidak hanya akurat dalam mengenali teks, tetapi juga mampu memahami struktur dokumen dan mengekstrak informasi relevan dengan presisi tinggi yang diperlukan untuk berbagai aplikasi \parencite{bille2005tree}.

\subsubsection{\emph{System Usability Scale} (SUS)}
\label{subsubsec:sus}

\emph{System Usability Scale} (SUS) adalah instrumen evaluasi usabilitas yang dikembangkan oleh Brooke pada tahun 1996 sebagai alat ukur cepat dan andal untuk mengevaluasi usabilitas sistem \parencite{brooke1996sus}. SUS terdiri dari sepuluh pertanyaan dengan skala \emph{Likert} lima poin yang dirancang untuk memberikan skor tunggal yang merepresentasikan penilaian subjektif pengguna terhadap usabilitas sistem. Instrumen ini telah menjadi salah satu metode evaluasi usabilitas yang paling banyak digunakan dalam interaksi manusia komputer karena kesederhanaan, reliabilitas, dan validitasnya \parencite{bangor2008empirical}.

SUS menggunakan sepuluh pertanyaan standar yang mencakup aspek-aspek fundamental usabilitas, termasuk efektivitas, efisiensi, dan kepuasan pengguna. Lima pertanyaan dinyatakan secara positif dan lima lainnya secara negatif untuk mengurangi kemungkinan respons bias dan meningkatkan keandalan pengukuran. Setiap pertanyaan dinilai pada skala 1 hingga 5, dengan 1 menunjukkan "sangat tidak setuju" dan 5 menunjukkan "sangat setuju". Pertanyaan SUS meliputi penilaian terhadap keinginan menggunakan sistem secara teratur, kompleksitas sistem yang dirasakan, kemudahan penggunaan, kebutuhan dukungan teknis, integrasi fungsi sistem, inkonsistensi, kemudahan pembelajaran, dan kepercayaan diri pengguna dalam menggunakan sistem \parencite{brooke1996sus}.

Perhitungan skor SUS mengikuti metodologi khusus untuk menghasilkan skor akhir antara 0 hingga 100. Untuk pertanyaan bernomor ganjil (pernyataan positif), skor dihitung dengan mengurangi 1 dari respons pengguna. Untuk pertanyaan bernomor genap (pernyataan negatif), skor dihitung dengan mengurangi respons pengguna dari 5. Jumlah total skor dari sepuluh pertanyaan kemudian dikalikan dengan 2,5 untuk menghasilkan skor SUS akhir. Persamaan \eqref{eq:sus-score} menunjukkan formula perhitungan skor SUS dengan $R_i$ adalah respons untuk pertanyaan $i$ \parencite{tullis2013measuring}.

\begin{equation}
    \label{eq:sus-score}
\text{SUS Score} = \left(\sum_{i=1,3,5,7,9} (R_i - 1) + \sum_{j=2,4,6,8,10} (5 - R_j)\right) \times 2.5
\end{equation}
\addcontentsline{loe}{myequations}{\protect\numberline{\theequation}Persamaan Skor SUS}

Interpretasi skor SUS menggunakan \emph{benchmark} yang telah ditetapkan berdasarkan penelitian ekstensif terhadap ribuan evaluasi usabilitas. Skor SUS di atas 68 dianggap diatas rata-rata, skor antara 68-80 menunjukkan usabilitas yang baik, skor 80-90 menunjukkan usabilitas yang sangat baik, dan skor di atas 90 menunjukkan usabilitas yang luar biasa. Skor di bawah 68 menunjukkan bahwa sistem memerlukan perbaikan usabilitas yang signifikan \parencite{bangor2009determining}. Penelitian oleh Sauro pada tahun 2011 menunjukkan bahwa skor SUS memiliki korelasi yang kuat dengan metrik usabilitas objektif seperti tingkat penyelesaian tugas dan efisiensi \parencite{sauro2011measuring}.

Keunggulan SUS sebagai instrumen evaluasi meliputi kemudahan administrasi, waktu pengisian yang singkat (2-5 menit), skor tunggal yang mudah dipahami, dan validitas yang telah terbukti melalui berbagai domain aplikasi \parencite{lewis2018sus}. SUS juga memungkinkan perbandingan usabilitas antar sistem yang berbeda dan \emph{benchmarking} terhadap standar industri.

Dalam konteks evaluasi sistem \emph{mobile} untuk ekstraksi data pembayaran, SUS menjadi instrumen yang relevan untuk mengukur sejauh mana pengguna Gen Z dapat menggunakan aplikasi dengan mudah dan efektif. Evaluasi usabilitas dengan SUS dapat mengidentifikasi apakah antarmuka aplikasi memenuhi ekspektasi pengguna dalam hal kemudahan penggunaan, efisiensi proses ekstraksi data, dan kepuasan pengguna secara keseluruhan terhadap sistem yang dikembangkan.




\section{Penelitian Terkait}
\label{sec:penelitian-terkait}

Subbab penelitian terkait memberikan rangkuman dari penelitian-penelitian yang pernah dilakukan pada kasus serupa. Subbab ini ditujukan sebagai referensi untuk mendapatkan pengetahuan alternatif solusi atau pendekatan penelitian serupa. Berikut adalah penelitian-penelitian terkait tersebut. 

\subsection{\textit{A Comprehensive Analysis of LayoutLM and Donut for Document Classification}}
\label{sec:penelitian-1}
Penelitian oleh \textciteyear{bajrami2023comprehensive}  akan melakukan analisis komparasi dari dua buah \textit{pre-trained model} berbasis \transformer{}, yaitu \layoutlm{} dan \donut, untuk klasifikasi dokumen. Penelitian ini melakukan perbandingan performa kedua model pada dua \dataset{} dengan karakteristik berbeda dengan melakukan evaluasi 
terhadap nilai \accuracy, \precision, dan \fscore{} dari kedua model. 

\subsubsection{Gambaran Umum}
Penelitian ini memiliki tujuan untuk membandingkan performa dari model \donut{} dan \layoutlm{} untuk melakukan klasifikasi dokumen. Studi ini akan 
melakukan komparasi menggunakan sejumlah data yang terdiri dari data-data yang tidak terstuktur (\textit{unstructured data}), seperti struk, \textit{invoice}, dan catatan tulis tangan.

\subsubsection{Analisis dan Metode}
\layoutlm{} adalah model yang mengombinasikan deteksi objek, pengenalan teks, dan analisis \textit{layout}. \layoutlm{} memiliki dependensi terhadap teknologi \ocr{} untuk melakukan ekstraksi teks dari dokumen berbentuk gambar. \donut{} adalah sebuah model \transformer{} untuk melakukan pemahaman dokumen yang telah bersifat \sotafull{} dan tidak memerlukan integrasi \ocr{}.

Analisis komparasi kedua model ini dilakukan pada dua jenis \dataset. \datasetfl{} pertama adalah sebuah \dataset{} dengan jumlah 10.000 sampel data yang terdiri dari dokumen perusahaan dengan kompleksitas yang lebih tinggi. Dataset 
kedua terdiri dari 50.000 sampel yang lebih terdiversifikasi. Kedua model di-\textit{fine-tuned} dengan menggunakan \textit{transfer learning} dengan \textit{pre-trained weights} dan evaluasi dilakukan dengan metrik \accuracy, \precision, dan \fscore.  

Pada \dataset{} 1, hasil yang didapatkan dari model \layoutlm{} memiliki akurasi yang lebih tinggi, yaitu pada angka 0,88 dibandingkan dengan penggunaan model \donut{} pada angka 0,74. Hal ini mengindikasikan bahwa \layoutlm{} lebih unggul untuk menangani dokumen dengan struktur yang lebih kompleks dibandingkan \donut{}. Pada \dataset{} 2, \donut{} memiliki angka akurasi yang lebih tinggi, yaitu 0,91, dibandingkan \layoutlm{} dengan angka 0,82 \parencite{bajrami2023comprehensive}.

\subsubsection{Kesimpulan}

\layoutlm{} dan \donut{} merupakan pilihan yang efektif untuk melakukan klasifikasi dokumen, tetapi dengan kebutuhan yang berbeda. \layoutlm{} lebih 
cocok untuk digunakan pada dokumen dengan kompleksitas yang lebih tinggi, namun hasil yang didapatkan lebih lama karena memiliki \textit{pipeline} yang lebih 
panjang akibat adanya proses \ocr{} terlebih dahulu. \donut{} lebih cocok untuk digunakan pada \dataset{} dengan diversifikasi dokumen yang lebih banyak dan dapat menghasilkan dengan lebih cepat akibat \emph{pipeline} yang lebih pendek akibat tidak perlu integrasi dengan \ocr. Dokumen yang memiliki \textit{noise} atau resolusi yang buruk akan mengurangi akurasi secara signifikan baik untuk \layoutlm{} atau \donut.
\subsection{\textit{An End-to-End OCR-Free Solution for Identity Document \linebreak Understanding}}
\label{sec:penelitian-2}
Penelitian yang dilakukan oleh \cite{carta2024end}  menunjukkan penggunaan model \donut{} untuk melakukan ekstraksi data dari dokumen identitas. Dokumen yang digunakan dapat berupa kartu identitas elektronik, kartu identitas, surat izin 
mengemudi, dan kartu kesehatan. Evaluasi dilakukan dengan membandingkan model yang telah mengalami proses \textit{fine-tuning} dan model basis yang ada dengan mempertimbangkan faktor \fscore, \mcerfull, \tedfull. 

\subsubsection{Gambaran Umum}
Sebuah penelitian oleh \cite{carta2024end} mempresentasikan hasil implementasi \donut{} untuk pengenalan dokumen identitas. Tujuan utama dari penelitian ini adalah menyediakan solusi untuk menjawab tantangan dalam melakukan ekstraksi dan verifikasi data dokumen identitas secara manual pada servis digital. Penelitian ini menggunakan teknologi terbaru pada \textit{Document Understanding} (DU) dan \ml{} untuk otomasi pengenalan dokumen identitas. Solusi yang digunakan menggunakan proses \textit{fine-tuning} dua tahap, yaitu model \textit{pre-trained} dan pelatihan pada \dataset{} sintetis dan asli. Hal ini dilakukan dengan tujuan untuk mengekstrak informasi dari kartu identitas secara efisien. Selama pengembangan, terdapat isu seperti data asli yang terbatas, kualitas gambar yang buruk, dan format identitas yang beragam. 

\subsubsection{Analisis dan Metode}
Solusi yang ditawarkan menggunakan arsitektur \transformer{} \donut{} yang mengutilisasi \swin{} sebagai \encoder{} dan BART sebagai \decoder{}. Arsitektur ini memberikan implementasi yang bebas dari \ocr{} untuk ekstraksi teks dan pemahaman \emph{layout}. Walaupun \donut{} adalah \textit{pre-trained model}, \donut{} tetap memerlukan dataset yang besar untuk melalui proses \emph{fine-tuning}. Terdapat dua fase \emph{fine-tuning} yang dilakukan, yaitu \emph{fine-tuning} dengan data sintetis dengan format, konten, dan \emph{layout} bervariasi dan \emph{fine-tuning} dengan data dokumen identitas asli untuk meningkatkan akurasi solusi yang ditawarkan. Dokumen identitas yang digunakan terdiri dari \emph{Electronic ID Card} (EIC), \emph{Identity Card} (IC), \emph{Health Card} (HC), dan \emph{Driving License} (DL). Evaluasi terhadap solusi dilakukan dengan menggunakan metrik \fscore, \mcerfull, dan \tedfull. Eksperimen dilakukan pada 20\% \emph{test data} dan 80\% \emph{train data} dengan hasil perbandingan implementasi \donut{} sebelum dan setelah \textit{fine-tuning} disajikan pada \autoref{tab:donut-comparison-on-id-documents}. 

\begin{table}[h!]
    \centering % Centers the table on the page
    \caption{Hasil perbandingan implementasi Donut sebelum dan setelah \textit{fine-tuning} \parencite{carta2024end}.}
    \label{tab:donut-comparison-on-id-documents}
    \begin{tabularx}{\textwidth}{|X|C|C|C|C|C|C|}
        \hline
        % --- First Header Row ---
        % The "Dokumen" cell spans 2 rows vertically.
        % The "Donut Base" and "Donut Synth" cells each span 3 columns horizontally.
        \multirow{2}{*}{Dokumen} & \multicolumn{3}{c|}{Donut Base} & \multicolumn{3}{c|}{Donut Synth} \\
        \cline{2-7} % Creates a horizontal line from column 2 to 7
        
        % --- Second Header Row ---
        % The first column is left blank because the multirow cell is occupying it.
        & TED & F1 & mCER & TED & F1 & mCER \\
        \hline% Double line to separate header from body
        
        % --- Table Body ---
        EIC & 0.821 & 0.441 & 0.136 & 0.893 & 0.554 & 0.084 \\ \hline
        IC  & 0.769 & 0.594 & 0.171 & 0.885 & 0.712 & 0.090 \\ \hline
        HC  & 0.959 & 0.891 & 0.039 & 0.986 & 0.943 & 0.014 \\ \hline
        DL  & 0.929 & 0.813 & 0.057 & 0.973 & 0.866 & 0.025 \\ \hline
    \end{tabularx}
\end{table}

\subsubsection{Kesimpulan}
Implementasi solusi dengan menggunakan arsitektur \donut{} dapat  melakukan ekstraksi informasi secara akurat dari dokumen identitas tanpa memerlukan implementasi \ocr{} yang konvensional. Proses \textit{fine-tuning} dua tahap sangat direkomendasikan karena ketersediaan data asli yang terbatas \parencite{carta2024end}. Eksperimen yang dilakukan mengonfirmasi proses \textit{fine-tuning} akan meningkatkan performa dari \emph{base model}.  
\subsection{\textit{The Future of Document Indexing: GPT and Donut Revolutionize Table of Content Processing}}
\label{sec:penelitian-3}
Penelitian yang dilakukan oleh \textciteyear{feyisa2024future} ini mempresentasikan bagaimana model dengan arsitektur \donut{} dan GPT-3.5 Turbo dapat digunakan untuk melakukan ekstraksi informasi dari dokumen tanpa perlu adanya pendekatan manual yang dilakukan. Dokumen yang digunakan merupakan data yang terstruktur dan bersifat kompleks. Hasil dari model yang digunakan disajikan dalam bentuk 
\textit{dashboard} dan dievaluasi dengan menggunakan akurasi setiap model yang digunakan.

\subsubsection{Gambaran Umum}
Penelitian ini menjelaskan mengenai bagaimana teknologi AI seperti \donut{} dan GPT-3.5 Turbo membuat gebrakan baru pada pemrosesan \emph{document indexing} dan daftar isi untuk dokumen-dokumen kompleks. Permasalahan yang ditemukan adalah ekstraksi informasi secara manual dari dokumen yang kompleks dan panjang sangat memakan waktu dan rawan kesalahan. 

\subsubsection{Analisis dan Metode}
Solusi yang digunakan adalah solusi berbasis AI dengan menggunakan model dengan arsitektur \donut{} untuk melakukan ekstraksi data dari gambar dan dokumen dan GPT-3.5 Turbo untuk melakukan retrukturisasi daftar isi. Sistem yang dirancang akan mengidentifikasi halaman daftar isi dan melakukan ekstraksi \emph{headings} serta \emph{subheadings}-nya. Hasil ekstraksi akan dikonversi menjadi data dengan format \json{} yang terstruktur untuk digunakan pada \emph{dashboard} atau \emph{database}.  

Data atau dokumen yang digunakan akan dikonversi menjadi gambar untuk \donut{} dan teks untuk GPT untuk kemudian digunakan untuk melakukan \emph{fine-tuning} model pada fase \emph{training model} \parencite{feyisa2024future}. Model yang telah di-\emph{fine-tune} ini akan melakukan identifikasi halaman daftar isi, melakukan ekstraksi, dan merestrukturisasi \emph{headings} serta \emph{subheadings} dalam format \json. Hasil yang telah diekstrak dalam format \json{} akan diintegrasikan ke dalam \emph{dashboard} yang telah dibuat dalam bentuk \emph{website}. Evaluasi terhadap model dilakukan dengan menggunakan \accuracy{}. Model dengan arsitektur Donut yang digunakan mencapai nilai \accuracy{} hingga 85\% dan GPT-3.5 Turbo mencapai \accuracy{} 89\%. 

\subsubsection{Kesimpulan}

Penggunaan \emph{Large Language Model} (LLM) dan teknologi \cv, seperti OpenAI GPT-3.5 Turbo dan \donut, memberikan solusi yang efisien untuk menyusun informasi dari dokumen besar secara otomatis. Dengan kemampuan untuk mengorganisasi data, seperti daftar isi dalam dokumen spesifikasi, teknologi ini mampu mengurangi risiko kesalahan, waktu, dan biaya yang berlebihan. 
% Pendekatan ini menjadi langkah inovatif yang mendukung pengelolaan dokumen teknis yang kompleks dan meningkatkan efisiensi dan akurasi dalam berbagai sektor industri. 
% \blankpage
\section{System Design and Implementation}

\subsection{System Architecture Overview}
The TrackMyBills system adopts a client-server architecture that separates user interface concerns from computational-intensive deep learning processing. The architecture consists of two primary components: a Flutter-based mobile application for user interaction and a FastAPI-based backend service for model inference. This separation allows the system to provide responsive user experience while leveraging cloud-based processing power for the deep learning models.

The mobile application handles all user-facing functionality including image capture, preprocessing, result visualization, and local data storage. The application implements a clean architecture pattern that separates presentation, domain, and data layers, facilitating maintainable code and testable components. User interactions flow through well-defined interfaces that abstract the complexity of backend communication and data management.

The backend service provides RESTful API endpoints for model inference, implementing both base and custom Donut models for different document types. The service architecture supports concurrent request processing and includes comprehensive error handling and logging for production deployment. Model loading and inference are optimized for response time while maintaining accuracy standards required for practical expense tracking applications.

\begin{figure}[htbp]
    \centerline{\includegraphics[width=0.45\textwidth]{images/use-case-diagram.png}}
    \caption{Use Case Diagram of TrackMyBills System}
    \label{fig:usecase}
\end{figure}

The use case diagram shown in Figure \ref{fig:usecase} illustrates the primary interactions between users and the system. Users can upload images from camera or gallery, crop images for optimal processing, initiate data extraction, review and edit extracted information, categorize transactions, and save records for future reference. Each use case is designed to minimize user effort while providing sufficient control over the extraction and categorization process.

\subsection{Mobile Application Design}
The TrackMyBills mobile application implements a user-centered design philosophy that prioritizes ease of use and visual clarity. The application workflow guides users through a streamlined process from image capture to transaction recording, minimizing the number of steps required to complete expense tracking tasks. The interface design follows Material Design principles while incorporating visual elements that appeal to Gen Z users.

The image capture and preprocessing module provides multiple input options including camera capture and gallery selection, accommodating different user preferences and scenarios. The cropping functionality allows users to focus the extraction process on relevant document areas, improving accuracy by eliminating background noise and irrelevant visual elements. The preprocessing pipeline includes automatic image enhancement and format standardization to optimize input for the deep learning models.

The data review and editing interface presents extracted information in an intuitive form layout that allows users to verify and correct any extraction errors. The interface includes intelligent form validation and suggestion features that help users complete missing information efficiently. Category selection is implemented through a user-friendly picker interface that supports both predefined categories and custom classification options.

Local data storage utilizes SQLite for offline functionality and data persistence, ensuring that user data remains accessible without network connectivity. The storage schema is designed to support future features such as expense analytics, budgeting, and data export functionality. Data synchronization capabilities are architected to support future cloud storage integration while maintaining user privacy and data ownership.

\subsection{Backend Service Architecture}
The DonutAPI backend service implements a microservices-inspired architecture that isolates model inference from other system concerns. The service exposes RESTful endpoints for document processing, supporting both synchronous and asynchronous processing modes depending on model complexity and expected processing time. The API design follows OpenAPI specifications, providing comprehensive documentation and enabling easy integration with client applications.

Model management within the backend service supports multiple Donut model instances optimized for different document types. The base model, fine-tuned on the CORD-v2 dataset, specializes in paper receipt processing and handles traditional merchant receipts with standardized layouts. The custom model, trained on the QRIS-TF dataset, focuses on digital payment documents from Indonesian financial applications including GoPay, SeaBank, Neobank, and BCA transfer receipts.

Request processing includes comprehensive input validation, image preprocessing, and result formatting to ensure consistent API behavior. Error handling mechanisms provide detailed feedback for various failure modes including invalid image formats, processing timeouts, and model inference errors. Logging and monitoring capabilities support production deployment and performance optimization.

\begin{figure}[htbp]
    \centerline{\includegraphics[width=0.45\textwidth]{images/component-diagram.png}}
    \caption{Component Diagram of the System}
    \label{fig:component}
\end{figure}

\subsection{Development Methodology}
The system development follows the Design Science Research Methodology (DSRM) framework, which provides a structured approach to developing and evaluating technology artifacts for solving identified problems. The methodology encompasses six phases: problem identification and motivation, objective definition for solution, design and development, demonstration, evaluation, and communication.

\begin{figure}[htbp]
    \centerline{\includegraphics[width=0.45\textwidth]{images/design-flow.png}}
    \caption{Planned Development Flow}
    \label{fig:devflow}
\end{figure}

The development flow illustrated in Figure \ref{fig:devflow} begins with comprehensive problem analysis focusing on Gen Z expense tracking challenges in the QRIS payment ecosystem. Objective definition establishes clear success criteria including accuracy thresholds for data extraction, usability benchmarks for user experience, and performance requirements for mobile deployment.

The design and development phase encompasses both data science and software engineering activities. Dataset preparation involves collecting and annotating payment documents from various sources, ensuring representation of real-world format diversity. Model development includes fine-tuning pre-trained Donut models on domain-specific datasets, with careful attention to overfitting prevention and generalization performance.

\subsection{Dataset Preparation and Model Training}
The research utilizes two complementary datasets to address different document processing requirements. The CORD-v2 dataset provides a foundation for general receipt processing, containing 1,000 annotated images of paper receipts with structured ground truth labels. This publicly available dataset enables comparison with existing research while providing sufficient data for transfer learning.

The custom QRIS-TF dataset addresses the specific requirements of Indonesian digital payment processing, containing 200 carefully curated images representing payment receipts from major Indonesian financial applications. Dataset creation involved systematic collection of QRIS payment receipts, bank transfer confirmations, and digital wallet transactions across different applications and merchant types. Annotation follows a structured schema that captures essential transaction information including amounts, timestamps, merchant identifiers, and transaction types.

Data preprocessing includes image normalization, format standardization, and augmentation techniques to improve model robustness. Augmentation strategies include rotation, scaling, and lighting variation to simulate real-world capture conditions. Training procedures implement early stopping and learning rate scheduling to optimize convergence while preventing overfitting on the relatively small custom dataset.

\begin{figure}[htbp]
    \centerline{\includegraphics[width=0.45\textwidth]{images/deployment-diagram.png}}
    \caption{Deployment Architecture}
    \label{fig:deployment}
\end{figure}

The deployment architecture shown in Figure \ref{fig:deployment} illustrates the production environment configuration. The mobile application communicates with the backend service through HTTPS REST API calls, ensuring secure data transmission. The backend service is containerized using Docker for consistent deployment across different cloud platforms. Model artifacts are versioned and managed separately to enable independent model updates without service disruption.

\chapter{Desain dan Implementasi Sistem Ekstraksi Data Struk dan Bukti Pembayaran dengan Model Donut}
\label{chapter:desain-implementasi}



\section{Tahapan Desain}
\label{sec:tahapan-desain}

\begin{figure}[htbp]
    \centering
    \includegraphics[width=1\textwidth]{images/design-flow.png}
    \caption{Alur kerja tahapan desain sistem}
    \label{fig:design-flow}
\end{figure}

\autoref{fig:design-flow} menunjukkan tahapan desain yang dilakukan untuk mencapai implementasi sistem. Tahapan desain utama mencakup mempersiapkan \dataset{} dan merancang arsitektur sistem. Tahapan desain yang dihasilkan akan menjadi dasar untuk membangun sistem pencatatan pengeluaran berbasis \emph{mobile}.

\subsection{Strategi Persiapan Dataset}
\label{subsec:strategi-persiapan-dataset}

Persiapan \dataset{} merupakan tahapan kritis dalam pengembangan sistem ekstraksi data pembayaran. Strategi persiapan \dataset{} dirancang untuk memastikan kualitas dan representativitas data yang akan digunakan untuk melatih model \donut{} yang telah disesuaikan dengan domain pembayaran Indonesia.

\datasetfl{} yang disiapkan mencakup gambar bukti pembayaran dari berbagai jenis aplikasi pembayaran digital dan struk kertas. \datasetfl{} struk kertas yang digunakan untuk melatih model \donut{} adalah \dataset{} CORD-V2\footnote{\href{https://huggingface.co/datasets/naver-clova-ix/cord-v2}{https://huggingface.co/datasets/naver-clova-ix/cord-v2}} (\emph{Consolidated Receipt Dataset for Post-OCR Parsing}). \datasetfl{} ini dibagi menjadi 800 data latih, 100 data validasi, dan 100 data uji. \datasetfl{} ini digunakan untuk melatih model \donut{} agar dapat mengenali dan mengekstrak informasi dari struk pembayaran. \datasetfl{} ini memiliki keterbatasan, yaitu memiliki bagian yang di-\emph{blur}. Keterbatasan ini membuat \donut{} mengalami kesulitan untuk memprediksi struk pembayaran yang penuh.
\begin{figure}[htbp]
    \centering
    \includegraphics[width=1\textwidth]{images/dataset-preparation-flow.png}
    \caption{Alur kerja persiapan \dataset{} QRIS dan transfer bank}
    \label{fig:dataset-preparation-flow}
\end{figure}

\datasetfl{} yang digunakan untuk melatih model \donut{} adalah \dataset{} QRIS dan transfer yang berisi gambar bukti pembayaran dari berbagai aplikasi pembayaran digital seperti SeaBank, Neobank, BCA, dan \gopay{} yang perlu dipersiapkan secara sistematis. \autoref{fig:dataset-preparation-flow} menunjukkan alur kerja sistematis dalam persiapan \dataset. Proses dimulai dengan melakukan transaksi melalui aplikasi pembayaran dan mengumpukan gambar bukti pembayaran dari berbagai sumber, yaitu aplikasi pembayaran digital (SeaBank, Neobank, BCA, dan \gopay). Total \dataset{} yang dikumpulkan adalah 350 gambar bukti pembayaran QRIS dan transfer dengan format dan jenis dokumen yang bervariasi.

Gambar bukti pembayaran yang telah dikumpulkan kemudian dianotasi secara manual. Proses anotasi dilakukan secara manual untuk menandai informasi penting pada gambar dan memastikan bahwa data yang dianotasikan sesuai. Atribut yang dianotasikan adalah sebagai berikut.
\begin{enumerate}
    \item \iden{}: Kode transaksi yang unik untuk setiap bukti pembayaran.
    \item \total{}: Jumlah pembayaran yang tercantum pada bukti pembayaran.
    \item \ttime{}: Waktu transaksi yang tercatat pada bukti pembayaran.
    \item \target{}: Nama penerima atau tujuan pembayaran.
    \item \app{}: Aplikasi yang digunakan untuk melakukan transaksi.
    \item \type{}: Jenis dokumen berupa QRIS atau transfer.
\end{enumerate}

Proses anotasi ini dilakukan dengan cermat untuk memastikan bahwa setiap informasi yang diperlukan dapat diekstraksi dengan akurat oleh model \donut{}. Hasil anotasi kemudian disimpan dalam format JSON dan kemudian dikonversi menjadi format yang sesuai untuk pelatihan model \donut{}, yaitu format JSONL. 

\datasetfl{} yang telah dianotasi ini kemudian dibagi menjadi tiga subset, yaitu data latih, data validasi, dan data uji. Pembagian dilakukan dengan mempertimbangkan distribusi jenis dan variasi jenis dokumen untuk memastikan bahwa setiap subset mencakup representasi yang baik dari keseluruhan \dataset. Data latih digunakan untuk \emph{fine-tuning} model, data validasi untuk proses pelatihan, sedangkan data uji digunakan untuk evaluasi akhir kinerja model. 

\emph{Fine-tuning} model kemudian dilakukan untuk memastikan data yang cukup dan tidak mengandung bias yang dapat menyebabkan \emph{overfitting} pada model. Proses pengumpulan data kemudian dihentikan setelah jumlah data dinilai cukup untuk melatih model \donut{} dengan baik.

\subsection{Merancang Arsitektur dan Desain Integrasi Sistem}
\label{subsec:merancang-aristektur-dan-desain-integrasi-sistem}

Subbab ini menjelaskan mengenai struktur dan organisasi sistem yang akan dibangun dari berbagai pandangan arsitektur yang digunakan untuk menggambarkan sistem. Rancangan sistem dibuat dengan mengacu pada kebutuhan fungsional dan non-fungsional yang telah 
dianalisis sebelumnya untuk memastikan sistem yang dibangun memenuhi tujuan serta kebutuhan pengguna.

Pendekatan yang digunakan untuk membuat desain arsitektur ini adalah 4+1 View Model. Model ini mengorganisasikan arsitektur sistem ke dalam lima pandangan yang berbeda, yaitu \emph{scenarios view}, \emph{logical view}, \emph{process view}, \emph{development view}, dan \emph{physical view}. Setiap pandangan memberikan perspektif yang berbeda terhadap sistem sehingga dapat memberikan gambaran yang komprehensif mengenai arsitektur sistem yang akan dibangun. Pendekatan ini memungkinkan proses pengembangan sistem menjadi lebih terstruktur dan sistematis.

\input{chapters/ta/chapter-4/tahapan-desain/architecture-views/use-case-view.tex}

\input{chapters/ta/chapter-4/tahapan-desain/architecture-views/logical-view.tex}

\input{chapters/ta/chapter-4/tahapan-desain/architecture-views/process-view.tex}


\input{chapters/ta/chapter-4/tahapan-desain/architecture-views/development-view.tex}

\input{chapters/ta/chapter-4/tahapan-desain/architecture-views/physical-view.tex}

\section{Hasil Desain}
\label{sec:hasil-desain}



\section{Kesimpulan dan Saran}

\subsection{Kesimpulan}
Penelitian ini berhasil mengembangkan sistem pencatatan pengeluaran berbasis mobile TrackMyBills yang menggunakan model Donut untuk ekstraksi data dari bukti pembayaran QRIS, transfer, dan struk tanpa OCR. Berdasarkan evaluasi komprehensif yang dilakukan, dapat disimpulkan sebagai berikut:

\begin{enumerate}
    \item \textbf{Fungsionalitas Sistem}: Seluruh 19 skenario pengujian fungsionalitas berhasil dijalankan dengan baik, menunjukkan bahwa sistem dapat menangani berbagai jenis input dan kasus penggunaan sesuai dengan spesifikasi yang ditetapkan.
    
    \item \textbf{Kinerja Model}: Kedua model yang dikembangkan menunjukkan performa yang memuaskan. Base model (CORD-v2) mencapai F1-score 84,68\% dengan MCER 18,85\%, sementara custom model (QRIS-TF) mencapai F1-score 81,50\% dengan MCER 17,20\% dan precision sempurna 100\%.
    
    \item \textbf{Pengalaman Pengguna}: Evaluasi menggunakan System Usability Scale (SUS) menghasilkan skor rata-rata 71,83, melampaui ambang batas standar usability (68). Sebanyak 73,3\% responden memberikan penilaian positif, dengan 80\% menyatakan akan menggunakan aplikasi untuk mempermudah pencatatan pengeluaran.
    
    \item \textbf{Kontribusi Teknologi}: Pendekatan end-to-end menggunakan model Donut terbukti efektif untuk ekstraksi data dokumen finansial tanpa memerlukan tahap OCR yang dapat menimbulkan error propagation.
    
    \item \textbf{Dampak Behavioral}: 60\% responden merasa akan menjadi lebih disiplin dalam mencatat pengeluaran dengan adanya aplikasi TrackMyBills, menunjukkan potensi dampak positif terhadap kebiasaan pengelolaan keuangan Gen Z.
\end{enumerate}

Hasil penelitian ini membuktikan bahwa teknologi Computer Vision dan Deep Learning dapat diimplementasikan secara efektif untuk mengatasi masalah pencatatan pengeluaran manual yang dihadapi Gen Z dalam era pembayaran digital.

\subsection{Saran Pengembangan Selanjutnya}
Berdasarkan hasil evaluasi dan limitasi yang ditemukan, penelitian selanjutnya dapat difokuskan pada area-area berikut:

\begin{enumerate}
    \item \textbf{Unified Model}: Mengembangkan satu model tunggal yang dapat menangani semua jenis dokumen pembayaran (QRIS, transfer, dan struk) dengan task prompt yang sesuai untuk mengurangi kompleksitas sistem.
    
    \item \textbf{Dataset Enhancement}: Memperbesar dan memperbanyak variasi dataset, terutama untuk dokumen pembayaran digital, agar model dapat menangani format yang lebih beragam dengan akurasi yang lebih tinggi.
    
    \item \textbf{Holistic Document Understanding}: Mengembangkan model yang dapat memahami dokumen secara holistik tanpa memerlukan cropping manual, dengan melatih pada dataset yang mencakup seluruh konteks dokumen.
    
    \item \textbf{Fitur Lanjutan}: Menambahkan fitur integrasi dengan layanan perbankan, penyimpanan cloud, dan analisis pengeluaran berbasis AI untuk memberikan insight yang lebih mendalam kepada pengguna.
    
    \item \textbf{Cross-Platform Support}: Mengembangkan versi iOS dan web application untuk menjangkau audiens yang lebih luas.
    
    \item \textbf{Real-world Deployment}: Melakukan studi longitudinal untuk mengukur dampak jangka panjang aplikasi terhadap kebiasaan pengelolaan keuangan pengguna dalam kondisi penggunaan sehari-hari.
\end{enumerate}

Pengembangan lebih lanjut diharapkan dapat meningkatkan akurasi ekstraksi, memperluas cakupan dokumen yang dapat diproses, dan memberikan solusi yang lebih komprehensif untuk kebutuhan pengelolaan keuangan digital Gen Z di Indonesia.

\chapter{Kesimpulan dan Saran}

\section{Kesimpulan}
\label{sec:kesimpulan}

\section{Saran}
\label{sec:saran}
Berdasarkan hasil evaluasi dan diskusi yang telah dilakukan, terdapat beberapa saran yang dapat diberikan untuk pengembangan sistem TrackMyBills ke depannya:
\begin{enumerate}
    \item 
\end{enumerate}
%---------------------------------------------------------------%

% Daftar pustaka
\printbibliography{}

% Setting judul lampiran
\titlespacing*{\chapter}{0pt}{0pt}{0pt}
\titlespacing*{\section}{0pt}{0pt}{*1}

% Setting judul anak lampiran - maintain 12pt consistency
\titleformat*{\section}{\normalfont\normalsize\bfseries}

\appendix

\chapter{Tautan Penting}
\label{chapter:tautan-penting}

\section{Tautan \emph{Script} Fine-Tuning}
\href{https://www.kaggle.com/code/vincentfranstyo/fine-tuned-for-qris-tf/}{https://www.kaggle.com/code/vincentfranstyo/fine-tuned-for-qris-tf/}

\section{Tautan \emph{Script} Evaluasi \emph{Base Model}}
\href{https://www.kaggle.com/code/vincentfranstyo/donut-fine-tuned-on-cord-v2-evaluation/}{https://www.kaggle.com/code/vincentfranstyo/donut-fine-tuned-on-cord-v2-evaluation/}

\section{Tautan \emph{Script} Evaluasi \emph{Custom Model}}
\href{https://www.kaggle.com/code/vincentfranstyo/qris-tf-eval-new}{https://www.kaggle.com/code/vincentfranstyo/qris-tf-eval-new}

\section{Tautan Repositori Kode Aplikasi \emph{Mobile} TrackMyBills}
\href{https://github.com/FYP-TrackMyBills/TrackMyBills}{https://github.com/FYP-TrackMyBills/TrackMyBills}

\section{Tautan Repositori Kode Layanan \emph{Backend} DonutAPI}
\href{https://github.com/FYP-TrackMyBills/Donut-Backend}
{https://github.com/FYP-TrackMyBills/Donut-Backend}

\section{Tautan Survei Evaluasi Pengguna}
\href{https://forms.gle/LwDiEJtcAL2gKyBfA}{https://forms.gle/LwDiEJtcAL2gKyBfA}

\section{Tautan Hasil Survei Pengguna}
\href{https://docs.google.com/spreadsheets/d/1a2Q1Kio0whC7-fwUD3y7AQzv5Cw0PaTOwEdsp9prtck/edit?resourcekey=&gid=529123852#gid=529123852}{Google Sheets: Hasil Survei Pengguna}
\chapter{Tangkapan Layar Antarmuka}
\label{chapter:tangkapan-layar-antarmuka}

\begin{tabular}{lll}
    \includegraphics[width=0.33\textwidth]{images/UI/choose-document-2.jpg} &
    \includegraphics[width=0.33\textwidth]{images/UI/extracting.jpg} &
    \includegraphics[width=0.33\textwidth]{images/UI/struk-edit-ta.jpg}
    \\
    \includegraphics[width=0.33\textwidth]{images/UI/expense-0.jpg} &
    \includegraphics[width=0.33\textwidth]{images/UI/expense-clear.jpg} &
    \includegraphics[width=0.33\textwidth]{images/UI/expense.jpg} \\
\end{tabular}

\begin{tabular}{lll}
    \includegraphics[width=0.33\textwidth]{images/UI/qris-cam.jpg} &
    \includegraphics[width=0.33\textwidth]{images/UI/qris-cat.jpg} &
    \includegraphics[width=0.33\textwidth]{images/UI/qris-a.jpg}
    \\
    \includegraphics[width=0.33\textwidth]{images/UI/qris-confirm.jpg} &
    \includegraphics[width=0.33\textwidth]{images/UI/qris-galery.jpg} &
    \includegraphics[width=0.33\textwidth]{images/UI/qris-mapping.jpg} \\
\end{tabular}

\begin{tabular}{lll}
    \includegraphics[width=0.33\textwidth]{images/UI/qris-recheck.jpg} &
    \includegraphics[width=0.33\textwidth]{images/UI/qris-review.jpg} &
    \includegraphics[width=0.33\textwidth]{images/UI/qris-t.jpg}
    \\
    \includegraphics[width=0.33\textwidth]{images/UI/qris-ta.jpg} &
    \includegraphics[width=0.33\textwidth]{images/UI/qris-td.jpg} &
    \includegraphics[width=0.33\textwidth]{images/UI/qris-tid.jpg} \\
\end{tabular}

\begin{tabular}{lll}
    \includegraphics[width=0.33\textwidth]{images/UI/qris-tn.jpg} &
    \includegraphics[width=0.33\textwidth]{images/UI/qris-tt.jpg} &
    \includegraphics[width=0.33\textwidth]{images/UI/qris-upload.jpg}
    \\
    \includegraphics[width=0.33\textwidth]{images/UI/struk-add.jpg} &
    \includegraphics[width=0.33\textwidth]{images/UI/struk-cat.jpg} &
    \includegraphics[width=0.33\textwidth]{images/UI/struk-confirm.jpg} \\
\end{tabular}

\begin{tabular}{lll}
    \includegraphics[width=0.33\textwidth]{images/UI/struk-crop.jpg} &
    \includegraphics[width=0.33\textwidth]{images/UI/struk-cropped.jpg} &
    \includegraphics[width=0.33\textwidth]{images/UI/struk-edit.jpg}
    \\
    \includegraphics[width=0.33\textwidth]{images/UI/struk-mapping.jpg} &
    \includegraphics[width=0.33\textwidth]{images/UI/struk-review.jpg} &
    \includegraphics[width=0.33\textwidth]{images/UI/struk-unit.jpg} \\
\end{tabular}
\begin{tabular}{lll}
    \includegraphics[width=0.33\textwidth]{images/UI/sharing-intent.jpg} &
    \includegraphics[width=0.33\textwidth]{images/UI/shared-confirm.jpg} &
    \includegraphics[width=0.33\textwidth]{images/UI/mock-data.jpg}
    \\
\end{tabular}
\chapter{Hasil Survei Pengguna}

\begin{figure}[htbp]
	\centering
	\includegraphics[width=0.5\textwidth]{resources/cover-ganesha.jpg}
	\caption{Contoh gambar}
\end{figure}
\chapter{Bentuk Anotasi}
\label{chap:bentuk-anotasi}

\section{Bentuk Anotasi Model \donut{} \emph{Fine-tuned} pada \datasetfl{} CORD-v2}
\begin{lstlisting}[style=jsonstyle]
{
  "gt_parse": {
    "menu": [
      {
        "nm": "REAL GANACHE",
        "cnt": "1",
        "price": "16,500"
      },
      {
        "nm": "EGG TART",
        "cnt": "1",
        "price": "13,000"
      }
    ],
    "total": {
      "total_price": "45,500",
    }
  }
}
\end{lstlisting}

\section{Bentuk Anotasi Model \donut{} \emph{Fine-tuned} pada \datasetfl{} QRIS-Transfer}
\begin{lstlisting}[style=jsonstyle]
{
    "file_name": "bri_qris_1.jpg",
    "ground_truth": {
        "gt_parse": {
            "total_amount": 261030,
            "transaction_time": "2025-04-07T15:08:34Z",
            "transaction_identifier": "825188939324",
            "type": "QRIS",
            "target_name": "Wheels Coffee Roasters BR",
            "application": "BRI"
        }
    }
}
\end{lstlisting}


\end{document}
